% @HEADER
% ***********************************************************************
%
%            Trilinos: An Object-Oriented Solver Framework
%                 Copyright (2001) Sandia Corporation
%
% Under terms of Contract DE-AC04-94AL85000, there is a non-exclusive
% license for use of this work by or on behalf of the U.S. Government.
%
% This library is free software; you can redistribute it and/or modify
% it under the terms of the GNU Lesser General Public License as
% published by the Free Software Foundation; either version 2.1 of the
% License, or (at your option) any later version.
%
% This library is distributed in the hope that it will be useful, but
% WITHOUT ANY WARRANTY; without even the implied warranty of
% MERCHANTABILITY or FITNESS FOR A PARTICULAR PURPOSE.  See the GNU
% Lesser General Public License for more details.
%
% You should have received a copy of the GNU Lesser General Public
% License along with this library; if not, write to the Free Software
% Foundation, Inc., 59 Temple Place, Suite 330, Boston, MA 02111-1307
% USA
% Questions? Contact Michael A. Heroux (maherou@sandia.gov)
%
% ***********************************************************************
% @HEADER

\documentclass[11pt,relax]{SANDreport}
\usepackage{graphicx}
\usepackage{amsmath,amsfonts,amsthm}
\usepackage{amssymb}
\usepackage{enumerate}
\usepackage{rotating}
\usepackage{times}

\def\choicebox#1#2{\noindent$\hphantom{th}$\parbox[t]{1.8in}{\sf
#1}\parbox[t]{4.5in}{#2}\\[0.8em]}

\author{Robert Hoekstra, Alan Williams, Michael Heroux\\
Computational Mathematics and Algorithms Department \\
Sandia National Laboratories \\
P.O. Box 5800 \\
Albuquerque, NM 87185-1110 \\[1cm]
Marzio Sala \\
Institute of Computational Science, \\
ETH Zurich, Switzerland}

\title{Serialization Tools for Distributed Linear Algebra Objects}
\SANDnum{SAND2006-XXXX}
\SANDauthor{
Robert Hoekstra, Alan Williams, Michael Heroux, and Marzio Sala}

\SANDprintDate{April 2006}
\SANDreleaseType{Unlimited Release}

\newcommand{\Trilinos}{Trilinos}
\newcommand{\TrilinosTM}{Trilinos \copyright}
\newcommand{\trilinos}{{\sc Trilinos}}
\newcommand{\ifpack}{{\sc Ifpack}}
\newcommand{\aztecoo}{{\sc AztecOO}}
\newcommand{\amesos}{{\sc Amesos}}
\newcommand{\epetra}{{\sc Epetra}}
\newcommand{\ml}{{\sc ML}}
\newcommand{\mb}[1]{{\mathbf {#1} }}
\newcommand{\teuchos}{{\sc Teuchos}}
\newcommand{\triutils}{{\sc Triutils}}
\newcommand{\metis}{{\sc METIS}}

\newcommand{\ie}{i.e., }
\newtheorem{remark}{Remark}

\begin{document}

\maketitle

\begin{abstract}
We describe  the I/O tools available in the EpetraExt package of Trilinos~\cite{trilinos-home-page}
\end{abstract}

\SANDmain

\tableofcontents

\newpage

% ============================================================================
\section{Introduction}
\label{sec:introduction}
% ============================================================================

Consider the following scenarios: 
\begin{enumerate}
\item A programmer is experiencing convergence
problems within his/her Newton-like nonlinear solver, because the
preconditioner does not yield convergence as expected. The application is
non-standard, custom-made, proprietary, and the source codes cannot be
distributed. 
\item During a time-dependent
simulation, a particular linear system solve phase does not behave as expected. 
This occurs after several
hours of computations, on a quite large problem, running on distributed memory
machines.
\end{enumerate}
Developers of both scenarios will benefit from being able to reproduce their
linear algebra objects {\sl outside} their codes, in a much simpler
environment, that does not exhibit
all the additional complexity typical of large application
codes. The process of moving objects from one application to the next is
typically called {\sl serialization}.

In the modern computer science context, serialization refers to
converting an object's detailed state data into a single byte stream to
transmit the object from its current location to a new location.  The most
important goal of serialization is to re-create an object that is identical to
the original object. This type of serialization is used mostly to transport an
object across a network, to persist objects to a file or database, or to
distribute identical objects to several applications or locations.
Serialization can also be used to transform an object into a human-readable
state. 
The processing of serializing an object is also known are {\sl deflating} or
{\sl marshalling} an object. The opposite operation of extracting a data
structure from a series of bytes, is called {\sl deserialization}, or {\sl
  unmarshalling} or {\sl inflating}. 

\smallskip

Tools for serialization/deserialization are important because they can
provide:
\begin{itemize}
\item A simple and robust way to make objects persistent;
\item A method for distributing objects across the network (e.g., with MPI);
\item An easy and convenient way to store or access standard problems for particular algorithms or techniques. 
\item A mechanism to send and receive objects from other nodes in the Grid;
\item A convenient format to upload or download linear algebra objects for
web-based solvers;
\item A tool to restart a simulation.
\end{itemize}
In this document, we are interested in tool to serialize and deserialize
distributed linear algebra objects, internally defined as Epetra
object~\cite{Epetra-Users-Guide}.  The serialization techniques we consider in
this document aim to define flexible file data formats. The Epetra objects we
will consider in this document are reported in Table \ref{tab:supported}.
These objects are characterized by the
following proporties:
\begin{itemize}
\item they are distributed across multiple processors, and therefore some for
of parallel I/O is required;
\item their data layout is determined by Epetra\_Map objects. The I/O should
be flexible enough to allow the clone object to have a different (tough
                                                                  compatible)
Epetra\_Map;
\item they require large chunks of memory;
\item the serialization and deserializaton process should allow I/O from
well-known formats, like MATLAB, MatrixMarket~\cite{boisvert97matrix}, and
Harwell/Boeing~\cite{duff89sparse}.
\end{itemize}
All the above objects
contain several pointers, which are too fragile to be saved. Therefore,
        the serialization process
must include procedures to convert the pointed memory into a local state, then
save this state into the final media. This means that the object has to be
converted into an intermediate state, which will be used to perform the
serialization itself. An analogous procedure occurs for deserialization. This
process is sometimes referred to as {\sl pointer unswizzling}.

\begin{table}
\begin{center}
\begin{tabular}{|l | c | c | c |}
\hline
object type           & H/B       & MM        & HDF5 \\
\hline
Epetra\_BlockMap      & O  & I/O & I/O \\
Epetra\_Map           & O  & I/O & I/O \\
Epetra\_IntVector     & -- & --  & I/O \\
Epetra\_Vector        & O  & I/O & I/O \\
Epetra\_MultiVector   & O  & I/O & I/O \\
Epetra\_CrsGraph      & -- & --  & I/O \\
Epetra\_RowMatrix     & O  &   O &   O \\
Epetra\_CrsMatrix     & O  & I/O & I/O \\
Epetra\_VbrMatrix     & O  &   O &   O \\
Teuchos::ParameterList& -- & --  & I/O \\
int                   & -- & --  & I/O \\
double                & -- & --  & I/O \\
void* array           & -- & --  & I/O \\
\hline
\end{tabular}
\caption{List of objects supported by the various I/O tools. ``H/B'' refers to
  the Harwell/Boeing format, ``MM'' to the MatrixMarket format, and ``HDF5''
    to the HDF5 format. `I/O' means that both input and output are supported,
       while `O' that only output, `--' means no support.}
\label{tab:supported}
\end{center}
\end{table}

\smallskip

This manuscript outlines the serialization and deserialization tools available
within the EpetraExt package of Trilinos. The design requirements are outlined
in Section~\ref{sec:design}.
Section~\ref{sec:hb} describes the
Harwell/Boeing I/O; section~\ref{sec:mm} presents the MatrixMarket I/O;
section~\ref{sec:hdf5} outlines the design and the implementation of the
EpetraExt::HDF5 I/O. 

% ============================================================================
\section{Design Overview}
\label{sec:design}
% ============================================================================

% ============================================================================
\subsection{Requirements}
% ============================================================================

The serialization techniques considered in this report aim to satisfy the
following requirements:
\begin{itemize}
\item 
{\sl Code portability}: the ASCII serialization only depends on ANSI
C/C++ capabilities; the binary serialization has a dependency on a well-know
library, based on the widely-available MPI-IO;
\item 
{\sl Deep copy behavior}: all data required to reconstruct the object is
serialized;
\item 
{\sl Data Portability}: ASCII and binary streams of bytes created on one
platform are readable on any other;
\item
{\sl Non-intrusive}: Serialization tools can be applied to unaltered classes,
  that is, to classes developed independently of the serialization tools.
        That is, it is not required for the classes to be serialized to be
        derived from a specific base class or implement specified member
        functions. 
\item
{\sl Simplicity}: The archive interface is simple enough to easily permit
creation of a new type of archive.
\item
{\sl Meta-data storage}
It is convenient to store {\sl metadata} together with the input and
output of a simulation. The simplest metadata is, for example, the name of the
object, or the class it belongs to. More sophisticated metadata can include 
details on the procedure used to build the object, like for instance the
quadrature formula used to assemble the matrix, or the parameters used to
generate the preconditioner.
\end{itemize}

% ============================================================================
\subsection{Implementation}
% ============================================================================

Probably, the most critical aspect is the distributed nature of the object,
  which calls for efficient parallel I/O.
Parallel I/O is a mechanism that allow many
processors to perform I/O at the same time, with the goal of exploring
parallelism in the parallel computer's communication network and I/O system.

The parallel programmer can take two different approaches to achieving
concurrency I/O operations. One approach is for each process to perform read
and write operations to a distinct file. While simple, this approach has
significant disadvantages: programs cannot easily be restarted on different
number of processors, the underlying file system has little information on
which to base optimization decisions, and files are not easily shared with
other programs. We decided to generate one single file, containing the entire
object. This is accomplished with different technical tools, depending on the
I/O format. For example, one can first reproduce the whole object on one
processor, then perform serial I/O, or let each processor write its part of
the object on the file. 

Another design choice is  on the storage type of the resulting file, which 
can be either human-readable (ASCII) or binary.
Human-readable objects are convenient when dealing with small objects.
Besides, they can easily edited or created with simple text editors, and at
least two formats have already been proposed in the literature: the
Harwell/Boeing file format, and the MatrixMarket file format. Another common
format, XML, is not considered here. The
disadvantage of human-readable file formats is that their ASCII content is not
space-efficient, and the I/O is typically much slower than binary I/O. To 
avoid these problems,
one can consider binary serialization. In order to make the resulting files
portable across different architectures, it is convenient to perform all the
binary I/O through some kind of library, and use some form of parallel I/O
techniques based on the widely available MPI-IO library.  This means that the
simple and fast procedure of directly copying the memory layout of the data
structure cannot work reliably for all architectures, and should instead be
replaced by more advanced techniques.

Another design choices must be considered.
The standard encoding method uses a simple translation of the field into a
byte steam, which contains all members of the object. This stream might
include other objects, referenced by the object currently under serialization;
all these objects must be in a serializable state as well. Alternatively, the
serialization must record enough information to instantiate these reference
objects without storing them in the stream. This process is sometimes refer to
as {\sl truncation}. We use truncation for all serialization tools presented in this
document.

% ============================================================================
\section{The Harwell-Boeing I/O}
\label{sec:hb}
% ============================================================================

Harwell-Boeing files containing real matrices can be read using the following
sequence of commands:
\begin{verbatim}
Epetra_Map* readMap;
Epetra_CrsMatrix* readA; 
Epetra_Vector* readx; 
Epetra_Vector* readb;
Epetra_Vector* readxexact;

Trilinos_Util_ReadHb2Epetra((char*)FileName_.c_str(), *comm_, readMap,
                            readA, readx, readb, readxexact);
\end{verbatim}
This I/O function is located in the Triutils package; the header file
\verb!Trilinos_Util.h! contains the signature of the function. Note that this
reader is only serial, and the objects have to be exported to distributed maps
in order to perform parallel computations.

% ============================================================================
\section{The Matrix-Market I/O}
\label{sec:mm}
% ============================================================================

EpetraExt offers a variety of Matrix-Market I/O tools.
The Matrix Market format is a simple, portable and human-readable (ASCII) 
format. The
specifications to create the corresponding ASCII files can be found at
\begin{verbatim}
http://math.nist.gov/MatrixMarket/
\end{verbatim}
where MATLAB files to perform I/O can be downloaded. 
The header files to be includes are
\begin{verbatim}
#include "EpetraExt_BlockMapIn.h"
#include "EpetraExt_BlockMapOut.h"
#include "EpetraExt_VectorIn.h"
#include "EpetraExt_VectorOut.h"
#include "EpetraExt_MultiVectorIn.h"
#include "EpetraExt_MultiVectorOut.h"
#include "EpetraExt_CrsMatrixIn.h"
#include "EpetraExt_RowMatrixOut.h"
\end{verbatim}
All classes offer both serial and parallel support.
To show the usage, let us consider for example the following objects:
\begin{verbatim}
Epetra_Map Map(10, 0, Comm);
Epetra_MultiVector X(Map, 2);
Epetra_MultiVector B(Map, 2);
Epetra_CrsMatrix A(Copy, Map, 0);  // create a simple diagonal matrix  
for (int i = 0 ; i < NumMyElements ; ++i)  
{
  int j = Map.GID(i);
  double value = 1.0;    
  EPETRA_CHK_ERR(A.InsertGlobalValues(j, 1, &value, &j));
}
A.FillComplete();
\end{verbatim}
These objects can be saved on file
in both serial and parallel computations by using the commands
\begin{verbatim}
EpetraExt::MapToMatrixMarketFile("Map.mm", Map);
EpetraExt::MultiVectorToMatrixMarketFile("X.mm", X);
EpetraExt::MultiVectorToMatrixMarketFile("B.mm", B);
EpetraExt::RowMatrixToMatrixMarketFile("A.mm", A);
\end{verbatim}
By using the \verb!mmread.m! MATLAB
script available at the Matrix Market web site, one can type within MATLAB
\begin{verbatim}
>> X = mmread('X.mm');
>> A = mmread('A.mm')                                       

A =

   (1,1)        1
   (2,2)        1
   (3,3)        1
   (4,4)        1
   (5,5)        1
\end{verbatim}
Alternatively, one can read the
previous files within Trilinos as follows:
\begin{verbatim}
Epetra_Map* newMap;
Epetra_MultiVector* newX;
Epetra_MultiVector* newB;
Epetra_CrsMatrix* newA;

EpetraExt::MatrixMarketFileToMap("Map.mm", Comm, newMap);
EpetraExt::MatrixMarketFileToMultiVector("X.mm", *newMap, newX);
EpetraExt::MatrixMarketFileToMultiVector("B.mm", *newMap, newB);
EpetraExt::MatrixMarketFileToCrsMatrix("A.mm", *newMap, newA);
\end{verbatim}
Note that one has to call method {\tt FillComplete()} to finalize the
structure of the Epetra\_CrsMatrix read by MatrixMarketFileToCrsMatrix.

% ============================================================================
\subsection{The Python Interface}
\label{sec:mm:python}
% ============================================================================

EpetraExt offers a Python interface to the MatrixMarket reader through the
PyTrilinos project~\cite{pytrilinos-la-guide,sala05pytrilinos}.  Example of
I/O using the Python interface is reported in Figure~\ref{fig:python}.

\begin{sidewaystable}
\begin{center}
\begin{tabular}{| c |}
\hline
\begin{minipage}{17cm}
\begin{verbatim}

from PyTrilinos import EpetraExt, Epetra
# Build a global communicator
comm    = Epetra.PyComm()
# Construct a vector x and populate with random values
n       = 10 * numProc
map     = Epetra.Map(n, 0, comm)
x       = Epetra.Vector(map)
x.Random()

# create a matrix and store it on file
A       = Epetra.CrsMatrix(Epetra.Copy, map2, 0)
Indices = Epetra.IntSerialDenseVector(1)
Values  = Epetra.SerialDenseVector(1)
for lrid in range(A.NumMyRows()):
  grid = A.GRID(lrid)
  Indices[0] = grid
  Values[0]  = grid
  A.InsertGlobalValues(grid, 1, Values, Indices)
A.FillComplete()

# Part I: writing

EpetraExt.BlockMapToMatrixMarketFile("map.mm", map)
EpetraExt.MultiVectorToMatrixMarketFile("x.mm", x)
EpetraExt.RowMatrixToMatrixMarketFile("A.mm", A)

# Part II: reading

(ierr, map2) = EpetraExt.MatrixMarketFileToMap("map.mm", comm)
(ierr, y) = EpetraExt.MatrixMarketFileToMultiVector("x.mm", map2)
(ierr, B) = EpetraExt.MatrixMarketFileToCrsMatrix("A.mm", map2)

\end{verbatim}
\end{minipage} \\
    \hline
\end{tabular}
\caption{Example of Python code using the EpetraExt MatrixMarket
  capabilities.}
\label{fig:python}
\end{center}
\end{sidewaystable}

% ============================================================================
\section{HDF5 I/O}
\label{sec:hdf5}
% ============================================================================

Matrix-Market files have several advantages, but their usage for large-scale
problems is limited by their size. Also, there is no read parallel I/O in the
Matrix-Market tools, and therefore a non-negligible performance penalty may
occur when large number of processors are involved. In this cases, it is
convenient to adopt a fully-parallel, binary reader, as done by the HDF5 class
or EpetraExt. This class performs I/O by adopting the HDF5
library, which can be downloaded  at the web address:
\begin{verbatim}
http://hdf.ncsa.uiuc.edu/HDF5/
\end{verbatim}
HDF5 is a general purpose library and file format for storing scientific data.
HDF5 can store two primary objects: datasets and groups. A dataset is
essentially a multidimensional array of data elements, and a group is a
structure for organizing objects in an HDF5 file. Using these two basic
objects, one can create and store almost any kind of scientific data
structure, such as images, arrays of vectors, and structured and unstructured
grids. 

While building Trilinos, it should be enough to specify the location of the
header files using \verb!--with-incdirs!, the location of the library using
\verb!--with-ldflags!, and the HDF5 library using \verb!--with-libs!. Note
that one might need to add \verb!-lz! to the list of linked libraries; this library
is typically already installed on all system;

The HDF5 class of EpetraExt has the following advantages:
\begin{itemize}
\item The file format is binary and portable;
\item The file format can be read and written in parallel;
HDF5 is based on MPI-IO, and allows for true parallel I/O;
\item MATLAB contains a built-in HDF5 data reader, making it easy to interface
Trilinos application with MATLAB, and vice-versa;
\item The number of processors reading a given
data does not have to coincide with that used to write the data.
\end{itemize}


The HDF5 class has the following limitations:
\begin{enumerate}
\item Objects stored in a file cannot be deleted; if you want to do that, you should read the 
  content of the file, then create a new file and store on it the information to keep;
\item It is not possible to overwrite distributed objects.
\end{enumerate}


% ============================================================================
\subsection{An Example of Usage}
\label{sec:hdf5:example}
% ============================================================================

This section reports several examples of usage of the EpetraExt::HDF5 class. 
First, one has to create an HDF5 class, then either {\tt Open()} or 
{\tt Create()} the file:
\begin{verbatim}
EpetraExt::HDF5 HDF5(Comm);
HDF5.Create("myfile.h5");
\end{verbatim}
Let us consider that {\tt Map} is an Epetra\_Map object, {\tt BlockMap} an
Epetra\_BlockMap object, {\tt LHS} and {\tt RHS} two Epetra\_MultiVector
objects, and {\tt Matrix} an Epetra\_RowMatrix. To write these objects on a
file, simply do
\begin{verbatim}
HDF5.Write("map-" + toString(Comm.NumProc()), *Map);
HDF5.Write("matrix", *Matrix);
HDF5.Write("solution vector", LHS);
HDF5.Write("right-hand side", RHS);
\end{verbatim}
The HDF5 class also offers I/O for Teuchos::ParameterList objects. Types
within the list that are not {\tt
  int}, {\tt double} and {\tt string} types or sublists are ignored.

In all the above instructions, the first input variable is a user-defined name
that identifies the object. Note that the {\sl same} file can contain an
arbitrary number of objects, of the same type or of different type, as long as
the associated names differ. It is therefore possible, for example, to store
all the Jacobians and the associated right-hand sides in a Newton-type
procedures.

An advantage of the HDF5 file format is that 
it is very easy to add "meta-data" to your data.
HDF5 naturally stores
datatype information and ``metadata'' describing the rank, dimension, and
other data properties. Data are stored in datasets which can be queried by
names to find out the size, layout, and other properties.
For example, a string
containing information about the machine where the code was executed, the
parameter list used for the preconditioner or the nonlinear solver, the
convergence history, can all be save in the same file;
In fact, the HDF5 library itself can be used to define very general data
formats; this class, instead, is only structured around the concept of {\sl
  groups}. 
A {\sl group}
is an entity, like for example a scalar value, an Epetra\_Map, or a
Teuchos::ParameterList. Within each group, different datasets describe the
content of the group. Class EpetraExt::HDF5 serializes each Epetra object by
writing all the required data, plus other metadata that can be useful to query
the object properties without reading (and therefore building) the object
itself. The list of metadata is reported in Table~\ref{tab:metadata}.
For example, an Epetra\_MultiVector is specified by
datasets {\tt NumVectors} and {\tt Values}, which contain the number of vectors, and
the numerical values, respectively. The {\tt comment} field of each group is a
character string that must match the class name. A snippet of code reading the
metadata of a Epetra\_CrsMatrix is as follows:
\begin{verbatim}
int NumGlobalRows, NumGlobalCols, NumGlobalNonzeros;
int NumGlobalDiagonals, MaxNumEntries;
double NormOne, NormInf;

ReadCrsMatrixProperties(GroupName, NumGlobalRows, NumGlobalCols,
                        NumGlobalNonzeros, NumGlobalDiagonals, 
                        MaxNumEntries, NormOne, NormInf);
\end{verbatim}
We refer to the up-to-date Doxygen documentation of the EpetraExt::HDF5 class
for more details on the property reading methods.

\begin{table}
\begin{center}
\begin{tabular}{|l | p{10cm} |}
\hline
Epetra object & Associated metadata \\
\hline
Epetra\_BlockMap & NumGlobalElements,
                   NumGlobalPoints,
                   IndexBase,
                   NumProc \\
\hline
Epetra\_Map & NumGlobalElements,
              IndexBase,
              NumProc \\
\hline
Epetra\_IntVector & GlobalLength \\
    \hline
Epetra\_MultiVector & GlobalLength,
                      NumVectors \\
\hline
Epetra\_CrsGraph & NumGlobalRows,
                   NumGlobalCols,
                   NumGlobalNonzeros,
                   NumGlobalDiagonals,
                   MaxNumIndices\\
\hline
Epetra\_CrsMatrix & NumGlobalRows,
                    NumGlobalCols,
                    NumGlobalNonzeros,
                    NumGlobalDiagonals,
                    MaxNumEntries,
                    NormOne,
                    NormInf \\
\hline
\end{tabular}
\caption{Metadata associated with each  Epetra object. These metadata can be
  queries by using the appropriate read method.}
\label{tab:metadata}
\end{center}
\end{table}

The metadata reported in Table~\ref{tab:metadata} are a subset of what is
stored in the HDF5 file. The actual content of the file can be visualized by
using the command
\begin{verbatim}
$ h5dump filename.h5
\end{verbatim}
distributed within the HDF5 library. Note also that an arbitrary number of
metadata can be associated with any group in the file.  A group can contain
more than one dataset, and can be a new group or an already existing group.
For example, to specify the numerical quadrature formula used to assemble the
matrix, one can do as follows:
\begin{verbatim}
HDF5.Write("matrix", "quadrature order", 3);
\end{verbatim}
In this case, dataset {\tt quadrature order} is associated to group {\tt
  matrix}, which already contains an Epetra\_CrsMatrix object. 
Alternatively, datasets can be assigned to a new group, in this case
{\tt my parameters}:
\begin{verbatim}
HDF5.Write("my parameters", "latitude", 12);
HDF5.Write("my parameters", "longitude", 67);
HDF5.Write("my parameters", "angle", 12.3);
\end{verbatim}
Another data type can be stored on file are arrays. The type of the array is
specified using native HDF5 datatypes. Two important types are {\tt
  H5T\_NATIVE\_INT} and {\tt H5T\_NATIVE\_DOUBLE}. For example:
\begin{verbatim}
vector<int> iarray(3); 
iarray[0] = 0, iarray[1] = 1; iarray[2] = 2;
HDF5.Write("my parameters", "int array", H5T_NATIVE_INT, 
           3, &iarray[0]);
\end{verbatim}
Note that all non-distributed data are supposed to be in the same state on all
processors. Without considering Epetra ojbects, EpetraExt::HDF5 allows only
one type of distributed object, as detailed in section \ref{sec:hdf5:arrays}.

\smallskip

Reading data is as easy as writing them. For example, to read an
Epetra\_CrsMatrix one can do as follows:
\begin{verbatim}
Epetra_CrsMatrix* NewMatrix = 0;
HDF5.Read("matrix", NewMatrix);
\end{verbatim}
Since no Epetra\_Map has been specified, 
the newly created object {\tt NewMatrix} is based on a linear map. If the DomainMap() and RangeMap() are
known and non-trivial, one can use
\begin{verbatim}
HDF5.Read("matrix", DomainMap, RangeMap, NewMatrix);
\end{verbatim}
Reading meta-data looks like:
\begin{verbatim}
HDF5.Read("my parameters", "latitude", new_latitude);
HDF5.Read("my parameters", "longitude", new_longitude);
HDF5.Read("my parameters", "int array", H5T_NATIVE_INT, 
          3, &new_iarray[0]);
\end{verbatim}

Class EpetraExt::HDF5 has few other public methods that can be used to
manipulate directly the HDF5 file. Method
{\tt CreateGroup()} creates the specified group in the file;
{\tt IsContained()} returns {\tt true} is the specified group is already 
contained in the file;
{\tt WriteComment()} allows to write any string as comment for a group.

Finally, when an error arised, an EpetraExt::Exception object is thrown.
Method Print() of the Exception class gives a description of what went wrong.

% ============================================================================
\subsection{I/O for Distributed Arrays}
\label{sec:hdf5:arrays}
% ============================================================================

EpetraExt::HDF5 allows to perform I/O on distributed arrays, that is, for an
array of global length {\tt GlobalLength} and local size {\tt MyLength}, where
this latter value can assume a different value on each processor. The
following example, to be executed on two processors, writes an array on
integers, which has 3 elements on processor 0 and 2 on processor 1.
\begin{verbatim}
Epetra_MpiComm Comm(MPI_COMM_WORLD);
int MyLength, GlobalLength = 5;
if (Comm.MyPID() == 0) MyLength = 3;
else                   MyLength = 2;
std::vector<int> data(MyLength); // populate the vector

EpetraExt::HDF5 HDF5(Comm);
HDF5.Create("myfile.h5");
HDF5.Write("my-array", "values", MyLength, GlobalLength,
           H5T_NATIVE_INT, &data[0]);
HDF5.Close();
\end{verbatim}
In the above example, the distributed data is saved in group {\tt my-array},
   using the dataset {\tt values}. The reading might be as follows, when using
   5 processors:
\begin{verbatim}
int data;
HDF5.Open("myfile.h5");
HDF5.Read("my-array", "values", 1, 5, H5T_NATIVE_INT, &data);
HDF5.Close();
\end{verbatim}

% ============================================================================
\subsection{The MATLAB Interface}
\label{sec:hdf5:matlab}
% ============================================================================

The build-in functions {\tt hdf5read, hdf5write, hdf5info} can be used to read
or write EpetraExt-compatible HDF5 files from MATLAB. Figure \ref{fig:matlab}
reports two sets of MATLAB commands. The first set shows how to read an 
Epetra\_CrsMatrix stored in the group {\tt matrix}, while the second present
a typical procedure to write on file a MATLAB sparse matrix
(in this case, a diagonal matrix of size 10). This matrix will be saved within
the group {\tt speye}.

\begin{sidewaystable}
\begin{tabular}{| c |}
\hline
\begin{minipage}{22cm}
\begin{verbatim}

>> % Part I: reading data
>> NumGlobalRows = double(hdf5read('myfile.h5', '/matrix/NumGlobalRows/'));
>> NumGlobalCols = double(hdf5read('myfile.h5', '/matrix/NumGlobalCols/'));
>> ROW = double(hdf5read('myfile.h5', '/matrix/ROW/'));
>> COL = double(hdf5read('myfile.h5', '/matrix/COL/'));
>> VAL = hdf5read('myfile.h5', '/matrix/VAL/');
>> matrix = sparse(ROW + 1, COL + 1, VAL, NumGlobalRows, NumGlobalCols);
>>
>> % Part II: writing data
>> n = 10;
>> A = speye(n, n);
>> [ROW,COL,VAL] = find(A);
>> hdf5write('matlab.h5', '/speye/__type__',           'Epetra_RowMatrix');
>> hdf5write('matlab.h5', '/speye/NumGlobalRows',      int32(n), 'WriteMode', 'append');
>> hdf5write('matlab.h5', '/speye/NumGlobalCols',      int32(n), 'WriteMode', 'append');
>> hdf5write('matlab.h5', '/speye/NumGlobalNonzeros',  int32(n), 'WriteMode', 'append');
>> hdf5write('matlab.h5', '/speye/NumGlobalDiagonals', int32(n), 'WriteMode', 'append');
>> hdf5write('matlab.h5', '/speye/MaxNumEntries',      int32(1), 'WriteMode', 'append');
>> hdf5write('matlab.h5', '/speye/NormOne',            1.0,      'WriteMode', 'append');
>> hdf5write('matlab.h5', '/speye/NormInf',            1.0,      'WriteMode', 'append');
>> hdf5write('matlab.h5', '/speye/ROW', int32(ROW - 1), 'WriteMode', 'append');
>> hdf5write('matlab.h5', '/speye/COL', int32(COL - 1), 'WriteMode', 'append');
>> hdf5write('matlab.h5', '/speye/VAL', VAL,            'WriteMode', 'append');
\end{verbatim}
\end{minipage} \\
\hline
\end{tabular}
\caption{Example of input/output for HDF5 from MATLAB.
The use of {\tt double()} is required by sparse, which does not accept {\tt
  int32} data.  }
\label{fig:matlab}
\end{sidewaystable}

% ============================================================================
\section*{Acknowledgments}
% ============================================================================

The authors would like to acknowledge the support of the ASCI and LDRD programs
that funded development of Trilinos. John Shadid, Prof. Peter Arbenz and Prof.
Petros Koumoutsakos are acknowledged for part of the financial coverage.

\medskip

% ============================================================================
\bibliographystyle{plain}
\bibliography{biblio,../../../../doc/CommonFiles/TrilinosBibliography}
% ============================================================================

\newpage

\noindent
{\bf
Distribution List for ``Serialization Tools for Distributed Linear Algebra Objects:''
}

\begin{itemize}
\item MS 9018 Central Technical Files, 8945-1
\item MS 0899 Technical Library, 9616
\item MS 0123 LDRD Donna Chavez, 1011
\end{itemize}

\end{document}
