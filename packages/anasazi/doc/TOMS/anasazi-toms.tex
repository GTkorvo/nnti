\documentclass[acmtoms]{acmtrans2m}

\newtheorem{theorem}{Theorem}[section]
\newtheorem{conjecture}[theorem]{Conjecture}
\newtheorem{corollary}[theorem]{Corollary}
\newtheorem{proposition}[theorem]{Proposition}
\newtheorem{lemma}[theorem]{Lemma}
\newdef{definition}[theorem]{Definition}
\newdef{remark}[theorem]{Remark}

\usepackage{amsmath}
\usepackage{amsfonts}
\usepackage{float}
\usepackage{graphicx,wrapfig}
\usepackage{path}
\usepackage{moreverb}
\usepackage{color}

\newcounter{algorithm}
\renewcommand\thealgorithm{\thesection.\arabic{algorithm}}
\newenvironment{algorithm}[2]
{
  \begin{center}
    \noindent
    \framebox{\hbox{
        \begin{minipage}{5in} \refstepcounter{algorithm}
          \vspace{5pt}
          {{\sc Algorithm} \thealgorithm:
            \textsf{\bfseries{#1}} % \hfill \mbox{ ~ }
            } {
            \slshape #2 }
        \end{minipage}
        }}
  \end{center}
  }

%\newcommand{\aspace}[1]{Anasazi::#1}
\newcommand{\aspace}[1]{\texttt{#1}}

\newcommand{\cbcomm}[1]{\textcolor{blue}{\emph{#1}}}
\newcommand{\fin}[1]{\textcolor{red}{\emph{#1}}}

\markboth{C. G. Baker et al}{Anasazi software for the numerical
solution of large-scale eigenvalue problems}

\title{Anasazi software for the numerical solution of large-scale eigenvalue problems}

\author{C. G. Baker and
U. L. Hetmaniuk and R. B. Lehoucq and H. K. Thornquist\\ Sandia
National Laboratories}

\begin{abstract}
Anasazi is a package within the Trilinos software project. Anasazi
uses modern software paradigms to implement algorithms for the
numerical solution of large-scale eigenvalue problems in templated
ANSI C++. The purpose of our paper is to describe the design and
development of Anasazi. Timings comparing the cost of Anasazi with
the popular FORTRAN 77 code ARPACK are given.
\end{abstract}

\category{G.1.3}{Numerical Analysis}{Numerical Linear Algebra}
\category{G.4}{Mathematical Software}{} \category{D.2.13}{Software
Engineering}{Reusable Software}

\terms{Algorithms, Design, Performance; Reliability, Theory}

\keywords{Eigenvalue problems, Numerical Algorithms, Generic
programming}

\begin{document}

\begin{bottomstuff}
Sandia is a multiprogram laboratory operated by Sandia Corporation,
a Lockheed Martin Company, for the United States Department of
Energy under contract DE-AC04-94AL85000. Authors' addresses: C. G.
Baker, U. L. Hetmaniuk, R. B. Lehoucq,
Sandia National Laboratories, Computational Mathematics \&
Algorithms, MS 1320, P.O.Box 5800, Albuquerque, NM 87185-1320; email
\{cgbaker,ulhetma,rblehou\}@sandia.gov. H. K. Thornquist, Sandia National
Laboratories, Electrical and Microsystem Modeling, MS 0316,
Albuquerque, NM 87185-0316; email hkthorn@sandia.gov.

\end{bottomstuff}

\maketitle

Anasazi is a package within the Trilinos
Project~\cite{Heroux:2005:OTP}. Anasazi uses modern software
paradigms to implement algorithms for the numerical solution of
large-scale eigenvalue problems in templated ANSI C++. We define a
large-scale eigenvalue problem to be one where a small number
(relative to the dimension of the problem) of eigenvalues and the
associated eigenspace are computed, and only knowledge of the
underlying matrix via application on a vector (or group of vectors)
is assumed.

An inspiration for Anasazi is the ARPACK~\cite{lesy:98} FORTRAN 77
software library. ARPACK implements one algorithm, namely an
implicitly restarted Arnoldi method~\cite{sore:92}. In contrast,
Anasazi provides a software framework, including the necessary
infrastructure, to implement a variety of algorithms.  Anasazi is an
extensible framework because the necessary linear algebra
infrastructure is made independent of the algorithms used for the
numerical solution of large-scale eigenvalue problems. We justify
our claims by implementing block variants of three popular
algorithms: a Davidson~\cite{davi:75} method, a
Krylov-Schur~\cite{stew:01} method, and an implementation of
LOBPCG~\cite{knya:01}.

ARPACK has proved to be a popular and successful FORTRAN 77 library
for the numerical solution of large-scale eigenvalue problems. A
crucial reason for the popularity of ARPACK is the use of a reverse
communication~\cite[p.~3]{lesy:98} interface for applying the
necessary matrix-vector products. This allows ARPACK to provide a
callback for the needed matrix-vector products in a simple fashion
within FORTRAN 77. Unfortunately, the reverse communication
interface is cumbersome, challenging to maintain, and does not allow
data encapsulation. Moreover, because ARPACK uses a procedural
programming paradigm where the matrix-vector operations rely upon
the physical representation of the data manipulated, ARPACK is
susceptible to design changes. Hence code reuse is limited, and
software complexity and maintenance are more cumbersome.

Procedural programming is often at odds with modern software
paradigms that include object-based, object-oriented, and generic
programming methods. The C programming language only allows an
object-based programming paradigm. In contrast, C++ supports
object-oriented techniques and generic programming (typically via
templates). One of the features of Anasazi is the use of generic
programming via static and dynamic polymorphism~\cite[Chapter
14]{VJ02}. Static polymorphism, via templating of the linear algebra
objects, allows algorithms in Anasazi to be written generically
(i.e., independent of the data). Dynamic polymorphism, via virtual
functions and object inheritance, allows eigensolvers to be
decoupled from functions such as orthogonalization and stopping
conditions; this functionality can then be decided at runtime. The
upshot of this decoupling is the facilitation of code reuse and
algorithmic modification.

%\begin{description}
%\item[object-based]
%(user defined) abstract data types, operator overloading,
%information hiding, separation of an interface from an
%implementation,
%\item[object-oriented]
%object-based methods, inheritance, polymorphism, dynamic (or late)
%binding,
%\item[generic programming]
%separation of data from the algorithms that operate on the data.
%\end{description}

%Anasazi uses generic programming and object-oriented techniques to
%provide a uniform interface and underlying infrastructure for
%developing a rich collection of eigensolvers.

The result of these design choices is to make Anasazi an extensible
and interoperable software. Extensibility is apparent in the
infrastructure's support for a significant class of large-scale
eigenvalue algorithms. Extensions can be made through the addition of
new algorithms or through modification of existing algorithms. This is
encouraged by promoting code modularization and multiple levels of
access to solvers and their data. Interoperability is enabled via the
treatment of both matrices and vectors as opaque objects---only
knowledge of the matrix and vectors via elementary operations is
necessary. This permits algorithms to be implemented in a generic
manner, requiring no knowledge of the underlying linear algebra types
or their specific implementations.

We emphasize that our interest is not solely in modern software
paradigms. Rather, our paper demonstrates that a rich collection of
block eigensolvers is easily implemented using modern programming
techniques. Our approach is algorithm-oriented \cite{muov:94}
because algorithms are front and center, followed by the software
abstractions. Moreover, our implementations are required to be
efficient and portable. We believe that Anasazi is the natural
successor to ARPACK, inheriting, and extending, the quality
practices employed by ARPACK.

Related software efforts that implement several algorithms for
large-scale eigenvalue (the reader is referred to~\cite{slepc:05}
for a software survey) problems are:
\begin{itemize}
\item
The SLEPc~\cite{slepc:06} library written in C for the solution of
large scale sparse eigenvalue problems on parallel computers. SLEPc
is an extension of PETSc~\cite{petsc-web-page} and can be used for
either Hermitian or non-Hermitian, standard or generalized,
eigenproblems;
\item
PRIMME~\cite{primme:06} is a C library to find a number of
eigenvalues and their corresponding eigenvectors of a real
symmetric, or complex Hermitian matrix.
\end{itemize}
Both of these efforts use an object-based programming paradigm and
so do not employ generic or object-oriented techniques.  We are not
aware of any other software implementing block algorithms for
large-scale eigenvalue problems using object-oriented or generic
programming techniques.

Our paper is organized as follows. Section
\ref{sec:algorithm-overview} describes a class of algorithms that
can be implemented within Anasazi. Section \ref{sec:framework}
reviews our software framework.  Section \ref{sec:benchmarking}
provides timings comparing ARPACK and Anasazi.


\section{Algorithm overview}
%%%%%%%%%%%%%%%%%%%%%%%%%%%%%%%%%%%%%%%%%%%%%%%%%%%%%%
\label{sec:algorithm-overview}

Anasazi software provides algorithms for computing a partial
eigen-decomposition for the generalized eigenvalue problem
\begin{equation}  \label{eq:intro.1}
  \mathbf{A} \mathbf{x} = \mathbf{B} \mathbf{x} \lambda , \qquad
  \mathbf{A}, \mathbf{B} \in \mathbb{C}^{n\times n}\ .
\end{equation}
The matrices $\mathbf{A}$ and $\mathbf{B}$ are large, possibly
sparse, and we assume that only their application to a block of
vectors is required. The reader is referred
to~\cite{saad:92,sore:02,stew:01book,vors:02} for background
information and references on the large-scale eigenvalue problem.

Algorithm~\ref{algo:obliqueRR} is a simple extension of the
Rayleigh-Ritz procedure given in~\cite[p.281]{stew:01book}. This
algorithm lists the salient steps found in the majority of
large-scale eigensolvers, namely subspace projection methods.

\begin{figure}[htb]
\begin{algorithm}{Rayleigh-Ritz Algorithm\label{algo:obliqueRR}}
{\smallskip
\begin{tabbing}
(nr)ss\=ijkl\=bbb\=ccc\=ddd\= \kill {\rm (1)} \>\> Let the matrices
$\mathbf{M}$, $\mathbf{U}$ and $\mathbf{V}$ be given
\\
{\rm (2)} \>\> Form the Rayleigh quotients $ \mathbf{V}^H
\mathbf{M}\Phi(\mathbf{A}) \mathbf{U}$ and $ \mathbf{V}^H \mathbf{M}
\Psi(\mathbf{B}) \mathbf{U}$\\
\>\> where $\Phi(\cdot)$ and $\Psi(\cdot)$ are matrix functions
\\
{\rm (3)} \>\> Compute an eigen-decomposition for the matrix
pencil\\
\>\>$ (\mathbf{V}^H \mathbf{M}\Phi(\mathbf{A}) \mathbf{U},
\mathbf{V}^H \mathbf{M}\Psi(\mathbf{B}) \mathbf{U})$
\\
{\rm (4)} \>\> Return the eigen-decomposition as an approximation\\
\>\>for the pencil $(\mathbf{A}, \mathbf{B})$
\\
\end{tabbing}
}
\end{algorithm}
\end{figure}

The matrices $\mathbf{U}$ and $\mathbf{V}$ are bases for the trial
and test subspaces $\mathcal{U}$ and $\mathcal{V}$, respectively.
When these two subspaces are distinct, then the Rayleigh-Ritz method is
called oblique. Otherwise, when $\mathcal{V} = \mathcal{U}$ the
orthogonal Rayleigh-Ritz method results. The functions $\Phi(\cdot)$
and $\Psi(\cdot)$ are often used to improve convergence to the
eigenvalues and eigenspace of interest. For example, a standard
approach is to reformulate (\ref{eq:intro.1}) as
$$
%\begin{equation}  \label{eq:si}
  \Phi(\mathbf{A}) \mathbf{x} \equiv
  (\mathbf{A}- \mathbf{B}\sigma)^{-1}\mathbf{B}\mathbf{x} =  \mathbf{x}
  (\lambda-\sigma)^{-1} \equiv \Psi(\mathbf{B})\mathbf{x} \nu ,
%\end{equation}
$$
where $\sigma \in \mathbb{C}$ is near the eigenvalues of interest.
This is an example of the shift-invert spectral transformation. The
matrix $\mathbf{M}$ is often used to denote an inner product; for
instance $\mathbf{M}$ can be set equal to $\mathbf{A}$ or
$\mathbf{B}$ when either matrix is Hermitian positive semi-definite.
A second example is to apply an equivalence transformation to
(\ref{eq:intro.1}) resulting in
$$
   \Phi(\mathbf{A}) \mathbf{x} \equiv
   \mathbf{N}^{-1}\mathbf{A} \mathbf{x} = \mathbf{N}^{-1}\mathbf{B} \mathbf{x} \lambda
   \equiv \Phi(\mathbf{B})\mathbf{x}\lambda
$$
where $\Psi = \Phi$ and set $\mathbf{M}=\mathbf{N}$.
%Of
%course, the assumption is made that linear systems with $\mathbf{A}-
%\mathbf{B}\sigma$ can be solved, either with a sparse direct method
%or a preconditioned iterative method.
%We define the \emph{eigenproblem} as (\ref{eq:intro.1}) and the
%\emph{eigen-operator}, or simply operator, as $OP$. Examples of the
%latter are $OP = \mathbf{A}$ or $OP = (\mathbf{A}-
%\mathbf{B}\sigma)^{-1}\mathbf{B}$. $OP$ will often utilize a
%preconditioned iterative method\footnote{We refer the reader
%to~\cite{Arbenz:2005:ACE} for an involved discussion of several
%approaches to solving the symmetric positive semi-definite
%(\ref{eq:intro.1}).}.

Several linear algebra operations are required to implement
large-scale eigenvalue computations. These include
\begin{itemize}
  \item Matrix-matrix applications: $\mathbf{A} \mathbf{U}$.
  \item Block inner products: $\mathbf{V}^H (\mathbf{A}\mathbf{U})$.
  \item Solution of typically much smaller eigenproblems (step~3).
\end{itemize}
%We define a \emph{multivector} as the data structure that holds a
%collection of vectors, or matrix $\mathbf{U}$.
Other linear algebra operations include methods for creating and
performing vectors and vector arithmetic. We list our primitives in
Section \ref{sec:framework}.

Algorithm~\ref{algo:obliqueRR} needs to be augmented with several
steps in order to result in an \emph{eigen-iteration}.  Algorithm
\ref{algo:eigen-iter} lists these additional steps, so defining an
eigen-iteration.

\begin{figure}[htb]
\begin{algorithm}{Eigen-iteration\label{algo:eigen-iter}}
{\smallskip
\begin{tabbing}
(nr)ss\=ijkl\=bbb\=ccc\=ddd\= \kill {\rm (1)} \>\> Update the
matrices $\mathbf{U}$ and $\mathbf{V}$\\
{\rm (2)} \>\> Determine whether any portion of the
eigen-decomposition\\
\>\> is of acceptable accuracy
\\
{\rm (3)} \>\> Deflate the accurate portions of the
eigen-decomposition
\\
{\rm (4)} \>\> Terminate the Eigen-iteration
\\
\end{tabbing} }
\end{algorithm}
\end{figure}

A distinguishing characteristic of the Rayleigh-Ritz algorithm is
the number of columns $m$ of $\mathbf{U}$ and $\mathbf{V}$. The size of the
bases $\mathbf{U}$ and $\mathbf{V}$ is either constant or
increasing. An example of the former is the gradient-based method
LOBPCG~\cite{knya:01}. Examples of the latter are the Davidson
algorithm~\cite{davi:75} and Krylov-based methods such as the
Arnoldi~\cite{arno:51} and Lanczos~\cite{lanc:50} methods.
Ultimately, the success of an algorithm depends crucially upon these
subspaces and the choice of bases representation, an issue that is
beyond the scope of this paper.

The algorithms that are currently available in Anasazi are:
\begin{enumerate}
  \item a block extension of a Krylov-Schur method~\cite{stew:01},
  \item a block Davidson method as described in~\cite{Arbenz:2005:ACE},
  \item an implementation of LOBPCG as described in~\cite{Hetmaniuk:2006:BSL}.
\end{enumerate}
Three remarks are in order. First, all three algorithms are
instances of the orthogonal Rayleigh-Ritz method. Therefore the
eigen-decomposition computed in step $(3)$ of
Algorithm~\ref{algo:obliqueRR} is equivalent to a Schur form when
(\ref{eq:intro.1}) is a regular pencil.

Our second remark is that only the Krylov-Schur method can be used
for non-Hermitian generalized eigenvalue problems. In contrast, all
three algorithms can be used for symmetric positive semi-definite
generalized eigenvalue problems.

Our third remark is that block methods are defined to be those that
apply $\mathbf{A}$ (or $\mathbf{B}$) to a collection of vectors.
This improves the ratio of floating-point operations to memory
reference and so better exploits the memory hierarchy.

This discussion illustrates that many distinct parts make up a
large-scale eigensolver code: orthogonalization, sorting tools, dense
linear algebra, convergence testing, multivector arithmetic, etc.
Anasazi presents a framework of algorithmic components, decoupling
operations where possible in order to simplify component verification,
encourage code reuse, and maximize flexibility in implementation.

%Our
%resulting implementations are therefore backward stable. Such a
%reliance allows us to conclude that we have computed a nearby
%partial Schur decomposition. The distance to the problem of interest
%is then bounded by the size of residual associated with the computed
%Schur decomposition.
%\begin{table}[htb]
%\caption{Necessary steps when implementing Algorithm
%\ref{algo:obliqueRR}} \label{table:additional-steps}
%\begin{center}
%\begin{tabular}{|ll|}\hline
%(1)& maintaining a basis $\mathbf{U}$ that is orthonormal to working
%precision\\
%(2) & restarting the iteration (need to define) when the number of
%basis vectors in $\mathbf{U}$ exceeds\\
%& storage requirements,\\
%(3) & termination criteria that determine when the approximate
%eigen-decomposition is of \\
%& acceptable accuracy, or when a sufficient number of iterations
%have been exceeded, or \\
%& when something has gone wrong\\
%(4) & a deflation mechanism so that portions of the approximate
%eigen-decomposition are set \\
%&aside when of acceptable accuracy\\
%\hline
%\end{tabular}
%\end{center}
%\end{table}



\section{Anasazi software framework}
%%%%%%%%%%%%%%%%%%%%%%%%%%%%%%%%%%%%%%%%%%%%%%%%%%%%%%
\label{sec:framework}

This section outlines the Anasazi software framework and motivates
the design decisions made in the development of Anasazi. Three
subsections describe the Anasazi operator/vector interface, the
eigensolver framework, and a review of the various classes in
Anasazi. The reader is referred to
\cite{Trilinos:anasazi,Trilinos-Tutorial} for software documentation
and a tutorial.

\subsection{The Anasazi Operator/Vector Interface}
%%%%%%%%%%%%%%%%%%%%%%%%%%%%%%%%%%%%%%%%%%%%%%%%%%%%%%
\label{sec:anasazi:opvec}

Anasazi utilizes abstract interfaces for matrix operators and
multivectors. This allows generic programming techniques to be used
when developing the numerical algorithms in Anasazi. In C++, generic
programming is traditionally implemented using virtual
functions or templates. The abstract numerical interfaces used
in Anasazi are supported via templates. Most classes in Anasazi accept
three template parameters:
\begin{itemize}
\item
a scalar type, describing the field over which the vectors and
operators are defined;
\item
a multivector type, that depends upon the scalar type, providing a
data structure that denotes a collection of vectors; and
\item
an operator type, that depends upon the multivector and scalar types,
providing linear operators used to define eigenproblems and
preconditioners.
\end{itemize}

Templating an eigensolver on multivector and scalar types enables software reuse. For
example, ARPACK implements the four subroutines---\texttt{SNAUPD}, \texttt{DNAUPD},
\texttt{CNAUPD}, and \texttt{ZNAUPD}---for solving non-Hermitian eigenproblems. Four
separate subroutines are provided for these four FORTRAN 77 floating point types (single
and double precision real, and single and double precision complex). Moreover, four
additional subroutines are needed for a distributed memory implementation (say using MPI).
In contrast, templating on multivector and scalar types allows Anasazi to maintain only
one C++ code. The multivector templating allows us to separate the eigenvalue algorithm
from the linear algebra data structures. The operator type templating is analogous to the
reverse communication interface used by ARPACK for providing matrix-vector products.

Because the underlying data types are unknown to the Anasazi
developer, algorithms are developed abstractly. Access to the
functionality of the underlying objects is provided via the classes
\aspace{MultiVecTraits} and \aspace{OperatorTraits}. These classes
implement the traits mechanism~\cite{myer:95} and specify the
operations that the multivector and operator classes must support in
order to be used within Anasazi. This mechanism hides the low-level
details of the underlying data structures, allowing the same
algorithmic implementation to be compatible with varying underlying
linear algebra objects (e.g., serial and parallel).

The methods defined by these traits classes are listed in
Table~\ref{tab:anasazi:mvt}. Most of the methods listed are
self-explanatory. The first three \aspace{MultiVecTraits} methods
are C++ \emph{virtual} constructors \cite[pp.~123--129]{meyers:96}
that create multivectors from a multivector provided by the user. Deep and shallow
copy denotes whether the object contains the storage for the
multivector entries or not. A shallow copy is useful when only a subset
of the columns of a multivector is required for computation.

\begin{table}
\begin{center}
  \caption{Methods provided by the \aspace{OperatorTraits} and \aspace{MultiVecTraits} interfaces.}
\label{tab:anasazi:mvt}
\begin{tabular}{| p{3cm} | p{8cm} |}
\hline
%%%
\multicolumn{2}{|c|}{\textbf{OperatorTraits}} \\\hline
\emph{Method name} & \emph{Description} \\\hline
Apply           & Applies an operator to a MultiVector, placing the
result in another MultiVector. \\\hline\hline
%%%
\multicolumn{2}{|c|}{\textbf{MultiVecTraits}} \\\hline
%%%
\emph{Method name} & \emph{Description} \\\hline
Clone           & Creates a new multivector containing a
specified number of columns.  \\\hline

CloneCopy & Creates a new multivector with a copy of the contents of
an existing multivector (deep copy). \\\hline

CloneView       & Creates a new multivector that shares the selected
contents of an existing multivector (shallow copy).  \\\hline

GetVecLength    & Returns the vector length of a multivector.
\\\hline

GetNumberVecs   & Returns the number of vectors in a multivector.
\\\hline

MvTimesMatAddMv & Applies a serial, dense matrix $M$ to
multivector $A$ and accumulates into another multivector $B$:\\
& $B \leftarrow \alpha A M + \beta B$.
\\\hline

MvAddMv         & Performs multivector AXPBY: $ B \leftarrow \alpha A + \beta B$.
\\\hline

MvTransMv & Computes the matrix $C \leftarrow \alpha A^H B$.
\\\hline

MvDot & Computes the vector $b$ where the components are the
individual dot-products of the $i$-th columns of $A$ and $B$, i.e.,
$b[i] = A[i]^H B[i]$.  \\\hline

MvScale         & Scales the columns of a multivector. \\\hline

MvNorm          & Computes the 2-norm of each vector of
$A$.  \\\hline

SetBlock        & Copies the vectors in $A$ to a subset of vectors in
$B$. \\\hline

MvRandom & Replaces the vectors in $A$ with random vectors.  \\\hline

MvInit & Replaces each element of the vectors in $A$ with $\alpha$.
\\\hline

MvPrint         & Prints the Multivector to an output stream.
\\\hline \hline
\end{tabular}
\end{center}
\end{table}

The use of \aspace{MultiVecTraits} and \aspace{OperatorTraits} requires that
specializations of these traits classes have been implemented for
given template arguments. Anasazi provides the following
specializations of these traits classes:
\begin{itemize}
  \item \aspace{Epetra\_MultiVector} and \aspace{Epetra\_Operator} (with scalar type
    \aspace{double}) allow Anasazi to be used with the Epetra~\cite{Trilinos:Epetra} linear
    algebra library provided with Trilinos.
  \item \aspace{Thyra::MultiVectorBase} and \aspace{Thyra::LinearOpBase} (with arbitrary scalar type)
        allow Anasazi to be used with any classes that implement the abstract interfaces
        provided by the Thyra~\cite{Trilinos:Thyra} package of Trilinos.
  \item \aspace{MultiVec} and \aspace{Operator} (with arbitrary scalar type)
        allow Anasazi to be used with any classes that implement
        the Anasazi abstract base classes \aspace{MultiVec} and \aspace{Operator}.
\end{itemize}

For scalar, multivectors and operators types not covered by these, specializations of
\aspace{MultiVecTraits} and \aspace{OperatorTraits} must be created. The benefit of the
traits mechanism is that it does not require that the chosen types are C++ classes.
Furthermore, it does not require rewriting the user's data types, as the traits class
specialization occurs external to the chosen types.

%The benefit of using a templated traits class over inheritance is
%that the latter requires the user to derive multivectors and
%operators from Anasazi-defined abstract base classes. Inheritance
%couples tightly Anasazi's  and the users' objects and so is a poor
%mechanism. Templated traits, however, promote the goals of
%extensibility and interoperability by a well-defined and narrow
%interface.

\subsection{The Anasazi Framework}
%%%%%%%%%%%%%%%%%%%%%%%%%%%%%%%%%%%%%%%%%%%%%%%%%%%%%%
\label{subsec:anasazi:solver_framework}

We explain how the Rayleigh-Ritz method of
Algorithm~\ref{algo:obliqueRR} and the additional steps listed in
Algorithm~\ref{algo:eigen-iter} are implemented within the Anasazi
framework.

In Anasazi, eigensolvers (encapsulating an iteration and its associated
state) are derived classes of the abstract base class
\aspace{Eigensolver}. An inheritance relationship was chosen for the following
reasons:
\begin{itemize}
  \item the abstract base class defines an interface used for checking the
    status of a solver by a status test;
  \item a concrete derived class will perform the iteration associated
    with a specific eigensolver algorithm; and
  \item a concrete derived class will act as a container for the state
    associated with its particular iteration.
\end{itemize}

The class \aspace{StatusTest} is used to specify stopping conditions for an
eigen-iteration. \aspace{Eigensolver} queries the \aspace{StatusTest} during its class
method \aspace{iterate()} to determine whether or not to continue iterating.  Concrete
subclasses of \aspace{StatusTest} provide particular stopping criteria. A typical
interaction between these two classes is illustrated in Figure~\ref{fig:comm}.

\begin{figure}[htb]
\begin{center}
\begin{boxedverbatim}
SomeEigensolver::iterate() {
  while ( somestatustest.checkStatus(this) != Passed ) {
    //
    // perform eigensolver iterations
    //
  }
  return;  // return back to caller
}
\end{boxedverbatim}
\end{center}
\caption{Example of communication between status test and eigensolver}
\label{fig:comm}
\end{figure}

Each \aspace{StatusTest} provides a virtual method,
\verb!checkStatus()!, which queries the methods provided by
\aspace{Eigensolver} and determines whether the solver meets the
criteria defined by a particular status test. After a solver returns
from \verb!iterate()!, the caller has the ability to access the
solver's state and the option re-initializing the solver with
a new state and continue iterating.

While this approach to interfacing with the solver is powerful, it can be overwhelming. It
requires the user to construct a number of support classes and to manage calls to
\verb!Eigensolver::iterate()!. The \aspace{SolverManager} class was developed to
encapsulate an instantiation of \aspace{Eigensolver}, providing additional functionality
and handling low-level interaction with the eigensolver that a user may not want to
specify. Solver managers are intended to be easy to use, while still providing the
features and flexibility needed to solve large-scale eigenvalue problems.

For example, the constructor of \aspace{BlockDavidsonSolMgr} accepts only two arguments:
an \aspace{Eigenproblem} specifying the eigenvalue problem to be solved and a
\texttt{ParameterList} of options specific to this solver manager. This solver manager
instantiates a \aspace{BlockDavidson} subclass of \aspace{Eigensolver}, along with the
status tests and other support classes needed by the eigensolver. To solve the eigenvalue
problem, the user simply calls the \verb!solve()! method of \aspace{BlockDavidsonSolMgr}.
The solver manager calls \verb!iterate()!, performs restarts and locking, and places the
final solution into the \aspace{Eigenproblem}.

Under this framework, users have a number of options for performing eigenvalue
computations with Anasazi:
\begin{itemize}
\item
use an existing solver manager. In this case, the user is limited to
the functionality provided by the existing solver managers.
\item
Develop a new solver manager for an existing eigensolver.
The user can extend the functionality provided by the eigensolver,
specifying custom configurations for status tests,
orthogonalization, restarting, locking,  etc.
\item
Implement a new eigensolver (and so extend Anasazi). The user can
write an eigensolver for an iteration that is not represented in
Anasazi. The user still has the benefit of the support classes
provided by Anasazi, and the knowledge that this effort can be
easily employed by anyone already familiar with Anasazi.
\end{itemize}


\subsection{Anasazi Classes}
%%%%%%%%%%%%%%%%%%%%%%%%%%%%%%%%%%%%%%%%%%%%%%%%%%%%%%
\label{subsec:anasazi:classes}

Anasazi is designed with extensibility in mind, so that users can
augment the package with any special functionality that may be
needed. However, the released version of Anasazi provides all
functionality necessary for solving a wide variety of problems. This
section lists and briefly describes the classes used in Anasazi.


We remark that Anasazi is largely independent of Trilinos. Anasazi
only relies on the Trilinos Teuchos package~\cite{Trilinos:Teuchos}
that provides a common suite of tools, such as: \texttt{RefCountPtr},
a reference-counting smart pointer~\cite{detlefs92garbage};
\texttt{ParameterList}, a list for algorithmic parameters of
varying data types; and the BLAS
\cite{Lawson:1979:BLA,Blackford:2002:USB} and LAPACK \cite{abbd:95}.

The abstract base class \aspace{Eigenproblem} is a container for the components and
solution of an eigenvalue problem. By requiring eigenproblems to derive from
\aspace{Eigenproblem}, Anasazi defines a minimum interface that can be expected of all
eigenvalue problems by the classes that will work with the problems (e.g., eigensolvers
and status testers). Anasazi provides users with a concrete implementation of
\aspace{Eigenproblem}, called \aspace{BasicEigenproblem}. This basic implementation
provides all the functionality necessary to describe both generalized and standard,
Hermitian and non-Hermitian linear eigenvalue problems.

The methods for storing and retrieving the results of the
eigenvalue computation in an \aspace{Eigenproblem} are:
\begin{verbatim}
const Eigensolution & Eigenproblem::getSolution();
void Eigenproblem::setSolution(const Eigensolution & sol);
\end{verbatim}

The \aspace{Eigensolution} class was developed in order to
facilitate setting and retrieving of solution data. This structure
contains the following information:
\begin{itemize}
  \item \verb!RefCountPtr< MV > Evecs! \\
   The computed eigenvectors.
 \item \verb!RefCountPtr< MV > Espace! \\
   An orthonormal basis for the computed eigenspace.
 \item \verb!std::vector< Value< ScalarType > > Evals! \\
   The computed eigenvalue approximations.
 \item \verb!std::vector< int > index! \\
   An index into \verb!Evecs! to enable compressed storage of eigenvectors for real, non-Hermitian problems.
 \item \verb!int numVecs! \\
   The number of computed eigenpair approximations.
\end{itemize}

Anasazi solver managers are expected to place the results of their computation in the
\aspace{Eigenproblem} class using an \aspace{Eigensolution}. However, a user working
directly with an eigensolver (i.e., not with a solver manager) will need to recover the
solution directly from the eigensolver state.

%The eigenvalues are always stored as two real values, even
%when templated on a complex data type or when the eigenvalues are
%real. Similarly, a basis for the eigenspace can always be represented
%by a multivector of width \verb!numVecs!, even for non-symmetric
%eigenproblems over the real field. However, the storage scheme for
%eigenvectors requires more finesse.

%When solving real symmetric eigenproblems, the eigenvectors can always
%be chosen to be real, and therefore can be stored in a single column
%of a real multivector. When solving eigenproblems over a complex
%field, whether Hermitian or non-Hermitian, the eigenvectors may be
%complex, but the multivector is defined over the complex field, so
%that this poses no problem. However, real non-symmetric problems can
%have complex eigenvectors, which prohibits a one-for-one storage
%scheme using a real multivector.  Fortunately, the eigenvectors in
%this scenario occur as complex conjugate pairs, so the pair can be
%stored in two real vectors. This permits a compressed storage scheme,
%which uses an index vector stored in the \aspace{Eigensolution},
%allowing conjugate pair eigenvectors to be easily retrieved from
%\verb!Evecs!.

%The integers in \aspace{Eigensolution}::index take one of three
%values: $\{0, +1, -1\}$. These values allow the eigenvectors to be
%retrieved as follows:
%\begin{itemize}
%  \item \aspace{index[i]== 0}: the $i$-th eigenvector is stored uncompressed in column $i$ of
%    \verb!Evecs!.
%  \item \aspace{index[i]==+1}: the $i$-th eigenvector is stored compressed, with the real
%    component in column $i$ of \verb!Evecs! and the \emph{positive} complex component
%    stored in column $i+1$ of \verb!Evecs!
%  \item \aspace{index[i]==-1}: the $i$-th eigenvector is stored compressed, with the real
%    component in column $i-1$ of \verb!Evecs! and the \emph{negative} complex component
%    stored in column $i$ of \verb!Evecs!
%\end{itemize}
%Because this storage scheme is only required for non-symmetric
%problems over the real field, all other eigenproblems will result in
%an index vector composed entirely of zeroes. For the real
%non-symmetric case, the $+1$ index will always immediately precede
%the corresponding $-1$ index.

%\begin{remark}
%  Solver managers all put the computed eigensolution into the eigenproblem class before
%  returning from \verb!solve()!. Eigensolvers do not; a user working directly with an
%  eigensolver will need to recover the solution directly from the eigensolver state.
%\end{remark}

The \aspace{Eigensolver} abstract base class defines the basic interface that must be met
by any eigensolver class in Anasazi. Specific eigensolvers are implemented as derived
classes of \aspace{Eigensolver}.

\aspace{Eigensolver} defines two significant types of methods: status methods and
solver-specific methods. The status methods are defined by the \aspace{Eigensolver}
abstract base class and represent the information that any status test can request from
any eigensolver. A list of these methods is given in
Table~\ref{tab:anasazi:genstatusmethods}.

% The class \aspace{Eigensolver}, like \aspace{Eigenproblem}, is
% templated on the scalar type, multivector type and operator type.

\begin{table}[htp]
\begin{center}
\caption{A list of generic status methods provided by
\aspace{Eigensolver}.} \label{tab:anasazi:genstatusmethods}
\begin{tabular}{| p{3cm} | p{6cm} |}
\hline
\emph{Method} & \emph{Description} \\
\hline
{\tt getNumIters}       & current number of iterations. \\
{\tt getRitzValues}     & most recent Ritz values. \\
{\tt getRitzVectors}    & most recent Ritz vectors. \\
{\tt getRitzIndex}      & Ritz index needed for indexing compressed Ritz vectors. \\
{\tt getResNorms}       & residual norms, with respect to the \aspace{OrthoManager}. \\
{\tt getRes2Norms}      & residual Euclidean norms. \\
{\tt getRitzRes2Norms}  & Ritz residual  Euclidean norms. \\
{\tt getCurSubspaceDim} & current subspace dimension. \\
{\tt getMaxSubspaceDim} & maximum subspace dimension. \\
{\tt getBlockSize}      & block size. \\
\hline
\end{tabular}
\end{center}
\end{table}

One of the tenets of object-oriented programming is data
encapsulation. This seems to contradict the need to be as efficient
as possible in scientific computing. To this end, each eigensolver
provides low-level methods for accessing and setting the state of
the solver:
\begin{itemize}
  \item \verb!getState()! - returns a solver-specific structure with read-only pointers to
    the current state of the solver.
  \item \verb!initialize(...)! - accepts a solver-specific structure enabling the user to
    initialize the solver with a particular state.
\end{itemize}
The combination of these two methods, along with the flexibility
provided by status tests, provides the user with a large degree of
control over eigensolver iterations.

% Using Anasazi by interfacing directly with eigensolvers is extremely
% powerful, but can be tedious. Solver managers provide a way for users
% to encapsulate specific solving strategies inside of an easy-to-use
% class. Novice users may prefer to use existing solver managers, while
% advanced user may prefer to write custom solver managers.

\aspace{SolverManager} defines only two methods: a constructor
accepting an \aspace{Eigenproblem} and a parameter list of
options specific to the solver manager; and a \verb!solve()! method, taking no
arguments and returning either \aspace{Converged} or
\aspace{Unconverged} (Figure~\ref{fig:examplesolve}).

\begin{figure}[htb]
\begin{center}
\begin{boxedverbatim}
// create an eigenproblem
RefCountPtr< Anasazi::Eigenproblem<ScalarType,MV,OP> > problem = ...;
// create a parameter list
ParameterList params;
params.set(...);
// create a solver manager
Anasazi::BlockDavidsonSolMgr<ScalarType,MV,OP> solman(problem,params);
// solve the eigenvalue problem
Anasazi::ReturnType ret = solman.solve();
// get the solution from the problem
Anasazi::Eigensolution<ScalarType,MV> sol = problem->getSolution();
\end{boxedverbatim}
\end{center}
\caption{Example: Solving an eigenvalue problem using a
\aspace{SolverManager}}
\label{fig:examplesolve}
\end{figure}

The goal of the solver manager is to instantiate a subclass of \aspace{Eigensolver}, along
with the necessary support objects. Another purpose of many solver managers is to manage
and initiate the repeated calls to the underlying solver's \verb!iterate()! method. For
solvers that increase the dimension of trial and test subspaces (e.g., Davidson and Krylov
subspace methods), the solver manager may also assume the task of restarting (so that
storage costs may be fixed). This decoupling of restarting from the eigensolver is
beneficial due to the numerous restarting techniques in use.

% These examples are meant to illustrate the flexibility that specific
% solver managers may have in implementing the \verb!solve()! routine.
% Some of these options might best be incorporated into a single
% solver manager, which takes orders from the user via the parameter
% list given in the constructor. Some of these options may better be
% contained in multiple solver managers, for the sake of code
% simplicity. It is even possible to write solver managers that
% contain other solvers managers; motivation for something like this
% would be to select the optimal solver manager at runtime based on
% some expert knowledge, or to create a hybrid method which uses the
% output from one solver manager to initialize another one.

Performing an eigen-iteration requires a number of support classes.  These are passed
through the objects constructor, defined by \aspace{Eigensolver} to take the form listed
in Figure~\ref{fig:constructor}.

\begin{figure}[htb]
\begin{center}
\begin{boxedverbatim}
Eigensolver(
   const RefCountPtr< Eigenproblem<ST,MV,OP> > &problem,
   const RefCountPtr< SortManager<ST,MV,OP>  > &sorter,
   const RefCountPtr< OutputManager<ST>      > &printer,
   const RefCountPtr< StatusTest<ST,MV,OP>   > &tester,
   const RefCountPtr< OrthoManager<ST,OP>    > &ortho,
   ParameterList                               &params
 );
\end{boxedverbatim}
\end{center}
\caption{Constructor for eigensolver}
\label{fig:constructor}
\end{figure}

These support classes are employed for the following purposes:
\begin{itemize}
  \item \verb!problem! - the eigenproblem to be solved; problem operators are
  defined.
  \item \verb!sorter! - the sort manager selects the
  eigenvalues of interest.
  \item \verb!printer! - the output manager dictates the verbosity level in addition to
    processing output streams.
  \item \verb!tester! - the status tester dictates when the solver terminates
  \verb!iterate()!.
  \item \verb!ortho! - the orthogonalization manager defines the inner product
   in addition to performing orthogonalization for the solver.
  \item \verb!params! - the parameter list specifies eigensolver-specific
  options.
\end{itemize}

The purpose of the \aspace{StatusTest} is to give the user or solver
manager flexibility in terminating the eigensolver iterations in
order to interact directly with the solver. For instance, typical
reasons for terminating the iteration are:
\begin{itemize}
  \item some convergence criterion has been satisfied;
  \item some portion of the subspace has reached sufficient accuracy to be
  deflated from the iterate or locked;
  \item the solver has performed a sufficient number of iterations.
\end{itemize}
The variation that exists for monitoring these and other conditions requires an abstract mechanism
controlling the iteration.

The following is a list of Anasazi-provided status tests:
\begin{itemize}
  \item \aspace{StatusTestMaxIters} - monitors the number of iterations
    performed by the solver; it can be used to halt the solver at some maximum number of iterations
    or even to require some minimum number of iterations.
  \item \aspace{StatusTestResNorm} - monitors the residual norms of the
    current iterate.
  \item \aspace{StatusTestOrderedResNorm} - monitors the residual
    norms of the current iterate, but only considers the residuals associated with the
    most significant eigenvalues.
  \item \aspace{StatusTestCombo} - a boolean combination of
    other status tests, creating near unlimited potential for complex status tests.
  \item \aspace{StatusTestOutput} - a wrapper around another
    status test, allowing for printing of status information on a call to
    \verb!checkStatus()!.
\end{itemize}

The purpose of a sort manager is to separate the eigensolver classes from the sorting
functionality required by those classes. This satisfies the flexibility principle sought
by Anasazi, by giving users the opportunity to perform the sorting in whatever manner is
deemed to be most appropriate. Anasazi defines an abstract class \aspace{SortManager} with
two methods, one for sorting real values and one for sorting complex values.  Anasazi
provides a concrete implementation called \aspace{BasicSort}.  This class provides basic
functionality for selecting significant eigenvalues: by largest or smallest real part, by
largest or smallest imaginary part, or by largest or smallest magnitude.

Orthogonalization and orthonormalization are commonly performed
computations in iterative eigensolvers. As explained in Section
\ref{sec:algorithm-overview}, all our current implementations are
orthogonal Rayleigh-Ritz methods where an orthonormal basis
representation is computed. The abstract base class
\aspace{OrthoManager} defines a small number of
orthogonalization-related operations, including choice of an inner
product (e.g., Euclidean, induced by a symmetric positive semi-definite
$\mathbf{B}$). Combined with the plethora of available methods for
performing these computations, Anasazi has left as much leeway to
the users as possible. To this end, Anasazi provides two concrete
orthogonalization managers:
\begin{itemize}
\item
  \aspace{BasicOrthoManager} - performs orthogonalization using
  multiple steps of classical Gram-Schmidt \cite{dgks:76}.
\item
  \aspace{SVQBOrthoManager} - performs orthogonalization using the
  SVQB orthogonalization technique described by Stathapoulos and
  Wu~\cite{Stathopoulos:2002:BOP}.
\end{itemize}

In order to perform the Rayleigh-Ritz analysis used by the
algorithms illustrating this section, Anasazi utilizes the classes
\aspace{Teuchos::BLAS} and \aspace{Teuchos::LAPACK}. The purpose of
these classes is to provide templated interfaces to the dense linear
algebra routines provided by the BLAS and LAPACK libraries.
Therefore, even such operations as dense matrix-matrix
multiplication are made independent of the scalar field defining the
eigenvalue problem. Users are therefore currently limited to
algorithms provided by LAPACK.

\section{Benchmarking}
%%%%%%%%%%%%%%%%%%%%%%%%%%%%%%%%%%%%%%%%%%%%%%%%%%%%%%
\label{sec:benchmarking}

\begin{table}
\caption{Comparing the overhead of Anasazi with ARPACK. --- denotes a measurement
below the clock resolution.}
\label{table:timings}
\begin{center}
\begin{tabular}{r|ll|ll|}
       \cline{2-5} %\hline
       % after \\: \hline or \cline{col1-col2} \cline{col3-col4} ...
        & \multicolumn{4}{c|}{Computing $50$ Arnoldi vectors} \\ \cline{2-5}
        & \multicolumn{2}{c|}{Matrix-vector} &
       \multicolumn{2}{c|}{Total}\\ \hline
       Matrix size & ARPACK & Anasazi & ARPACK & Anasazi \\ \hline %\cline{2-5}
       %2500 & 0.010 & 0.018 & 0.47 & 0.54 \\
       10000 & --- & 0.01 & 0.14 & 0.15 \\
       62500 & 0.04 & 0.09 & 1.20 & 1.17 \\
       250000 & 0.15 & 0.32 & 4.98 & 4.79 \\
       1000000 & 0.66 & 1.23 & 19.2 & 18.8 \\
       \hline
        & \multicolumn{4}{c|}{Computing $100$ Arnoldi vectors} \\ \cline{2-5}
        & \multicolumn{2}{c|}{Matrix-vector} &
       \multicolumn{2}{c|}{Total}\\ \hline
       Matrix size & ARPACK & Anasazi & ARPACK & Anasazi \\ \hline %\cline{2-5}
       %2500 & 0.010 & 0.018 & 0.47 & 0.54 \\
       10000 & 0.03 & 0.02 & 0.53 & 0.55 \\
       62500 & 0.03 & 0.17 & 4.37 & 4.29 \\
       250000 & 0.34 & 0.64 & 17.8 & 17.5 \\
       1000000 & 1.27 & 2.40 & 68.4 & 67.1 \\
       \hline
        & \multicolumn{4}{c|}{Computing $150$ Arnoldi vectors} \\ \cline{2-5}
        & \multicolumn{2}{c|}{Matrix-vector} &
       \multicolumn{2}{c|}{Total}\\ \hline
       Matrix size & ARPACK & Anasazi & ARPACK & Anasazi \\ \hline %\cline{2-5}
       %2500 & 0.010 & 0.018 & 0.47 & 0.54 \\
       10000 & 0.03 & 0.04 & 1.15 & 1.22 \\
       62500 & 0.14 & 0.26 & 9.53 & 9.39 \\
       250000 & 0.50 & 0.96 & 38.1 & 38.0 \\
       1000000 & 1.97 & 3.56 & 149 & 146 \\
       \hline
     \end{tabular}
\end{center}
\end{table}



We now discuss the important issue of comparing Anasazi and ARPACK
on a model problem. Our interest is in accessing any overhead of
Anasazi and ARPACK, C++ and FORTRAN 77 software.

We benchmarked Anasazi's \aspace{BlockKrylovSchurSolMgr} (with a block size
of one) and ARPACK's \aspace{dnaupd} that compute approximations to the
eigenspace of a non-symmetric matrix. Our goal was to benchmark the
cost of computing $50, 100, 150$ Arnoldi vectors for a finite
difference approximation to a two dimensional convection diffusion
problem. Both codes use the DGKS \cite{dgks:76} method for
maintaining the numerical orthogonality of the Arnoldi basis
vectors.  The Intel 9.1 C++ and FORTRAN compilers were used with
compiler switches ``-O2 -xP'' on an Intel Pentium D, 3GHz, 1MB L2
cache, 2GB main, Linux/FC5 PC.

The operator application in Anasazi records approximately twice as much time as the ARPACK
implementation. This is because the Anasazi code used an Epetra sparse matrix
representation, while the ARPACK implementation applies the block tridiagonal
matrix via a stencil. Note that the operator application comprised only a small
portion of the clock time in these tests. The performance of the Anasazi library in
computing the Arnoldi vectors is similar to that of ARPACK. Our conclusion is that a
well-designed library in C++ is as efficient as a FORTRAN 77 library.

\section{Acknowledgments}
We thank Roscoe Bartlett, Mike Heroux, Roger Pawlowski, Eric Phipps,
and Andy Salinger for many helpful discussions.

\bibliography{anasazi-toms}
\bibliographystyle{acmtrans}


\begin{received}
????
\end{received}


\end{document}
