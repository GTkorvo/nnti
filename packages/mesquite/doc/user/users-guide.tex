\documentstyle[psfig]{article}

\pagestyle{empty}

\title{ {\LARGE\bf Mesquite User's Guide}}

\author{Patrick Knupp, Lori Freitag, Darryl Melander, Mike Brewer \\
The Sandia National Laboratories \\
Albuquerque NM USA \\
and \\
Thomas Leurent \\
Argonne National Laboratory \\
Argonne, IL 60490 USA}

\date{}


\begin{document}

\maketitle
\begin{abstract} 
MESQUITE is a software library for improving mesh quality. 
\end{abstract}

\tableofcontents

\listoffigures

\listoftables

\section{Introduction to Mesh Quality Improvement} 

\subsection{General Overview}
What do we mean by mesh quality. Why is it important? A priori vs. a posteriori
quality control.

Goals:
\begin{itemize}
\item To provide a vision for Mesquite (what it is and what it will become),
\item To cover both noice and expert users. The novice should know what to
do after reading a small percentage of the guide (perhaps a third); more 
advanced concepts are also included but are done later in the text.
\item To provide practical instructions on using Mesquite API, compiling,
linking, TSTT interface,
\item To provide insight into the optimization algorithms
\end{itemize}

\subsection{Mesquite Philosophy}
The hippocratic oath: we will do no harm (can we do an undo or should the 
user keep track of the original mesh?).  What are the boundaries of Mesquite?
What it is not going to do. What do we mean by a mesh quality improvement 
algorithm. Including the best known algorithms for a particular goal (not 
just a collection of hacked tools). Provides both a useful tool that is easy
to use for application scientists and is extensible to provide a platform for
mesh quality improvment research. Give an example API file (script for 
Laplace smoothing - novice). Show the expert view of the world. Readers can
use script to build from. Use example to help explain our philosophy.

\subsection{The Mesquite Vision}
Comprehensive, robust, flexible, efficient. In core to applications, not 
just a preprocessign step.

\section{Mesh Improvement Goals}
Why have different mesh improvement goals? What are they (shape, untangle, 
alignment, morph, etc.). Improvement goals currently supported (shape, untangle).

\section{Methods - Theory and Algorithms}

\subsection{Vertex Movement Schemes}
Why use an optimization-based approach. Posing mesh quality as an optimization
problem (metrics, objective function templates and goals, L1, L2, Linf).
Optimization Algorithms (brief description, enough to have the API discussion 
make sense): variational, discrete, solvers (CG, active set), local vs. global,
fitting Laplacian and other common smoothers into this framework.

\subsection{Topology Modifiers}
Swapping algorithms.

\section{Basic Mesquite Usage}

\subsection{Telling Mesquite About Your Mesh}
TSTT and Mesquite interface overview. Particular interfaces needed. 
Testing your TSTT interface implementation.

\subsection{Wrapper Classes}
Provides sophisticated functionality via simple interface. List the ones
provided (goal, stopping criterion, customization options).

\subsection{Compilling and Linking Mesquite}
Software dependencies, compiling AOMD, TSTT interfaces (need the .h file),
other (gmake, makedepend)?  Makefiles (examples that link Mesquite to an 
application code). Compiler flags and options (debugging, profiling). 
Platforms supported. Optional functionalities (debugging, profiling, 
error handling). 

\subsection{Hints for Performance Profiling}
Compiler options, aggregation.

\section{Software Design}

\subsection{Overview}
Why did we choose C++ (extensibility)? Achieving performance in the 
numerical kernals.

\subsection{Mesquite Classes}
QualityMetric, Objective Function Templates, Vertex Movers, Instruction Queue,
Stopping Criterion, MeshSet and PatchData

\subsection{Directory Structure}

\subsection{Example Programs}

\subsection{Testing Philosophy}

\section{Advanced Use of Mesquite}

\subsection{The Simplified Geometry Engine}

\subsection{Creating a Quality Metric}
What is a quality metric? Define full range, acceptable range, what is means 
if the metric uses a map, nodal-invariance, well-posed metrics, averaging 
methods, sample points. Give a list of the individual metrics we support. 
Provide a key to the tables.

\begin{table}[h]
\begin{center}
\begin{tabular}{|l|c|c|c|c|c|}
\hline
Metric Name & Type & Dimension & Full Range & Ideal & Degenerate \\ \hline
Inverse Mean Ratio & Shape & 2D, 3D & $[0,1]$ & 1 & 0 \\ 
Mean Ratio &  &  &  &  &  \\ 
Condition Number &  &  &  &  &  \\ 
Untangle &  &  &  &  &  \\ 
Aspect Ratio Gamma &  &  &  &  &  \\ 
\hline
\end{tabular}
\caption{\label{QualityMetrics1} Mesquite Quality Metrics Summary, Part I}
\end{center}
\end{table}

\begin{table}[h]
\begin{center}
\begin{tabular}{|l|c|c|c|c|c|c|}
\hline
Metric Name & Map? & Averaging & Sample Pts & Elements & Gradient & Source \\ \hline
Inverse Mean Ratio & No & Harmonic & Vertices & TQTH & Analytic & \\ 
Mean Ratio &  &  &  &  &  \\ 
Condition Number &  &  &  &  &  \\ 
Untangle &  &  &  &  &  \\ 
Aspect Ratio Gamma &  &  &  &  &  \\ 
\hline
\end{tabular}
\caption{\label{QualityMetrics2} Mesquite Quality Metrics Summary, Part II}
\end{center}
\end{table}

\noindent {\bf Mean Ratio Metric} \newline
Give description, formulas. 

\subsection{Creating an Objective Function}
List of all the templates. What is the difference between a template and
an obective function? For each element of the mesh (or for each 
free-node of the mesh), let there be an associated quality metic, 
$\mu_m$.  Define the vector 
\begin{equation}
{\bf U} = [ \mu_1, \mu_2, \ldots, \mu_M ]
\end{equation}
where $M$ is the number of elements or free-nodes in the mesh. \newline

\noindent Available templates in Mesquite: \newline

\noindent {\bf The $\ell_p$ Template} \newline
Given $1 \leq p < \infty$, let
\begin{equation}
\| {\bf U} \|_p = ( \sum_{m=1}^M \mid \mu_m \mid^p )^{1/p}
\end{equation}
be the $\ell_p$ template. \newline

\noindent {\bf The $\ell_p^p$ Template} \newline
Given $1 \leq p < \infty$, let $\| {\bf U} \|_p^p$ be the 
$\ell_p^p$ template. \newline

\noindent {\bf The $\ell_{\infty}$ Template} \newline
The $\ell_{\infty}$ template is
\begin{equation}
\| {\bf U} \|_{\infty} = \max_{m=1,\ldots,M} \mid \mu_m \mid
\end{equation}

These are currently the only tempates available in Mesquite.
We soon hope to add the 'max' template:
\begin{equation}
MAX = \max_{m=1,\ldots,M} ( \mu_m )
\end{equation}

Which of these templates has numerical/analytic gradient/Hessian?

\subsection{Creating a Vertex Mover}
List solvers and appropriate templates. Is a solver the same things as a 
vertex mover? List options.

\begin{table}[h]
\begin{center}
\begin{tabular}{|l|c|c|c|c|c|}
\hline
Solver Name & Description & Templates & L/G & Num/Anal & Fixed Vertices? \\ \hline
Steepest Descent & Clasical & $\ell_p$ & Both & Both & Yes \\
Conjugate Gradient & & & & & \\
Feasible Newton & & & & & \\
Active Set & & Own & & & \\
Laplace & & None & & & \\
Randomize & & None & & & \\
\hline
\end{tabular}
\caption{\label{Solvers} Mesquite Solver Summary}
\end{center}
\end{table}


\subsection{Creating a Topology Modifier}

\subsection{Termination Criteria}
List, along with uses.

\subsection{Culling Algorithms}

\subsection{Quality Assessment}

\subsection{The Instruction Queue}
Setting the master quality improver, setting up and using preconditioners,
using quality assessors, running the instruction queue.

\subsection{Composite Metrics}

\subsection{Composite Objective Functions}
The only ones currently available are CompositeMult, CompositeAdd,
CompositeScalarMult, and CompositeScalarAdd.

\subsection{Running in Parallel}

\section{Extending Mesquite}
Adding a new quality metric, adding a new template, adding a new vertex 
mover.

\section{Troubleshooting}

\section{Caveats, Limitations, and Disclaimers}

\section{User Support}
Mailing lists, web site, examples, tutorials (pointer to slides?), 
instructions on downloading the software (open source, preparing 
derivative works), referencing Mesquite.

\section{The Mesquite Team}
Development team, external contributors.

\section{The Mesquite Development Plan}
Timeline for proposed additions to the software.

\section{Acknowledgements}
Funding sources (DOE SciDAC Program, Chuck Romine), Related activities 
(TSTT, Cubit, Opt-MS), helpful folks (RPI for AOMD, TSTT for MDB, 
LLNL for Overture tests, etc.

\begin{thebibliography}{999}
\bibitem{Hawaii}
L. Freitag, P. Knupp, T. Leurent, and D. Melander, {\it MESQUITE Design: Issues in the Development of a Mesh Quality Improvement Toolkit}, p159-168, Proceedings of the 8th Intl. Conference on Numerical Grid Generation in Computational Field Simulations, Honolulu 2002.

\bibitem{unt}
P. Knupp, {\it Hexahedral and Tetrahedral Mesh Untangling}, Engineering with Computers, Vol. 17, No. 3, pp261-268, 2001.

\bibitem{Nocedal}
J. Nocedal and S. Wright, {\it Numerical Optimization}, Springer, New York, 1999.

\end{thebibliography}

\newpage
\appendix{\large {\bf Appendix A: The Conjugate Gradient Algorithm }} \newline
\label{append_A}
\newpage
\appendix{\large {\bf Appendix B: The Feasible Newton Algorithm }} \newline
\label{append_B}
\newpage
\appendix{\large {\bf Appendix C: The Active Set Method }} \newline
\label{append_C}

\end{document}

