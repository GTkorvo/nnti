% alternative names
%\chapter{Telling Mesquite About Your Mesh}
\chapter{Mesquite Interactions with Your Mesh and Geometry}
\label{sec:meshes}

Mesquite is responsible for gathering information from the
application's mesh and geometry.  Because this must be as efficient as
possible, considerable attention has been given to the interface
between Mesquite and the application code.  Because Mesquite is
designed as a library to work on a broad assortment of mesh and
element types on complex geometrical domains, a general,
data-structure neutral API is needed.  In general, Mesquite requires
access to basic information about the mesh such as the number of
vertices and elements in the mesh, the vertex locations, and the
element connectivities.  To move the vertex locations, Mesquite needs
to set the vertex coordinate positions, and eventually, to perform
swapping operations Mesquite will need to add and delete various mesh
entities.  In addition, for smoothing meshes on complex surfaces,
access to operations on the underlying solid model such as normal
information and closest point information are required to ensure
vertices are constrained to the surface.

\section{Reading in a Mesh File} \label{sec:meshFiles}

The simplest way to start using mesquite is to read a mesh file in one of the format currently
supported by mesquite. Currently, mesquite has a class that implements its mesh access interface and
supports the VTK \cite{VTKbook, VTKuml} and EXODUS formats: \texttt{MeshImpl}, as shown in the tutorial (section
\ref{sec:tutMesh}). The functions \texttt{MeshImpl::read\_vtk()}
and \texttt{MeshImpl::read\_exodus()} can be used as template to read instead your own mesh format. This is not
recommended, since ultimately you want to run mesquite dynamically to interact with your mesh
database and implementing the Mesquite Mesh Interface (section \ref{sec:msq_mesh}) should require the same amount of work. 

If you plan to link in with the EXODUS proprietary library, you will need to define the following
compile option in Mesquite's Makefile.customize: 

  [{\bf EXODUS\_LIB\_DIR}] is an optional variable that holds the location of the Exodus II library. 
({\it E.g.,} EXODUS\_LIB\_DIR = \$\{MSQ\_BASE\_DIR\}/external/exodus/lib).

\section{Implementing the Mesquite Mesh Interface} \label{sec:msq_mesh}

%Most users will want to skip this section and avoid the associated work. There are several manners
%to give the mesh data to Mesquite, and implementing the Mesquite mesh interface is the one that
%requires the most coding work. 
%
%Let's review the alternative in the first place. For very simple meshes, Mesquite has an integrated
%mesh database that can read in mesh files in specific format, such as VTK (see section \ref{sec:meshFiles}). 
%On the other hand, Mesquite is compatible with the TSTT mesh interface: this means that if you are
%running a code that has implemented the TSTT mesh interface, you can link Mesquite with your code
%and perform all mesh improvements through the TSTT mesh interface (see section \ref{sec:TSTT}).
%Implementing the TSTT mesh interface also has the advantage that your code will be compatible with
%the TSTT mesh interface standard: this is a very good investment. 
%
%If you want to implement the specific Mesquite mesh interface, you probably have very specific
%reasons in mind ... and you know what you are doing. In this section, we will give you the
%documentation for the abstract classes. This will allow you to implement the interface. 

\subsection{Mesh Interface Concepts}

To expose this information, Mesquite defines a set of interfaces 
(C++ abstract base classes) that are specifically designed for mesh
quality improvement needs.  There are four such interfaces: \texttt{Mesh},
\texttt{VertexIterator}, \texttt{ElementIterator}, and \texttt{MeshDomain}.
\begin{itemize}
\item \texttt{Mesh}: The Mesh interface represents the set of mesh
entities that are to be operated on.  It is through this interface
that one retrieves information about the mesh and its entities.
Examples of functionality provided by this interface include:
retrieving the number of elements in the mesh, determining which
elements contain a particular vertex, and modifying vertex
coordinates.
\item \texttt{VertexIterator}: The VertexIterator provides access to each
vertex in a mesh.  A VertexIterator is obtained from a Mesh object,
and is used to iterate through the list of all vertices in the Mesh
from which it was obtained.
\item \texttt{ElementIterator}: The ElementIterator provides access to
each element in a mesh.  Other than the type of entity it exposes, it
is identical to the VertexIterator.
\item \texttt{MeshDomain}: The MeshDomain represents the set of geometric
domains to which the mesh may be constrained. Its implementation is \emph{not} systematically 
required. See section \ref{sec:geometry} for more details. 
%The MeshDomain
%interface enables an application to restrict the locations to which a
%vertex can be moved, such as constraining a vertex to a surface.
%Through the MeshDomain interface, Mesquite's algorithms can also
%obtain a domain's normal vector, which aides validity checking and 
%decision making during the quality improvement process.
\end{itemize}
These interfaces are data-structure neutral and use only primitive
data types; an application may implement the Mesquite interfaces
without changing its existing mesh data structures.  Instead of
representing mesh entities with complex data structures or with typed
pointers, entities are identified with opaque values called handles.
Each mesh entity has a unique handle value, but otherwise handles have
no intrinsic meaning to Mesquite.  

\subsection{Implementation Details}

The implementation details must be looked up in the doxygen documentation located in mesquite/doc/developer.

\subsection{Testing Your Implementation}

The implementation of the Mesquite mesh interface is considerably eased by using the unit testing
framework. From the mesquite/testSuite/unit directory, change the MeshInterfaceTest.cpp file to
instantiate your mesh interface implentation (i.e. replace occurences of \texttt{MeshImpl} with your
class). Run gmake and run the test:
\begin{verbatim} 
./msq_test MeshInterfaceTest
\end{verbatim} 
This should help you pinpoint the vast majority of implementation errors. 


\section{Linking Mesquite with Implementations of the TSTT Mesh Interface} 
\label{sec:TSTT}

Mesquite can instead use the mesh interfaces currently
being developed through the TSTT center.  This interface definition
effort focuses on providing access to information pertaining to low
level mesh objects such as vertices, edges, faces, and regions through
both array-based and iterator-based mechanisms.  It is designed to
support existing packages such as CUBIT, NWGrid, PAOMD, and Overture. 
The TSTT interface achieves language interoperability
through use of the SIDL/Babel tools from LLNL \cite{babel}.
Achieving consensus within a large group of participants is
paramount (see \cite{Cubit-website}, \cite{overture}, \cite{aomd-imr},
and \cite{NWGrid-website}).  
%This interface definition effort is
%evolving, and the Mesquite team is actively participating to ensure
%that our needs for mesh quality improvement are adequately and
%efficiently addressed.  
A TSTT-based implementation of the Mesquite
interfaces is available.  As such, any tool that exposes its
mesh through the TSTT interfaces can be used with Mesquite without
additional development.

Note that the Mesquite-specific interface described in section \ref{sec:msq_mesh} is fully compatible
with the current TSTT mesh and geometry interfaces, and in fact,
Mesquite's approach to data structure neutrality is directly derived
from the TSTT interfaces.  Although similar in spirit to the TSTT
interface, the Mesquite-specific interface is not as general, and 
therefore consist of fewer
functions and does not require additional tools such as Babel.

\section{The Geometry Engine}
\label{sec:geometry}

Certain smoothing procedures in Mesquite require some geometry information. The mean ratio quality
metric for example requires the normal of a 2D element in a surface mesh. Also, boundary
vertices can be snapped back to the surface they belong to during a smoothing procedure, instead of
simply fixing the boundary vertices (and therefore lowering the quality of the final mesh). A simple abstract
class, \texttt{MeshDomain}, must be inherited and implemented by the user in order to provide the
geometry information to Mesquite.  This class contains only two virtual abstract functions:
\texttt{MeshDomain::normal\_at} and \texttt{MeshDomain::snap\_to} for which the details are given in
the doxygen documentation (in the directory mesquite/doc/developer).

Note that MeshDomain does not systematically needs to be implemented. For example it is possible to
improve a 3D mesh with any metric by fixing the boundary vertices. If the surface mesh is of good
quality but the volume mesh is poor, Mesquite can generates a high quality mesh without accessing
any geometry information. 


\subsection{The Simplified Geometry Engine}

We provide implementations of the MeshDomain class for basic geometries, such as a plan
(\texttt{PlanarDomain}) and a sphere (\texttt{SphericalDomain}). Those classes allows the user to use
Mesquite straight away without any implementation. They are introduced in the tutorial section
\ref{sec:tutMesh} and are described in details in the doxygen documentation.

\section{Visualizing the Mesh}

\begin{verbatim}
advices. pointer to VTK doc. 
pointer to VTK -> Exodus II converter
\end{verbatim}
