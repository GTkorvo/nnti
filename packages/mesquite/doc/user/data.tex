% alternative names
%\chapter{Telling Mesquite About Your Mesh}
\chapter{Accessing Mesh and Geometry}
\label{sec:meshes}

The application must provide Mesquite with data on which to operate.  The two
fundamental classes of information Mesquite requires are:
\begin{itemize}
\item Mesh vertex coordinates and element connectivity, and
\item Constraints on vertex movement.
\end{itemize}
For some types of optimization algorithms, Mesquite may also require the ability 
to associate arbitrary data (tags) with mesh entities.

The interfaces Mesquite uses to access this data are \texttt{Mesquite::Mesh}
and \texttt{Mesquite::MeshDomain}.  They are abstract classes defined in
\texttt{include/MeshInterface.hpp}.  An instance of the \texttt{Mesquite::Mesh}
interface is required, while providing a \texttt{Mesquite::MeshDomain}
implementation is optional.

\section{Mesh Data} \label{sec:MeshData}

All methods for accessing mesh data are declared in the \texttt{Mesquite::Mesh}
interface.  It is expected that most applications will provide a custom
implementation of this interface, composed of call-backs to the application
allowing Mesquite direct in-memory access to application mesh data.  However,
Mesquite includes several alternate methods of accessing mesh data that will be
described later in this chapter.

Mesquite accesses mesh data using `handles'.  There must be a unique handle
space for all vertices, and a separate unique handle space for all elements. 
That is, there must be a one-to-one mapping between handle values and vertices,
and a one-to-one mapping between handle values and elements.  The storage type of
the handles is specified by \texttt{Mesquite::Mesh::VertexHandle} and
\texttt{Mesquite::Mesh::ElementHandle}.  The handle types are of sufficient size
to hold either a pointer or an index, allowing the underlying implementation of
the \texttt{Mesquite::Mesh} interface to be either pointer-based or index-based. 
All mesh entities are referenced using handles.  For example, the
\texttt{Mesquite::Mesh} interface declares a method to retrieve the list of all
vertices as an array of handles and a method to update the coordinates of a
vertex where the vertex is specified as a handle.

An implementation of the \texttt{Mesquite::Mesh} interface must also provide
implementations of the \texttt{Mesquite::ElementIterator} and
\texttt{Mesquite::VertexIterator} interfaces.  Mesquite uses these interfaces
to avoid the need for allocating very large arrays of handles when performing
Nash-game optimizations on very large meshes.  It is recommended that
implementations of the \texttt{Mesquite::Mesh} interface provide implementations
of these iterator interfaces.  However, if the storage of large arrays of handles
is not deemed to be important, an implementation of the \texttt{Mesquite::Mesh}
interface may use the implementations defined in \texttt{ArrayIterator.hpp}. 
These default implementations use the methods in the \texttt{Mesquite::Mesh}
interface to retrieve a list of all vertex (or element) handles and iterate over
the resulting array.

Mesquite does not utilize or understand 1D (edge or bar) elements.  Thus
implementations of the \texttt{Mesquite::Mesh} interface are expected to return
only surface (triangle, quadrilateral, etc.) or volume (tetrahedron, hexahedron,
etc.) elements from all queries for elements.

In addition to providing methods to query mesh data, the \texttt{Mesquite::Mesh}
interface is expected to implicitly define the active mesh that Mesquite is to
optimize.  If Mesquite is expected to optimize a subset of the mesh (e.g. the
mesh in one of several geometric volumes), the methods in the interface that
implement queries of global mesh properties are expected to limit their results
to the active subset of the mesh.  Such queries include:
\begin{itemize}
\item get all elements in the mesh
\item get all vertices in the mesh
\item iterate over all vertices in the mesh
\item iterate over all elements in the mesh
\item get elements adjacent to a vertex.
\end{itemize}

It is recommended that the application invoke the Mesquite optimizer for subsets
of the mesh rather than the entire mesh whenever it makes sense to do so.  For
example, if a mesh of two geometric volumes is to be optimized and all mesh
vertices lying on geometric surfaces are constrained to be fixed (such vertices
will not be moved during the optimization) then optimizing each volume separately
will produce the same result as optimizing both together.  

\section{Vertex Constraints and Geometric Domains} \label{sec:geom}

Vertex locations can be constrained via one of two mechanisms.  Vertices can
either be designated as `fixed' or constrained to a geometric domain.  The
`fixed' designation supersedes any associated geometric domain.  For an
optimization problem to be well defined, it is expected that any vertices on the
boundary of a mesh be constrained via one of these mechanisms.  Mesquite is 
implemented using $\mathbb{R}^{3}$ vertex coordinates.  A planar geometric domain
is typically used to define an $\mathbb{R}^{2}$ optimization in Mesquite 
(see Section \ref{sec:MsqGeom}).

The location of a `fixed' vertex is never modified by Mesquite.  Typically, the 
vertices on the boundary of the active mesh are designated as `fixed'.  The fixed
designation for vertices is communicated to Mesquite using the
\texttt{vertices\_get\_fixed\_flag} method in the \texttt{Mesquite::Mesh} interface.

Vertex positions may be constrained to a geometric domain by providing Mesquite
with an optional instance of the \texttt{Mesquite::MeshDomain} interface.  This
interface provides two fundamental capabilities: mesh-geometry classification,
and interrogation of local geometric properties.  The methods defined in the 
\texttt{Mesquite::MeshDomain} interface combine both queries into single
operation.  Queries are passed a mesh entity handle (see Section \ref{sec:MeshData}),
and are expected to interrogate the geometric domain that the specified mesh
entity is classified to.

If Mesquite is used to optimize the mesh of a B-Rep solid model (the typical CAD
system data model), then the domain is composed of geometric vertices,
curves, surfaces, and volumes.  Curves are bounded by end vertices, surfaces are
bounded by loops (closed chains) of curves, and volumes are bounded by groups of
surfaces.  Mesquite expects each surface element (triangle, quadrilateral, etc.)
to be associated with a 2D domain (surface).  Vertices may be associated with
a geometric entity that either contains adjacent mesh elements or bounds the
geometric entity containing the adjacent elements. Mesquite does not use 
geometric volumes.  A query for the closest location on the domain for a vertex 
or element whose classification is a geometric volume should simply return the 
input position.  

It is possible to define an optimization problem such that mesh classification
data need not be provided in a \texttt{Mesquite::MeshDomain} implementation. 
This is done by optimizing the mesh associated with each simple geometric
component of the domain separately, with the boundary vertices flagged as fixed.
The following pseudo-code illustrates such an approach for a B-Rep type geometric
domain:
\begin{verbatim}
  for each geometric vertex
    make associated vertex as fixed
  end-for
  for each curve
    do any application-specific optimization of curve node placement
    mark associated mesh vertices as fixed
  end-for
  for each surface
    define Mesquite::MeshDomain for surface geometry
    invoke Mesquite to optimize surface mesh
    mark all associated mesh vertices as fixed
  end-for
  for each volume
    invoke Mesquite to optimize volume mesh w/ no Mesquite::MeshDomain
  end-for
\end{verbatim}

\section{Extra Data Associated With Mesh} \label{tags}

Mesquite contains several tools capable of querying and storing other
data that is associated with individual mesh entities.  For example, Mesquite
can pre-calculate and store target matrices for use in target-based local
metrics.  This is accomplished
using a `tag' model similar to that defined by the TSTT\cite{tstt} interfaces.
A tag is identified by a tuple composed of a tag handle and an entity handle. 
The tag handle corresponds to a definition including a name, data type, and size.
 The entity handle specifies which mesh entity the tag value is associated with. 
 
Methods for interacting with tag data are declared in the \texttt{Mesquite::Mesh}
interface.  If the optimization algorithm used does not require this
functionality, the implementation of the \texttt{Mesquite::Mesh} interface
need not provide useful implementations of these methods.
 
\section{Reading Mesh From Files} \label{sec:meshFiles}

Mesquite provides a concrete implementation of the \texttt{Mesquite::Mesh} named
\texttt{Mesquite::MeshImpl}.  This implementation is capable of reading mesh from
VTK\cite{VTKbook, VTKuml} and optionally ExodusII files. See Sections 
\ref{sec:depends} and \ref{sec:compiling} for more 
information regarding the optional support for ExodusII files.

The `fixed' flag for vertices is can be specified in VTK files by defining a
SCALAR POINT\_DATA attribute with values of 0 or 1, where 1 indicates that the
corresponding vertex is fixed.  The \texttt{Mesquite::MeshImpl} class is capable
of reading and storing tag data (see Section \ref{tags}) using VTK attributes and
field data.  The current implementation writes version 3.0 of the VTK file format
and is capable of reading any version of the file format up to 3.0.  

\section{Accessing Mesh In Arrays} \label{sec::ArrayMesh}

One common representation of mesh in applications is composed of simple 
coordinate and index arrays.  Mesquite provides the \texttt{ArrayMesh} implementation of the \texttt{Mesquite::Mesh} interface to allow Mesquite
to access such array-based mesh definitions directly.  The mesh must be
defined as follows:
\begin{itemize}
\item Vertex coordinates must be stored in an array of double-precision
      floating-point values.  The coordinate values must be interleaved,
      meaning that the x, y, and z coordinate values for a single vertex
      are contiguous in memory.
\item The mesh must be composed of a single element type.
\item The element connectivity (vertices in each element) must be stored
      in an interleaved format as an array of long integers.  The vertices
      in each element are specified by an integer \texttt{i}, where the location       of the coordinates of the corresponding vertex is located at position
      \texttt{3*i} in the vertex coordinates array.
\item The fixed/boundary state of the vertices must be stored in an array
      of integer values, where a value of 1 indicates a fixed vertex and a 
      value of 0 indicates a free vertex.  The values in this array must
      be in the same order as the corresponding vertex coordinates in the
      coordinate array.
\end{itemize}

The following is a simple example of using the ArrayMesh object to pass
Mesquite a mesh containing four quadrilateral elements.
\begin{verbatim}
  /*** define some mesh ***/
    /* vertex coordinates */
  double coords[] = { 0, 0, 0,
                      1, 0, 0,
                      2, 0, 0,
                      0, 1, 0,
                     .5,.5, 0,
                      2, 1, 0,
                      0, 2, 0,
                      1, 2, 0,
                      2, 2, 0 };
    /* quadrileratal element connectivity (vertices) */
  long quads[] = { 0, 1, 4, 3,
                   1, 2, 5, 4,
                   3, 4, 7, 6,
                   4, 5, 8, 7 };
    /* all vertices except the center on are fixed */
  int fixed[] = { 1, 1, 1,
                  1, 0, 1,
                  1, 1, 1 };
  
  /*** create an ArrayMesh to pass the above mesh into Mesquite ***/
  
  ArrayMesh mesh( 3,            /* 3D mesh (three coord values per vertex) */
                  9,            /* nine vertices */
                  coords,       /* the vertex coordinates */ 
                  fixed,        /* the vertex fixed flags */
                  4,            /* four elements */
                  QUADRILATERAL,/* elements are quadrilaterals */
                  quads );      /* element connectivity */
  
  /*** smooth the mesh ***/
  
    /* Need surface to constrain 2D elements to */
  PlanarDomain domain( PlanarDomain::XY );

  MsqError err;
  ShapeImprovementWrapper shape_wrapper( err );
  if (err) {
    std::cout << err << std::endl;
    exit (2);
  }
  
  shape_wrapper.run_instructions( &mesh, &domain, err );
  if (err) {
    std::cout << "Error smoothing mesh:" << std::endl
              << err << std::endl;
  }
  
  /*** Output the new location of the center vertex ***/
  std::cout << "New vertex location: ( "
            << coords[12] << ", " << coords[13] << ", " << coords[14]
            << " )" << std::endl;
\end{verbatim}

NOTE:  When using the \texttt{ArrayMesh} interface, the application is responsible for managing the storate of the mesh data.  The \texttt{ArrayMesh}
 does NOT copy the input mesh.  

\section{The TSTT Interfaces} \label{sec:TSTT}

Mesquite can access mesh and domain data using the TSTT \cite{tstt} interfaces 
defined by the SciDAC ISICs.  The TSTT interfaces are language-interoperable 
interfaces defined using SIDL, the Scientific Interface Definition Language
\cite{babel}.  See \texttt{http://www.tstt-scidac.org/software/} for more
about the TSTT interfaces.

Mesquite provides the \texttt{Mesquite::MeshTSTT} implementation of the 
\texttt{Mesquite::Mesh} interface as an adapter for using the TSTT Mesh
interface.  The \texttt{Mesquite::GeomTSTT} class is a
\texttt{Mesquite::MeshDomain} adapter for the TSTT geometry and relation
interfaces.  If mesh-geometry classification is not required (see Section
\ref{sec:geom}), an instance of the \texttt{Mesquite::GeomTSTT} class can be
created that relies only on the TSTT geometry interface.  

The \texttt{Mesquite::MeshTSTT} adapter expects a single-integer tag named 
``MesquiteVertexFixed'' with a value of 1 for all fixed vertices.

\section{Simple Geometric Domains} \label{sec:MsqGeom}

Mesquite provides several implementations of the
\texttt{Mesquite::MeshDomain} interface for simple geometric primitives.  These
include
\begin{itemize}
\item \texttt{PlanarDomain}: An unbounded planar domain.
\item \texttt{SphericalDomain}: A closed spherical domain.
\item \texttt{CylindarDomain}: An unbounded cylindrical domain.
\item \texttt{BoundedCylindarDomain}: A bounded cylindrical domain.
\end{itemize}
The \texttt{PlanarDomain} is often used to map $\mathbb{R}^{2}$ optimization
problems to $\mathbb{R}^{3}$.  The others are used primarily for testing
purposes.  The \texttt{BoundedCylindarDomain} provides some simplistic
mesh-geometry classification capabilities.  The others do not provide any
classification functionality.  
