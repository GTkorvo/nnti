\chapter{Mesquite Wrapper Descriptions}
\label{sec:wrappers}

Applications which desire to access Mesquite capabilities without delving 
into the low-level API can invoke wrappers to perform basic mesh quality 
improvement tasks that, except for a few user-defined inputs, are fully 
automatic. The wrappers target classic mesh optimization problems that occur 
repeatedly across many applications. See section 3.1.3 for an example of how
to invoke a wrapper. 
This chapter provides a summary of the current Mesquite wrappers. \newline

\noindent Note that the wrappers do not, by themselves, completely define
the optimization problem.  The user still has to set the fixed/free flags,
and the values of the termination criteria.  \newline

\section{Laplace-smoothing} \label{sec:LaplacianIQ}

\noindent {\it Name}: LaplacianIQ \newline
\noindent {\it Purpose}: Produce a {\it smooth} mesh. \newline
\noindent {\it Notes}: This is a local patch relaxation-solver. A 'smart' 
Laplacian solver is also available in Mesquite, but it is not used in this 
wrapper.  \newline
\noindent {\it Limitations/assumptions}: No invertibility guarantee. \newline 
\noindent {\it Input Termination Criterion}: Stop after 10 global iterations. \newline \newline

\noindent Under the Cover: \newline
\noindent {\it Hardwired Parameters}: None \newline
\noindent {\it Mesh/Element Type}: Any supported type. \newline
\noindent {\it Global/Local}: Local Patch \newline



\section{Shape-Improvement} \label{sec:ShapeImprovementWrapper}

\noindent {\it Name}: ShapeImprovementWrapper \newline
\noindent {\it Purpose}: Make the shape of an element as close as possible to 
that of the ideal/regular element shape. For example, make triangular and 
tetrahedral elements equilateral. \newline
\noindent {\it Notes}: If the initial mesh is inverted, then an attempt to 
untangle the mesh is made. If the untangling effort fails, then the wrapper
aborts.  \newline
\noindent {\it Limitations/assumptions}:  \newline 
\noindent {\it Input Termination Criterion}: CPU time limit (not used if input 
value is non-positive) or L2 norm of the 
gradient of the objective function (default = 1.e-06). Relative successive 
improvement (1e-04). \newline \newline

\noindent Under the Cover: \newline
\noindent {\it Hardwired Parameters}: Untangle Beta = 1.e-8 \newline
\noindent {\it Metric}: IdealWeightInverseMeanRatio \newline
\noindent {\it Objective Function}: L2 norm of element metrics \newline
\noindent {\it Mesh/Element Type}: Any supported type. \newline
\noindent {\it Solver}: Feasible Newton \newline
\noindent {\it Global/Local}: Global \newline


%\section{The Parallel Shape-Improvement Wrapper} \label{sec:ParallelShapeImprovementWrapper}

%\noindent {\it Name}: ParallelShapeImprovementWrapper \newline
%\noindent {\it Purpose}: Same as the ShapeImprovementWrapper, but via synchronous 
%mesh optimization, so that the mesh is 'smooth' across processor boundaries. \newline



\section{Minimum Edge-Length Improvement} \label{sec:PaverMinEdgeLengthWrapper}

\noindent {\it Name}: PaverMinEdgeLengthWrapper \newline
\noindent {\it Purpose}: Make all the edges in the mesh of equal length while 
moving toward the ideal shape. Intended for explicit PDE codes whose 
time-step limination is governed by the minimum edge-length in the mesh. \newline
\noindent {\it Notes}: Based on Target-matrix paradigm. \newline
\noindent {\it Limitations/assumptions}: Initial mesh must be non-inverted. User does not want to preserve or create anisotropic elements. \newline 
\noindent {\it Input Termination Criterion}: maximum iterations (default=50), maximum absolute vertex movement \newline \newline

\noindent Under the Cover: \newline
\noindent {\it Hardwired Parameters}: None \newline
\noindent {\it Metric}:  Target2DShapeSizeBarrier or Target3DShapeSizeBarrier \newline
\noindent {\it Tradeoff Coefficient}: 1.0 \newline
\noindent {\it Objective Function}: Linear Average over the Sample Points \newline
\noindent {\it Mesh/Element Type}: Any supported type.  \newline
\noindent {\it Solver}: Trust Region \newline
\noindent {\it Global/Local}: Global \newline

 
\section{Improve the Shapes in a Size-adapted Mesh} \label{sec:SizeAdaptShapeWrapper}

\noindent {\it Name}: SizeAdaptShapeWrapper \newline
\noindent {\it Purpose}: Make the shape of an element as close as possible to that
of the ideal element shape, {\it while preserving, as much as possible, the size of each element in the mesh}. To be used on isotropic initial meshes that are already size-adapted. \newline
\noindent {\it Notes}: Based on Target-matrix Paradigm. \newline
\noindent {\it Limitations/assumptions}: Initial mesh must be non-inverted. 
User wants to preserve sizes of elements in initial mesh and does not want to preserve or 
create anisotropic elements.  \newline 
\noindent {\it Input Termination Criterion}: maximum iterations (default=50), maximum absolute vertex movement \newline \newline

\noindent Under the Cover: \newline
\noindent {\it Hardwired Parameters}: None \newline
\noindent {\it Metric}: Target2DShapeSizeBarrier or Target3DShapeSizeBarrier \newline
\noindent {\it Tradeoff Coefficient}: 1.0 \newline
\noindent {\it Objective Function}: Linear Average over the Sample Points \newline
\noindent {\it Mesh/Element Type}: Any supported type. \newline
\noindent {\it Solver}: Trust Region \newline
\noindent {\it Global/Local}: Global \newline


\section{Improve Sliver Tets in a Viscous CFD Mesh} \label{sec:ViscousCFDTetShapeWrapper}

\noindent {\it Name}: ViscousCFDTetShapeWrapper \newline
\noindent {\it Purpose}: Improve the shape of sliver elements in the far-field 
of a CFD mesh while 
preserving an existing layer of sliver elements in the boundary layer. \newline
\noindent Notes: Based on Target-matrix paradigm.  \newline
\noindent {\it Limitations/assumptions}: Tetrahedral meshes only.  \newline 
\noindent {\it Input Termination Criterion}: Iteration Count (default=50) or Maximum Absolution Vertex Movement \newline \newline
 
\noindent Under the Cover: \newline
\noindent {\it Hardwired Parameters}: In tradeoff coefficient model. \newline
\noindent {\it Metric}: Target2DShape+Target2DShapeSizeOrient (or 3D) (or Barrier) \newline
\noindent {\it Tradeoff Coefficient}: Based on element dihedral angle \newline
\noindent {\it Objective Function}: Linear average over Sample Points \newline
\noindent {\it Mesh/Element Type}: Tetrahedra \newline
\noindent {\it Solver}: Trust Region \newline
\noindent {\it Global/Local}: Global \newline


