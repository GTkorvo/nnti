\chapter{Mesquite Wrapper Descriptions}
\label{sec:wrappers}

Applications which desire to access Mesquite capabilities without delving 
into the low-level API can invoke wrappers to perform basic mesh quality 
improvement tasks that, except for a few user-defined inputs, are fully 
automatic. The wrappers target classic mesh optimization problems that occur 
repeatedly across many applications. See section 3.1.3 for an example of how
to invoke a wrapper. 
This chapter provides a summary of the current Mesquite wrappers. \newline

\noindent Note that the wrappers do not, by themselves, completely define
the optimization problem.  The user still has to set the fixed/free flags,
and the values of the termination criteria.  \newline

\section{Laplace-smoothing} \label{sec:LaplaceWrapper}

\noindent {\it Name}: LaplaceWrapper \newline
\noindent {\it Purpose}: Produce a {\it smooth} mesh. \newline
\noindent {\it Notes}: This is a local patch relaxation-solver. A 'smart' 
Laplacian solver is also available in Mesquite, but it is not used in this 
wrapper.  \newline
\noindent {\it Limitations/assumptions}: No invertibility guarantee. \newline 
\noindent {\it Input Termination Criterion}: Stop after 10 global iterations. \newline \newline

\noindent Under the Cover: \newline
\noindent {\it Hardwired Parameters}: None \newline
\noindent {\it Mesh/Element Type}: Any supported type. \newline
\noindent {\it Global/Local}: Local Patch with Culling \newline



\section{Shape-Improvement} \label{sec:ShapeImprover}

\noindent {\it Name}: ShapeImprover \newline
\noindent {\it Purpose}: Make the shape of an element as close as possible to 
that of the ideal/regular element shape. For example, make triangular and 
tetrahedral elements equilateral.  The wrapper will use a non-barrier metric
until the mesh contains no inverted elements.  If the initial mesh was not tangled this phase will not modify the mesh.  If no limit is imposed on CPU
time or time remains a second phase using a barrier metric will further 
optimize the mesh with the guarantee that no elements will become inverted. \newline
\noindent {\it Notes}: Will return failure if mesh contains inverted elements after first phase.\newline
\noindent {\it Limitations/assumptions}:  \newline 
\noindent {\it Input Termination Criterion}: CPU time limit (not used if input 
value is non-positive) or fraction of mean edge length (default is 0.005). \newline \newline

\noindent Under the Cover: \newline
\noindent {\it Metric}: TMPQualityMetric(Shape/ShapeBarrier) \newline
\noindent {\it Objective Function}: Algebraic mean of quality metric values \newline
\noindent {\it Mesh/Element Type}: Any supported type. \newline
\noindent {\it Solver}: Conjugate Gradient \newline
\noindent {\it Global/Local}: Global \newline


\section{Untangler} \label{sec:Untangler}

\noindent {\it Name}: UntangleWrapper \newline
\noindent {\it Purpose}: Untangle elements.  Prioritizes untangling over
element shape or other mesh quality measures.  \newline
\noindent {\it Notes}: A second optimization to improve element quality
after untangling is often necessary.  \newline
\noindent {\it Limitations/assumptions}: There is no guarantee that the optimal mesh computed using this wrapper will, in fact, be untangled.  \newline 
\noindent {\it Input Termination Criterion}: CPU time limit (not used if input 
value is non-positive) or fraction of mean edge length (default is 0.005).  It
also terminates if all elements are untangled, such that it should not modify
an input mesh with no inverted elements. \newline \newline

\noindent Under the Cover: \newline
\noindent {\it Metric}: TUntangleBeta or TUntangleMu(TSizeNB1) or TUntangleMu(TShapeSizeNB1) \newline
\noindent {\it Objective Function}: Algebraic mean of quality metric values \newline
\noindent {\it Mesh/Element Type}: Any supported type. \newline
\noindent {\it Solver}: Steepest Descent \newline
\noindent {\it Global/Local}: Local with culling, optionally Jacobi \newline


\section{Minimum Edge-Length Improvement} \label{sec:PaverMinEdgeLengthWrapper}

\noindent {\it Name}: PaverMinEdgeLengthWrapper \newline
\noindent {\it Purpose}: Make all the edges in the mesh of equal length while 
moving toward the ideal shape. Intended for explicit PDE codes whose 
time-step limination is governed by the minimum edge-length in the mesh. \newline
\noindent {\it Notes}: Based on Target-matrix paradigm. \newline
\noindent {\it Limitations/assumptions}: Initial mesh must be non-inverted. User does not want to preserve or create anisotropic elements. \newline 
\noindent {\it Input Termination Criterion}: maximum iterations (default=50), maximum absolute vertex movement \newline \newline

\noindent Under the Cover: \newline
\noindent {\it Hardwired Parameters}: None \newline
\noindent {\it Metric}:  Target2DShapeSizeBarrier or Target3DShapeSizeBarrier \newline
\noindent {\it Tradeoff Coefficient}: 1.0 \newline
\noindent {\it Objective Function}: Linear Average over the Sample Points \newline
\noindent {\it Mesh/Element Type}: Any supported type.  \newline
\noindent {\it Solver}: Trust Region \newline
\noindent {\it Global/Local}: Global \newline

 
\section{Improve the Shapes in a Size-adapted Mesh} \label{sec:SizeAdaptShapeWrapper}

\noindent {\it Name}: SizeAdaptShapeWrapper \newline
\noindent {\it Purpose}: Make the shape of an element as close as possible to that
of the ideal element shape, {\it while preserving, as much as possible, the size of each element in the mesh}. To be used on isotropic initial meshes that are already size-adapted. \newline
\noindent {\it Notes}: Based on Target-matrix Paradigm. \newline
\noindent {\it Limitations/assumptions}: Initial mesh must be non-inverted. 
User wants to preserve sizes of elements in initial mesh and does not want to preserve or 
create anisotropic elements.  \newline 
\noindent {\it Input Termination Criterion}: maximum iterations (default=50), maximum absolute vertex movement \newline \newline

\noindent Under the Cover: \newline
\noindent {\it Hardwired Parameters}: None \newline
\noindent {\it Metric}: Target2DShapeSizeBarrier or Target3DShapeSizeBarrier \newline
\noindent {\it Tradeoff Coefficient}: 1.0 \newline
\noindent {\it Objective Function}: Linear Average over the Sample Points \newline
\noindent {\it Mesh/Element Type}: Any supported type. \newline
\noindent {\it Solver}: Trust Region \newline
\noindent {\it Global/Local}: Global \newline


\section{Improve Sliver Tets in a Viscous CFD Mesh} \label{sec:ViscousCFDTetShapeWrapper}

\noindent {\it Name}: ViscousCFDTetShapeWrapper \newline
\noindent {\it Purpose}: Improve the shape of sliver elements in the far-field 
of a CFD mesh while 
preserving an existing layer of sliver elements in the boundary layer. \newline
\noindent Notes: Based on Target-matrix paradigm.  \newline
\noindent {\it Limitations/assumptions}: Tetrahedral meshes only.  \newline 
\noindent {\it Input Termination Criterion}: Iteration Count (default=50) or Maximum Absolution Vertex Movement \newline \newline
 
\noindent Under the Cover: \newline
\noindent {\it Hardwired Parameters}: In tradeoff coefficient model. \newline
\noindent {\it Metric}: Target2DShape+Target2DShapeSizeOrient (or 3D) (or Barrier) \newline
\noindent {\it Tradeoff Coefficient}: Based on element dihedral angle \newline
\noindent {\it Objective Function}: Linear average over Sample Points \newline
\noindent {\it Mesh/Element Type}: Tetrahedra \newline
\noindent {\it Solver}: Trust Region \newline
\noindent {\it Global/Local}: Global \newline


\section{Deforming Domain} \label{sec:DeformingDomain}

\noindent {\it Name}: DeformingDomainWrapper \newline
\noindent {\it Purpose}:  Use initial mesh on undeformed geometric domain
to guide optimization of mesh moved to deformed geometric domain.  \newline
\noindent {\it Notes}: Uses a non-barrier metric which means that the
wrapper could potentially invert/tangle elements.  \newline
\noindent {\it Limitations/assumptions}:  Application responsible for explicit handling of mesh on geometric curves and points.  Initial mesh before moving to deformed domain must be available.\newline 
\noindent {\it Input Termination Criterion}: CPU time limit (not used if input 
value is non-positive) or fraction of mean edge length (default is 0.005). \newline \newline

\noindent Under the Cover: \newline
\noindent {\it Metric}: TMPQualityMetric(TShapeNB1 or TShapeSizeNB1 or TShapeSizeOrientNB1) \newline
\noindent {\it Objective Function}: Algebraic mean of quality metric values \newline
\noindent {\it Mesh/Element Type}: Any supported type. \newline
\noindent {\it Solver}: Steepest Descent \newline
\noindent {\it Global/Local}: Local with culling \newline
