\chapter{Introduction to Mesquite} \label{sec:intro}

\section{Overview of Mesh Quality}

\hskip 0.25in {\it Mesh quality} refers to geometric properties of a mesh such as 
local volume, smoothness, shape, and orientation that, if not properly 
controlled, 
can adversely affect solution accuracy or computational efficiency of numerical simulations. In this section we give an overview of the role of mesh quality 
in the context of computer simulations of physical phenomena. \newline

Simulation of many phenomena in the physical world involves computing 
numerical 
solutions to partial differential equations (PDE's). Commonly used approaches 
to computing  numerical solutions such as finite volume and finite 
element methods require the use of approximations to the continuum operators 
in the PDE and a mesh or grid to subdivide the physical domain into small 
subregions. Together, the approximations and the mesh define a discretization. 
The difference between the exact solution to the PDE and the numerical solution is known as the discretization error. A {\it convergent} 
discretization means that the discretization error will asymptotically 
approach zero as the characteristic mesh size 'h' 
approaches zero. Decreasing mesh size to reduce discretization error to 
nearly zero is often impractical in realistic simulations due to limited 
computing resources. One way to increase the accuracy of simulations with the 
same computer resources is to {\it adapt} the mesh to the domain and to the 
numerical solution. In adaptive refinement, the local mesh volume (or size) 
is made smaller in locations where the local discretization error is large and 
is made larger in locations where the error is small. In local h-refinement, 
mesh volume is made smaller by locally subdividing the mesh. In 
r-refinement, mesh volume is made smaller by moving mesh nodes closer together.
Geometric adaptation can also be important in improving simulation accuracy. In regions of high domain curvature one adapts the mesh to the domain geometry by creating locally smaller mesh sizes. We see, then, that local mesh size (or volume) is a critical parameter in determining the accuracy of a simulation. \newline

Aside from local mesh size, several other geometric mesh properties can affect 
solution accuracy. These include mesh smoothness, local mesh angles, aspect 
ratio, and orientation. For example, in some discretization methods there will be a 
loss of accuracy if the mesh is not smooth.  In other cases, aspect ratios and 
orientation must be carefully adapted to the solution in order to maintain a 
certain level of accuracy. Simulations using meshes or domains that evolve 
in time (such as in ALE simulations) usually require that initially good 
geometric mesh properties be retained throughout the simulation time period. 
It is thus often important to control other geometric mesh properties in addition to local mesh size within an adaptive simulation. \newline

In addition to solution accuracy, geometric mesh properties can also affect the amount of computer time required to obtain the numerical solution. Simulation codes usually employ iterative solvers to solve systems of equations and thus obtain numerical solutions to PDE's. The rate at which these solvers converge is determined by the spectral radius of a certain matrix. The spectral radius of the matrix is affected by, among other things, geometric properties of the mesh. Poor mesh quality can thus adversely impact solution efficiency. \newline

Adaptive meshing techniques require an initial mesh to begin the adaptation procedure. Poor quality of the initial mesh (relative to the adapted mesh) can be difficult to overcome or, at least, reduce the efficiency of 
the adaptive procedure. For example, if the initial mesh contains locally inverted elements, these can often be fixed before the adaptive procedure begins. As another example, if it is known \`{a} priori that small angles will be needed on the boundary of the domain to obtain reasonable simulation accuracy, one should try to first create the small angles in the initial mesh to improve the efficiency of the subsequent adaptive meshing procedure. \newline


Many simulations, particularly those in industry, are performed in a 
non-adaptive setting. That is to say, an initial mesh is generated and
used throughout the calculation. The mesh is not changed as the solution
is computed. Mesh quality remains important for such calculations. First, 
for complicated geometric domains it is often difficult to obtain good 
initial mesh quality. This is particularly true for non-simplicial meshes 
but can be true for simplicial meshes as well. A common requirement is that the mesh be smooth. Many simulation codes will not run to completion if the initial mesh contains a local volume which is negative. These must be eliminated before a simulation can begin. Analysts performing 
non-adaptive calculations often have considerable experience in using a variety
of meshes on their problem and have a good \`{a} priori idea of what constitutes
good mesh quality for a given problem. They thus desire to control the usual
geometric mesh properties of the non-adapted mesh carefully. 



\section{How Mesh Quality Is Improved}
Mesh quality can and should be considered during many stages of 
the mesh generation process from de-featuring CAD models to  
creation and adaptation of the mesh. Thus, for example, certain
non-essential features of a CAD model, if eliminated, would go a 
long way to improving the quality of the mesh, depending upon
the meshing scheme. Other critical meshing parameters which can 
affect mesh quality include geometric domain partitions, interval size
and count, interaction of meshes within large assemblies of parts, 
biasing requirements, corner picking, etc. Choices made during the 
mesh generation phase of an analysis may have a large impact on 
initial mesh quality.  Mesh quality can thus be improved by changing the 
way in which the domain is meshed. \newline

Once the meshing stage is completed, one can improve mesh quality
by techniques such as vertex movement and local topology modification.
In vertex movement schemes, one seeks to reposition existing mesh vertices to 
achieve better quality. If vertex movement is undertaken within an adaptive 
setting, it is commonly referred to as r-refinement. 
Classic examples of vertex movement methods 
include Laplace smoothing \cite{F88} and Winslow smoothing \cite{Winslow}. 
It is helpful, in vertex movement schemes, to first be 
able to measure mesh quality so that one can explain in what sense one 
has improved it. Given a {\it metric} to measure mesh quality, 
one can formulate a numerical 
optimization problem which guides vertex movement to find the optimal 
mesh and thus improve its quality.  Numerical 
optimization methods recently
developed for unstructured meshes include \cite{Opt-MS,Kn00,FrKn01,
FeasNewt,bjoe:swap,bjoe:chain-swap,es92}. We refer the reader to ref xxx
which is a survey of mesh quality improvement methods which have appeared 
in the literature. \newline

A large number of mesh quality metrics have been devised to measure 
mesh quality. Many of these metrics are independent of any solution 
properties and are thus not useful in adaptive meshing. However, there
are a number of weighted quality metrics which can be tied to the 
numerical solution via error indicators or other information for adaptive 
meshing. Examples of weighted metrics which have or could be used in 
adaptive calculations include those of Brackbill (ref), Knupp (ref), etc. \newline

Another way to improve mesh quality is to use local topological modification methods in which mesh vertices or elements are locally created and/or destroyed. These methods are very successful when applied to simplicial meshes, often within an adaptive context.  Local topology modification is less effective on non-simplicial meshes. \newline

Mesh quality improvement remains an important on-going research area. 
There remain, for example, open questions with regard to metrics which 
can be used in adaptive settings, theoretical questions on problem 
formulation, and how to obtain improved meshes quickly. \newline

Although mesh quality improvement algorithms have been widely implemented 
in both meshing and applications codes, it has always been difficult to 
improve the quality of a mesh created in one software package using an 
improvement algorithm which has been implemented in another.  This difficulty
and others have inspired the creation of the Mesquite software library. 
This library is described in the next section. \newline


\section{Mesquite Goals}
Mesquite (Mesh Quality Improvement Toolkit) is designed to provide a
stand-alone, portable, comprehensive suite of mesh quality improvement
algorithms.  The design flexible so that they can be applied to many
different mesh element types and orders and referenced to both
isotropic and anisotropic ideal elements.  Mesquite provides a robust
and effective mesh improvement toolkit that allows both meshing
researchers application scientists to benefit from the latest
developments in mesh quality control and improvement. \newline

Mesquite design goals are derived from a mathematical framework and
are focused on providing a versatile, comprehensive, inter-operable,
robust, and efficient library of mesh quality improvement algorithms
that can be used by the non-expert and extended and customized by
experts.  In this section we highlight the current status of Mesquite
in several of our design goal areas. \newline


{\bf Versatile.}  Mesquite works on structured, unstructured, and
hybrid meshes in both two and three dimensions. The design permits
improvements to meshes composed of triangular, tetrahedral,
quadrilateral, and hexahedral elements. Prismatic, pyramidal, and
polyhedral elements can be easily added.  It currently incorporates
only methods for node movement; plans for topology modification and
hybrid improvement strategies lie in the future.  Node movement
strategies include both local patch-based iteration schemes for one or
a few free vertices and global objective functions which improve all
vertices simultaneously. Mesquite will be applicable to both adaptive
and nonadaptive meshing and to both low- and high-order discretization
schemes, but currently works with non-adaptive meshes containing
linear elements. \newline

{\bf Comprehensive.}  Mesquite will address a large variety of mesh
quality improvement goals including mesh volume control (sizing,
invertibility), mesh angles, aspect ratios, and orientation. Specific
goals include mesh untangling, mesh smoothing, shape improvement,
anisotropic smoothing, mesh rezoning for ALE, mesh alignment, and
deforming mesh algorithms. These goals can be pursued in both adaptive
and non-adaptive settings. The software is customizable, enabling
users to insert their own quality metrics, objective functions, and
algorithms and also provides mechanisms for creating combined
approaches that use one or more improvement algorithms. \newline

%{\bf Effective.}  Mesquite uses state-of-the-art algorithms and
%metrics to guarantee improvement in mesh quality.  Because the
%definition of mesh quality is application specific, we provide quality
%metrics that allow the user to untangle meshes, improve mesh
%smoothness, element size, and shape. In the future these metrics will
%be referenced to permit non-isotropic smoothing and adaptivity. \newline

{\bf Inter-operable.}  To ensure that Mesquite is inter-operable with a
large number of mesh generation packages, we use the common
interfaces for mesh query currently under development by the TSTT
center.  These interfaces provide uniform access to mesh geometry and
topology and will be implemented by all TSTT center software including
several DOE-supported mesh generation packages.  We are working with
the TSTT interface design team to ensure that Mesquite has efficient
access to mesh and geometry information through strategies such as
information caching and agglomeration.  We are also participating in
the design of interfaces needed to support topological changes
generated by mesh swapping and flipping algorithms and to constrain
vertices to the surface of a geometrical model. \newline

{\bf Efficient.}  The outer layers of Mesquite use 
object-oriented design in C++ while the inner kernels use
optimizable coding constructs such as arrays and inlined
functions.  To ensure efficient use of computationally intensive
optimization algorithms, we employ inexpensive smoothers, such as
Laplacian smoothing, as ``preconditioners'' for the more expensive
optimization techniques.  In addition, mesh culling algorithms can be
used to smooth only those areas of the mesh that require improvement.
Considerable attention has been devoted to understanding and
implementing a variety of termination criteria that can be used to
control the computational cost of the optimization algorithms. \newline

{\bf Robust.} Sound software engineering principles and robust numerical 
algorithms are employed in Mesquite. 
%Code interrupts due to null pointers and zero-divides will be handled gracefully.  
A comprehensive suite of test problems and a unit testing framework have
been developed to verify the correct execution of the code. \newline

Mesquite is not intended to be a mesh generation tool. It can serve as 
a post-processor to a mesh generation procedure, a mesh pre-processor to a 
non-adaptive simulation code, or as an algorithm for in-core adaptive mesh 
quality improvement. As a software library, Mesquite is intended to be
linked to either a meshing code or to a simulation code. \newline

\section{Mesquite Concepts} \label{sec:concepts}

Mesquite software design is based on a mathematical 
framework that improves mesh quality by solving an optimization 
problem to guide the movement of mesh vertices \cite{formal}. 
The user inputs a mesh or 
submesh consisting of vertices, elements, and the relationships between them. 
The quality of each vertex or 
element in the mesh is described by a local quality metric that is a function 
of a subset of the mesh vertices. The global quality of the mesh is formed by 
taking the global norm or the average of the local mesh qualities. The global 
quality is thus a function of the positions of all the mesh vertices. If this 
function can be used in a well-posed minimization problem (e.g., it is 
bounded below and has one or more local minimums), mesh vertices are moved 
by Mesquite toward the vertex positions of the optimal mesh, thus improving 
the quality according to the criterion defined by the local quality metric. 
By changing the local quality metric one can achieve a variety of mesh quality improvement goals such as mesh untangling, shape improvement, and size adaptation. \newline

Users of Mesquite should have in mind a goal or set of goals which define 
the quality of the mesh which is to be improved. The goal determines which
quality metric or metrics one will use in the optimization problem. Other 
user inputs will include an objective function template which describes 
the norm or average they wish to use in defining the global mesh quality. 
For example, an L-infinity norm will tend to improve the worst-case local 
quality while an L-2 norm will improve the RMS quality of the global mesh. 
Once the global quality (objective function) is defined, the user can 
select a numerical optimization scheme (solver) within Mesquite such as a 
steepest descent, conjugate gradient, or feasible Newton method. A variety of 
termination criteria can be selected singly or in combination to tell the 
solver when to halt. These are useful in controlling the trade-off between
the accuracy of the minimization procedure vs. how much CPU is consumed. 
There is also an important flag that determines whether the optimization 
problem will be solved via a succession of optimizations on local patches 
followed by a complete pass over the global mesh or if it will be solved using 
a global patch in which all mesh vertices are moved simultaneously. Advantages
and disadvantages of each of these approaches is currently under study.\newline

Sometimes hybrid mesh optimization schemes are useful, for example, in 
first untangling a mesh and then improving the shape of its elements. For 
sequences of optimization problems Mesquite uses the concept of an 
instruction queue.  The queue determines the order in which the optimization
problems are solved, using the output from the previous optimization step 
as the input to the next optimization step. The queue defines a master 
quality improver that defines the ultimate mesh quality improvement goal.
The queue can also be used to include steps to assess mesh quality say 
before and after each optimization step within the queue.  The quality 
assessor measures various aspects of quality in the mesh and may include 
other quality metrics besides the one used to define the optimization problem.
\newline

Optimization problems can be solved directly by minimizing the objective 
function or indirectly by positioning mesh vertices at a stationary point
of the global objective function. Stationary points are defined by setting 
the gradient of the objective function to zero. The indirect method is akin 
to iteratively solving a system of linear (or nonlinear) equations. 
Currently, such systems are solved in Mesquite and other mesh quality 
software by using the local patch method that is akin 
to a Gauss-Seidel iteration. The prime example of this in Mesquite is 
Laplace smoothing. In the 
future we may include methods for solving global systems of equations 
in Mesquite to obtain solutions more quickly. 
In the past, some mesh smoothing algorithms have been formulated as a 
local iterative method that cannot be derived  
by setting the gradient of an objective function to zero. Such methods are
frowned upon in Mesquite since one cannot state what mesh quality metric is
improved.  However, if such methods are included in future version sof Mesquite, they will be done in a manner similar to the local Laplace smoothing 
algorithm in Mesquite. \newline

\noindent The following notation is used in the rest of this manual
\begin{itemize}
\item The mesh is assumed to consist of $N$ elements and $M$ free vertices.  
Let $n=1,2,\ldots,N$.
\item Let $q$ be a scalar which defines an element-based {\it quality} metric. 
The quality of the $n^{th}$ element in the mesh is given by the scalar 
$q_n$. Element quality is a function of the coordinates ${\bf x}_n$ 
of the vertices belonging to the element, i.e., $q_n = q({\bf x}_n)$
\item Let $Q \in R^N$ be the vector $[q_1,q_2,\ldots,q_n]$ of element 
qualities over the mesh. Let $f$ be a function from $R^N$ to $R$. When  
$f$ is applied to $Q$, we call $f(Q)$ an {\it objective function template}.
\item Because each of the element qualities depends on the coordinates of
the vertices which it contains, the vector $Q$ is a function of the coordinates
of all of the free vertices ${\bf x} \in R^{3M}$ in the mesh, i.e., $Q=Q({\bf x})$. Finally, form $F({\bf x})= f \circ Q({\bf x}) = f(Q({\bf x}))$ as a 
function from $R^{3M}$ to $R$.  The function $F$ is the mesh quality 
{\it objective function}. 
\item $\nabla F \in R^{3M}$ is the {\it gradient} of the objective function 
with respect to the coordinates of the free vertices. Let ${\cal H} F= \nabla (\nabla F)$ be the {\it Hessian} of the objective function.  The Hessian is a 
$3M \times 3M$ matrix. 
\end{itemize}

\begin{figure}[htb]
\begin{center}
\begin{tabular}{c}
\psfig{figure=./msq-paradigm.eps,width=4.7in}
\end{tabular}
\end{center}
\caption{\em The Mesquite Paradigm \label{Paradigm} }
\end{figure}

\section{How to use this User's Manual}
This user's manual 
\begin{itemize}
\item provides an introduction to mesh quality and basic Mesquite concepts (chapter \ref{sec:intro}), 
\item instructs novice users on how to download and compile
Mesquite. A tutorial is given of Mesquite simplified user's interface and Mesquite's detailed API (chapter \ref{sec:basics}).
\item describes how to load a mesh in Mesquite via files (section \ref{sec:meshFiles}), 
\item provides instructions on using the extensive TSTT interface or a Mesquite mesh specific mesh
      interface to load a mesh dynamicallu in Mesquite (sections \ref{sec:msq_mesh}, \ref{sec:TSTT}).
\item describes Mesquite interactions with domain geometry (Section 3.4),
\item Exposes in details the concepts and the mechanisms of the advanced API (chapter \ref{sec:API}), and 
\item instructs the user on how to add their own instances of quality 
metrics, objective functions, and solvers (chapter \ref{sec:extensions})
\end{itemize}

Consult the doxygen documentation for the API reference as well as details on the software. There
are two sets of doxygen documentations available:
\begin{itemize}
\item The developer doxygen doc is located in mesquite/doc/developer/. From that directory, you
      must run 'doxygen Mesquite.dox'.
\item The user doxygen doc (API doc) is located in mesquite/doc/user/doxygen. From that directory, you
      must run 'doxygen Mesquite-user.dox'.
\end{itemize}
The doxygen command will generate two directories: an html directory containing the file
index.html that you can open with your web browser, and a latex directory containing a Makefile that
will generate a dvi file. 
