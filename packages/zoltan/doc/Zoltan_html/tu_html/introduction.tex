%
% What this document is, what prior knowledge is assumed, and
% brief outline of what is contained in it.
% 
% PLEASE FEEL FREE TO RE-WRITE ANY OF THIS!!
%
\chapter{Introduction}

This document provides an introduction to the Zoltan Toolkit,
for Zoltan version 2.1.  Zoltan is a parallel load balancing library
targeted to the high performance computing community.  The Zoltan
library can be
linked with C, C++ and Fortran applications and it requires MPI.  
Most of the examples in this tutorial are C language examples.

The first chapter (\ref{cha:lb})
will discuss the load balancing problem in general.
Zoltan's role in addressing the load balancing problem is to 
efficiently partition,
or repartition, the problem data across mulitple processes while an
application is running, when requested to do so by the application.

The following chapter (\ref{cha:partitioning})
will describe in more detail the partitioning
methods available in Zoltan, and will provide a guide
to choosing the method or methods best suited to your problem.

In chapter ~\ref{cha:using} we briefly explain the interface
through which your application employs Zoltan.  This topic is
convered in more detail in the
Zoltan User's Guide available at
\url{http://www.cs.sandia.gov/Zoltan/ug_html/ug.html}.
An example may be worth a thousand words, so you may wish to 
skip this chapter and look through the examples in the last section first.

The final chapter of this document (~\ref{cha:ex}) provides
source code examples for simple applications that use Zoltan.
These examples may be found in the \textbf{examples} directory
of the Zoltan source code.  You can obtain that source code,
along with installation instructions, at
\url{http://www.cs.sandia.gov/web1400/1400_download.html}.

You will find additional documents and publications at the
Zoltan web site at \url{http://www.cs.sandia.gov/Zoltan/}.

% Should we have a pointer to a list of publications as well.
