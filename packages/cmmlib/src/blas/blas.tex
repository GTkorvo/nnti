\documentclass{article}

\begin{document}

\section{Vector and Matrix description and storage.}

  C and C++ offer structures and classes that could be used
to describe vectors and matrices. C and C++ also have
a built-in array notation for these entities. After
trying these facilities extensively, my current implementation
in the library does not use them. Generally what I found is
that a simpler approach ends up with less code and clearer
code. This simpler approach can be coupled to a variety of
higher level descriptions.

  In the library a vector is just a pointer to memory, and
vector elements are packed contiguously. An element is
referred to by a base plus an offset, and a vector is passed
as an argument by passing a base and a length. 

  A matrix is also just a pointer to memory, with two offsets
describing an element. The address of element $i, j$ is the
base address plus the overall offset: $ b + i + j\cdot l$, where $l$ is
the "leading dimension". This storage approach means that a
matrix is described by four quantities: $b$,$m$,$n$, and $l$.
This corresponds to "column major" storage.

\end{document}





