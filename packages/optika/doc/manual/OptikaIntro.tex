\section{Introduction}
Creating a good user interface for a scientific application can be a tedious and involved task. It can also distract
the application developer from the main goal of the application.
Optika is an attempt to remedy this situation in a generic fashion. By drastically simplifying the process
of GUI creation, Optika allows application developers to spend less time on their User Interface and more time
on the core of their application. At the same time, it gives the users of these applications a much more 
intuitive method for supplying input via a GUI.

The purpose of this paper is to show in detail how Optika can be used to create
significantly better user interfaces for scientific applications. We will include
numerous examples throughout.
