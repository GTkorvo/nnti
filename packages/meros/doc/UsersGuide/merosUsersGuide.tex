% Meros User's Guide
%
% Formatted for Sandia SAND Report
\typeout{Using SANDmath LaTeX Package: 6/12/2007, VEH}

\documentclass[oneeqnum,onefignum,onetabnum,10pt]{SANDreport}
\usepackage{amsmath}
\usepackage{amsfonts}
\usepackage{amssymb}
\usepackage{epsfig}
\usepackage{ifthen}

\usepackage{mathptmx}

\usepackage{theorem}
\usepackage{latexsym}
\usepackage{thp}
\usepackage{subfigure}
\def\vdown{\\[4pt]}

\setlength{\topmargin}{0in}
\setlength{\oddsidemargin}{.10in}
\setlength{\evensidemargin}{.10in}
\setlength{\headheight}{1pt}
\setlength{\headsep}{10pt}
\setlength{\topskip}{1pt}
\setlength{\textheight}{8.5in}
\setlength{\textwidth}{6.25in}

\definecolor{AztecOOBlue}{rgb}{0,0,0}
\def\TrilinosTM{{\color{AztecOOBlue} \bf
      \textsf{Trilinos}}\textsuperscript{\tiny{\texttrademark}}}
\def\MerosTM{{\color{AztecOOBlue} \bf
      \textsf{Meros}}\textsuperscript{\tiny{\texttrademark}}}
\newcommand{\InlineCommand}[1]{
  {\hspace{0.01 in}} {\tt #1} {\hspace{0.01 in}}}

\input{epsf}

\title{DRAFT: Meros User's Guide\footnote{For \MerosTM{} Version 2.0 in
\TrilinosTM{} Release 8.0}}

%\thanks{This work was partially supported
%by the DOE Office of Science MICS Program and by the ASC Program at
%Sandia National Laboratories.  Sandia is a multiprogram laboratory
%operated by Sandia Corporation, a Lockheed Martin Company, for the
%United States Department of Energy's National Nuclear Security
%Administration under contract DE-AC04-94AL85000.}}

\author{
Victoria E. Howle\thanks{Sandia National Laboratories, PO Box 969, MS 9159
    Livermore, CA 94551, {\tt vehowle}\protect@{\tt sandia.gov}.}
  \and
  Robert Shuttleworth\thanks{Applied Mathematics and Scientific Computing Program and Center for Scientific
Computation and Mathematical Modeling,
University of Maryland, College Park, MD 20742. {\tt rshuttle}\protect@{\tt math.umd.edu}}
  \and
  Ray Tuminaro\thanks{Sandia National Laboratories, PO Box 969, MS 9159,
    Livermore, CA 94551, {\tt rstumin}\protect@{\tt sandia.gov}.}
}

\date{}

\SANDnum{SAND2007-xxxx}
\SANDprintDate{August 2007}
\SANDauthor{Victoria E. Howle, Robert Shuttleworth, Ray Tuminaro}

\begin{document}

\maketitle
\begin{abstract}
meros abstract
\end{abstract}




\bibliographystyle{siam}
%\pagestyle{plain}

\SANDmain

\clearpage
\tableofcontents
%\listoffigures
%\listoftables

%\chapter{Introduction to Mesquite} \label{sec:intro}

\section{Overview of Mesh Quality}

\hskip 0.25in {\it Mesh quality} refers to geometric properties of a mesh such as 
local volume, smoothness, shape, and orientation that, if not properly 
controlled, 
can adversely affect solution accuracy or computational efficiency of numerical simulations. In this section we give an overview of the role of mesh quality 
in the context of computer simulations of physical phenomena. \newline

Simulation of many phenomena in the physical world involves computing 
numerical 
solutions to partial differential equations (PDE's). Commonly used approaches 
to computing  numerical solutions such as finite volume and finite 
element methods require the use of approximations to the continuum operators 
in the PDE and a mesh or grid to subdivide the physical domain into small 
subregions. Together, the approximations and the mesh define a discretization. 
The difference between the exact solution to the PDE and the numerical solution is known as the discretization error. A {\it convergent} 
discretization means that the discretization error will asymptotically 
approach zero as the characteristic mesh size ``h'' 
approaches zero. Decreasing mesh size to reduce discretization error to 
nearly zero is often impractical in realistic simulations due to limited 
computing resources. One way to increase the accuracy of simulations with the 
same computer resources is to {\it adapt} the mesh to the domain and to the 
numerical solution. In adaptive refinement, the local mesh volume (or size) 
is made smaller in locations where the local discretization error is large and 
is made larger in locations where the error is small. In local h-refinement, 
mesh volume is made smaller by locally subdividing the mesh. In 
r-refinement, mesh volume is made smaller by moving mesh nodes closer together.
Geometric adaptation can also be important in improving simulation accuracy. In regions of high domain curvature one adapts the mesh to the domain geometry by creating locally smaller mesh sizes. We see, then, that local mesh size (or volume) is a critical parameter in determining the accuracy of a simulation. \newline

Aside from local mesh size, several other geometric mesh properties can affect 
solution accuracy. These include mesh smoothness, local mesh angles, aspect 
ratio, and orientation. For example, in some discretization methods there will be a 
loss of accuracy if the mesh is not smooth.  In other cases, aspect ratios and 
orientation must be carefully adapted to the solution in order to maintain a 
certain level of accuracy. Simulations using meshes or domains that evolve 
in time (such as in ALE simulations) usually require that initially good 
geometric mesh properties be retained throughout the simulation time period. 
It is thus often important to control other geometric mesh properties in addition to local mesh size within an adaptive simulation. \newline

In addition to solution accuracy, geometric mesh properties can also affect the amount of computer time required to obtain the numerical solution. Simulation codes usually employ iterative solvers to solve systems of equations and thus obtain numerical solutions to PDE's. The rate at which these solvers converge is determined by the spectral radius of a certain matrix. The spectral radius of the matrix is affected by, among other things, geometric properties of the mesh. Poor mesh quality can thus adversely impact solution efficiency. \newline

Adaptive meshing techniques require an initial mesh to begin the adaptation procedure. Poor quality of the initial mesh (relative to the adapted mesh) can be difficult to overcome or, at least, reduce the efficiency of 
the adaptive procedure. For example, if the initial mesh contains locally inverted elements, these can often be fixed before the adaptive procedure begins. As another example, if it is known \`{a} priori that small angles will be needed on the boundary of the domain to obtain reasonable simulation accuracy, one should try to first create the small angles in the initial mesh to improve the efficiency of the subsequent adaptive meshing procedure. \newline


Many simulations, particularly those in industry, are performed in a 
non-adaptive setting. That is to say, an initial mesh is generated and
used throughout the calculation. The mesh is not changed as the solution
is computed. Mesh quality remains important for such calculations. First, 
for complicated geometric domains it is often difficult to obtain good 
initial mesh quality. This is particularly true for non-simplicial meshes 
but can be true for simplicial meshes as well. A common requirement is that the mesh be smooth. Many simulation codes will not run to completion if the initial mesh contains a local volume which is negative. These must be eliminated before a simulation can begin. Analysts performing 
non-adaptive calculations often have considerable experience in using a variety
of meshes on their problem and have a good \`{a} priori idea of what constitutes
good mesh quality for a given problem. They thus desire to control the usual
geometric mesh properties of the non-adapted mesh carefully. 



\section{How Mesh Quality Is Improved}
Mesh quality can and should be considered during many stages of 
the mesh generation process from de-featuring CAD models to  
creation and adaptation of the mesh. Thus, for example, certain
non-essential features of a CAD model, if eliminated, would go a 
long way to improving the quality of the mesh, depending upon
the meshing scheme. Other critical meshing parameters which can 
affect mesh quality include geometric domain partitions, interval size
and count, interaction of meshes within large assemblies of parts, 
biasing requirements, corner picking, etc. Choices made during the 
mesh generation phase of an analysis may have a large impact on 
initial mesh quality.  Mesh quality can thus be improved by changing the 
way in which the domain is meshed. \newline

Once the meshing stage is completed, one can improve mesh quality
by techniques such as vertex movement and local topology modification.
In vertex movement schemes, one seeks to reposition existing mesh vertices to 
achieve better quality. If vertex movement is undertaken within an adaptive 
setting, it is commonly referred to as r-refinement. 
Classic examples of vertex movement methods 
include Laplace smoothing \cite{F88} and Winslow smoothing \cite{Winslow}. 
It is helpful, in vertex movement schemes, to first be 
able to measure mesh quality so that one can explain in what sense one 
has improved it. Given a {\it metric} to measure mesh quality, 
one can formulate a numerical 
optimization problem which guides vertex movement to find the optimal 
mesh and thus improve its quality.  Numerical 
optimization methods recently
developed for unstructured meshes include \cite{Opt-MS,Kn00,FrKn01,
FeasNewt,bjoe:swap,bjoe:chain-swap,es92}. We refer the reader to ref xxx
which is a survey of mesh quality improvement methods which have appeared 
in the literature. \newline

A large number of mesh quality metrics have been devised to measure 
mesh quality. Many of these metrics are independent of any solution 
properties and are thus not useful in adaptive meshing. However, there
are a number of weighted quality metrics which can be tied to the 
numerical solution via error indicators or other information for adaptive 
meshing. Examples of weighted metrics which have or could be used in 
adaptive calculations include those of Brackbill (ref), Knupp (ref), etc. \newline

Another way to improve mesh quality is to use local topological modification methods in which mesh vertices or elements are locally created and/or destroyed. These methods are very successful when applied to simplicial meshes, often within an adaptive context.  Local topology modification is less effective on non-simplicial meshes. \newline

Mesh quality improvement remains an important on-going research area. 
There remain, for example, open questions with regard to metrics which 
can be used in adaptive settings, theoretical questions on problem 
formulation, and how to obtain improved meshes quickly. \newline

Although mesh quality improvement algorithms have been widely implemented 
in both meshing and applications codes, it has always been difficult to 
improve the quality of a mesh created in one software package using an 
improvement algorithm which has been implemented in another.  This difficulty
and others have inspired the creation of the Mesquite software library. 
This library is described in the next section. \newline


\section{Mesquite Goals}
Mesquite (Mesh Quality Improvement Toolkit) is designed to provide a
stand-alone, portable, comprehensive suite of mesh quality improvement
algorithms.  The design flexible so that they can be applied to many
different mesh element types and orders and referenced to both
isotropic and anisotropic ideal elements.  Mesquite provides a robust
and effective mesh improvement toolkit that allows both meshing
researchers application scientists to benefit from the latest
developments in mesh quality control and improvement. \newline

Mesquite design goals are derived from a mathematical framework and
are focused on providing a versatile, comprehensive, inter-operable,
robust, and efficient library of mesh quality improvement algorithms
that can be used by the non-expert and extended and customized by
experts.  In this section we highlight the current status of Mesquite
in several of our design goal areas. \newline


{\bf Versatile.}  Mesquite works on structured, unstructured, and
hybrid meshes in both two and three dimensions. The design permits
improvements to meshes composed of triangular, tetrahedral,
quadrilateral, and hexahedral elements. Prismatic, pyramidal, and
polyhedral elements can be easily added.  It currently incorporates
only methods for node movement; plans for topology modification and
hybrid improvement strategies lie in the future.  Node movement
strategies include both local patch-based iteration schemes for one or
a few free vertices and global objective functions which improve all
vertices simultaneously. Mesquite will be applicable to both adaptive
and nonadaptive meshing and to both low- and high-order discretization
schemes, but currently works with non-adaptive meshes containing
linear elements. \newline

{\bf Comprehensive.}  Mesquite will address a large variety of mesh
quality improvement goals including mesh volume control (sizing,
invertibility), mesh angles, aspect ratios, and orientation. Specific
goals include mesh untangling, mesh smoothing, shape improvement,
anisotropic smoothing, mesh rezoning for ALE, mesh alignment, and
deforming mesh algorithms. These goals can be pursued in both adaptive
and non-adaptive settings. The software is customizable, enabling
users to insert their own quality metrics, objective functions, and
algorithms and also provides mechanisms for creating combined
approaches that use one or more improvement algorithms. \newline

%{\bf Effective.}  Mesquite uses state-of-the-art algorithms and
%metrics to guarantee improvement in mesh quality.  Because the
%definition of mesh quality is application specific, we provide quality
%metrics that allow the user to untangle meshes, improve mesh
%smoothness, element size, and shape. In the future these metrics will
%be referenced to permit non-isotropic smoothing and adaptivity. \newline

{\bf Inter-operable.}  To ensure that Mesquite is inter-operable with a
large number of mesh generation packages, we use the common
interfaces for mesh query currently under development by the TSTT
center.  These interfaces provide uniform access to mesh geometry and
topology and will be implemented by all TSTT center software including
several DOE-supported mesh generation packages.  We are working with
the TSTT interface design team to ensure that Mesquite has efficient
access to mesh and geometry information through strategies such as
information caching and agglomeration.  We are also participating in
the design of interfaces needed to support topological changes
generated by mesh swapping and flipping algorithms and to constrain
vertices to the surface of a geometrical model. \newline

{\bf Efficient.}  The outer layers of Mesquite use 
object-oriented design in C++ while the inner kernels use
optimizable coding constructs such as arrays and inlined
functions.  To ensure efficient use of computationally intensive
optimization algorithms, we employ inexpensive smoothers, such as
Laplacian smoothing, as ``preconditioners'' for the more expensive
optimization techniques.  In addition, mesh culling algorithms can be
used to smooth only those areas of the mesh that require improvement.
Considerable attention has been devoted to understanding and
implementing a variety of termination criteria that can be used to
control the computational cost of the optimization algorithms. \newline

{\bf Robust.} Sound software engineering principles and robust numerical 
algorithms are employed in Mesquite. 
%Code interrupts due to null pointers and zero-divides will be handled gracefully.  
A comprehensive suite of test problems and a unit testing framework have
been developed to verify the correct execution of the code. \newline

Mesquite is not intended to be a mesh generation tool. It can serve as 
a post-processor to a mesh generation procedure, a mesh pre-processor to a 
non-adaptive simulation code, or as an algorithm for in-core adaptive mesh 
quality improvement. As a software library, Mesquite is intended to be
linked to either a meshing code or to a simulation code. \newline

\section{Mesquite Concepts} \label{sec:concepts}

Mesquite software design is based on a mathematical 
framework that improves mesh quality by solving an optimization 
problem to guide the movement of mesh vertices \cite{formal}. 
The user inputs a mesh or 
submesh consisting of vertices, elements, and the relationships between them. 
The quality of each vertex or 
element in the mesh is described by a local quality metric that is a function 
of a subset of the mesh vertices. The global quality of the mesh is formed by 
taking the global norm or the average of the local mesh qualities. The global 
quality is thus a function of the positions of all the mesh vertices. If this 
function can be used in a well-posed minimization problem (e.g., it is 
bounded below and has one or more local minimums), mesh vertices are moved 
by Mesquite toward the vertex positions of the optimal mesh, thus improving 
the quality according to the criterion defined by the local quality metric. 
By changing the local quality metric one can achieve a variety of mesh quality improvement goals such as mesh untangling, shape improvement, and size adaptation. \newline

Users of Mesquite should have in mind a goal or set of goals which define 
the quality of the mesh which is to be improved. The goal determines which
quality metric or metrics one will use in the optimization problem. Other 
user inputs will include an objective function template which describes 
the norm or average they wish to use in defining the global mesh quality. 
For example, an L-infinity norm will tend to improve the worst-case local 
quality while an L-2 norm will improve the RMS quality of the global mesh. 
Once the global quality (objective function) is defined, the user can 
select a numerical optimization scheme (solver) within Mesquite such as a 
steepest descent, conjugate gradient, or feasible Newton method. A variety of 
termination criteria can be selected singly or in combination to tell the 
solver when to halt. These are useful in controlling the trade-off between
the accuracy of the minimization procedure vs. how much CPU is consumed. 
There is also an important flag that determines whether the optimization 
problem will be solved via a succession of optimizations on local patches 
followed by a complete pass over the global mesh or if it will be solved using 
a global patch in which all mesh vertices are moved simultaneously. Advantages
and disadvantages of each of these approaches is currently under study.\newline

Sometimes hybrid mesh optimization schemes are useful, for example, in 
first untangling a mesh and then improving the shape of its elements. For 
sequences of optimization problems Mesquite uses the concept of an 
instruction queue.  The queue determines the order in which the optimization
problems are solved, using the output from the previous optimization step 
as the input to the next optimization step. The queue defines a master 
quality improver that defines the ultimate mesh quality improvement goal.
The queue can also be used to include steps to assess mesh quality say 
before and after each optimization step within the queue.  The quality 
assessor measures various aspects of quality in the mesh and may include 
other quality metrics besides the one used to define the optimization problem.
\newline

Optimization problems can be solved directly by minimizing the objective 
function or indirectly by positioning mesh vertices at a stationary point
of the global objective function. Stationary points are defined by setting 
the gradient of the objective function to zero. The indirect method is akin 
to iteratively solving a system of linear (or nonlinear) equations. 
Currently, such systems are solved in Mesquite and other mesh quality 
software by using the local patch method that is akin 
to a Gauss-Seidel iteration. The prime example of this in Mesquite is 
Laplace smoothing. In the 
future we may include methods for solving global systems of equations 
in Mesquite to obtain solutions more quickly. 
In the past, some mesh smoothing algorithms have been formulated as a 
local iterative method that cannot be derived  
by setting the gradient of an objective function to zero. Such methods are
frowned upon in Mesquite since one cannot state what mesh quality metric is
improved.  However, if such methods are included in future version sof Mesquite, they will be done in a manner similar to the local Laplace smoothing 
algorithm in Mesquite. \newline

\noindent The following notation is used in the rest of this manual
\begin{itemize}
\item The mesh is assumed to consist of $N$ elements and $M$ free vertices.  
Let $n=1,2,\ldots,N$.
\item Let $q$ be a scalar which defines an element-based {\it quality} metric. 
The quality of the $n^{th}$ element in the mesh is given by the scalar 
$q_n$. Element quality is a function of the coordinates ${\bf x}_n$ 
of the vertices belonging to the element, i.e., $q_n = q({\bf x}_n)$
\item Let $Q \in R^N$ be the vector $[q_1,q_2,\ldots,q_n]$ of element 
qualities over the mesh. Let $f$ be a function from $R^N$ to $R$. When  
$f$ is applied to $Q$, we call $f(Q)$ an {\it objective function template}.
\item Because each of the element qualities depends on the coordinates of
the vertices which it contains, the vector $Q$ is a function of the coordinates
of all of the free vertices ${\bf x} \in R^{3M}$ in the mesh, i.e., $Q=Q({\bf x})$. Finally, form $F({\bf x})= f \circ Q({\bf x}) = f(Q({\bf x}))$ as a 
function from $R^{3M}$ to $R$.  The function $F$ is the mesh quality 
{\it objective function}. 
\item $\nabla F \in R^{3M}$ is the {\it gradient} of the objective function 
with respect to the coordinates of the free vertices. Let ${\cal H} F= \nabla (\nabla F)$ be the {\it Hessian} of the objective function.  The Hessian is a 
$3M \times 3M$ matrix. 
\end{itemize}

\begin{figure}[htb]
\begin{center}
\begin{tabular}{c}
\psfig{figure=./msq-paradigm.eps,width=4.7in}
\end{tabular}
\end{center}
\caption{\em The Mesquite Paradigm \label{Paradigm} }
\end{figure}

\section{How to use this User's Manual}
This user's manual 
\begin{itemize}
\item provides an introduction to mesh quality and basic Mesquite concepts (chapter \ref{sec:intro}), 
\item instructs novice users on how to download and compile
Mesquite. A tutorial is given of Mesquite simplified user's interface and Mesquite's detailed API (chapter \ref{sec:basics}).
\item describes how to load a mesh in Mesquite via files (section \ref{sec:meshFiles}), 
\item provides instructions on using the extensive TSTT interface or a Mesquite mesh specific mesh
      interface to load a mesh dynamicallu in Mesquite (sections \ref{sec:msq_mesh}, \ref{sec:TSTT}).
\item describes Mesquite interactions with domain geometry (Section 3.4),
\item Exposes in details the concepts and the mechanisms of the advanced API (chapter \ref{sec:API}), and 
\item instructs the user on how to add their own instances of quality 
metrics, objective functions, and solvers (chapter \ref{sec:extensions})
\end{itemize}

Consult the doxygen documentation for the API reference as well as details on the software. There
are two sets of doxygen documentations available:
\begin{itemize}
\item The developer doxygen doc is located in mesquite/doc/developer/. From that directory, you
      must run 'doxygen Mesquite.dox'.
\item The user doxygen doc (API doc) is located in mesquite/doc/user/doxygen. From that directory, you
      must run 'doxygen Mesquite-user.dox'.
\end{itemize}
The doxygen command will generate two directories: an html directory containing the file
index.html that you can open with your web browser, and a latex directory containing a Makefile that
will generate a dvi file. 

\section{Related Documents}

Documentation of the Mesquite API can be generated from comments in the Mesquite
source code using the Doxygen utility 
(\url{http://www.stack.nl/~dimitri/doxygen/}).
To generate HTML and LaTeX copies of this documentation, execute the command 
{\tt doxygen Mesquite-user.dox} in the {\tt doc/usr/doxygen} subdirectory
of the Mesquite source.

Further information and related documentation are available on the 
Mesquite home page, located at:
\url{http://www.cs.sandia.gov/optimization/knupp/Mesquite.html}.

\section{Notation}
\begin{verbatim} 
% a_really_long_command 
\end{verbatim}
The character \verb!%! indicates any LINUX or UNIX shell prompt.
Function names are shown as {\sf LSCSchurFactory}.  Names of packages
or libraries as reported in small caps, as {\sc Epetra}. Mathematical
entities are shown in italics.


\section{Introduction}
Meros is a segregated preconditioning package. Provides scalable block
preconditioning for problems that coupled simultaneous solution
variables such as Navier-Strokes problems. 


Adding a citation to test bib\cite{ElmanSilvesterWathen.book}.  Meros
is a segregated preconditioning package within Trilinos. Meros
provides scalable block preconditioning for problems that couple
simultaneous solution variables such as Navier-Strokes problems.

The initial focus of Meros is on incompressible Navier-Stokes.

This guide describes the use of a block preconditioner within the
Meros package.  The block preconditioners can be used to solve block
linear systems of the type
\begin{equation}
\label{E:blocks}
\tilde{A} =
\begin{bmatrix}
F & B^T \\
\hat{B} & -C \\
\end{bmatrix}
\end{equation}
where the matrix $\tilde{A}$ is a user supplied matrix of size $n
\times n$ that arises from linearization and discretization of the
incompressible Navier-Stokes equations.  We will denote the velocity
and pressure degrees of freedom as $v$ and $p$ respectively.

More specifically, $F$ is the convection-diffusion-like operator of
size $v \times v$, $B^T$ the pressure gradient of size $v \times p$ ,
$B$ the divergence operator of size $p \times v$ , and $C$ is a
stabilization matrix of size $p \times p$.  Depending on the
discretization $C$ might be the zero matrix.

Why do we want block preconditioners?

\begin{itemize}
\item Want the scalability and mesh-independence of multigrid 
\item Difficult to apply multigrid to the whole system 
\item Segregate blocks and apply multigrid separately to subproblems 
\item Requires a good Schur complement approximation 
\end{itemize}

\section{Block Methods}\label{index_methods}
Meros 1.0 includes the following classes of methods:

\begin{itemize}
\item Approximate Commutator Methods
\begin{enumerate}
 \item Pressure Convection-Diffusion (Fp) methods 
 \item Least Squares Commutator (LSC) methods 
\end{enumerate}
\item Pressure-Projection Methods
\begin{enumerate}
 \item SIMPLE - Semi Implicit Methods for Pressure Linked Equations 
 \item SIMPLEC 
\end{enumerate}
\end{itemize}

\subsection{Block Preconditioning Overview}
This guide describes the use of a block preconditioner within the
Meros package.  The block preconditioners in Meros are designed for
block linear systems of the type
$$ A = 
\begin{bmatrix}
F & B^T \\
B & -C
\end{bmatrix}
\begin{bmatrix}
u \\
p
\end{bmatrix}
=
\begin{bmatrix}
f \\
g
\end{bmatrix}
$$
that usually arise from linearization and discretization of the
incompressible Navier-Stokes equations. When using Meros, the sub
matrices $F, B, B^T, C$ and the vectors $f$ and $g$ are user supplied
and $u$ and $p$ are vectors to be computed.  Meros is intended to be
used on large block sparse linear systems arising from partial
differential equation (PDE) discretizations.  While technically any
block linear system can be considered, Meros should be used on linear
systems that correspond to things that work well with multigrid
methods (e.g. elliptic PDEs).


\subsection{Base Factory Class}
This base factory is for general block preconditioners with Schur
complement approximations. The methods are based on a LD\ U
factorization of the saddlepoint system. The LDU factors of a saddle
point system are given as follows:

$ \begin{bmatrix} A & B^T \\ B & C \end{bmatrix} = \begin{bmatrix} I &
 \\ BF^{-1} & I \end{bmatrix} \begin{bmatrix} F & \\\ & -S
 \end{bmatrix} \begin{bmatrix} I & F^{-1} B^T \\ & I \end{bmatrix}, $

where $S$ is the Schur complement $S = B F^{-1} B^T - C$.





\subsection{Pressure Convection-Diffusion (PCD)}
Factory for building pressure convection-diffusion style block
preconditioner. This class of preconditioners were originally proposed
by Kay, Loghin, and Wathen (ref) and Silvester, Elman, Kay, and Wathen
(ref).

Meros 1.0 currently implements the PCD preconditioner, a.k.a. Fp
preconditioner. 

The LDU factors of a saddle point system are given as follows:

\begin{equation}
  \left[ \begin{array}{cc} A & B^T \\ B & C \end{array} \right]
     = \left[ \begin{array}{cc} I & \\ BF^{-1} & I \end{array} \right]
       \left[ \begin{array}{cc} F & \\  & -S \end{array} \right]
       \left[ \begin{array}{cc} I & F^{-1} B^T  \\  & I \end{array} \right],
\end{equation}

where $S$ is the Schur complement $S = B F^{-1} B^T - C$.  A
pressure convection-diffusion style preconditioner is then given by

\begin{equation}
  P^{-1} =
       \left[ \begin{array}{cc} F & B^T \\ & -\tilde S \end{array} \right]^{-1}
       = 
       \left[ \begin{array}{cc} F^{-1} &  \\  & I \end{array} \right]
       \left[ \begin{array}{cc} I & -B^T \\  & I \end{array} \right]
       \left[ \begin{array}{cc} I &  \\  & -\tilde S^{-1} \end{array} \right]
\end{equation}
where for $\tilde S$ is an approximation to the Schur complement S.

To apply the above preconditioner, we need a linear solver on the
(0,0) block and an approximation to the inverse of the Schur
complement.

To build a concrete preconditioner object, we will also need a 2x2
block Thyra matrix or the 4 separate blocks as either Thyra or Epetra
matrices.  If Thyra, assumes each block is a Thyra EpetraMatrix.





\subsection{Least Squares Commutator (LSC)}
Factory for building least squares commutator style block
preconditioner.  

Note that the LSC preconditioner assumes that we are using
a stable discretization an a uniform mesh.

The LDU factors of a saddle point system are given as follows:

\begin{equation}
  \left[ \begin{array}{cc} A & B^T \\ B & C \end{array} \right]
     = \left[ \begin{array}{cc} I & \\ BF^{-1} & I \end{array} \right]
       \left[ \begin{array}{cc} F & \\  & -S \end{array} \right]
       \left[ \begin{array}{cc} I & F^{-1} B^T  \\  & I \end{array} \right],
\end{equation}
where $S$ is the Schur complement $S = B F^{-1} B^T - C$.
A pressure convection-diffusion style preconditioner is then given by
     
\begin{equation}
     P^{-1} =
       \left[ \begin{array}{cc} F & B^T \\ & -\tilde S \end{array} \right]^{-1}
       = 
       \left[ \begin{array}{cc} F^{-1} &  \\  & I \end{array} \right]
       \left[ \begin{array}{cc} I & -B^T \\  & I \end{array} \right]
       \left[ \begin{array}{cc} I &  \\  & -\tilde S^{-1} \end{array} \right]
\end{equation}
where for $\tilde S$ is an approximation to the Schur complement S.

To apply the above preconditioner, we need a linear solver on the
(0,0) block and an approximation to the inverse of the Schur
complement.

To build a concrete preconditioner object, we will also need a 2x2
block Thyra matrix or the 4 separate blocks as either Thyra or Epetra
matrices.  If Thyra, assumes each block is a Thyra EpetraMatrix.




\subsection{SIMPLE}

\section{Examples}

\subsection{Getting Started with Meros}
In this section, we show how to use Meros as a preconditioner.

LSC Example:


\begin{verbatim}
// @HEADER
// ***********************************************************************
// 
//              Meros: Segregated Preconditioning Package
//                 Copyright (2004) Sandia Corporation
// 
// Under terms of Contract DE-AC04-94AL85000, there is a non-exclusive
// license for use of this work by or on behalf of the U.S. Government.
// 
// This library is free software; you can redistribute it and/or modify
// it under the terms of the GNU Lesser General Public License as
// published by the Free Software Foundation; either version 2.1 of the
// License, or (at your option) any later version.
//  
// This library is distributed in the hope that it will be useful, but
// WITHOUT ANY WARRANTY; without even the implied warranty of
// MERCHANTABILITY or FITNESS FOR A PARTICULAR PURPOSE.  See the GNU
// Lesser General Public License for more details.
//  
// You should have received a copy of the GNU Lesser General Public
// License along with this library; if not, write to the Free Software
// Foundation, Inc., 59 Temple Place, Suite 330, Boston, MA 02111-1307
// USA
// Questions? Contact Michael A. Heroux (maherou@sandia.gov) 
// 
// ***********************************************************************
// @HEADER

// saddle_lsc.cpp

// Example program that reads in Epetra matrices from files, creates a
// Meros Least Squares Commutator (LSC) preconditioner and does a
// solve.

#include "Teuchos_ConfigDefs.hpp"
#include "Teuchos_MPISession.hpp"
#include "Teuchos_GlobalMPISession.hpp"
#include "Teuchos_DefaultComm.hpp"
#include "Teuchos_ParameterList.hpp"
#include "Teuchos_ParameterXMLFileReader.hpp"
#include "Teuchos_RefCountPtr.hpp"

#include "Thyra_SolveSupportTypes.hpp"
#include "Thyra_LinearOpBase.hpp"
#include "Thyra_LinearOpBaseDecl.hpp"
#include "Thyra_VectorDecl.hpp"
#include "Thyra_VectorImpl.hpp" 
#include "Thyra_VectorSpaceImpl.hpp"
#include "Thyra_LinearOperatorDecl.hpp"
#include "Thyra_LinearOperatorImpl.hpp"
#include "Thyra_SpmdVectorBase.hpp"
#include "Thyra_DefaultZeroLinearOp.hpp"
#include "Thyra_DefaultBlockedLinearOpDecl.hpp"
#include "Thyra_LinearOpWithSolveFactoryHelpers.hpp"
#include "Thyra_PreconditionerFactoryHelpers.hpp"
#include "Thyra_DefaultInverseLinearOpDecl.hpp"
#include "Thyra_DefaultInverseLinearOp.hpp"
#include "Thyra_PreconditionerFactoryBase.hpp"
#include "Thyra_DefaultPreconditionerDecl.hpp"
#include "Thyra_DefaultPreconditioner.hpp"
#include "Thyra_SingleRhsLinearOpWithSolveBase.hpp"
#include "Thyra_AztecOOLinearOpWithSolveFactory.hpp"

#ifdef HAVE_MPI
#include "Epetra_MpiComm.h"
#include "mpi.h"
#else
#include "Epetra_SerialComm.h"
#endif

#include "Epetra_Comm.h"
#include "Epetra_Map.h"
#include "Epetra_CrsMatrix.h"
#include "Epetra_Vector.h"

#include "EpetraExt_CrsMatrixIn.h"
#include "EpetraExt_VectorIn.h"

#include "Thyra_EpetraLinearOp.hpp"
#include "Thyra_EpetraThyraWrappers.hpp"

#include "AztecOO.h"
#include "Thyra_AztecOOLinearOpWithSolveFactory.hpp"
#include "Thyra_AztecOOLinearOpWithSolve.hpp"

#include "Meros_ConfigDefs.h"
#include "Meros_Version.h"
#include "Meros_LSCPreconditionerFactory.h"
#include "Meros_LSCOperatorSource.h"
#include "Meros_AztecSolveStrategy.hpp"
#include "Meros_InverseOperator.hpp"
#include "Meros_ZeroOperator.hpp"
#include "Meros_IdentityOperator.hpp"
#include "Meros_LinearSolver.hpp"

using namespace Teuchos;
using namespace EpetraExt;
using namespace Thyra;
using namespace Meros;


int main(int argc, char *argv[]) 
{
  GlobalMPISession mpiSession(&argc, &argv);

  // DEBUG 0 = no extra tests or printing
  // DEBUG 1 = print out basic diagnostics as we go.
  //           prints outer saddle system iterations, but not inner solves
  // DEBUG > 1 = test usage of some of the operators before proceeding
  //           prints inner and outer iterations
  int DEBUG = 1;

  // Get stream that can print to just root or all streams!
  Teuchos::RefCountPtr<Teuchos::FancyOStream>
    out = Teuchos::VerboseObjectBase::getDefaultOStream();

  //  Epetra_Comm* Comm;
#ifdef HAVE_MPI
  Epetra_MpiComm Comm(MPI_COMM_WORLD);
  const int myRank = Comm.MyPID();
  const int numProcs = Comm.NumProc();
#else
  Epetra_SerialComm Comm;
  const int myRank = 0;
  const int numProcs = 1;
#endif

  if(DEBUG > 0)
    {
      if (myRank == 0)
	{
	  cout << "Proc " << myRank << ": " 
	       << "Number of processors = " 
	       << numProcs << endl;
	  cout << "Proc " << myRank << ": " 
	       << Meros::Meros_Version() << endl;
	}
    }
  
  try
    {
      /* ------ Read in epetra matrices and rhs ------------------- */

      // Using Q1 matrix example; see Meros/examples/data/q1
      // (2,2 block (C block) is zero in this example)
  
      // Make necessary Epetra maps.
      // Need a velocity space map and a pressure space map.
      const Epetra_Map* velocityMap = new Epetra_Map(578, 0, Comm); 
      const Epetra_Map* pressureMap = new Epetra_Map(192, 0, Comm); 
 

      // Read matrix and vector blocks into Epetra_Crs matrices
      Epetra_CrsMatrix* FMatrix(0);
      char * filename = "../../../../../packages/meros/example/data/q1/Aq1.mm";
      MatrixMarketFileToCrsMatrix(filename,
                                  *velocityMap,
                                  *velocityMap, *velocityMap,
                                  FMatrix);

      // cerr << "If you get Epetra ERROR -1 here, the file was not found"
      //  << "Need to make symbolic link to data directory" 
      //  << endl;

      Epetra_CrsMatrix* BtMatrix(0);
      filename = "../../../../../packages/meros/example/data/q1/Btq1.mm";
      MatrixMarketFileToCrsMatrix(filename,
                                  *velocityMap, 
                                  *velocityMap, *pressureMap,
                                  BtMatrix);

      Epetra_CrsMatrix* BMatrix(0);
      filename = "../../../../../packages/meros/example/data/q1/Bq1.mm";
      MatrixMarketFileToCrsMatrix(filename,
                                  *pressureMap,
                                  *pressureMap, *velocityMap,
                                  BMatrix);


      Epetra_Vector* rhsq1_vel(0);
      filename = "../../../../../packages/meros/example/data/q1/rhsq1_vel.mm";
      MatrixMarketFileToVector(filename,
                               *velocityMap, 
                               rhsq1_vel);

      Epetra_Vector* rhsq1_press(0);
      filename = "../../../../../packages/meros/example/data/q1/rhsq1_press.mm";
      MatrixMarketFileToVector(filename,
                               *pressureMap, 
                               rhsq1_press);

      FMatrix->FillComplete();
      BtMatrix->FillComplete(*pressureMap, *velocityMap);
      BMatrix->FillComplete(*velocityMap, *pressureMap);


      // Wrap Epetra operators into Thyra operators. To do this, we
      // first wrap as Thyra core operators, then convert to the
      // handle layer LinearOperators.
      
      RefCountPtr<LinearOpBase<double> >
        tmpF = rcp(new EpetraLinearOp(rcp(FMatrix,false)));
      const LinearOperator<double> F = tmpF;

      RefCountPtr<LinearOpBase<double> >
        tmpBt = rcp(new EpetraLinearOp(rcp(BtMatrix,false)));
      const LinearOperator<double> Bt = tmpBt;

      RefCountPtr<LinearOpBase<double> >
        tmpB = rcp(new EpetraLinearOp(rcp(BMatrix,false)));
      const LinearOperator<double> B = tmpB;


      // Wrap Epetra vectors into Thyra vectors similarly by first
      // wrapping to the Thyra core vector layer, then converting to
      // the handle layer

      RefCountPtr<const Thyra::VectorSpaceBase<double> > epetra_vs_press
        = Thyra::create_VectorSpace(rcp(pressureMap,false));
      RefCountPtr<const Thyra::VectorSpaceBase<double> > epetra_vs_vel
        = Thyra::create_VectorSpace(rcp(velocityMap,false));


      RefCountPtr<VectorBase<double> > rhs1
        = create_Vector(rcp(rhsq1_press, false), epetra_vs_press);
      RefCountPtr<VectorBase<double> > rhs2
        = create_Vector(rcp(rhsq1_vel, false), epetra_vs_vel);

      // Convert the vectors to handled vectors
      RefCountPtr<VectorBase<double> > tmp1 = rhs1;
      const Vector<double> rhs_press = tmp1;

      RefCountPtr<VectorBase<double> > tmp2 = rhs2;
      const Vector<double> rhs_vel = tmp2;


      if(DEBUG > 1){
        // Test the wrapped operators
        cerr << "F matrix: description  " << F.description() << endl;
        cerr << "F domain: " << F.domain().dim() 
             << ",  F range: " << F.range().dim() 
             << endl;
        cerr << "B matrix: description  " << B.description() << endl;
        cerr << "B domain: " << B.domain().dim() 
             << ",  B range: " << B.range().dim() 
             << endl;
        cerr << "Bt matrix: description  " << Bt.description() << endl;
        cerr << "Bt domain: " << Bt.domain().dim() 
             << ",  Bt range: " << Bt.range().dim() 
             << endl;

        Vector<double> testvec1 = Bt * rhs_press;
        cerr << "Bt * rhs_press = " << norm2(testvec1) <<endl;

	cerr << "domain size of B " << B.domain().dim() << endl;
	cerr << "range size of B " << B.range().dim() << endl;
	cerr << "size of testvec1 " << dim(testvec1) << endl;

        Vector<double> testvec2 = B * testvec1;
        cerr << "B * Bt * rhs_press = " << norm2(testvec2) <<endl;

        Vector<double> testvec3 = F * rhs_vel;
        cerr << "F * rhs_vel = " << norm2(testvec3) <<endl;

        Vector<double> testvec4 = B * F * Bt * rhs_press;
        cerr << "BFBt * rhs_press = " << norm2(testvec4) <<endl;

      }



      // Get Thyra velocity and pressure spaces from the operators.
      VectorSpace<double> velocitySpace = F.domain();
      VectorSpace<double> pressureSpace = Bt.domain();

      if(DEBUG > 0){
        cerr << "P" << myRank 
             << ": vel space dim = " << velocitySpace.dim() << endl;
        cerr << "P" << myRank 
             << ": press space dim = " << pressureSpace.dim() << endl;
      }

      // Make a zero operator on the small (pressure) space since the
      // 2,2 block (C) is zero in this example.
      // RefCountPtr<Thyra::LinearOpBase<double> > tmpZ = 
      // rcp(new DefaultZeroLinearOp<double>(tmpBt->domain(), 
      //                                   tmpBt->domain()));
      // const LinearOperator<double> Z = tmpZ;
      const LinearOperator<double> Z;

      // Build the block saddle operator with F, Bt, B, and Z.  This
      // operator will be blockOp = [F Bt; B zero] (in Matlab
      // notation).
      LinearOperator<double> blockOp = block2x2(F, Bt, B, Z);

      if(DEBUG > 1) 
        {
          // Test getting subblock components out of the block operator.
          LinearOperator<double> testF1 = blockOp.getBlock(0,0);
          LinearOperator<double> testBt1 = blockOp.getBlock(0,1);
          LinearOperator<double> testB1 = blockOp.getBlock(1,0);
          cerr << "Checking blocks after extracting them out the block op"
               << endl;
          cerr << "testF domain: " << testF1.domain().dim() << endl;
          cerr << "testF range: " << testF1.range().dim() << endl;
          cerr << "testBt domain: " << testBt1.domain().dim() << endl;
          cerr << "testBt range: " << testBt1.range().dim() << endl;
          cerr << "testB domain: " << testB1.domain().dim() << endl;
          cerr << "testB range: " << testB1.range().dim() << endl;

          Vector<double> testvec1 = testBt1 * rhs_press;
          cerr << "Bt * rhs_press = " << norm2(testvec1) <<endl;
	  
          Vector<double> testvec2 = testB1 * testvec1;
          cerr << "B * Bt * rhs_press = " << norm2(testvec2) <<endl;
	  
          Vector<double> testvec3 = testF1 * rhs_vel;
          cerr << "F * rhs_vel = " << norm2(testvec3) <<endl;
	  
          Vector<double> testvec4 = testB1 * testF1 * testBt1 * rhs_press;
          cerr << "BFBt * rhs_press = " << norm2(testvec4) <<endl;
        }

      // Get the domain and range product spaces from the block operator. 
      VectorSpace<double> domain = blockOp.domain();
      VectorSpace<double> range = blockOp.range();

      // Build a product vector for the rhs and set the velocity and
      // pressure components of the rhs with the vectors we previously
      // read in from files and wrapped in Thyra.
      Vector<double> rhs = range.createMember();
      rhs.setBlock(0,rhs_vel);
      rhs.setBlock(1,rhs_press);

      // Build a solution vector and initialize it to zero.
      Vector<double> solnblockvec = domain.createMember();
      zeroOut(solnblockvec);


      // We now have Thyra block versions of the saddle point system,
      // rhs, and solution vector. Next we build the Meros LSC
      // preconditioner.


      /* -------- Build the Meros preconditioner factory ---------*/

      // Build a Least Squares Commutator (LSC) block preconditioner
      // with Meros
      // 
      // | inv(F) 0 | | I  -Bt | | I        |
      // | 0      I | |     I  | |   -inv(X)|
      // 
      // where inv(X) = inv(B*Bt) * B * F * Bt * inv(B*Bt)
      // (velocity mass matrix is the identity in this example)
      //
      // We'll do this in 4 steps:
      // 1) Build an AztecOO ParameterList for inv(F) solve
      // 2) Build an AztecOO ParameterList for inv(B*Bt) solve
      //    The Schur complement approximation inverse requires solves
      //    on the composed operator B*Bt.
      // 3) Make an LSCOperatorSource with blockOp  (and Qu if needed)
      // 4) Build the LSC block preconditioner factory 


      // 1) Build an AztecOO ParameterList for inv(F) solve
      //    This one corresponds to (unpreconditioned) GMRES

      RefCountPtr<ParameterList>
        aztecFParams = rcp(new ParameterList("aztecOOFSolverFactory"));

      RefCountPtr<LinearOpWithSolveFactoryBase<double> > aztecFLowsFactory;

      if(DEBUG> 1)
        {
          // Print out valid parameters and the existing default params.
          aztecFLowsFactory = rcp(new AztecOOLinearOpWithSolveFactory());
          cerr << "\naztecFLowsFactory.getValidParameters():\n" << endl;
          aztecFLowsFactory->getValidParameters()->print(cerr, 0, true, false);
          cerr << "\nPrinting initial parameters. " << endl;
          aztecFLowsFactory->setParameterList(aztecFParams);
          aztecFLowsFactory->getParameterList()->print(cerr, 0, true, false);
        }


      // forward solve settings
      aztecFParams->sublist("Forward Solve").set("Max Iterations", 100);
      aztecFParams->sublist("Forward Solve").set("Tolerance", 10e-8);
      // aztecOO solver settings
      aztecFParams->sublist("Forward Solve")
        .sublist("AztecOO Settings").set("Aztec Solver", "GMRES");
      aztecFParams->sublist("Forward Solve")
        .sublist("AztecOO Settings").set("Aztec Preconditioner", "none");
      aztecFParams->sublist("Forward Solve")
        .sublist("AztecOO Settings").set("Size of Krylov Subspace", 100);


      if(DEBUG > 1)
        {
          // turn on AztecOO output
          aztecFParams->sublist("Forward Solve")
            .sublist("AztecOO Settings").set("Output Frequency", 10);
        }

      if(DEBUG > 1)
        {
          // Print out the parameters we just set
          aztecFLowsFactory->setParameterList(aztecFParams);
          aztecFLowsFactory->getParameterList()->print(cerr, 0, true, false);
        }


      // 2) Build an AztecOO ParameterList for inv(Ap) solve
      //    This one corresponds to unpreconditioned CG.

      RefCountPtr<ParameterList>
        aztecBBtParams = rcp(new ParameterList("aztecOOBBtSolverFactory"));

      // forward solve settings
      aztecBBtParams->sublist("Forward Solve").set("Max Iterations", 100);
      aztecBBtParams->sublist("Forward Solve").set("Tolerance", 10e-8);
      // aztecOO solver settings
      aztecBBtParams->sublist("Forward Solve")
        .sublist("AztecOO Settings").set("Aztec Solver", "CG");
      aztecBBtParams->sublist("Forward Solve")
        .sublist("AztecOO Settings").set("Aztec Preconditioner", "none");


      if(DEBUG > 1)
        {
          // turn on AztecOO output
          aztecBBtParams->sublist("Forward Solve")
            .sublist("AztecOO Settings").set("Output Frequency", 10);
        }


      if(DEBUG > 1)
        {
          // Print out the parameters we just set
          RefCountPtr<LinearOpWithSolveFactoryBase<double> > 
            aztecBBtLowsFactory = rcp(new AztecOOLinearOpWithSolveFactory());
          aztecBBtLowsFactory->setParameterList(aztecBBtParams);
          aztecBBtLowsFactory->getParameterList()->print(cerr, 0, true, false);
        }


      // 3) Make an LSCOperatorSource that contains  blockOp
      //    The velocity mass matrix Qu is the identity in this example.
      RefCountPtr<const LinearOpSourceBase<double> > myLSCopSrcRcp 
        = rcp(new LSCOperatorSource(blockOp));


      if(DEBUG > 1) 
        {
          // Test getting subblock components out of the operator source
          RefCountPtr<const LSCOperatorSource> lscOpSrcPtr 
            = rcp_dynamic_cast<const LSCOperatorSource>(myLSCopSrcRcp);  
	  
          // Retrieve operators from the LSC operator source
          RefCountPtr<const LinearOpBase<double> > tmpBlockOp 
            = lscOpSrcPtr->getOp();
          ConstLinearOperator<double> blockOp = tmpBlockOp;
          ConstLinearOperator<double> testF2 = blockOp.getBlock(0,0);
          ConstLinearOperator<double> testBt2 = blockOp.getBlock(0,1);
          ConstLinearOperator<double> testB2 = blockOp.getBlock(1,0);	 
	  
          cerr << "Checking blocks after extracting them out the block op"
               << endl;
          cerr << "testF domain: " << testF2.domain().dim() << endl;
          cerr << "testF range: " << testF2.range().dim() << endl;
          cerr << "testBt domain: " << testBt2.domain().dim() << endl;
          cerr << "testBt range: " << testBt2.range().dim() << endl;
          cerr << "testB domain: " << testB2.domain().dim() << endl;
          cerr << "testB range: " << testB2.range().dim() << endl;

          Vector<double> testvec11 = testBt2 * rhs_press;
          cerr << "Bt * rhs_press = " << norm2(testvec11) <<endl;
	  
          Vector<double> testvec22 = testB2 * testvec11;
          cerr << "B * Bt * rhs_press = " << norm2(testvec22) <<endl;
	  
          Vector<double> testvec33 = testF2 * rhs_vel;
          cerr << "F * rhs_vel = " << norm2(testvec33) <<endl;
	  
          Vector<double> testvec44 = testB2 * testF2 * testBt2 * rhs_press;
          cerr << "BFBt * rhs_press = " << norm2(testvec44) <<endl;
        }


      // 4) Build the LSC block preconditioner factory.
      RefCountPtr<PreconditionerFactoryBase<double> > merosPrecFac
        = rcp(
	      new LSCPreconditionerFactory(
		rcp(new Thyra::AztecOOLinearOpWithSolveFactory(aztecFParams)),
		rcp(new Thyra::AztecOOLinearOpWithSolveFactory(aztecBBtParams))
		)
	      );
      
      RefCountPtr<PreconditionerBase<double> > Prcp 
        = merosPrecFac->createPrec();
      
      merosPrecFac->initializePrec(myLSCopSrcRcp, &*Prcp);


      // Checking that isCompatible and uninitializePrec at least
      // compile and throw the intended exceptions for now.
      // merosPrecFac->uninitializePrec(&*Prcp, &myLSCopSrcRcp);
      // bool passed;
      // passed = merosPrecFac->isCompatible(*(myLSCopSrcRcp.get()));


      /* --- Now build a solver factory for outer saddle point problem --- */

      // Set up parameter list and AztecOO solver
      RefCountPtr<ParameterList> aztecSaddleParams 
        = rcp(new ParameterList("aztecOOSaddleSolverFactory"));
      
      RefCountPtr<LinearOpWithSolveFactoryBase<double> >
        aztecSaddleLowsFactory = rcp(new AztecOOLinearOpWithSolveFactory());
      
      double saddleTol = 1.0e-6;

      // forward solve settings
      aztecSaddleParams->sublist("Forward Solve").set("Max Iterations", 500);
      aztecSaddleParams->sublist("Forward Solve").set("Tolerance", saddleTol);
      // aztecOO solver settings
      aztecSaddleParams->sublist("Forward Solve")
        .sublist("AztecOO Settings").set("Aztec Solver", "GMRES");
      aztecSaddleParams->sublist("Forward Solve")
        .sublist("AztecOO Settings").set("Aztec Preconditioner", "none");
      aztecSaddleParams->sublist("Forward Solve")
        .sublist("AztecOO Settings").set("Size of Krylov Subspace", 500);

      if(DEBUG > 0)
        {
          // turn on AztecOO output
          aztecSaddleParams->sublist("Forward Solve")
            .sublist("AztecOO Settings").set("Output Frequency", 1);
        }

      aztecSaddleLowsFactory->setParameterList(aztecSaddleParams);

      if(DEBUG > 0)
        {
          // Print out the parameters we've set.
          aztecSaddleLowsFactory->getParameterList()->print(cerr, 0, 
                                                            true, false);
        }



      // Set up the preconditioned inverse object and do the solve!
      RefCountPtr<LinearOpWithSolveBase<double> > rcpAztecSaddle 
        = aztecSaddleLowsFactory->createOp();

      // LinearOperator<double> epetraBlockOp = makeEpetraOperator(blockOp);

      // initializePreconditionedOp<double>(*aztecSaddleLowsFactory, 
      //			 epetraBlockOp.ptr(), 
      //				 Prcp,
      //				 &*rcpAztecSaddle );

      //       initializePreconditionedOp<double>(*aztecSaddleLowsFactory, 
      // 					 blockOp.ptr(), 
      // 					 Prcp,
      // 					 &*rcpAztecSaddle );
      
      //       RefCountPtr<LinearOpBase<double> > tmpSaddleInv 
      // 	= rcp(new DefaultInverseLinearOp<double>(rcpAztecSaddle));
      
      //       LinearOperator<double> saddleInv = tmpSaddleInv;
      //       saddleInv.description();

      RefCountPtr<const LinearOpBase<double> > tmpPinv 
        = Prcp->getRightPrecOp();
      ConstLinearOperator<double> Pinv = tmpPinv;

      LinearSolveStrategy<double> azSaddle 
        = new AztecSolveStrategy(*(aztecSaddleParams.get()));

      ConstLinearOperator<double> saddleInv 
	= new InverseOperator<double>(blockOp * Pinv, azSaddle);

      // Do the solve!
      solnblockvec = saddleInv * rhs;

      // Check our results.
      Vector<double> residvec = blockOp * Pinv * solnblockvec - rhs;

      cerr << "norm of resid " << norm2(residvec) << endl;
      
      double normResvec = norm2(residvec);

      if(normResvec < 10.0*saddleTol)
	{
	  cerr << "Example PASSED!" << endl;
	  return 0;
	}
      else
	{
	  cerr << "Example FAILED!" << endl;
	  return 1;
	}



    } // end of try block


  catch(std::exception& e)
    {
      cerr << "Caught exception: " << e.what() << endl;
    }

  MPISession::finalize();

} // end of main()


\end{verbatim}


\appendix

\section{Quick Start}
This section is intended for the user who wants to build meros very quickly.
It's assumed that you've already have a local copy of
trilinos\footnote{Please refer to the web page {\sc
  http://software.sandia.gov/trilinos} on how to obtain a copy of
    trilinos.}.
Using the instructions here, your build of trilinos will have the following
libraries: aztecoo, epetra, epetraext, ML, newpackage,
nox teuchos, and thyra.
\begin{itemize}
\item \verb!cd! into the trilinos~directory.
\item Make a build directory, e.g., \verb!mkdir LINUX!.
\item \verb!cd LINUX!.
\item Configure trilinos:
  \begin{enumerate}
  \item   If you do not want to use MPI:
\begin{verbatim} 
{ ../configure --enable-teuchos --enable-amesos} 
\end{verbatim}
  \item To use MPI:
\begin{verbatim} 
{ ../configure  --enable-teuchos \\
  --enable-amesos --with-mpi-compilers=/usr/local/mpich/bin} 
\end{verbatim}
where your path to the mpi-compilers is specified.
  \end{enumerate}

\item Build trilinos: \verb!make!.
\item If your build finished without errors, you should see the directory\\
\verb!Trilinos/LINUX/packages/!, with subdirectories below that for
each individual library.   Meros's subdirectory, \verb!meros!, should contain
files \verb!config.log!, \verb!config.status!, \verb!Makefile!, and
\verb!Makefile.export!, and directories \verb!src! and \verb!examples!.
Directory \verb!src! contains object files and \verb!libmeros.a!.
Directory \verb!examples! contains executables with extension \verb!.exe!,
symbolic links to the corresponding source code, and object files.
Directory \verb!test! is intended primarily for developers and can be ignored.
\item Look in \verb!Trilinos/LINUX/packages/meros/examples! for examples of how
to use Meros. File \verb!Trilinos/packages/meros/examples/README! suggests how to
use the examples.
\end{itemize}


\section{Meros \InlineCommand{configure} Options}
Sample configure script for Meros:

\begin{verbatim}
BUILD_DIR=`pwd`

../configure \
--enable-mpi \
--with-mpi-compilers="/usr/local/mpich-1.2.7_gcc3.4.3/bin" \
CXXFLAGS="-g -O0 -Wreturn-type" \
--with-libs="-lexpat" \
--disable-default-packages \
--enable-teuchos --enable-teuchos-extended \
--enable-teuchos-complex --enable-teuchos-abc --enable-teuchos-expat \
--enable-thyra \
--enable-thyra-examples \
--enable-epetra \
--enable-epetra-thyra \
--enable-epetraext \
--enable-epetraext-thyra \
--enable-ml \
--enable-ml-thyra \
--enable-aztecoo \
--enable-aztecoo-thyra \
--enable-ifpack \
--enable-ifpack-thyra \
--enable-nox \
--enable-stratimikos \
--enable-meros \
--enable-meros-examples \
--with-gnumake \
--with-install="/usr/bin/install -c -p" \
--prefix=$BUILD_DIR
\end{verbatim}


\bibliography{merosrefs}
    
%%
% This is an example of how to create the distribution page. Some
% distributions are required by Sandia; e.g. the housekeeping copies.
% Depending on the type of report; e.g. CRADA, Patent Caution, etc.
% additional distribution lines may have to be added. See the
% "Guide for Preparing SAND Reports"
%
% SANDdistribution takes CA or NM as an optional argument. If given,
% the approrpiate housekeeping copies are inserted autmatically.
% Inside the SANDdistribution environment, several commands can be used
% insert the distributions for CRADA, LDRD, etc. See example below.
%
% You can leave the CA or NM option off and not use any of the SANDdist*
% commands. This will allow you to create a distribution list manually.
%
\begin{SANDdistribution}[NM]
    % Housekeeping copies necessary for every unclassified report:
    % \SANDdistCRADA	% If this report is about CRADA work
    % \SANDdistPatent	% If this report has a Patent Caution or Patent Interest
    % \SANDdistLDRD	% If this report is about LDRD work

    % Some external Addresses

    \SANDdistExternal{1}{Roscoe A. Bartlett\\ Oak Ridge National Laboratory \\ P.O. Box 2008 \\ Oak Ridge, Tennessee 37831 \\ United States }
    % \SANDdistExternal{3}{Some Address\\ and street\\City, State}
    % \SANDdistExternal{12}{Another Address\\ On a street\\City, State\\U.S.A.}
    %\bigskip

    % The following MUST BE between the external and internal distributions!
    % \SANDdistClassified % If this report is classified

    % Internal Addresses
    \SANDdistInternal{1}{9018}{Central Technical Files}{8945-1}
    \SANDdistInternal{2}{0899}{Technical Library}{9610}
    \SANDdistInternal{2}{0612}{Review \& Approval Desk}{4916}
    \SANDdistInternal{1}{0897}{Todd S. Coffey}{1543}
    \SANDdistInternal{1}{1320}{Curtis C. Ober}{1442}
    \SANDdistInternal{1}{1318}{Roger P. Pawlowski}{1444}
    
    % Example of a mail channel use (instead of a mail stop)
    % \SANDdistInternalM{1}{M9999}{Someone}{01234}

\end{SANDdistribution}


\end{document}
