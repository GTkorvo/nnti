\section{Methods}
\label{index_methods}
%Adding a citation to test bib\cite{ElmanSilvesterWathen.book}.  
\subsection{Overview}
Outline:
\begin{itemize}
\item Overview of problems Meros is intended to solve and the methods
  included
\item Overview of PreconditionerFactory and how Meros extends from
  Thyra
\end{itemize}


Meros is a segregated preconditioning package within
Trilinos~\cite{Trilinos-Overview}.  Meros
provides scalable block preconditioning for problems that couple
simultaneous solution variables such as Navier-Strokes problems.

The initial focus of Meros is on preconditioners for the
incompressible Navier-Stokes equations:
\begin{equation} \label{NS}
  \begin{array}{rcl}
    \eta {\bf u}_t -\nu \nabla^2 {\bf u}
    + ({\bf u} \cdot {\rm grad})\, {\bf u} +
    {\rm grad}\, p &= &{\bf f} \\
    -{\rm div}\, {\bf u} &= &0
  \end{array}
\end{equation}
on $\Omega \subset \mathbb{R}^d$, $d=2$ or $3$.  Here, ${\bf u}$
is the $d$-dimensional velocity field, which is assumed to satisfy
suitable boundary conditions on $\partial \Omega$, $p$ is the
pressure, and $\nu$ is the kinematic viscosity, which is
inversely proportional to the Reynolds number.  
The value $\eta=0$ corresponds to the steady-state problem and $\eta=1$
to the  case of unsteady flow. 
Linearization and discretization of (\ref{NS}) by finite elements, finite 
differences or finite volumes leads to a sequence of linear systems of 
equations of the form
\begin{equation} \label{NSdiscretelinearStabilized}
\left[  
  \begin{array}{lc}
    F  & \ B^T   \\ B  & -C
  \end{array}
 \right]
  \begin{bmatrix}
    {\bf u}\\ p
  \end{bmatrix}
  =
  \begin{bmatrix}
    {\bf f}\\ g
  \end{bmatrix}.
\end{equation}
These systems, which are the primary focus of the Meros
preconditioners, must be solved 
at each step of a nonlinear (Picard or Newton) iteration, or at
each time step.   Here, $B$ and $B^T$ are matrices corresponding
to discrete divergence and gradient operators, respectively and
$F$ operates on the discrete velocity space. 

This guide describes the use of a block preconditioner within the
Meros package.  The block preconditioners can be used to solve block
linear systems of the type in (\ref{NSdiscretelinearStabilized}).
We will denote the velocity
and pressure degrees of freedom as $v$ and $p$ respectively.

The user must supply the four subblocks $F$, $B^T$, $B$, and $C$. If
the matrix $C$ is not provided, it is assumed to be the zero matrix.
$F$ is the convection-diffusion-like operator of size $v \times v$,
$B^T$ the pressure gradient of size $v \times p$ , $B$ the divergence
operator of size $p \times v$ , and $C$ is a stabilization matrix of
size $p \times p$.  Depending on the discretization $C$ might be the
zero matrix. For {\em div-stable} discretizations, $C=0$.  For mixed
approximation methods that do not uniformly satisfy a discrete inf-sup
condition, the matrix $C$ is a nonzero {\em stabilization operator}.

Block preconditioners a useful for a number of reasons:


\begin{itemize}
\item Want the scalability and mesh-independence of multigrid 
\item Difficult to apply multigrid to the whole system 
\item Segregate blocks and apply multigrid separately to subproblems 
\item Requires a good Schur complement approximation 
\end{itemize}



Meros 1.0 includes the following classes of methods:

\begin{itemize}
\item Approximate Commutator Methods
\begin{enumerate}
 \item Pressure Convection-Diffusion (Fp) methods 
 \item Least Squares Commutator (LSC) methods 
\end{enumerate}
\item Pressure-Projection Methods
\begin{enumerate}
 \item SIMPLE - Semi Implicit Methods for Pressure Linked Equations 
 \item SIMPLEC 
\end{enumerate}
\end{itemize}














This guide describes the use of a block preconditioner within the
Meros package.  The block preconditioners in Meros are designed for
block linear systems of the type
$$ A = 
\begin{bmatrix}
F & B^T \\
B & C
\end{bmatrix}
\begin{bmatrix}
u \\
p
\end{bmatrix}
=
\begin{bmatrix}
f \\
g
\end{bmatrix}
$$
that usually arise from linearization and discretization of the
incompressible Navier-Stokes equations. When using Meros, the sub
matrices $F, B, B^T, C$ and the vectors $f$ and $g$ are user supplied
and $u$ and $p$ are vectors to be computed.  Meros is intended to be
used on large block sparse linear systems arising from partial
differential equation (PDE) discretizations.  While technically any
block linear system can be considered, Meros should be used on linear
systems that correspond to things that work well with multigrid
methods (e.g. elliptic PDEs).



The methods in Meros are based on an LDU
factorization of the saddlepoint system. The LDU factors of a saddle
point system are given as follows:

$ \begin{bmatrix} A & B^T \\ B & C \end{bmatrix} = \begin{bmatrix} I &
 \\ BF^{-1} & I \end{bmatrix} \begin{bmatrix} F & \\\ & -S
 \end{bmatrix} \begin{bmatrix} I & F^{-1} B^T \\ & I \end{bmatrix}, $

where $S$ is the Schur complement $S = B F^{-1} B^T - C$.

\subsection{Pressure Convection-Diffusion (PCD)}
\begin{itemize}
\item Mathematical overview
\item PCDPreconditionerFactory and PCDOperatorSource
\end{itemize}


In this section we describe the factory for building a pressure
convection-diffusion style block preconditioner. This class of
preconditioners were originally proposed by Kay, Loghin, and
Wathen~\cite{KayLoghinWathen} and Silvester, Elman, Kay, and
Wathen~\cite{SilvesterElmanKayWathen}.

Meros currently implements the PCD preconditioner, a.k.a. Fp
preconditioner. 

The LDU factors of a saddle point system are given as follows:

\begin{equation}
  \left[ \begin{array}{cc} A & B^T \\ B & C \end{array} \right]
     = \left[ \begin{array}{cc} I & \\ BF^{-1} & I \end{array} \right]
       \left[ \begin{array}{cc} F & \\  & -S \end{array} \right]
       \left[ \begin{array}{cc} I & F^{-1} B^T  \\  & I \end{array} \right],
\end{equation}

where $S$ is the Schur complement $S = B F^{-1} B^T - C$.  A
pressure convection-diffusion style preconditioner is then given by

\begin{equation}
  P^{-1} =
       \left[ \begin{array}{cc} F & B^T \\ & -\tilde S \end{array} \right]^{-1}
       = 
       \left[ \begin{array}{cc} F^{-1} &  \\  & I \end{array} \right]
       \left[ \begin{array}{cc} I & -B^T \\  & I \end{array} \right]
       \left[ \begin{array}{cc} I &  \\  & -\tilde S^{-1} \end{array} \right]
\end{equation}
where for $\tilde S$ is an approximation to the Schur complement S.

To apply the above preconditioner, we need a linear solver on the
(0,0) block and an approximation to the inverse of the Schur
complement.

To build a concrete preconditioner object, we will also need a 2x2
block Thyra matrix or the 4 separate blocks as either Thyra or Epetra
matrices.  If Thyra, assumes each block is a Thyra EpetraMatrix.

\subsection{Least Squares Commutator (LSC)}
\begin{itemize}
\item Mathematical overview
\item LSCPreconditionerFactory and LSCOperatorSource
\end{itemize}


Factory for building least squares commutator style block
preconditioner.  

Note that the LSC preconditioner assumes that we are using
a stable discretization an a uniform mesh.

The LDU factors of a saddle point system are given as follows:

\begin{equation}
  \left[ \begin{array}{cc} A & B^T \\ B & C \end{array} \right]
     = \left[ \begin{array}{cc} I & \\ BF^{-1} & I \end{array} \right]
       \left[ \begin{array}{cc} F & \\  & -S \end{array} \right]
       \left[ \begin{array}{cc} I & F^{-1} B^T  \\  & I \end{array} \right],
\end{equation}
where $S$ is the Schur complement $S = B F^{-1} B^T - C$.
A pressure convection-diffusion style preconditioner is then given by
     
\begin{equation}
     P^{-1} =
       \left[ \begin{array}{cc} F & B^T \\ & -\tilde S \end{array} \right]^{-1}
       = 
       \left[ \begin{array}{cc} F^{-1} &  \\  & I \end{array} \right]
       \left[ \begin{array}{cc} I & -B^T \\  & I \end{array} \right]
       \left[ \begin{array}{cc} I &  \\  & -\tilde S^{-1} \end{array} \right]
\end{equation}
where for $\tilde S$ is an approximation to the Schur complement S.

To apply the above preconditioner, we need a linear solver on the
(0,0) block and an approximation to the inverse of the Schur
complement.

To build a concrete preconditioner object, we will also need a 2x2
block Thyra matrix or the 4 separate blocks as either Thyra or Epetra
matrices.  If Thyra, assumes each block is a Thyra EpetraMatrix.


\subsection{SIMPLE}
\begin{itemize}
\item Mathematical overview
\item SIMPLEPreconditionerFactory and SIMPLEOperatorSource
\end{itemize}
