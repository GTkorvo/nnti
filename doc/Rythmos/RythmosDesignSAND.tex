\documentclass[pdf,ps2pdf,11pt]{SANDreport}
\usepackage{pslatex}

%Local stuff
\usepackage{graphicx}
\usepackage{latexsym}
%
% Command to print 1/2 in math mode real nice
%
\newcommand{\myonehalf}{{}^1 \!\!  /  \! {}_2}

%
% Command to print over/under left aligned in math mode
%
\newcommand{\myoverunderleft}[2]{ \begin{array}{l} #1 \\ \scriptstyle #2 \end{array} }

%
% Command to number equations 1.a, 1.b etc.
%
\newcounter{saveeqn}
\newcommand{\alpheqn}{\setcounter{saveeqn}{\value{equation}}
\stepcounter{saveeqn}\setcounter{equation}{0}
\renewcommand{\theequation}
	{\mbox{\arabic{saveeqn}.\alph{equation}}}}
\newcommand{\reseteqn}{\setcounter{equation}{\value{saveeqn}}
\renewcommand{\theequation}{\arabic{equation}}}

%
% Shorthand macros for setting up a matrix or vector
%
\newcommand{\bmat}[1]{\left[ \begin{array}{#1}}
\newcommand{\emat}{\end{array} \right]}

%
% Command for a good looking \Re in math enviornment
%
\newcommand{\RE}{\mbox{\textbf{I}}\hspace{-0.6ex}\mbox{\textbf{R}}}

%
% Commands for Jacobians
%
\newcommand{\Jac}[2]{\displaystyle{\frac{\partial #1}{\partial #2}}}
\newcommand{\jac}[2]{\partial #1 / \partial #2}

%
% Commands for Hessians
%
\newcommand{\Hess}[2]{\displaystyle{\frac{\partial^2 #1}{\partial #2^2}}}
\newcommand{\hess}[2]{\partial^2 #1 / \partial #2^2}
\newcommand{\HessTwo}[3]{\displaystyle{\frac{\partial^2 #1}{\partial #2 \partial #3}}}
\newcommand{\hessTwo}[3]{\partial^2 #1 / (\partial #2 \partial #3)}



%\newcommand{\Hess2}[3]{\displaystyle{\frac{\partial^2 #1}{\partial #2 \partial #3}}}
%\newcommand{\myHess2}{\frac{a}{b}}
%\newcommand{\hess2}[3]{\partial^2 #1 / (\partial #2 \partial #3)}

%
% Shorthand macros for setting up a single tab indent
%
\newcommand{\bifthen}{\begin{tabbing} xxxx\=xxxx\=xxxx\=xxxx\=xxxx\=xxxx\= \kill}
\newcommand{\eifthen}{\end{tabbing}}

%
% Shorthand for inserting four spaces
%
\newcommand{\tb}{\hspace{4ex}}

%
% Commands for beginning and ending single spacing
%
\newcommand{\bsinglespace}{\renewcommand{\baselinestretch}{1.2}\small\normalsize}
\newcommand{\esinglespace}{}



% If you want to relax some of the SAND98-0730 requirements, use the "relax"
% option. It adds spaces and boldface in the table of contents, and does not
% force the page layout sizes.
% e.g. \documentclass[relax,12pt]{SANDreport}
%
% You can also use the "strict" option, which applies even more of the
% SAND98-0730 guidelines. It gets rid of section numbers which are often
% useful; e.g. \documentclass[strict]{SANDreport}

% ---------------------------------------------------------------------------- %
%
% Set the title, author, and date
%

\title{
Rythmos: A set of abstract and concrete C++ classes for the solution and
analysis of DAEs and ODEs
}
\author{
Roscoe A. Bartlett \\ Optimization/Uncertainty Estim \\ \\
Todd Coffey \\ Computational Math/Algorithms \\ \\
... \\ \\
Sandia National Laboratories\footnote{
Sandia is a multiprogram laboratory operated by Sandia Corporation, a
Lockheed-Martin Company, for the United States Department of Energy
under Contract DE-AC04-94AL85000.}, Albuquerque NM 87185 USA, \\
}
\date{}

% ---------------------------------------------------------------------------- %
% Set some things we need for SAND reports. These are mandatory
%
\SANDnum{SAND2005-xxxx}
\SANDprintDate{January 12, 2005}
\SANDauthor{
Roscoe A. Bartlett \\ Optimization/Uncertainty Estim \\ \\
Todd Coffey \\ Computational Math/Algorithms \\ \\
... \\ \\
}

% ---------------------------------------------------------------------------- %
% The following definitions are optional. The values shown are the default
% ones provided by SANDreport.cls
%
%\SANDreleaseType{Unlimited Release}
\SANDreleaseType{Not approved for release outside Sandia}

% ---------------------------------------------------------------------------- %
% The following definition does not have a default value and will not
% print anything, if not defined
%
%\SANDsupersed{SAND1901-0001}{January 1901}

% ---------------------------------------------------------------------------- %
%
% Start the document
%
\begin{document}
\raggedright

\maketitle

% ------------------------------------------------------------------------ %
% An Abstract is required for SAND reports
%

%
\begin{abstract}
%
Blah blah blah ...
%
\end{abstract}
%

% ------------------------------------------------------------------------ %
% An Acknowledgment section is optional but important, if someone made
% contributions or helped beyond the normal part of a work assignment.
% Use \section* since we don't want it in the table of context
%
\clearpage
\section*{Acknowledgment}
The authors would like to thank ...

The format of this report is based on information found
in~\cite{Sand98-0730}.

% ------------------------------------------------------------------------ %
% The table of contents and list of figures and tables
% Comment out \listoffigures and \listoftables if there are no
% figures or tables. Make sure this starts on an odd numbered page
%
\clearpage
\tableofcontents
\listoffigures
%\listoftables

% ---------------------------------------------------------------------- %
% An optional preface or Foreword
%\clearpage
%\section{Preface}
%Although muggles usually have only limited experience with
%magic, and many even dispute its existence, it is worthwhile
%to be open minded and explore the possibilities.

% ---------------------------------------------------------------------- %
% An optional executive summary
%\clearpage
%\section{Summary}
%Once a certain level of mistrust and scepticism has
%been overcome, magic finds many uses in todays science
%and engineering. In this report we explain some of the
%fundamental spells and instruments of magic and wizardry. We
%then conclude with a few examples on how they can be used
%in daily activities at national Laboratories.

% ---------------------------------------------------------------------- %
% An optional glossary. We don't want it to be numbered
%\clearpage
%\section*{Nomenclature}
%\addcontentsline{toc}{section}{Nomenclature}
%\begin{itemize}
%\item[alohomora]
%spell to open locked doors and containers
%\end{itemize}

% ---------------------------------------------------------------------- %
% This is where the body of the report begins; usually with an Introduction
%
\setcounter{secnumdepth}{3}
\SANDmain % Start the main part of the report

\section{Introduction}

Here we design and describe a set of C++ interfaces and concrete
implementations for the solution of a broad class of transient ordinary
differential equations (ODEs) and differential algebraic equations (DAEs) in a
consistent manner.

\section{Requirements}

Here we describe requirements in two general categories: strong and weak
(i.e.\ wants).

\subsection{Strong requirements}

Here we list requirements that are strongly needed or desired in the time
integrator software.

\begin{enumerate}

{}\item Provide a single interface that applications can implement that allows
for the development of many different types of popular explicit and implicit
initial-value ODE and DAE solvers.

{}\item Provide a single interface where an application can seamlessly select
from one of a number of available time integration methods.

{}\item Exploit as much as possible a common set of software interfaces and
utilities to combine support for ODEs and DAEs together.  Minimize specialized
code and algorithms that apply only to special classes of ODEs and DAEs.

{}\item Provide an automated support framework for direct and adjoint
sensitivity methods for ODEs and DAEs.

{}\item Present an DAE/ODE interface for applications to implement that
upwardly supports the computation of (discrete) direct and adjoint
sensitivities and other types of information.

{}\item For direct sensitivity computations for spatially discretized PDEs,
allow the use of a continuous sensitivity equation instead of a discrete
approximation if desired.

{}\item For adjoint sensitivity computations for spatially discretized PDEs,
allow the use of a continuous adjoint instead of the discrete adjoint if
desired.

{}\item Provide support for general checkpointing of a time integration method
and a uniform way to deal with different time integration methods.

{}\item Require checkpointing of data to result in being able to go back and
continue a computation exactly as if the method would have continued on as
usual.

{}\item Use the basic checkpointing capability to implement checkpointing
methods for nonlinear adjoint sensitivity computations in a way that is
independent of specific applications.

{}\item Allow applications detailed control over the time stepping algorithms
to support breakpoints (e.g.\ for discontinuities) and pauses (e.g.\ for code coupling). 

{}\item Allow basic time stepping algorithm building blocks to be composed
together to build various types of operator splitting methods for complex
physics and multi-physics coupling.

{}\item Provide numerical algorithms to solve non-stiff and stiff ODEs (and
implicit ODEs)

{}\item Provide numerical algorithms to solve fully-implicit and semi-explicit
index-1 DAEs

{}\item Support both application provided and general methods for
preconditioning.

{}\item Provide concrete implementations for the following transient integration methods:
  \begin{enumerate}
  {}\item implicit/explicit BDF (multi-step) \\
        including variable-order variable-step as per CVODE/IDA \cite{CVODE,IDA}.
  {}\item generalized-alpha \cite{GeneralizedAlpha}.
  {}\item explicit Runge Kutta (one-step) \cite{ERKMethods}.
  {}\item Trapezoid \cite{Trapezoid}.
  {}\item Pseudo-Transient Continuation \cite{PTC,PTCDAE}.
  \end{enumerate}

{}\item Operate efficiently on massively parallel distributed memory
computers such as ``Red Storm".

{}\item Document transient integration algorithms with references.

{}\item Document and allow access to internal algorithm coefficients
(so-called ``magic-numbers").

{}\item Provide a well-defined and well-documented way to add and modify
existing transient integration algorithms without requiring modification of
existing source code (this is critical for maintainability).

{}\item Provide an interface to estimating (local/global)
accuracy/error in the solution.

{}\item Allow users to give an upper bound for (local/global) accuracy/error
in the solution that time integrators must satisfy.  This is needed for
certain types of error-controlling optimization algorithms.

{}\item Be compatible with automatic differentiation (AD).

{}\item ???

\end{enumerate}

\subsection{Weak requirements (wants)}

Here we list requirements that have a much lower priority and may be
considered optional.

\begin{enumerate}

{}\item Support the following transient integration methods:
  \begin{enumerate}
  {}\item implicit Runge Kutta (one-step) \cite{IRKMethods}.
  {}\item general predictor-corrector methods.
  {}\item discontinuous-Galerkin in time \cite{DGTime}.
  {}\item finite-element in time \cite{FETime}
  \end{enumerate}

{}\item Be compatible with space/time discretizations.

{}\item As much as possible, build step-size and order adaptivity strategies
independent from specific time integration methods.

{}\item Allow for use of CVODE, CVODES, IDA and IDAS at some level [???].

\end{enumerate}

\section{Mathematical formulation of DAE/ODE for DAE/ODE solvers}
\label{rythmos:scn:mathformulation}

Here we describe the basic mathematical form of a general nonlinear DAE (or
ODE) for the purpose of presenting it to a forward time integrator.  At the
most general level of abstraction, we will consider the solution of fully
implicit DAEs of the form
%
\begin{eqnarray}
f(\dot{x},x,t) & = & 0, \; \mbox{for} \; t \in [t_0, t_f], \label{rythmos:eqn:dae} \\
x(t_0) & = & x_0 \label{rythmos:eqn:dae:ic}
\end{eqnarray}
\begin{tabbing}
\hspace{4ex}where\hspace{5ex}\= \\
\>	$x\:\in\:\mathcal{X}$ is the vector of differential state variables, \\
\>	$\dot{x} = d(x)/d(t)\:\in\:\mathcal{X}$ is the vector of temporal derivatives of $x$, \\
\>	$t, t_0, t_f\:\in\:\RE$ are the current, initial, and the final times respectively, \\
\>	$f(\dot{x},x,t)\:\in\:\mathcal{X}^2\times\RE\rightarrow\mathcal{F}$ defines the DAE vector function, \\
\>	$\mathcal{X} \:\subseteq\:\RE^{n_x}$ is the vector space of the state variables $x$, and \\
\>	$\mathcal{F} \:\subseteq\:\RE^{n_x}$ is the vector space of the output of the DAE function $f(\ldots)$.
\end{tabbing}

Here we have been careful to define the Hilbert vector spaces $\mathcal{X}$
and $\mathcal{F}$ for the involved vectors and vector functions.  The
relevance of defining these vector spaces is that they come equipped with a
definition of the space's scalar product (i.e.\ $<u,v>_{\mathcal{X}}$ for
$u,v\in\mathcal{X}$) which should be considered and used when designing the
numerical algorithms.

The above general DAE can be specialized to more specific types of problems
based on the nature of Jacobian matrices ${}\partial f / {}\partial
{}\dot{x}\in\mathcal{F}|\mathcal{X}$ and ${}\partial f / {}\partial
{}\dot{x}\in\mathcal{F}|\mathcal{X}$.  Here we use the notation
$\mathcal{F}|\mathcal{X}$ to define a linear operator space that maps vectors
from the space $\mathcal{X}$ to the space $\mathcal{F}$.  Note that the
adjoint of such linear operators are defined in terms of these vector spaces
(i.e.\ $<A u,v>_{\mathcal{F}} = <A^H v,u>_{\mathcal{X}}$ where
$A\in\mathcal{F}|\mathcal{X}$ and $A^H$ denotes the adjoint operator for $A$).
Here we assume that these first derivatives exist for the specific intervals
of  $t\in[t_0,t_f]$ of which such an time integrator algorithm will be
directly applied the the problem.

The precise class of the problem is primarily determined by the nature of the
Jacobian ${}\partial f / {}\partial {}\dot{x}$:
%
\begin{itemize}
%
{}\item ${}\partial f / {}\partial {}\dot{x} = I {}\in {}\mathcal{F}|\mathcal{X}$ yields an explicit ODE
%
{}\item ${}\partial f / {}\partial {}\dot{x}$ full rank yields an implicit ODE
%
{}\item ${}\partial f / {}\partial {}\dot{x}$ rank deficient yields a general
DAE.
\end{itemize}
%
In addition, the ODE/DAE may be linear or nonlinear and DAEs are classified by
their index \cite{BCP}.  It is expected that a DAE will be able to tell a 
integrator what type of problem it is (i.e.\ explicit ODE, implicit ODE,
general DAE) and which, if any, of the variables are linear in the problem.
This type of information can be exploited in a time integration algorithm.

{}\textbf{Ross: We need notation for taking advantage of which ``variables are
linear in the problem."}

Another formulation we will consider is the semi-explicit DAE formulation:
\begin{equation}
\label{rythmos:eqn:dae:semiexplicit} 
\begin{array}{rcl}
\dot{y} & = & f(y,z,t) \; \mbox{for} \; t \in [t_0, t_f], \\
0       & = & g(y,z,t) \; \mbox{where} \; x = [y, z], \\
x(t_0)  & = & x_0.
\end{array}
\end{equation}

For each transient computation, the formulation will be cast into the general
form in (\ref{rythmos:eqn:dae})--(\ref{rythmos:eqn:dae:ic}) to demonstrate how
this general formulation can be exploited in many different contexts.

\section{Mathematics of Basic Time Integration Strategies}

In this section we show how time steps are formulated and computed for a
variety of popular explicit and implicit time integration methods.  In general,
all of these time integration methods are based on some basic smoothness
assumptions of the underlying DAE/ODE but the details of the theoretical
background for these methods will not be covered here.

We first consider explicit methods followed by implicit methods.

\subsection{Formulation for Explicit Time Steppers for ODEs}

Explicit integration methods are primarily only attractive for non-stiff explicit and
implicit ODEs but some classes of DAEs can be considered as well (i.e.\ by
nonlinearly eliminating the algebraic variables from the semi-explicit DAE formulation 
(\ref{rythmos:eqn:dae:semiexplicit}) \cite{BCP}).  For this discussion, we will also assume
that the DAE has been written in the explicit ODE form (i.e.\ ${}\partial f /
{}\partial {}\dot{x} = I$).  Note that implicit ODEs can always be written as
explicit ODEs by multiplying the implicit ODE from the left with $({}\partial
f / {}\partial {}\dot{x})^{-1}$ as:
%
\begin{eqnarray*}
f(\dot{x},x,t) & = & 0 \\
& \Rightarrow \\
\frac{\partial f}{\partial \dot{x}} \dot{x} + \hat{f}(x,t) & = & 0 \\
& \Rightarrow \\
\left( \frac{\partial f}{\partial \dot{x}} \right)^{-1}
\left( \frac{\partial f}{\partial \dot{x}} \dot{x} + \hat{f}(x,t) \right) & = & 0 \\
& \Rightarrow \\
\dot{x} & = & -\left( \frac{\partial f}{\partial \dot{x}} \right)^{-1} \hat{f}(x,t) \\
& = & \bar{f}(x,t) \\
\end{eqnarray*}
%
where ${}\dot{x} = \bar{f}(x,t)$ is the new explicit form of the ODE that is
considered by the explicit time integration strategies below.

The above transformation of course requires that the matrix $({}\partial f /
{}\partial {}\dot{x})^{-1}$ be fairly easily invertible.

{}\textbf{Ross: See if redefining the scalar product in terms of the mass
matrix allow us to avoid an explicit solve with the mass matrix to form an
explicit ODE.}

Below, we describe some popular explicit integration methods for ODEs.

\subsubsection{Explicit Runge-Kutta methods}

\subsubsection{Explicit multi-step methods}

\subsubsection{Explicit ??? methods}

\subsection{Formulation for Implicit Time Steppers for ODEs and DAEs}

Here we consider several different classes of implicit time stepping methods.
For each class of method we show the set of general nonlinear equations that
defines a single time step and then show how a linearized form of the
equations may be formed to be solved by a Newton-type nonlinear equation
solver.

In particular, for each method, we will show how to define a set of nonlinear
equations of the form
%
\begin{equation}
r(z) = 0
\label{rythmos:eqn:r}
\end{equation}
%
such that when solved will define an implicit time step from $t_k$ to
$t_{k+1}$, where $\Delta t = t_{k+1} - t_k$ denotes the time-step.  In
addition, for each method, we will show how to use the DAE residual evaluation
$(\dot{x},x,t) {}\rightarrow f$ to define the nonlinear time step equation
(\ref{rythmos:eqn:r}).  At the highest level, the time step method only
requires very general convergence criteria for the time step equation
(\ref{rythmos:eqn:r}) and therefore great flexibility is allowed in how the
time step equation is solved.  In general, the system in (\ref{rythmos:eqn:r})
must be solved such that $||x_{k+1} - x^*(t_{k+1})|| < \eta$, where
$x_{k+1}\in\mathcal{X}$ is the computed step for the state,
$x^*(t_{k+1})\in\mathcal{X}$ is the exact state solution at $t_{k+1}$, and
$\eta$ is the maximum allowable local truncation error defined by the user.

Even through the time step equation can be solved by a variety of means, a
large class of DAEs/ODEs can also potentially provide support for a general
Newton-type method for solving these equations and can therefore leverage
general software for solving such problems (e.g.\ NOX).  The foundation of
Newton-like methods is the ability to solve linear systems similar to the
Newton system
%
\begin{equation}
\frac{\partial r}{\partial z} \Delta z = - r(z_l)
\end{equation}
%
where $z_l$ is the current candidate solution of the nonlinear equations
(which also defines the point where $\partial r / {}\partial z$ is evaluated)
and $\Delta z = r_{l+1} - r_l$ is the update direction.  Line-search Newton
methods then define an update to the solution along the direction $\Delta z$.
The essential functionality needed to perform a Newton-like method are the the
abilities to evaluate the nonlinear residual $z {}\rightarrow r$ and to
(approximately) solve linear systems involving the Jacobian matrix ${}\partial
r / {}\partial z$.  For each type of implicit time integration method, we will
show, if possible, how to perform solves with ${}\partial r / {}\partial z$ by
performing solves with the matrix
%
\begin{equation}
M = \alpha \frac{\partial f}{\partial \dot{x}} + \beta \frac{\partial f}{\partial x},
\label{rythmos:eqn:M}
\end{equation}
%
evaluated at points $(\dot{x},x,t)$ selected by the time integration method
and where $\alpha\in\RE$ and $\beta\in\RE$ is some constants also defined by
the time integration method.  Note that the matrix $M$ above in
(\ref{rythmos:eqn:M_bdf}) may not necessarily exactly represent ${}\partial r
/ {}\partial z$ and $z$ and $r$ may not simply lie in the vector spaces
$\mathcal{X}$ and $\mathcal{F}$ respectively; but in many cases, they will.

\subsubsection{Implicit multi-step BDF methods}

The first class of methods that we will consider are so called multi-step
backward difference formula (BDF) methods.  In these methods, an $p$-step
approximation for the time derivative $\dot{x}$ for the new time point
$t_{k+1}$ is formed based on a linear combination of state values $y_{k+1-j}$
for next, current and previous time values $t_{k+1-j}$, for $j = 0 {}\ldots p$
and the form of the approximation is
%
\begin{equation}
\dot{x}_{k+1} = \frac{1}{\Delta t} \sum_{j=0}^{p} \gamma_j \: x_{k+1-j}
\label{rythmos:eqn:bdf_x_dot}
\end{equation}
%
where $\Delta t = t_{k+1} - t_k$ and $\gamma_j$, for $j=0 {}\ldots p$, are the
BDF method coefficients \cite{AscherPetzold}.

The nonlinear time step equation to advance the solution from $t_k$ to
$t_{k+1}$ is then formed by substituting $\dot{x} = \dot{x}_{k+1}$ in
(\ref{rythmos:eqn:bdf_x_dot}), $x = x_{k+1}$ and $t = t_{k+1}$ into
(\ref{rythmos:eqn:dae}) to obtain
%
\begin{equation}
f\left( \frac{1}{\Delta t} \left[ \gamma_0 x_{k+1} + \sum_{j=1}^{p} \gamma_j \: x_{k+1-j} \right],x_{k+1},t_{k+1}\right) = 0.
\label{rythmos:eqn:bdf_dae_ne}
\end{equation}
%
One can immediately identify the BDF time step equations
(\ref{rythmos:eqn:bdf_dae_ne}) with the general form of the time step
equations (\ref{rythmos:eqn:r}) and with unknown solution variables $z =
x_{k+1}$.  All of the other state variables $x_{k+1-j}$, for $j = 1 {}\ldots
p$, are given.

Note that the first-order BDF method with $p=1$, $\gamma_0 = 1$ and $\gamma_1 =
-1$ is simply the standard backward Euler time integration method \cite{AscherPetzold}.

When considering a general Newton-like method for solving
(\ref{rythmos:eqn:bdf_dae_ne}), note that the Newton Jacobian of these
equations is
%
\begin{equation}
\frac{\partial r}{\partial z}
= \frac{\gamma_0}{\Delta t} \frac{\partial f}{\partial \dot{x}} + \frac{\partial f}{\partial x},
\label{rythmos:eqn:M_bdf}
\end{equation}
%
which is evaluated at the point $\dot{x}$ in (\ref{rythmos:eqn:bdf_x_dot}), $x
= x_{k+1}$ and $t = t_{k+1}$.  One can immediately identify
(\ref{rythmos:eqn:M_bdf}) with the general form of the matrix $M$ in
(\ref{rythmos:eqn:M}) where $\alpha = {}\gamma_0 / \Delta t$ and $\beta = 1$.
Note that the Jacobian (\ref{rythmos:eqn:M_bdf}) is the exact Jacobian for the
nonlinear time step equations; this will not be true for some of the other
methods.

\subsubsection{Generalized Theta methods}

The next fairly straightforward class of methods are the so called $\theta$
(Theta) methods \cite{HairerWanner}.  Theta methods define a time step equation by requiring
enforcement of the DAE equation at an intermediate time in the range
$t_{\theta} {}\in [t_k,t_{k+1}]$ where $t_{\theta} = t_k + \theta ( t_{k+1} -
t_k )$ and $\theta {}\in [0,1]$.  Then, Theta-averaged values for $x$ and $t$
are then used to form the implicit time step equation
%
\begin{equation}
f\left( \frac{x_{k+1} - x_{k}}{\Delta t},x_k + \theta ( x_{k+1} - x_{k} ), t_k + \theta \Delta t \right) = 0
\label{rythmos:eqn:theta_dae_ne}
\end{equation}
%
which is then solved for the unknown state $x_{k+1}$.  The values of $\theta =
0$, $\myonehalf$ and $1$ give the well known forward Euler, midpoint
rule, and backward Euler methods \cite{HairerWanner}.

{}\textbf{ToDo: Todd, is this correct?  I could not find a specific reference
for a general DAE but this seems not to be consistent with the midpoint rule
for ODEs and described in [Burden].}

\subsubsection{Trapezoidal Methods}

The next class of methods that we consider are Trapezoidal methods.
Trapezoidal methods are essentially equivalent to the trapezoidal rule for
numerical quadrature (i.e.\ integrating under a curve) and takes the form of
the averaging of the DAE at $t_k$ and $t_{k+1}$ as
%
\begin{equation}
\frac{1}{2} \left[
f\left( \frac{x_{k+1} - x_{k}}{\Delta t}, x_{k+1} , t_{k+1} \right)
+ f\left( \frac{x_{k+1} - x_{k}}{\Delta t}, x_k , t_k \right)
\right]
 = 0
\label{rythmos:eqn:trap_dae_ne}
\end{equation}
%
which defines a set of nonlinear equations that are solved for the unknown
state variables $x_{k+1}$.  The straightforward Newton Jacobian for the
equations in (\ref{rythmos:eqn:trap_dae_ne}) is
%
\begin{equation}
\frac{\partial r}{\partial z}
= \frac{1}{\Delta t} \left[
\left( \frac{\partial f}{\partial \dot{x}} \right)_{k+1}
+ \left( \frac{\partial f}{\partial \dot{x}} \right)_{k}
\right]
+ \left( \frac{\partial f}{\partial x} \right)_{k+1},
\label{rythmos:eqn:drdz_trap}
\end{equation}
%
where we use the shorthand notation of $(\ldots)_{k+1}$ and $(\ldots)_k$ to
mean that an object is evaluated at the points $(\dot{x},x,t) =
((x_{k+1}-x_k)/\Delta t, x_{k+1}, t_{k+1})$ and $(\dot{x},x,t) =
((x_{k+1}-x_k)/\Delta t, x_k, t_k)$ respectively.  Note that the form of the
Jacobian in (\ref{rythmos:eqn:drdz_trap}) does not match the form of
(\ref{rythmos:eqn:M}) that we are trying to utilize.  However, a common trick
in these types of methods to simply replace $x_{k+1}$ and $t_{k+1}$ in the
second and third arguments of $((x_{k+1}-x_k)/\Delta t, x_{k+1}, t_{k+1})$
with $x_k$ and $t_k$ in the Jacobian which gives the approximate Jacobian
%
\begin{equation}
M
= \frac{2}{\Delta t} \frac{\partial f}{\partial \dot{x}}
+ \frac{\partial f}{\partial x},
\label{rythmos:eqn:trap_M}
\end{equation}
%
where now all of these terms are evaluated at the point $(\dot{x},x,t) =
((x_{k+1}-x_k)/\Delta t, x_k, t_k)$.  It is easy to see that this is the same
form as the Jacobian in (\ref{rythmos:eqn:M}) where $\alpha = 2/\Delta t$ and
and $\beta = 1$.  The Jacobian in (\ref{rythmos:eqn:trap_M}) is not an exact
Jacobian for the equations in (\ref{rythmos:eqn:trap_dae_ne}) but can produce
good Newton steps in many cases.  However, an inexact Jacobian such as this
can cause a problem when sensitivity computations are being performed (see
Section
\ref{rythmos:scn:transientsensitivity}).

\subsubsection{Generalized Alpha methods}

\cite{GeneralizedAlpha}
{}\textbf{Todd: Can you fill this in?}

\subsubsection{Implicit Runge-Kutta methods}

We now consider a class of powerful and popular one-step methods for solving
implicit DAEs, implicit Runge-Kutta (RK) methods.  The most general form
of implicit RK methods requires the simultaneous solution of $p$ sets
of coupled nonlinear equations that take the form
%
\begin{equation}
r_i(z) = f\left( \dot{x}_i, x_k + \Delta t \sum_{j=1}^{p} a_{ij} \dot{x}_j,
t_k + c_i \Delta t \right) = 0,
\; \mbox{for} \; i = 1 \ldots p
\label{rythmos:eqn:irk_dae_ne}
\end{equation}
%
where $\dot{x}_i$ are essentially approximations to the derivatives
$\dot{x}(t_k + c_i \Delta t)$ called {}\textit{stage derivatives} and $z = [
{}\dot{x}_1, {}\dot{x}_1, {}\ldots, {}\dot{x}_p ]^T$ are the unknowns in this
set of equations.  After this set of coupled equations is solved, the state
solution $x_{k+1}$ is given as the linear combination
%
\begin{equation}
x_{k+1} = x_k + \Delta t \sum_{i=1}^{p} b_i \dot{x}_j.
\end{equation}

Implicit RK methods are classified by both the order $p$ and the selection of
the constants $a_{ij}$, $c_i$ and $b_i$.  It is customary to consider these
constants in the form of the {}\textit{Butcher diagram}
%
\[
\begin{array}{c|c}
c & A \\
\hline
  & b^T
\end{array}
\; \; = \; \;
\begin{array}{c|cccc}
c_1 & a_{11} & a_{12} & \cdots & a_{1p} \\
c_2 & a_{21} & a_{22} & \cdots & a_{2p} \\
\vdots & \vdots & \vdots & \ddots & \vdots \\
c_p & a_{p1} & a_{p2} & \cdots & a_{pp} \\
\hline
 & b_1 & b_2 & \cdots & b_p
\end{array}
\]
%
where $A = [a_{ij}]$, $b = [b_1, b_2, {}\ldots, b_p]^T$, and $c =
[c_1, c_2, {}\ldots, c_p]^T$.

It is clear how to form the residual for the fully coupled system for $r(z)$ in
(\ref{rythmos:eqn:irk_dae_ne}) just from individual evaluations $(\dot{x},x,t)
{}\rightarrow f$ but how the Newton system for such a system is solved will
vary greatly based on the structure and properties of the Butcher matrix $A$.

\subsubsection*{Fully implicit IRK methods}

Fully implicit IRK methods present somewhat of a problem for developing
general software since they involve the need to solve a fully coupled system
of $p$ sets of equations of the form (\ref{rythmos:eqn:irk_dae_ne}).  Each
block ${}\partial r_i / {}\partial z_j = {}\partial r_i / {}\partial
{}\dot{x}_j$ of the full Jacobian ${}\partial r / {}\partial z$ is represented
as
%
\begin{equation}
\frac{\partial r_i}{\partial \dot{x}_j}
= \frac{\partial f}{\partial \dot{x}}
+ \Delta t a_{ij} \frac{\partial f}{\partial x},
\; \mbox{for} \; i = 1 \ldots p, \; j = 1 \ldots p
\label{rythmos:eqn:dridzi_trap}
\end{equation}
%
which is evaluated at the points $(\dot{x},x,t) = ( {}\dot{x}_i, x_k +
{}\Delta t {}\sum_{j=1}^{p} a_{ij} {}\dot{x}_j, t_k + c_i {}\Delta t )$.  Note
that ${}\partial r_i / {}\partial z_j = {}\partial r_i / {}\partial
{}\dot{x}_j$ in (\ref{rythmos:eqn:dridzi_trap}) is of the same form as
(\ref{rythmos:eqn:M}) where $\alpha = 1$ and $\beta = \Delta t a_{ij}$.

When considering a iterative method for solving systems with the block
operator matrix ${}\partial r / {}\partial z$, it is easy to see how to use
the general matrix $M$ in (\ref{rythmos:eqn:M}) to implement a matrix-vector
product, but it is not obvious how to precondition such a system.  Clearly a
block diagonal preconditioner could be used but the effectiveness of such a
preconditioning strategy is open to question.  Other preconditioning strategies
are also possible just given the basic block operators and this is an open
area of research.

In some cases, however, it may be possible to form a full matrix object for
${}\partial r / {}\partial z$ but this is not something that can be expected
for most applications.

\subsubsection*{Semi-explicit IRK methods}

Semi-explicit IRK methods are those IRK methods where the Butcher matrix $A$
is lower diagonal and therefore gives rise to a block lower triangular
Jacobian matrix ${}\partial r / {}\partial z$.  For these types of methods,
the nonlinear equations in (\ref{rythmos:eqn:dridzi_trap}) can be solved one
at a time for $i = 1 {}\ldots p$ which is easily accommodated using a
Newton-type method where the Newton Jacobian for each $i$ is given by
(\ref{rythmos:eqn:dridzi_trap}), which is of our basic general form
(\ref{rythmos:eqn:M}).

\subsubsection*{Diagonally-implicit IRK methods}

The next specialized class of IRK methods that we consider are
diagonally-implicit IRK methods where the Butcher coefficients in $A$ and $c$
give rise to a lower triangular Jacobian ${}\partial r / {}\partial z$ (and
hence are also semi-explicit IRK methods) that has the same nonsingular matrix
block of the form in (\ref{rythmos:eqn:dridzi_trap}) along the diagonal.
This, of course, requires that $a_{11} = a_{22} = {}\ldots = a_{pp}$ and
$c_{1} = c_{2} = {}\ldots = c_{p}$.  In this class of IRK methods, significant
savings may be achieved since a single set of linear solver objects (i.e.\
operators and preconditioners) can be utilized for the solution of the fully
coupled system.  In fact, it may even be possible to utilize multi-vector
applications of the operator and preconditioner for matrices of the form
(\ref{rythmos:eqn:M}) which can be supported by many applications.

{}\textbf{ToDo: I need to write up some more ideas here to see how to take
advantage of multi-vector operations when possible and if it makes since to do
so.}

\subsubsection{Discontinuous Galerkin in time}

\cite{DGTime}
{}\textbf{Todd: Scott Collis mentioned this a something that can be important
from a variety of standpoints but I don't have any references for this yet.}

\subsubsection{Finite-elements in time }

\cite{FETime}
{}\textbf{Todd: Scott Collis mentioned this a something that can be important
from a variety of standpoints but I don't have any references for this yet.}

\subsubsection{Implicit ??? methods}

{}\textbf{Todd: Are there any other implicit methods here?}

%\subsection{General Solution Strategies for Implicit Nonlinear Time Step Equations}

\section{Staggered DAEs/ODEs}
\label{rythmos:sec:staggered-daes}

A typical situation that occurs is when a set of two or more DAEs/ODEs are
solved in which the solutions are propagated from one to the other.  Here we
define this type of DAE and show how these can be effectively dealt with.  In
this section we focus on just two staggered DAEs/ODEs for simplicity but the
same principles apply to more than two staggered DAEs/ODEs.

The model that we will use here is of two DAEs/ODEs where the first DAE
%
\begin{eqnarray}
%
c_1(\dot{y}_1(t),y_1(t),t) & = & 0,
\; t \in \left[ t_0, t_f \right], \label{rythmos:eqn:stag:c1} \\
y_1(0) & = & y_{0,1}, \label{rythmos:eqn:stag:c1:ic} \\
%
\end{eqnarray}
%
is solved and who's solution $y_1(t)$is used to define a second set of DAEs of
the form
%
\begin{eqnarray}
%
c_2(\dot{y}_2(t),y_2(t),y_1(t),t) & = & 0,
\; t \in \left[ t_0, t_f \right], \label{rythmos:eqn:stag:c2} \\
y_2(0) & = & y_{0,2}(y_1(t_0)), \label{rythmos:eqn:stag:c2:ic}
%
\end{eqnarray}
%
which are solved for $y_2(t)$, where $y_1\in\mathcal{Y}_1$,
$\dot{y}_1\in\mathcal{Y}_1$, $y_2\in\mathcal{Y}_2$,
$\dot{y}_2\in\mathcal{Y}_2$, $c_1(\dot{y}_1(t),y_1(t),t) {}\in
{}\mathcal{Y}_1^2 {}\times {}\RE {}\rightarrow {}\mathcal{C}_1$,
$c_2(\dot{y}_2(t),y_2(t),t) {}\in {}\mathcal{Y}_2^2 {}\times {}\mathcal{Y}_1
{}\times {}\RE {}\rightarrow {}\mathcal{C}_1$, $y_{0,1}\in\mathcal{Y}_1$, and
$y_{0,2}(y_1(t)) {}\in {}\mathcal{Y}_1 {}\rightarrow {}\mathcal{Y}_2$.

These types of two-level staggered DAEs comes up over and over again in the
following discussion.  Staggered DAEs are used extensively in sensitivity
computations and show up in other contexts as well such as multi-physics
solutions.  Note that staggered DAEs, as shown above, are much simpler to
consider than fully coupled sets of DAEs (i.e.\ where each DAE is dependent on
the state of the other).  Issues related to fully coupled DAEs are a whole
other matter entirely.

The staggered DAEs shown above can be solved in a number of ways.  The first
way to solve these staggered DAEs is to integrate them more-or-less together
but to solve for the time-step for
(\ref{rythmos:eqn:stag:c1})--(\ref{rythmos:eqn:stag:c1:ic}) first and then
solve for the time-step for
(\ref{rythmos:eqn:stag:c2})--(\ref{rythmos:eqn:stag:c2:ic}) second.  This
approach requires minimally that the solution for $y_1(t)$ be communicated in
some way to the second DAE
(\ref{rythmos:eqn:stag:c2})--(\ref{rythmos:eqn:stag:c2:ic}).  Just this
requirement in itself has a number of difficult issues associated with it.  We
discuss these and other issues in more detail latter, after we discuss the
second approach.

The second approach for solving the two staggered DAEs is to combine them into
one implicit DAE of the form in (???) where
%
\begin{eqnarray*}
x(t) & = & {\bmat{c} y_1(t) \\ y_2(t) \emat} \\
f(\dot{x}(t),x(t),t) & = & {\bmat{c} c_1(\dot{y}_1(t),y_1(t),t) \\  c_2(\dot{y}_2(t),y_2(t),y_1(t),t) \emat} \\
x_0 & = & {\bmat{c} y_{0,1} \\ y_{0,2}(y_{0,1}) \emat}
\end{eqnarray*}
%
where $\mathcal{X}=\mathcal{Y}_1\times\mathcal{Y}_2$ is a product vector
space, and then solve them with a single time stepper algorithm.  Many of the
implicit time stepper methods described in Section ??? would then need to
solve the implicit nonlinear system $r(z)=0$ that arises and a Newton-like
method would require solves with the composite matrix
%
\begin{equation}
\alpha \frac{\partial f}{\partial \dot{x}} + \beta \frac{\partial f}{\partial x}
=
{\bmat{cc}
\alpha \frac{\partial c_1}{\partial \dot{y}_1} + \beta \frac{\partial c_1}{\partial x_1}
& 0 \\
\beta \frac{\partial c_2}{\partial x_1}
& \alpha \frac{\partial c_2}{\partial \dot{y}_2} + \beta \frac{\partial c_2}{\partial x_2} \\
\emat}.
\end{equation}
%
This block matrix is lower block triangular with nonsingular blocks along the
diagonal which are the customary matrices for each individual DAE.  The first
potential problem with the above block matrix is the lower left block $\beta
({}\partial c_2 / {}\partial x_1)$ which is not required by any of the
individual DAE solvers (but is required by a direct sensitivity solver, see
Section ???).  Therefore, an exact Newton method requires a type of derivative
computation that is not required when the DAEs are solved individually.
However, in an iterative method, only the action of $({}\partial c_2 /
{}\partial x_1) V$ are to arbitrary multivectors $V\in\mathcal{C}_2|\RE^m$ are
needed which can be approximated using directional finite differences.

There are many different issues to consider when comparing and contrasting
these two basic approaches for solving staggered DAEs:
%
\begin{itemize}
%
{}\item {}\textbf{Communication of state information}: The one big advantage
of the combined DAE approach using a single time stepper algorithm is that the
time stepper itself can easily and transparently handle the communication of
$y_1$ from the first DAE to the second DAE.  For example, consider a
higher-level RK method where stage derivatives must be computed a Gauss
points.  In a single time stepper method, this is a simple matter, but is more
difficult to handle when using two different time stepping algorithms (which
may have some data encapsulation).  This issue can be resolved if a general
interpolation mechanism where exposed by a time stepper where $y_1(t)$ at any
time in the time-step interval where easily accessible.
%
{}\item {}\textbf{Step-size and error control}: When the combined DAE is
integrated by a single time stepping method, step-size and error control is
automatically handled for both of the DAE solutions.  This may or may not be
desirable.  However, when the two DAEs are integrated separately using two
different time stepping methods, these two methods may not make the same
decisions as to the step sizes to take and this can cause a problem for a
variety of standpoints.  However, if the first time stepper where driven by
the interpolation interface for extracting $y_1(t)$ as called by the second
time integrator, then steps-size control would not be an issue.  In addition,
handling the errors separately may be an advantage since the these two DAEs
may have different solution error requirements.

Note: We need to look at the theory for the propagation of errors in a
staggered time stepper in order to be faithful about satisfying user-defined
error tolerances.  For example, if the first DAE is solved inexactly, then
this error will be propagated to the second DAE and will be compounded by the
error in solving the second DAE.  Therefore, one can not simply look at just
the errors in the two DAE separately if one wants to insure an accurate error
bound on the second DAE.
%
{}\item {}\textbf{Linearity, nonlinearity and relative cost}: One of the
biggest advantages of solving the staggered DAEs using separate time stepping
algorithms (or at least using a specialized nonlinear solver within a time
step) is that one of the DAEs may be linear while the other nonlinear and one
DAE may be much more expensive than the other to solve.  For example, if the
first DAE is nonlinear and small, but the second DAE is linear but much larger
and more expensive to solve than the first, then a combined Newton approach
will result in much more expense than if the two DAE time steps are computed
independently.  However, if the first DAE is inexpensive and well-conditioned
compared to the second DAE, then there is little to no extra expense in
solving the DAEs as one large combined DAE.
%
{}\item {}\textbf{State derivatives}: As mentioned above, the combined DAE
approach using a single Newton method requires derivative computations
involving $({}\partial c_2 / {}\partial x_1)$ which are not required when
solving the DAEs separately.
%
\end{itemize}
%

\section{Transient Sensitivity Computations}
\label{rythmos:sec:trans-sens}

In this section we describe the nuts and bolts associated with direct and
adjoint sensitivity methods that are derived in
Appendix~\ref{rythmos:app:sens-derivations}.

The general DAE model for sensitivity computations that we consider here takes
the form
%
\begin{eqnarray}
c\left( \dot{y}(t), y(t), p, v(t), t \right) & = & 0,
\; t \in \left[ t_0, t_f \right] \label{rythmos:eqn:sens:c} \\
y(0) & = & y_0(p) \label{rythmos:eqn:sens:c:ic}
\end{eqnarray}
\begin{tabbing}
\hspace{4ex}where\hspace{1ex}\= \\
\>	$y(t) \:\in\:\mathcal{Y}$ are the transient state variables defined on $t\in[t_0,t_f]$, \\
\>	$\dot{y}(t) = d(y)/d(t)\:\in\:\mathcal{Y}$ is the vector of temporal derivatives of $y(t)$ defined on $t\in[t_0,t_f]$, \\
\>	$p \:\in\:\mathcal{P}$ are steady-state auxiliary variables, \\
\>	$v(t) \:\in\:\mathcal{V}$ are transient auxiliary variables defined on $t\in[t_0,t_f]$, \\
\>	$c(\dot{y}(t), y(t), p, v(t), t) :
		\mathcal{Y}^2 \times \mathcal{P} \times \mathcal{V} \times \RE
		\rightarrow \mathcal{C}$ are the state DAEs, \\
\>	$y_0(p) \:\in\:\mathcal{P} \rightarrow \mathcal{Y}$ define the initial conditions for $y(t_0)$ as a function of $p$, \\
\>	$\mathcal{Y} \:\subseteq\:\RE^{n_y}$ is the vector space of the state variables $y(t)$, \\
\>	$\mathcal{P} \:\subseteq\:\RE^{n_p}$ is the vector space of the steady-state auxiliary variables $p$, \\
\>	$\mathcal{V} \:\subseteq\:\RE^{n_v}$ is the vector space of the transient auxiliary variables $v(t)$, and \\
\>	$\mathcal{C} \:\subseteq\:\RE^{n_y}$ is the vector space of the output of the DAE function $c(\ldots)$.
\end{tabbing}

The auxiliary variables $p$ and $v(t)$ in (\ref{rythmos:eqn:sens:c}) represent
adjustable variables that can be manipulated for some purpose and, once
selected, determine a fully specified set of DAEs that can be solved forward
in time for the state variables $y(t)$.  The form of the DAE in
(\ref{rythmos:eqn:sens:c}), which has auxiliary variables, arises in a number
of contexts other than the sensitivity computations being considered here.

Note that the input $v(t_0:t_f)$ is really an infinite dimensional
input in general but in practice will be a finite-dimensional
discretizations.

We now define a composite set of auxiliary response functions for which
sensitivities will be computed
%
\begin{equation}
d(y(t_0:t_f),p,v(t_0:t_f))
= \int_{t_0}^{t_f} g(y(t),p,v(t),t) dt + h(y(t_f),p),
\label{rythmos:eqn:sens:d}
\end{equation}
\begin{tabbing}
\hspace{4ex}where\hspace{1ex}\= \\
\>	$g(y(t),p,v(t),t) : \:
		\mathcal{Y} \times \mathcal{P} \times \mathcal{V} \times \RE
		\rightarrow \mathcal{G}$ are distributed response functions defined on $t\in[t_0,t_f]$, \\
\>	$h(y(t_f),p) : \: \mathcal{Y} \times \mathcal{P}
		\rightarrow \mathcal{G}$ are terminal response functions defined only at $t=t_f$, and \\
\>	$\mathcal{G} \:\subseteq\:\RE^{n_g}$ is the vector space of the auxiliary response functions.
\end{tabbing}

Note that the domain space for the overall response
$d(y(t_0:t_f),p,v(t_0:t_f)$ is infinite since $y(t_0:t_f)$ and
$v(t_0:t_f)$ are infinite in the continuous formulation.

The state DAEs in
(\ref{rythmos:eqn:sens:c})--(\ref{rythmos:eqn:sens:c:ic}) may be used
to define the implicit state function $y(p,v(t_0:t),t)$, where
$v(t_0:t)$ is short hand for the selection of $v(t)$ (or a discrete
approximation in a practical implementation) in the range
$t\in[t_0,t]$.  The full implicit state solution for $y(t)$ in
$t\in[t_0,t_f]$ is signified as $y(p,v(t_0:t_f),t_0:t_f)$.

The implicit state solution $y(p,v(t_0:t),t)\in\mathcal{Y}$ substituted into
(\ref{rythmos:eqn:sens:d}) gives the reduced auxiliary response functions
%
\begin{equation}
\hat{d}(p,v(t_0:t_f))
= \int_{t_0}^{t_f} \hat{g}(p,v(t),t) dt + \hat{h}(p),
\label{rythmos:eqn:sens:d_hat}
\end{equation}
\begin{tabbing}
\hspace{4ex}where\hspace{1ex}\= \\
\>	$\hat{g}(p,v(t)) = g(y(p,v(t_0:t),t),p,v(t),t) : \:
		\mathcal{P} \times \mathcal{V} \times \RE	\rightarrow \mathcal{G}$, and \\
\>	$\hat{h}(p) = h(y(p,v(t_0:t),t),p) : \: \mathcal{P}	\rightarrow \mathcal{G}$.
\end{tabbing}

Below, we will consider direct and adjoint approaches to the computation of
the first derivative object
%
\begin{equation}
\frac{\partial \hat{d}}{\partial p} \in \mathcal{G}|\mathcal{P}
\label{rythmos:eqn:d_d_hat_d_p}
\end{equation}
%
and adjoint approaches for the computation of
%
\begin{equation}
\frac{\partial \hat{d}}{\partial v(t)} \in \mathcal{G}|\mathcal{V}, \; t \in [t_i,t_f].
\label{rythmos:eqn:d_d_hat_d_v}
\end{equation}
%
Note that it is generally not computationally tractable to compute a discrete
approximation to the first derivative object (\ref{rythmos:eqn:d_d_hat_d_v})
involving the transient auxiliary variables $v(t)$ because of the large
discretized dimension of $v(t_i:t_f)$.

\subsection{Transient direct sensitivity computations}

The direct sensitivity method for computing the first derivative object
$\partial {}\hat{d} / {}\partial p$, as derived in
Appendix~\ref{rythmos:app:direct-sens-derivation}, involves first solving the
$n_p$ direct sensitivity equations
%
\begin{eqnarray}
%
\frac{\partial c}{\partial \dot{y}} \dot{S} + \frac{\partial c}{\partial y} S + \frac{\partial c}{\partial p}
& = & 0, \; t \in \left[ t_0, t_f \right], \label{rythmos:eqn:sens:direct-c} \\
S(t_0) & = & \frac{\partial y_0}{\partial p}. \label{rythmos:eqn:sens:direct-c:ic}
\end{eqnarray}
%
followed by the computation of the reduced derivatives
%
\begin{equation}
\frac{\partial \hat{d}}{\partial p} 
= \int_{t_0}^{t_f} \left( \frac{\partial g}{\partial y} S +  \frac{\partial g}{\partial p} \right) dt
+ \left. \left(  \frac{\partial h}{\partial y} S + \frac{\partial h}{\partial p} \right) \right|_{t=t_f},
\label{rythmos:eqn:sens:d_d_hat_d_p_2}
\end{equation}
%
\begin{tabbing}
\hspace{4ex}where\hspace{1ex}\= \\
\> $S = \frac{\partial y}{\partial p} {}\in {}\mathcal{Y}|\mathcal{P}$ is the
direct sensitivity of the state at $t\in[t_0,t_f]$, \\
\> $\dot{S} = \left(\frac{\partial}{\partial t} \frac{\partial y(t)}{\partial p}\right)
{}\in {}\mathcal{Y}|\mathcal{P}$ is the time derivative of $S$ at $t\in[t_0,t_f]$, \\
\> $\frac{\partial c}{\partial {}\dot{y}} {}\in {}\mathcal{C}|\mathcal{Y}$ is defined on
$t\in[t_0,t_f]$, \\
\> $\frac{\partial c}{\partial y} {}\in {}\mathcal{C}|\mathcal{Y}$ is defined on $t\in[t_0,t_f]$, \\
\> $\frac{\partial c}{\partial p} {}\in {}\mathcal{C}|\mathcal{P}$ is defined on $t\in[t_0,t_f]$, \\
\> $\frac{\partial y_0}{\partial p} {}\in {}\mathcal{Y}|\mathcal{P}$ is defined only at $t=t_0$, \\
\> $\frac{\partial {}\hat{g}}{\partial p} {}\in {}\mathcal{G}|\mathcal{P}$ is defined on $t\in[t_0,t_f]$, \\
\> $\frac{\partial {}\hat{h}}{\partial p} {}\in {}\mathcal{G}|\mathcal{P}$ is defined only at $t=t_f$, and \\
\> $\frac{\partial {}\hat{d}}{\partial p} {}\in {}\mathcal{G}|\mathcal{P}$ is defined independent of time.
\end{tabbing}

This computation is performed by integrating the state
(\ref{rythmos:eqn:sens:c})--(\ref{rythmos:eqn:sens:c:ic}) together with the
direct sensitivity equations
(\ref{rythmos:eqn:sens:direct-c})--(\ref{rythmos:eqn:sens:direct-c:ic}) and
the integral in (\ref{rythmos:eqn:sens:d_d_hat_d_p_2}) is computed while the
DAEs are being integrated.

There are a variety of approaches for integrating the the state and direct
sensitivity equations (see [???]) but arguably the the more efficient approach
is to solve for the (nonlinear) time step for the state equations
(\ref{rythmos:eqn:sens:c}) first followed by computing the linear time step
for the direct sensitivities (\ref{rythmos:eqn:sens:direct-c}) and the
integral in (\ref{rythmos:eqn:sens:d_d_hat_d_p_2}).

A few important properties of the direct sensitivity equations are worth
noting: they are linear, and the sensitivity DAE for each sensitivity
parameter $p_i$ ($i=1\ldots{}n_p$) are independent.

Note that the evaluation of the residual for the direct sensitivity equations
in (\ref{rythmos:eqn:sens:direct-c}) requires the computation of ?????

Some tips from DASPK 3.0 [???] for solving this problem are:
\begin{itemize}
%
{}\item Solve state equation, applying error control first, to convergence for a
time step for the state.
%
{}\item Next, solve sensitivity equations using a Newton method (even though the problem
is linear) as this gives a more stable solution.
%
{}\item Consider truncation error for each set of parameters independently and take
the max for the overall error control.
%
\end{itemize}

Some tips from CVODES [???] for solving this problem are:
\begin{itemize}
%
{}\item The linear sensitivity systems are also solved using a Newton method,
mainly to allow the use of out-of-date preconditioners.
%
{}\item Step size control may or may not consider the sensitivity parameters.
%
{}\item Error control of the integration of the quadrature variables does
not seem to be taken into account.
%
\end{itemize}

Note that for the purposes of a general time stepper algorithm that accepts
the general DAE formulation $f(\dot{x},x,t)$, it is easy to cast the direct
sensitivity equations in that form by first defining the composite vector
%
\begin{equation}
x =
{\bmat{c}
S_{(:,1)} \\ \vdots \\ S_{(:,n_p)}
\emat}
\in \mathcal{Y}^{n_p}
\end{equation}
%
consisting of the columns $S_{(:,i)}$ of $S$ and then defining a similar
composite DAE function
%
\begin{equation}
f(\dot{x},x,t) =
{\bmat{c}
%
\frac{\partial c}{\partial \dot{y}} \dot{S}_{(:,1)}
+ \frac{\partial c}{\partial y} S_{(:,1)}
+ \frac{\partial c}{\partial p_{(1)}} \\
%
\vdots \\
%
\frac{\partial c}{\partial \dot{y}} \dot{S}_{(:,n_p)}
+ \frac{\partial c}{\partial y} S_{(:,n_p)}
+ \frac{\partial c}{\partial p_{(n_p)}}
%
\emat}
\in \mathcal{Y}^{n_p} \rightarrow \mathcal{C}^{n_p}.
\label{rythmos:eqn:direct-sens-daes}
\end{equation}
%
The matrix $M$ in (\ref{rythmos:eqn:M}) for
(\ref{rythmos:eqn:direct-sens-daes}) is given by
%
\begin{equation}
\alpha \frac{\partial f}{\partial \dot{x}} + \beta \frac{\partial f}{\partial x} = 
{\bmat{cccc}
\alpha \frac{\partial c}{\partial \dot{y}} + \beta \frac{\partial c}{\partial y} \\
& \alpha \frac{\partial c}{\partial \dot{y}} + \beta \frac{\partial c}{\partial y} \\
& & \ddots \\
& & & \alpha \frac{\partial c}{\partial \dot{y}} + \beta \frac{\partial c}{\partial y}
\emat}
in \mathcal{C}^{n_p}|\mathcal{Y}^{n_p}
\label{rythmos:eqn:direct-sens-daes-jac}
\end{equation}
%
which is a block diagonal matrix with the same block along the diagonal and
all quantities are evaluated at the same trial point $(\dot{y},y,u,t)$ as
defined by the time integration algorithm.  Note that in the software
implementation using Thyra, it is easy to cast back and forth between a
multi-vector and vector view of the same data and it is therefore easy to take
advantage of multi-vector computations even through the time integrator has no
idea that multi-vectors are being used under the covers.

Note that with the addition of quadrature variables $w(t)\in\mathcal{G}$ and
the associated quadrature ODEs $\dot{w}(t) = g(y(t),p,v(t),t)$, the
sensitivity problem can be transformed to remove an explicit integration of
the the distributed response function $g(y(t),p,v(t),t)$ from the problem
seen.  This approach is used in CVODES and DASPK 3.0 to some extent and this
has advantages and disadvantages.  This creates essentially another combined
set of staggered DAEs that are solved by a single time stepper algorithm.
This may imply that the error in the quadrature equations are considered when
taking a time step for the state equations.  This may or may not be desirable.

Note that the structure and properties of the direct sensitivity equations in
(\ref{rythmos:eqn:sens:direct-c}) can exploited in a few different ways
as described in the following subsections.

\subsubsection{Block linear solvers}

Different blocks of sensitivities (i.e.\ for $i = i_k {}\ldots i_{k+1}$) can
be solved at one time using block direct or iterative linear solvers.  The use
of such block solvers in CVODES, IDAS or DASPK 3.0 is not possible given the
structure of these codes.  The use of block iterative linear solver (such as
block GMRES in Belos [???])  may speed up the calculation by an order of
magnitude or more compared to solving the linear systems one RHS at a time.

Because of the use of block linear solvers, our direct sensitivity DAE solvers
may achieve as much as an order of magnitude of speedup or more over a solver
such as CVODES that only allows for solving sensitivity systems one RHS at a
time (see the function {}\texttt{cv\_lsolveS(...)} in [???CVODES User's
Guide???]).

\subsubsection{Exploitation of parallelism for blocks for sensitivities}

Since each direct sensitivity equation is independent of the others, they can
be solved in parallel blocks and parallel clusters of processors.  For
example, consider a DAE that comes from a discretized PDE who's state is
distributed over 10 and has $n_p=100$ parameters to compute sensitivities for.
The solution of the direct sensitivity equations can be solved on a single
chuck of 10 processors where all 100 DAEs are solved simultaneously using a
block solver.  However, the direct sensitivity DAEs can also be solved on 10
chucks of 10 processors each (i.e.\ 100 overall processors) where blocks of 10
direct sensitivity DAEs are solved on each chunk of processors.  However, when
using this approach, note that in general the state DAE has to be integrated
on each of the chunks of processors along with the direct sensitivity DAEs and
this will severely impact parallel speedup.  However, if the state DAE is
linear then the state is not needed to define the direct sensitivity DAEs and
therefore no state computation is needed in order to solve the direct
sensitivity DAEs.  In this case, better parallel speedup can be achieve to
larger numbers of chunks of processors.  However, parallel speedup will be
limited when the numbers of sensitivity DAEs becomes too small an each chunk
of processors since the block linear solver will slow down (because of poorer
flop rates due to reductions in cache efficiency and global communication
overhead).

\subsection{Transient adjoint sensitivity computations}

The most general statement of the adjoint equations associated with the state
equations in (\ref{rythmos:eqn:sens:c})--(\ref{rythmos:eqn:sens:c:ic}) and the
response functions in (\ref{rythmos:eqn:sens:d}), as derived in Appendix~???,
is
%
\begin{eqnarray}
- \frac{d}{dt}\left( \frac{\partial c}{\partial \dot{y}}^H \Lambda \right)
+  \frac{\partial c}{\partial y}^H \Lambda + \frac{\partial g}{\partial y}^H
& = & 0, \; t \in \left[ t_0, t_f \right],
\label{rythmos:app:eqn:sens:adj} \\
\left.\left( \frac{\partial c}{\partial \dot{y}}^H \Lambda \right)\right|_{t=t_f}
& = & - \left. \frac{\partial h}{\partial y}^H \right|_{y=y(t_f)},
\label{rythmos:app:eqn:sens:adj:fc}
\end{eqnarray}
%
where $\Lambda\in\mathcal{C}|\mathcal{G}$ is the multi-vector of Lagrange
multipliers at $t\in[t_0,t_f]$.  As noted in Appendix~??? and [???], the above
form of the adjoint may be unstable for some DAEs.  Here we consider index-0
and index-1 DAEs where $\partial c / {}\partial {}\dot{y}$ is constant and
non-constant.

Note that for index-0 DAEs (which includes implicit and explicit ODEs) that
the mass matrix $\partial c / {}\partial {}\dot{y}$ is non-singular and
therefore the final condition (\ref{rythmos:app:eqn:sens:adj:fc}) becomes
trivially easy to solve as it just requires an inversion of $\partial c /
{}\partial {}\dot{y}$ (assuming that the mass matrix is easy to invert).

\subsubsection{Non-augmented adjoint equations for constant mass matrices}

When $\partial c / {}\partial {}\dot{y}$ is constant, then
(\ref{rythmos:app:eqn:sens:adj}) reduces to
%
\begin{eqnarray}
- \frac{\partial c}{\partial \dot{y}}^H \dot{\Lambda}
+  \frac{\partial c}{\partial y}^H \Lambda + \frac{\partial g}{\partial y}^H
& = & 0, \; t \in \left[ t_0, t_f \right],
\label{rythmos:app:eqn:sens:adj} \\
\left.\left( \frac{\partial c}{\partial \dot{y}}^H \Lambda \right)\right|_{t=t_f}
& = & - \left. \frac{\partial h}{\partial y}^H \right|_{y=y(t_f)},
\label{rythmos:app:eqn:sens:adj:fc}
\end{eqnarray}
%
where $\dot{\Lambda}=d(\Lambda)/d(t)\in\mathcal{C}|\mathcal{G}$.

The adjoint DAE can be cast in the standard form $f(\dot{x},x,t)$ understood
by a general time integrator by first defining the composite vector
%
\begin{equation}
x =
{\bmat{c}
\Lambda_{(:,1)} \\ \vdots \\ \Lambda_{(:,n_g)}
\emat}
\in \mathcal{C}^{n_g}
\end{equation}
%
consisting of the columns $\Lambda_{(:,j)}$ of $\Lambda$ and then defining the
associated composite DAE function
%
\begin{equation}
f(\dot{x},x,t) =
{\bmat{c}
%
- \frac{\partial c}{\partial \dot{y}}^H \dot{\Lambda}_{(:,1)}
+  \frac{\partial c}{\partial y}^H \Lambda_{(:,1)} + \frac{\partial g_{(1)}}{\partial y}^H
%
\vdots \\
%
- \frac{\partial c}{\partial \dot{y}}^H \dot{\Lambda}_{(:,n_g)}
+  \frac{\partial c}{\partial y}^H \Lambda_{(:,n_g)} + \frac{\partial g_{(n_g)}}{\partial y}^H
%
\emat}
\in \mathcal{Y}^{n_p} \rightarrow \mathcal{C}^{n_p}.
\label{rythmos:eqn:adj-sens-daes}
\end{equation}
%
Note also that a transformation in the sensitivity equations of $\tau = t_f -
t$ is needed to allow the integration backward in time using a forward time
integrator.

The matrix $M$ in (\ref{rythmos:eqn:M}) for
(\ref{rythmos:eqn:adj-sens-daes}) is given by
%
\begin{equation}
\alpha \frac{\partial f}{\partial \dot{x}} + \beta \frac{\partial f}{\partial x}
= 
{\bmat{ccc}
\alpha \frac{\partial c}{\partial \dot{y}}^H + \beta \frac{\partial c}{\partial y}^H \\
& \ddots \\
& & \alpha \frac{\partial c}{\partial \dot{y}}^H + \beta \frac{\partial c}{\partial y}^H
\emat}
\in \mathcal{C}^{n_p}|\mathcal{Y}^{n_p}
\label{rythmos:eqn:adj-sens-daes-jac}
\end{equation}
%
which is a block diagonal matrix with the same block along the diagonal and
all quantities are evaluated at the same trial point $(\dot{y},y,u,t)$ as
defined by the time integration algorithm.  An important property of this
block matrix is that each block along the diagonal is the adjoint of the
matrix ${}\alpha (\partial c / {}\partial {}\dot{y}) + {}\beta (\partial c /
{}\partial y)$ which is needed by the forward implicit solvers and the direct
sensitivity solvers.  In general, the adjoint time stepper can be implemented
just from the same sensitivity objects as used for the forward and direct
sensitivity solves with the added requirement that adjoint operations with
these object be supported.

Also note that the block diagonal form of
(\ref{rythmos:eqn:adj-sens-daes-jac}) immediately lends itself for use with a
block linear solver.  When $n_g$ is larger, then a block linear solver may
provide significant speedups over a single RHS linear solver implementation
such as is the case in CVODES and IDAS (see ??? in [???] for example).

\subsubsection{Augmented adjoint equations for non-constant mass matrices}

When $\partial c / {}\partial {}\dot{y}$ is a function of $t$ and/or $y$ then
the adjoint equations in
(\ref{rythmos:app:eqn:sens:adj})--(\ref{rythmos:app:eqn:sens:adj:fc}) are
transformed into the {}\textit{augmented adjoint equations} which take the
form
%
\begin{eqnarray}
\frac{d}{dt}\left( \bar{\Lambda} \right)
+  \frac{\partial c}{\partial y}^H \Lambda + \frac{\partial g}{\partial y}^H
& = & 0, \; t \in \left[ t_0, t_f \right],
\label{rythmos:app:eqn:sens:aug-adj-de} \\
\bar{\Lambda} + \frac{\partial c}{\partial \dot{y}}^H
& = & 0, \; t \in \left[ t_0, t_f \right],
\label{rythmos:app:eqn:sens:aug-adj-ae} \\
\left. \bar{\Lambda} \right|_{t=t_f}
& = & \left. \frac{\partial h}{\partial y}^H \right|_{y=y(t_f)}.
\label{rythmos:eqn:sens:aug-adj:fc}
\end{eqnarray}

The augmented adjoint system can be stated in the standard form understood by
general time integrators by first defining the composite vector
%
\begin{equation}
x =
{\bmat{c}
{\bmat{c} \bar{\Lambda}_{(:,1)} \\ \Lambda_{(:,1)} \emat} \\
\vdots \\
{\bmat{c} \bar{\Lambda}_{(:,n_g)} \\ \Lambda_{(:,n_g)} \emat} \\
\emat}
\in \mathcal{C}^{2 n_g}
\end{equation}
%
consisting of the columns $\bar{\Lambda}_{(:,j)}\in\mathcal{C}$ and
$\Lambda_{(:,j)}\in\mathcal{C}$ of $\bar{\Lambda}$ and $\Lambda$,
respectively, and then defining the associated composite DAE function
%
\begin{equation}
f(\dot{x},x,t) =
{\bmat{c}
%
{\bmat{c}
\frac{d}{dt}\left( \bar{\Lambda}_{(:,1)} \right)
+ \frac{\partial c}{\partial y}^H \Lambda_{(:,1)} + \frac{\partial g_{(1)}}{\partial y}^H
\\
\bar{\Lambda}_{(:,1)} + \frac{\partial c}{\partial \dot{y}}^H \Lambda_{(:,1)}
\emat} \\
%
\vdots \\
%
{\bmat{c}
\frac{d}{dt}\left( \bar{\Lambda}_{(:,n_g)} \right)
+ \frac{\partial c}{\partial y}^H \Lambda_{(:,n_g)} + \frac{\partial g_{(n_g)}}{\partial y}^H
\\
\bar{\Lambda}_{(:,n_g)} + \frac{\partial c}{\partial \dot{y}}^H \Lambda_{(:,n_g)}
\emat} \\
%
\emat}
\in \rightarrow \mathcal{C}^{2 n_p}|\mathcal{Y}^{2 n_p}.
\label{rythmos:eqn:aug-adj-sens-daes}
\end{equation}
%
Note also that a transformation in the sensitivity equations of $\tau = t_f -
t$ is needed to allow the integration backward in time using a forward time
integrator.

The matrix $M$ in (\ref{rythmos:eqn:M}) for
(\ref{rythmos:eqn:aug-adj-sens-daes}) is given by
%
\begin{equation}
\alpha \frac{\partial f}{\partial \dot{x}} + \beta \frac{\partial f}{\partial x}
= 
{\bmat{ccc}
%
{\bmat{cc}
\alpha I & \beta \frac{\partial c}{\partial y}^H \\
\beta I &  \beta \frac{\partial c}{\partial \dot{y}}^H
\emat} \\
%
& \ddots \\
%
& & {\bmat{cc}
\alpha I & \beta \frac{\partial c}{\partial y}^H \\
\beta I &  \beta \frac{\partial c}{\partial \dot{y}}^H
\emat}
%
\emat}
\in \mathcal{Y}^{2 n_p}|\mathcal{C}^{2 n_p}
\label{rythmos:eqn:aug-adj-sens-daes-jac}
\end{equation}
%
which is a block diagonal matrix with the same block along the diagonal and
all quantities are evaluated at the same trial point $(\dot{y},y,u,t)$ as
defined by the time integration algorithm.  At first glance, the blocks
along the diagonal
%
\begin{equation}
{\bmat{cc}
\alpha I & \beta \frac{\partial c}{\partial y}^H \\
\beta I &  \beta \frac{\partial c}{\partial \dot{y}}^H
\emat} \in \mathcal{Y}^{n_p}|\mathcal{C}^{n_p}
\label{rythmos:eqn:aug-adj-sens-jac}
\end{equation}
%
do not seem to correspond to the adjoint of the general form shown for the
matrix $M$ in (\ref{rythmos:eqn:M}) that we are trying to exploit.  However,
to show how to utilize the general form of $M$ in (\ref{rythmos:eqn:M}),
consider the solution of any a system of the form
%
\begin{equation}
{\bmat{cc}
\alpha I & \beta \frac{\partial c}{\partial y}^H \\
\beta I &  \beta \frac{\partial c}{\partial \dot{y}}^H
\emat}
{\bmat{c} v_1 \\ v_2 \emat}
=
{\bmat{c} b_1 \\ b_2 \emat}
\label{rythmos:eqn:aug-adj-sens-system}
\end{equation}
%
for arbitrary vectors $v_1, v_2 {}\in\mathcal{C}$ and $b_1, b_2
{}\in\mathcal{Y}$.  Apply the transformation
%
\[
{\bmat{cc}
-\frac{\beta}{\alpha} I & I \\
 & I
\emat}
\left[
{\bmat{cc}
\alpha I & \beta \frac{\partial c}{\partial y}^H \\
\beta I &  \beta \frac{\partial c}{\partial \dot{y}}^H
\emat}
{\bmat{c} v_1 \\ v_2 \emat}
=
{\bmat{c} b_1 \\ b_2 \emat}
\right]
\]
%
yields the transformed system
%
\begin{equation}
{\bmat{cc}
 & \bar{\alpha} \frac{\partial c}{\partial \dot{y}}^H + \bar{\beta} \frac{\partial c}{\partial y}^H \\
\beta I &  \beta \frac{\partial c}{\partial \dot{y}}^H
\emat}
{\bmat{c} v_1 \\ v_2 \emat}
=
{\bmat{c}  b_2 - \bar{\gamma} b_1 \\ b_2 \emat}
\label{rythmos:eqn:aug-adj-sens-trans-system}
\end{equation}
%
where $\bar{\alpha}=\beta$, $\bar{\beta}=\beta^2/\alpha$, and
$\bar{\gamma}=\beta/2$.  This system can then be solved as
%
\[
v_2 = \left( \bar{\alpha} \frac{\partial c}{\partial \dot{y}}^H + \bar{\beta} \frac{\partial c}{\partial y}^H \right)^{-1}
\left(  b_2 - \bar{\gamma} b_1 \right)
\]
%
followed by
%
\[
v_1 = \frac{1}{\beta} \left( b_2 - \beta \frac{\partial c}{\partial \dot{y}}^H v_2 \right).
\]
%
Therefore, the Schur complement
%
\[
\bar{\alpha} \frac{\partial c}{\partial \dot{y}}^H + \bar{\beta} \frac{\partial c}{\partial y}^H
\]
%
that is solved for is of the general form needed for the forward and direct
sensitivity and no new functionality over what is needed for the non-augmented
adjoint DAE form is required for the adjoint system.  Solving the augmented
DAE system only requires a few extra operations but requires twice the
time-step storage since both $\bar{\Lambda}$ and $\Lambda$ need to be
maintained.

Again, these computations can be rearranged so that multi-vector operations
and block solvers can be utilized resulting in significant speedups over a
single vector RHS version such as in CVODES and IDAS.

\subsubsection{Adjoint reduced gradient computations}

As the adjoint is being computed, the integral
%
\begin{equation}
\frac{\partial \hat{d}}{\partial p}^H =
\int_{t_0}^{t_f} \left(
    \frac{\partial g}{\partial p}^H
    + \frac{\partial c}{\partial p}^H \Lambda
  \right) dt
  + \left. \frac{\partial h}{\partial p}^H \right|_{t=t_f}
  - \left. \left( \frac{\partial y_0}{\partial p}^H \frac{\partial c}{\partial \dot{y}}^H \Lambda \right) \right|_{t=t_0}
\label{rythmos:eqn:sens:d_d_hat_d_p}
\end{equation}
%
and the point-wise gradient
%
%
\begin{equation}
\frac{\partial \hat{d}}{\partial v(t)}^H
= \frac{\partial g}{\partial v(t)}^H + \frac{\partial c}{\partial v(t)}^H \Lambda(t),
t\in[t_0,t_f]
\label{rythmos:eqn:sens:d_d_hat_d_v_t}
\end{equation}
%
can both be computed.  Note that these computations depend on the state $y(t)$
as well as on the adjoint $\Lambda(t)$ in general and therefore the state and
the (local) adjoint history are both needed.  If the integral equation is
computed using quadrature variables, the getting access to $y(t)$ and
$\Lambda(t)$ is automatic but if these equations are integrated using a
different time stepping algorithm (working backward in time), then more
general interpolation access to $y(t)$ and $\Lambda(t)$ will be required.

\section{Stability analysis and eigen problems}

{}\textbf{ToDo: Andy, can you fill this section in or tell me what the
requirements are?}

\section{General DAE formulation and requirements}

In this section we state the requirements for a general DAE to support various
time integration methods, direct and adjoint sensitivity computations, and
other types of numerical problems.  The most fundamental implementation of the
interface described here will be provided by the underlying application.  Our
goal here is to model the requirements at a level of abstraction most
convenient for application developers.  However, many other composite DAE
objects will also exist that will be based on the basic DAE abstractions.

Here we define the concept of a parameterized general DAE (or ODE) that takes
the form
%
\begin{eqnarray}
c\left( \dot{y}(t), y(t), \{u_l\}, t \right) & = & 0,
\; t \in \left[ t_0, t_f \right] \label{rythmos:eqn:gen:c} \\
y(t_0) & = & y_0(\{u_l\}) \label{rythmos:eqn:gen:c:ic}
\end{eqnarray}
\begin{tabbing}
\hspace{4ex}where\hspace{1ex}\= \\
\>	$y(t) \:\in\:\mathcal{Y}$ are the transient state variables, \\
\>	$\dot{y}(t) = d(y)/d(t)\:\in\:\mathcal{Y}$ is the vector of temporal derivatives of $y(t)$, \\
\>	$\{u_l\}$ is a set of auxiliary parameter vectors with $|\{u_l\}| = N_u$ members, \\
\>	$u_l\in\mathcal{U}_l$, $l=1 {}\ldots N_u$, is the $l^{\mbox{th}}$ subvector of auxiliary parameters, \\
\>	$c(\dot{y}(t), y(t), \{u_l\}, t) :
		\mathcal{Y}^2 \times \mathcal{U}_1 \times \ldots \times \mathcal{U}_{N_u} \times \RE
		\rightarrow \mathcal{C}$ are the state DAEs, \\
\>	$y_0(\{u_l\}) \:\in\:\mathcal{U}_1 \times \ldots \times \mathcal{U}_{N_u} \rightarrow \mathcal{Y}$
		define the initial conditions for $y(t_0)$ as a function of $\{u_l\}$, \\
\>	$\mathcal{Y} \:\subseteq\:\RE^{n_y}$ is the vector space of the state variables $y(t)$, \\
\>	$\mathcal{U}_l \:\subseteq\:\RE^{n_{u,l}}$ is the vector space of $u_l$ ($l = 1 {}\ldots N_u$), \\
\>	$\mathcal{C} \:\subseteq\:\RE^{n_y}$ is the vector space of the output of the DAE function $c(\ldots)$.
\end{tabbing}

The DAE formulation above in
(\ref{rythmos:eqn:gen:c})--(\ref{rythmos:eqn:gen:c:ic}) differs slightly than
the sensitivity DAE formulation in
(\ref{rythmos:eqn:sens:c})--(\ref{rythmos:eqn:sens:c:ic}) in that the latter
is more specific to sensitivity computations while the former is more general
and more amiable to a general software implementation.  The DAE in
(\ref{rythmos:eqn:gen:c})--(\ref{rythmos:eqn:gen:c:ic}) is meant to represent
the most general DAE formulation that allows the composition of multiple DAEs
(for example in a staggered mode).  Different subvectors of auxiliary
parameters $u_l\in\mathcal{U}_l$ can be mapped into different types of
quantities in different types of numerical methods.  Some of the kinds of
roles that an individual subset vector $u_l\in\mathcal{U}_l$ of auxiliary
parameters can represent are:
%
\begin{itemize}
%
{}\item Steady-state parameters for sensitivity computations (i.e.\ $u_1 =
p$ in (\ref{rythmos:eqn:sens:c}))
%
{}\item Transient parameters for sensitivity computations (i.e.\ $u_2 = v(t)$
in (\ref{rythmos:eqn:sens:c}))
%
{}\item Coupling variables between different DAEs (e.g.\ $u_1 = y_1$ in
(\ref{rythmos:eqn:stag:c2}))
%
{}\item Continuation parameters (i.e.\ to aid in the solution of the implicit
nonlinear time step equations $r(z)=0$ in (\ref{rythmos:eqn:r}) or in the
solution of initial conditions)
%
{}\item Uncertain parameters (i.e.\ to be manipulated in an uncertainty
quantification study)
%
\end{itemize}

From the DAE developer's point of view, how a particular set of parameters is
used in some higher-level numerical algorithm does not greatly influence what
is required from the DAE to allow the manipulation of these parameters or in
how derivatives for those variables are computed.  The general DAE model in
(\ref{rythmos:eqn:gen:c})--(\ref{rythmos:eqn:gen:c:ic}) can be used, for
example, to represent
(\ref{rythmos:eqn:sens:c})--(\ref{rythmos:eqn:sens:c:ic}) in a sensitivity
computation, or can represent each DAE in a staggered set of DAEs, or any
other type of DAE.

For the purpose of sensitivity computations we will also define a general
set of auxiliary response functions of the form
%
\begin{equation}
g_j(y,\{\hat{u}_l\},t) \in \mathcal{U}_1 \times \ldots \times \mathcal{U}_{\hat{N}_u} \rightarrow \mathcal{G}_j, j=1 \ldots N_g, 
\label{rythmos:eqn:gen:g}
\end{equation}
%
The purpose of defining a general set of response functions in
(\ref{rythmos:eqn:gen:g}) is to provide a concise description of the
requirements for various types of computations.  The reason for defining
different sets of response functions $g_j(\ldots)$ is that any particular
member may represent a high-dimensional function with a specialized scalar
product for its vector space $\mathcal{G}_j$.  Note that computing
sensitivities for a a very high dimensional vector function $g_j(\ldots)$ is
intractable for an adjoint approach but is very a direct sensitivity approach
(note: the steady-state interface for nonlinear problems should be changed in
this way to support large dimensional response functions as well).

Note that the evaluations for $c(\dot{y},y,\{u_l\},t)$ and $g_j(y,\{u_l\}t)$
(for $j = 1 {}\ldots N_g$) may share common subcomputations and therefore
adding provisions to allow efficient simultaneous computations of all of the
objects is advisable.  In addition, the derivative computations described
below will automatically compute the function values when AD is used so a
means to allow the simultaneous computation of derivative objects along with
function values is needed for optimal performance.  Also, if the DAE
implementation does copy the state then it would be advisable to tell the DAE
when the state being passed in has changed or not.

Here we divide up the requirements for general DAE in
(\ref{rythmos:eqn:gen:c})--(\ref{rythmos:eqn:gen:c:ic}) and the response
functions in (\ref{rythmos:eqn:gen:g}) into three different categories:
forward solve, direct sensitivities and adjoint sensitivities.

\subsection{Requirements for forward solve}

Here we describe requirements for
(\ref{rythmos:eqn:gen:c})--(\ref{rythmos:eqn:gen:c:ic}) needed to
implement a basic class of implicit forward integration methods.

\begin{itemize}

\item\textbf{Zero-order requirements (explicit and implicit time steppers)}

\begin{itemize}
%
{}\item Initial condition evaluation: Given $u_l\in\mathcal{U}_l$ (for
$l=1\ldots{}N_u$) evaluate the initial condition function
\[
\{u_l\} {}\rightarrow y_0(\{u_l\})\in\mathcal{Y}.
\]
%
{}\item DAE residual evaluation: Given selections for $\dot{y}\in\mathcal{Y}$,
$y\in\mathcal{Y}$, $u_l\in\mathcal{U}_l$ (for $l=1\ldots{}N_u$), and
$t\in\RE$, evaluate
\[
(\dot{y},y,\{u_l\},t) {}\rightarrow c(\dot{y},y,\{u_l\},t)\in\mathcal{C}.
\]
%
{}\item Auxiliary response evaluation: Given selections for $y\in\mathcal{Y}$,
$u_l\in\mathcal{U}_l$ (for $l=1\ldots{}N_u$), and $t\in\RE$ evaluate
\[
\{u_l\} {}\rightarrow g(\{u_l\},t)\in\mathcal{G}_j, j = 1 \ldots N_g.
\]
%
\end{itemize}

\item\textbf{First-order requirements (implicit time steppers and sensitivity computations)}

\begin{itemize}
%
{}\item Formation of $M = {}\alpha ({}\partial c / {}\partial {}\dot{y}) +
{}\beta ({}\partial c / {}\partial y)$: Given selections for
$\dot{y}\in\mathcal{Y}$, $y\in\mathcal{Y}$, $u_l\in\mathcal{U}_l$ (for
$l=1\ldots{}N_u$), $t\in\RE$, $\alpha\in\RE$, and $\beta\in\RE$ evaluate an
abstract nonsingular object
\[
(\dot{y},y,\{u_l\},t) {}\rightarrow M
= {}\alpha \frac{\partial c}{\partial {}\dot{y}} + \beta \frac{\partial c}{\partial y}
{}\in {}\mathcal{C}|\mathcal{Y}.
\]
Note the special cases of $(\alpha,\beta)=(1,0)$ and $(\alpha,\beta)=(0,1)$
allow a client to access the operators $({}\partial c / {}\partial {}\dot{y})$
and $({}\partial c / {}\partial y)$ separately.  Of course the operator
returned for $M = ({}\partial c / {}\partial {}\dot{y})$ for the case of
$(\alpha,\beta)=(1,0)$ is not guaranteed to be nonsingular.  Only operators
$M$ for reasonable finite nonzero values of $\alpha$ and $\beta$ are
guaranteed to be nonsingular.
%
{}\item Allow reuse of factorizations, preconditioners etc.\ form some old
nonsingular operator $M_{\mbox{old}}$ in the reformation of $M = {}\alpha
({}\partial c / {}\partial {}\dot{y}) + {}\beta ({}\partial c / {}\partial
y)$: Given a previously computed nonsingular operator $M_{\mbox{old}}$, create
a new operator $M$ at a new point $(\dot{y},y,t)$ what reuses any
factorizations or preconditioners from $M_{\mbox{old}}$ but represents the
correct analytic operator.  This capability is needed to allow a time stepper
algorithm to have some control over when preconditioners or factorizations are
updated.  However, the newly computed operator $M$ must still provide a
sufficient analytic approximation to the true matrix which is needed for
sensitivity computations.
%
{}\item Application of $M = {}\alpha ({}\partial c / {}\partial {}\dot{y}) +
{}\beta ({}\partial c / {}\partial y)$: Given a pre-formed object for $M =
{}\alpha ({}\partial c / {}\partial {}\dot{y}) + {}\beta ({}\partial c /
{}\partial y)$, provide the ability to apply the operator as
\[
Y = M X
\]
where $X\in\mathcal{Y}|\RE^m$ is some arbitrary RHS $m$-column multi-vector
and $Y\in\mathcal{C}|\RE^m$ is the LHS multi-vector.  The purpose of stating
the multi-vector form is to encourage the development of block methods that
are useful for other types of more advanced algorithms.
%
{}\item Solve with $M = {}\alpha ({}\partial c / {}\partial {}\dot{y}) +
{}\beta ({}\partial c / {}\partial y)$: Given a pre-formed object for $M =
{}\alpha ({}\partial c / {}\partial {}\dot{y}) + {}\beta ({}\partial c /
{}\partial y)$, then provide the ability to (approximately) solve for systems
of the form
\[
X = M^{-1} B
\]
where $B\in\mathcal{Y}|\RE^m$ is some arbitrary RHS $m$-column multi-vector
and $X\in\mathcal{C}|\RE^m$ is the LHS solution multi-vector.  The purpose of
stating the linear solve requirement in multi-vector form is to encourage the
development of block linear solver approaches that are useful for other types
of more advanced algorithms.  However, note that the ability to solve single
linear systems of the form
\[
x = M^{-1} b,
\]
where $b\in\mathcal{Y}$ and $x\in\mathcal{C}$, automatically satisfies this
requirement.  The exact specification of what is meant to (approximately)
solve systems of this nature must be accurately specified. Note that a status
test that will work for composite and block linear solvers is needed which is
different than for a straightforward iterative or direct linear solver.
%
{}\item{} [Optional] Formation and maintenance of multiple $M = {}\alpha
({}\partial c / {}\partial {}\dot{y}) + {}\beta ({}\partial c / {}\partial y)$
objects: Some types of numerical algorithms can be implemented much more
efficiently if more than one instance of $M = {}\alpha ({}\partial c /
{}\partial {}\dot{y}) + {}\beta ({}\partial c / {}\partial y)$ can be computed
and maintained simultaneously.  This may be needed, for instance, in an
implicit RK method as described in Section ???.
%
\end{itemize}

\end{itemize}

\subsection{Requirements for first-order direct sensitivities}

In addition for the requirements for forward solves described in the previous
section, here we describe an additional set of requirements for
(\ref{rythmos:eqn:gen:c})--(\ref{rythmos:eqn:gen:c:ic}) needed to
implement the implicit direct sensitivity methods as described in Section
\ref{rythmos:scn:transientsensitivity:direct}.

\begin{itemize}

{}\item Initial condition direct sensitivity: Given $u_l\in\mathcal{U}_l$ (for
$l=1\ldots{}N_u$) and some $U_l\in\mathcal{U}_l|\RE^m$ compute the action of
\[
(\{u_l\},U_l) {}\rightarrow \frac{\partial y_0}{\partial u_l} U_l \in \mathcal{Y}|\RE^m
\; \mbox{for} \; l = 1 {}\ldots N_u,
\]
or alternatively evaluate the sensitivity themselves as a multi-vector
\[
\{u_l\} {}\rightarrow \frac{\partial y_0}{\partial u_l} \in \mathcal{Y}|\mathcal{U}_l
\; \mbox{for} \; l = 1 {}\ldots N_u.
\]

{}\item DAE auxiliary direct sensitivities: Given selections for
$\dot{y}\in\mathcal{Y}$, $y\in\mathcal{Y}$, $u_l\in\mathcal{U}_l$ (for
$l=1\ldots{}N_u$), $t\in\RE$, and some $U_l\in\mathcal{U}_l|\RE^m$ compute
the action of
\[
(\dot{y},y,\{u_l\},t,U_l) {}\rightarrow \frac{\partial c}{\partial u_l} U_l \in \mathcal{C}|\RE^m,
\; \mbox{for} \; l = 1 {}\ldots N_u,
\]
or alternatively compute the sensitivities themselves as a multi-vector
\[
(\dot{y},y,\{u_l\},t) {}\rightarrow \frac{\partial c}{\partial u_l} \in \mathcal{C}|\mathcal{U}_l,
\; \mbox{for} \; l = 1 {}\ldots N_u.
\]

{}\item Auxiliary response function direct sensitivities: Given selections for
$y\in\mathcal{Y}$, $u_l\in\mathcal{U}_l$ (for $l=1\ldots{}N_u$), $t\in\RE$ ,
and some $U_l\in\mathcal{U}_l|\RE^m$ compute the action of
\[
(y,\{u_l\},U_l) {}\rightarrow \frac{\partial g_j}{\partial u_l} U_l \in \mathcal{G}|\RE^m,
\; \mbox{for} \; l = 1 {}\ldots N_u, j = 1 \ldots N_g
\]
or alternatively compute the sensitivities themselves as a multi-vector
\[
(y,\{u_l\},U_l) {}\rightarrow \frac{\partial g_j}{\partial u_l} \in \mathcal{G}|\mathcal{U}_l,
\; \mbox{for} \; l = 1 {}\ldots N_u, j = 1 \ldots N_g.
\]

\end{itemize}

Note that these three different types of forward derivative computations can
always be approximated using directional finite differences of $\{u_l\}
{}\rightarrow y_0$, $(\dot{y},y,\{u_l\},t) {}\rightarrow c$ and
$(\{\hat{u}_l\},t) {}\rightarrow g$, respectively.  Therefore, numerical
methods that only require forward sensitivities as described above can always
use directional finite differences and therefore require really extra
functionality beyond what is needed for the forward solve.  Some numerical
methods, such as sensitivity computations, do require some degree of exactness
from derivative computations used for $M = {}\alpha ({}\partial c / {}\partial
{}\dot{y}) + {}\beta ({}\partial c / {}\partial y)$ and the other derivative
objects described above.

\subsection{Requirements for first-order adjoint sensitivities}

In this section we describe an extra set of requirements for implementing
adjoint methods.  Basically, these requirements state that the derivative
objects described for the direct sensitivity methods need to also support
adjoints.

\begin{itemize}

{}\item Initial condition adjoint sensitivity: Given $u_l\in\mathcal{U}_l$ (for
$l=1\ldots{}N_u$) and some $V\in\mathcal{C}|\RE^m$ compute the action of
\[
(\{u_l\},U_l) {}\rightarrow \frac{\partial y_0}{\partial u_l}^H V \in \mathcal{U}_l|\RE^m
\; \mbox{for} \; l = 1 {}\ldots N_u,
\]
or alternatively evaluate the sensitivity themselves as a multi-vector
\[
\{u_l\} {}\rightarrow \frac{\partial y_0}{\partial u_l} \in \mathcal{Y}|\mathcal{U}_l
\; \mbox{for} \; l = 1 {}\ldots N_u.
\]
%
{}\item Application of adjoint operator $M^H = {}\alpha ({}\partial c /
{}\partial {}\dot{y}) + {}\beta ({}\partial c / {}\partial y)^H$: Given a
pre-formed object for $M = {}\alpha ({}\partial c / {}\partial {}\dot{y}) +
{}\beta ({}\partial c / {}\partial y)$, provide the ability to apply the
operator as
\[
Y = M^H X
\]
where $X\in\mathcal{C}|\RE^m$ is some arbitrary RHS $m$-column multi-vector
and $Y\in\mathcal{Y}|\RE^m$ is the LHS multi-vector.  The purpose of
stating the multi-vector form is to encourage the development of block methods
that are useful for other types of more advanced algorithms.
%
{}\item Solve with adjoint operator $M^H = {}\alpha ({}\partial c / {}\partial
{}\dot{y}) + {}\beta ({}\partial c / {}\partial y)^H$: Given a pre-formed
object for $M = {}\alpha ({}\partial c / {}\partial {}\dot{y}) + {}\beta
({}\partial c / {}\partial y)$, then provide the ability to (approximately)
solve for systems of the form
\[
X = M^{-H} B
\]
where $B\in\mathcal{C}|\RE^m$ is some arbitrary RHS $m$-column multi-vector
and $X\in\mathcal{Y}|\RE^m$ is the LHS solution multi-vector.  The purpose of
stating the linear solve requirement in multi-vector form is to encourage the
development of block linear solver approaches that are useful for other types
of more advanced algorithms.  However, note that the ability to solve single
linear systems of the form
\[
x = M^{-H} b,
\]
where $b\in\mathcal{C}$ and $x\in\mathcal{Y}$, automatically satisfies this
requirement.  The exact specification of what is meant to (approximately)
solve systems of this nature must be accurately specified. Note that a status
test that will work for composite and block linear solvers is needed which is
different than for a straightforward iterative or direct linear solver.
%
{}\item DAE auxiliary adjoint sensitivities: Given selections for
$\dot{y}\in\mathcal{Y}$, $y\in\mathcal{Y}$, $u_l\in\mathcal{U}_l$ (for
$l=1\ldots{}N_u$), $t\in\RE$, and some $V\in\mathcal{C}|\RE^m$ compute
the action of
\[
(\dot{y},y,\{u_l\},t,V) {}\rightarrow \frac{\partial c}{\partial u_l}^H V \in \mathcal{U}_l|\RE^m,
\; \mbox{for} \; l = 1 {}\ldots N_u,
\]
or alternatively compute the sensitivities themselves as a multi-vector
\[
(\dot{y},y,\{u_l\},t) {}\rightarrow \frac{\partial c}{\partial u_l} \in \mathcal{C}|\mathcal{U}_l,
\; \mbox{for} \; l = 1 {}\ldots N_u.
\]

{}\item Auxiliary response function adjoint sensitivities: Given selections for
$y\in\mathcal{Y}$, $u_l\in\mathcal{U}_l$ (for $l=1\ldots{}N_u$), $t\in\RE$ ,
and some $V\in\mathcal{G}_j|\RE^m$ compute the action of
\[
(y,\{u_l\},U_l) {}\rightarrow \frac{\partial g_j}{\partial u_l}^H V \in \mathcal{U}_l|\RE^m,
\; \mbox{for} \; l = 1 {}\ldots N_u, j = 1 \ldots N_g
\]
or alternatively compute the sensitivities themselves as a multi-vector
\[
(y,\{u_l\},U_l) {}\rightarrow \frac{\partial g_j}{\partial u_l} \in \mathcal{G}|\mathcal{U}_l,
\; \mbox{for} \; l = 1 {}\ldots N_u, j = 1 \ldots N_g.
\]

\end{itemize}

Note that unlike the direct sensitivity requirements, these adjoints can not
be simulated using directional finite differences.  In general, the best way
to perform these adjoint computations is a tailored use of the reverse mode of
AD.  Or, if the matrix and multi-vector objects can be created explicitly,
then they can just be transposed (or conjugate transpose in the complex case)
to provide the needed adjoint operations.

\section{General adaptive error-control strategies}

RAB's comments on this:
\begin{itemize}
%
{}\item It seems that error control, step size control and order control is
very integral in a specific class of DAE and ODE solvers.  Therefore, as a
first pass, it may just make sense to just make an atomic time stepping
algorithm define the time step equations and perform step-size and order
control.
%
{}\item After some more thought, there is little reason to allow mixing and
matching of time stepping methods for predictors and correctors.  While the
predictor is essentially an explicit time stepping method there are other ways
of reusing common code than to reuse entire time stepping objects for
predictors.
%
{}\item External clients should be able to replace the error and step-size
control algorithms in addition to being able to adjust the ``magic''
parameters.
%
{}\item For staggered DAEs, an overall error control strategy should be
composable just from local truncation error estimates from atomic time
stepping methods.
%
\end{itemize}

\section{Checkpointing and Interpolation Storage Management Strategies}

Checkpointing a transient iterative computation (or any other expensive
iterative computation) is an important capability in large-scale scientific
computing.  A checkpoint is a snap shoot of the state of an iterative
computation that can be saved to a hard disk and then read back in later to
restart the computation from the checkpoint from such that it results in the
same computations as if the computations where never interrupted.  There are
various levels of checkpoints but in its most basic form we use the term
checkpoint to mean that the state is written in at a level such then when read
back in to a time integrator that is results in the exact same
{}\underline{binary} computations that would have resulted if the computation
was not restarted.

Checkpoints are primarily needed for the following use cases:
%
\begin{itemize}
%
{}\item {}\textbf{Fault tolerance}: By periodically checkpointing an iterative
solve on an MPP, if one or more processors goes down, the computation can be
restarted on a different set of processors from the last checkpoint.  Note
that in this case, one can only expect the same binary output from the
simulation if using the same number and same type of processors and if the
results of MPI global reductions is the same for the new processor topology.
Also note that this type of checkpointing requires that the checkpoints be
saved in such a way that they can be retrieved if one or more processors goes
down.  This means that using local disk storage on a processor is less
attractive than using a globally shared disk array.
%
{}\item {}\textbf{Recomputation of the state for adjoint computations}:
Another big motivator for checkpointing capability is the need to recompute
the state equation (\ref{rythmos:eqn:sens:c})--(\ref{rythmos:eqn:sens:c:ic})
when the adjoint sensitivity equation
(\ref{rythmos:app:eqn:sens:adj})--(\ref{rythmos:app:eqn:sens:adj:fc}) is
solved backwards in time.  This topic is discussed in more detail below.
These types of checkpoints may best be stored on a local disk on each
processor in an MPP which results in faster I/O.  Note that this type of
checkpointing and the more traditional fault-tolerant checkpointing, where
checkpoints are written to a nonlocal disk array, may both be appropriate in
an adjoint computation.
%
\end{itemize}

In general, each time stepper algorithm will be responsible for specifying how
to checkpoint it's algorithm with the least possible data such that a restart
will result in the same binary results.

Related to checkpointing in the need to interpolate an integrated solution.
Interpolation is needed for the following use cases:
%
\begin{itemize}
%
{}\item {}\textbf{Allowing clients to pick out solutions at specific time
points}: There are situations where clients need to extract the state solution
at specific time points in order to, for instance, compare to some other
estimate of the state such as in a least-squares problem.  Another example of
this is in extracting the state at some final time $t_f$ which the time
integrator may not necessary stop exactly on.  For example, if the Jacobian is
being reused for the last time step it may be more efficient to actually
integrate past $t_f$ and then let the user pick out $t_f$ using some
interpolation.
%
{}\item {}\textbf{Interpolation of the state for the backward adjoint
computations}: In general, the backward integration of the adjoint equation
(\ref{rythmos:app:eqn:sens:adj})--(\ref{rythmos:app:eqn:sens:adj:fc}) will not
fall directly on the same time steps used to compute the state equation
(\ref{rythmos:eqn:sens:c})--(\ref{rythmos:eqn:sens:c:ic}), especially if
adaptive time-stepping is used.  Therefore, there must be some general means
for a time integrator to record a history of its solution in an interpolation
buffer that can then be queried for values of the sate $y(t)$ and perhaps even
the time derivative $\dot{y}(t)$ at specific times $t$.  Interpolation will be
discussed in more detail later.
%
{}\item {}\textbf{Interpolation of DAE solutions in decoupled staggered DAE
time steppers}: On way to solve staggered DAEs, such as described in
Section~\ref{rythmos:sec:staggered-daes}, is to use separate time integrators
for the first and second DAEs.  This allows different time stepping algorithms
tailored to each set of DAEs and result in an overall better time integration
scheme.  This however, requires that the second DAE be able to read the state
from the first DAE in a general way.  If the first time stepper algorithm can
provide an interpolation buffer that can be queried for $y_1(t)$ and/or
$\dot{y}_1(t)$ by the second DAE, then the two time stepper algorithms can be
completely decoupled.  Note that the interpolation buffer needed in this use
case is likely much shorter than would be needed for an interpolation of the
state solution as required by an adjoint time integrator.
%
\end{itemize}

A few ideas regarding checkpointing and interpolation are:
\begin{itemize}
%
{}\item A time stepper (or a series of staggered time steppers) restarted form
a checkpoint should produce the exact same (binary) output as if the
integration never stopped.  This makes a checkpoint a special time event that
affects a lot of other things.
%
{}\item A time stepper should be able to write out $(y(t),\dot{y}(t))$ at the
end if each time step so that a Hermite interpolant can be achieved.  Or, an
abstract interpolation object (which may be specific to a time integrator)
should be constructed and maintained by a time integrator.
%
\end{itemize}

\subsection{Interpolation methods}

While there are many possible interpolation techniques possible for
interpolating the solution of a DAE, the Hermite interpolation seems to be the
most popular~\cite{sundials}.  The most simple Hermite interpolation stores
both $y(t_k)$ and $\dot{y}(t_k)$ at various points in time $t_k$ and then the
values of $y(t)$ (and $\dot{y}(t)$) are interpolated using a piecewise cubic
polynomial formed from $y(t_k)$, $\dot{y)k}(t)$, $y(t_{k+1})$, and
$\dot{y}(t_{k+1})$, where $t\in[t_k,t_{k+1}]$.  The Hermite interpolate is a
local computation and is zero-order and first-order continuous.

Note that interpolation methods may also take into account the accuracy needed
in the adjoint so that a more course interpolation buffer can be stored (Note:
ask Andrei about this).

\subsection{Checkpointing for adjoint sensitivity computations}

The adjoint solution of
(\ref{rythmos:app:eqn:sens:adj})--(\ref{rythmos:app:eqn:sens:adj:fc}) requires
access to the state $y(t)$ as it is integrated backwards in time.  If there is
sufficient storage, then a complete interpolation buffer for $y(t)$ for
$t\in[t_0,t_f]$ can be stored while the state equation
(\ref{rythmos:eqn:sens:c})--(\ref{rythmos:eqn:sens:c:ic}) is integrated
forward in time.  This state interpolation buffer can then be used to
integrate the adjoint backward in time.  However, if there is not sufficient
disk space to store a sufficiently accurate interpolation buffer, then this
approach will not work and a checkpoint strategy must be employed.

Checkpointing strategies\footnote{Note that the term ``checkpoint'' is being
somewhat overloaded here in the context of transient adjoint problems and is
more entailed than the basic definition of a checkpoint of a single state and
previously defined.} for adjoint problems store a series of snapshots and then
recompute the state again one or more times in smaller time segments in order
to build smaller temporary interpolation buffers that can then be used to
integrate the adjoint backward in time one piece at a time.

The question to ask is, ``Given a limited interpolation buffer (to be stored
in core) and a maximum number of checkpoints that can be stored (as allowed by
the disk storage), how should checkpoints be set that result in the minimum
about of recomputation of the forward state equation?''  This question is
answered the checkpointed algorithm of Griwank~\cite{347846} and is
encapsulated in a simple program called {}\texttt{revolve()}~\cite{347846}.
The algorithm implemented in this simple driver program directs the forward
checkpointing and reverse adjoint computations in such a way as to provably
minimize the amount of needed recomputation of the forward problem.  The key
to using {}\texttt{revolve()} is to realize that {}\texttt{revolve}'s concept
of a ``step'' is the length of the forward computation that can be stored in
an incore interpolation buffer or as a set of computations such that the
operation count swaps individual checkpoint management.
Figure~\ref{rythmos:fig:checkpoint_graph} shows a plot of the relative
increase in cost in recomputing the forward solution for various ratios of the
number of checkpoint stored with respect to the total number of ``steps'' (as
defined by {}\texttt{resolve}).  Amazingly, as little as a 1\% ratio of
checkpoints to ``steps'' only increases the number of forward computation by
2.7!  It is not until the ratio of checkpoints to ``time'' steps goes below
0.1\% does the cost start to dramatically increase and asymptotically
approaches infinity as the ratio goes to zero.

{\bsinglespace
\begin{figure}
\begin{center}
%\fbox{
\includegraphics*[scale=0.75]{checkpoint_graph}
%}
\end{center}
\caption{
\label{rythmos:fig:checkpoint_graph}
Plot increased cost of computing adjoint using checkpointed scheme implemented
by {}\texttt{resolve()} algorithm relative to cost of a single forward solve
for where the ratio of the number of checkpoints to the total time steps is
the x-axis where 5000 total time steps where used.  }
\end{figure}
\esinglespace}

Note: The {}\texttt{revolve()} function has several basic problems -- such as
the use of static data, global namespace pollution, explicit calls to
{}\texttt{printf()}, the use of ``magic numbers'', and non-intuitive calling
sequences -- that makes it inappropriate for large-scale production programs.
However, this code can easily be refactored into a more general C++ class that
allows for multiple instances and better usability.  The question is how to
build general software around a simple utility such as this.

\subsection{Checkpoint compression strategies}

If one is willing to give up binary restart-ability for the state computation
and is willing to allow some greater degree error in the adjoint sensitivity
computation, then compression strategies for the state can be used to even
further reduce memory storage.  For all DAEs, compression in time is possible
where interpolates are spaced out according to some error bound.  For DAEs
that come from discretized PDEs, compression in space and even the
computation of a courser state equation is possible if only inexact
adjoints are needed.

Note: This may be a good angle for a research proposal.  How do we combine
knowledge about inexactness with an adjoint time integrator and allow for
compressed interpolation and checkpointing strategies and control of adaptive
step-control error?  This leads into an inexactness proposal.

\section{Example applications}

Notes: The purpose of this section is to describe some important Sandia
applications and describe how the propose time integration tools may be used
with these applications.

\subsection{Xyce}

\subsection{Premo}

\subsection{Charon}

\subsection{Sundance}

\section{Conclusions}

% ---------------------------------------------------------------------- %
% References
%
\clearpage
\bibliographystyle{plain}
\bibliography{RythmosDesignSAND}
\addcontentsline{toc}{section}{References}

% ---------------------------------------------------------------------- %
% Appendices should be stand-alone for SAND reports. If there is only
% one appendix, put \setcounter{secnumdepth}{0} after \appendix
%
\appendix

\section{Derivation of transient sensitivity computations}
\label{rythmos:app:sens-derivations}

Here we derive direct and adjoint sensitivity methods for general DAEs as
described in Section~\ref{rythmos:sec:trans-sens}.  Restated, the general DAE
model is
%
\begin{eqnarray}
c\left( \dot{y}(t), y(t), p, v(t), t \right) & = & 0,
\; t \in \left[ t_0, t_f \right] \label{rythmos:apdx:eqn:sens:c} \\
y(0) & = & y_0(p) \label{rythmos:apdx:eqn:sens:c:ic}
\end{eqnarray}
%
and the auxiliary response functions, for which sensitivities are computed, are
%
\begin{equation}
d(y(t_0:t_f),p,v(t_0:t_f))
= \int_{t_0}^{t_f} g(y(t),p,v(t),t) dt + h(y(t_f),p),
\label{rythmos:apdx:eqn:sens:d}
\end{equation}
%
where $y(t){}\:\in\:\mathcal{Y}$, $\dot{y}(t) = d(y)/d(t)\:\in\:\mathcal{Y}$,
$p{}\:\in\:\mathcal{P}$, $v(t){}\:\in\:\mathcal{V}$, $c(\dot{y}(t), y(t), p,
v(t), t)
:{}\mathcal{Y}^2{}\times{}\mathcal{P}{}\times{}\mathcal{V}{}\times{}\RE
{}\rightarrow{}\mathcal{C}$,
$y_0(p){}\:\in\:\mathcal{P}{}\rightarrow{}\mathcal{Y}$, $g(y(t),p,v(t),t)
:{}\: {}\mathcal{Y}{}\times{}\mathcal{P}{}\times{}\mathcal{V}{}\RE
{}\rightarrow{}\mathcal{G}$, $h(y(t_f),p) :{}\:
{}\mathcal{Y}{}\times{}\mathcal{P} {}\rightarrow{}\mathcal{G}$,
$\mathcal{G}{}\:\subseteq\:\RE^{n_g}$, $\mathcal{Y}{}\:\subseteq\:\RE^{n_y}$,
$\mathcal{P}{}\:\subseteq\:\RE^{n_p}$, $\mathcal{V}{}\:\subseteq\:\RE^{n_v}$,
and $\mathcal{C}{}\:\subseteq\:\RE^{n_y}$.

The implicit state solution for the DAEs in
(\ref{rythmos:apdx:eqn:sens:c})--(\ref{rythmos:apdx:eqn:sens:c:ic}) gives
$y(p,v(t_0:t),t)\in\mathcal{Y}$, where $v(t_0:t)$ is short hand for the
selection of $v(t)$ (or a discrete approximation in a practical
implementation) in the range $t\in[t_0,t_f]$.  The full implicit state
solution for $y(t)$ in $t\in[t_0,t_f]$ is signified as
$y(p,v(t_0:t_f),t_0:t_f)$.

Given the implicit state function, a reduced set of auxiliary response
functions are defined as
%
\begin{equation}
\hat{d}(p,v(t_i:t_f))
= \int_{t_0}^{t_f} \hat{g}(p,v(t),t) dt + \hat{h}(p),
\label{rythmos:apdx:eqn:sens:d_hat}
\end{equation}
\begin{tabbing}
\hspace{4ex}where\hspace{1ex}\= \\
\>	$\hat{g}(p,v(t),t) = g(y(p,t),p,v(t),t)) : \:
		\mathcal{P} \times \mathcal{V} \times \RE \rightarrow \mathcal{G}$ is defined on $t\in[t_0,t_f]$, and \\
\>	$\hat{h}(p) = h(y(p,t_f),p)) : \:
		\mathcal{P} \rightarrow \mathcal{G}$ is defined only at $t=t_f$.
\end{tabbing}

\subsection{Derivation of direct sensitivities}
\label{rythmos:app:direct-sens-derivation}

For the derivation of direct sensitivities we will ignore the
transient parameters $v(t)$ for $t\in[t_0,t_f]$ since direct
sensitivity methods for such problems are typically impractical.

Note: For some applications where $n_v$ is not too large and where
$v(t_0:t_f)$ is given a very course discretization in $t\in[t_0,t_f]$,
computing direct sensitivities with respect to the discretized $v$ can
be practical.

The direct sensitivity problem is easily stated by differentiating
(\ref{rythmos:apdx:eqn:sens:d_hat}) with respect to $p$ to obtain
%
\begin{equation}
\frac{\partial \hat{d}}{\partial p} 
= \int_{t_0}^{t_f} \left( \frac{\partial g}{\partial y} \frac{\partial y}{\partial p} +  \frac{\partial g}{\partial p} \right) dt
+ \left. \left(  \frac{\partial h}{\partial y} \frac{\partial y}{\partial p} + \frac{\partial h}{\partial p} \right) \right|_{t=t_f},
\label{rythmos:apdx:eqn:sens:d_d_hat_d_p}
\end{equation}
\begin{tabbing}
\hspace{4ex}where\hspace{1ex}\= \\
\>	$\frac{\partial \hat{g}}{\partial p} \in \mathcal{G}|\mathcal{P}$ is defined on $t\in[t_0,t_f]$, \\
\>	$\frac{\partial \hat{h}}{\partial p} \in \mathcal{G}|\mathcal{P}$ is defined only at $t=t_f$, \\
\>	$\frac{\partial y}{\partial p} \in \mathcal{Y}|\mathcal{P}$ is the direct sensitivity of the state
    at $t\in[t_0,t_f]$, and \\
\>	$\frac{\partial \hat{d}}{\partial p} \in \mathcal{G}|\mathcal{P}$ is defined independent of time.
\end{tabbing}

The direct sensitivity $\partial y / {}\partial p$ is computed by solving a
set of $n_p$ independent direct sensitivity equations which are obtained by
differentiating
(\ref{rythmos:apdx:eqn:sens:c})--(\ref{rythmos:apdx:eqn:sens:c:ic}) with
respect to $p$ to give
%
\begin{eqnarray}
%
\frac{\partial c}{\partial \dot{y}} \left(\frac{\partial}{\partial t} \frac{\partial y(t)}{\partial p} \right)
+ \frac{\partial c}{\partial y} \left(\frac{\partial y(t)}{\partial p}\right)
+ \frac{\partial c}{\partial p} & = & 0, \; t \in \left[ t_0, t_f \right], \label{rythmos:apdx:eqn:sens:direct-c} \\
\frac{\partial y(t_0)}{\partial p} & = & \frac{\partial y_0}{\partial p}. \label{rythmos:apdx:eqn:sens:direct-c:ic}
\end{eqnarray}
\begin{tabbing}
\hspace{4ex}where\hspace{1ex}\= \\
\>	$\frac{\partial c}{\partial \dot{y}} \in \mathcal{C}|\mathcal{Y}$ is defined on $t\in[t_0,t_f]$, \\
\>	$\frac{\partial c}{\partial y} \in \mathcal{C}|\mathcal{Y}$ is defined on $t\in[t_0,t_f]$, \\
\>	$\frac{\partial c}{\partial p} \in \mathcal{C}|\mathcal{P}$ is defined on $t\in[t_0,t_f]$, and \\
\>	$\frac{\partial y_0}{\partial p} \in \mathcal{Y}|\mathcal{P}$ is defined only at $t=t_0$.
\end{tabbing}

These $n_p$ independent sensitivity equations are solved for ${}\partial y /
{}\partial p {}\in\mathcal{Y}|\mathcal{P}$ forward in time from $t=t_0$.  The
integral in (\ref{rythmos:apdx:eqn:sens:d_d_hat_d_p}) can be evaluated along
with the integration of
(\ref{rythmos:apdx:eqn:sens:c})--(\ref{rythmos:apdx:eqn:sens:c:ic}) and
(\ref{rythmos:apdx:eqn:sens:direct-c})--(\ref{rythmos:apdx:eqn:sens:direct-c:ic}).
Note that (\ref{rythmos:apdx:eqn:sens:c})--(\ref{rythmos:apdx:eqn:sens:c:ic})
and
(\ref{rythmos:apdx:eqn:sens:direct-c})--(\ref{rythmos:apdx:eqn:sens:direct-c:ic})
are a staggered set of DAE equations in that
(\ref{rythmos:apdx:eqn:sens:c})--(\ref{rythmos:apdx:eqn:sens:c:ic}) are solved
first followed by
(\ref{rythmos:apdx:eqn:sens:direct-c})--(\ref{rythmos:apdx:eqn:sens:direct-c:ic}).

\subsection{Derivation of adjoint sensitivities using direct sensitivity weak form}

Here we describe a derivation for the adjoint equation and the gradient
expressions that is based on basic principles that have a simple and strong
foundation.

\subsubsection{Derivation of adjoint equation and sensitivities for steady-state parameters}

In this section we begin with the same basic expressions of the reduced
derivative ${}\partial {}\hat{d} / {}\partial p$ in
(\ref{rythmos:apdx:eqn:sens:d_d_hat_d_p}) and the direct sensitivity equations
for ${}\partial y / {}\partial p$ in
(\ref{rythmos:apdx:eqn:sens:direct-c})--(\ref{rythmos:apdx:eqn:sens:direct-c:ic})
and then perform basic manipulations in order to arrive at the adjoint
equations and the adjoint sensitivities for ${}\partial {}\hat{d} /
{}\partial p$.

We begin our derivation of the adjoint equation and sensitivities for
steady-state parameters by writing the weak form of the direct sensitivity
equations (\ref{rythmos:apdx:eqn:sens:direct-c}) which is
%
\begin{equation}
\int_{t_0}^{t_f} \Lambda^H \left(
\frac{\partial c}{\partial \dot{y}} \left(\frac{\partial}{\partial t} \frac{\partial y(t)}{\partial p} \right)
+ \frac{\partial c}{\partial y} \left(\frac{\partial y(t)}{\partial p}\right)
+ \frac{\partial c}{\partial p}
\right) dt = 0,
\label{rythmos:apdx:eqn:sens:weak-direct-c}
\end{equation}
%
where at this point $\Lambda(t)\in\mathcal{C}|\mathcal{G}$ is any appropriate
weighting function that is also consistent with the initial conditions
(\ref{rythmos:apdx:eqn:sens:direct-c:ic}).  Later, $\Lambda$ will be chosen to
be the adjoint variables but for now it is nothing more than an arbitrary
weighting function for the purpose of stating the weak form.  The solution to
the weak form of (\ref{rythmos:apdx:eqn:sens:weak-direct-c}) is every bit as
valid and is indeed more general than the strong form in
(\ref{rythmos:apdx:eqn:sens:direct-c}) and we lose nothing by considering the
weak form [???].

Next, we substitute the integration by parts
%
\begin{equation}
\int_{t_0}^{t_f} \left[ \left( \Lambda^H \frac{\partial c}{\partial \dot{y}} \right) \frac{d}{dt}\left( \frac{\partial y}{\partial p} \right) \right] dt
= \left. \left( \Lambda^H \frac{\partial c}{\partial \dot{y}} \frac{\partial y}{\partial p} \right) \right|_{t_0}^{t_f}
- \int_{t_0}^{t_f} \left[ \frac{d}{dt}\left( \Lambda^H \frac{\partial c}{\partial \dot{y}} \right) \frac{\partial y}{\partial p} \right] dt
\end{equation}
%
into (\ref{rythmos:apdx:eqn:sens:weak-direct-c}) and rearrange which yields
%
\begin{equation}
\int_{t_0}^{t_f} \left( \Lambda^H \frac{\partial c}{\partial p} \right) dt
+ \int_{t_0}^{t_f} \left[
    - \frac{d}{dt}\left( \Lambda^H \frac{\partial c}{\partial \dot{y}} \right)
    + \Lambda^H \frac{\partial c}{\partial y}
  \right] \frac{\partial y}{\partial p} dt
+ \left. \left( \Lambda^H \frac{\partial c}{\partial \dot{y}} \frac{\partial y}{\partial p} \right) \right|_{t_0}^{t_f}
= 0
\label{rythmos:apdx:eqn:sens:weak-direct-c-2}
\end{equation}
%
At this point in the derivation we decide to restrict the possible set of
functions for $\Lambda(t)$ by forcing $\Lambda(t)$ to satisfy the differential
equation
%
\begin{equation}
- \frac{d}{dt}\left( \Lambda^H \frac{\partial c}{\partial \dot{y}} \right)
+  \Lambda^H \frac{\partial c}{\partial y} = - \frac{\partial g}{\partial y}, \; t \in \left[ t_0, t_f \right].
\label{rythmos:apdx:eqn:sens:adj-trans}
\end{equation}
%
All that we have done in (\ref{rythmos:apdx:eqn:sens:adj-trans}) is to make
what appears to be an arbitrary choice to help narrow the weighting function
$\Lambda(t)$ from the infinite set of possible choices.  It will be clear
below why this chose is a convenient one.  Note that the choice in
(\ref{rythmos:apdx:eqn:sens:adj-trans}) does not in and of itself uniquely
determine $\Lambda(t)$ as no boundary conditions have yet be specified.  When
we move toward the end of the derivation, a natural choice for a boundary
condition that uniquely specifies $\Lambda(t)$ will fall out.

The next step in the derivation is somewhat non-direct but is based on simple
principles that have simple and clear mathematical foundations.  Here we
substitute (\ref{rythmos:apdx:eqn:sens:adj-trans}) into
(\ref{rythmos:apdx:eqn:sens:weak-direct-c-2}), add
%
\[
\left. \left( \frac{\partial h}{\partial y} \frac{\partial y}{\partial p} \right) \right|_{t=t_f}
\]
%
to both side of the equation, and then rearrange to yield
%
\begin{eqnarray}
\int_{t_0}^{t_f} \left( \frac{\partial g}{\partial y} \frac{\partial y}{\partial p} \right) dt
+ \left. \left( \frac{\partial h}{\partial y} \frac{\partial y}{\partial p} \right) \right|_{t=t_f}
& = &  \int_{t_0}^{t_f} \left( \Lambda^H \frac{\partial c}{\partial p} \right) dt
- \left. \left( \Lambda^H \frac{\partial c}{\partial \dot{y}} \frac{\partial y}{\partial p} \right) \right|_{t=t_0}
\nonumber \\
& & - \left. \left[ \left(
    \Lambda^H \frac{\partial c}{\partial \dot{y}}
    +  \frac{\partial h}{\partial y}
  \right)  \frac{\partial y}{\partial p} \right] \right|_{t=t_f}.
\label{rythmos:apdx:eqn:sens:weak-direct-c-3}
\end{eqnarray}
%
By rearranging (\ref{rythmos:apdx:eqn:sens:d_d_hat_d_p}) to
%
\begin{equation}
\frac{\partial \hat{d}}{\partial p} 
= \left[
    \int_{t_0}^{t_f} \left(
      \frac{\partial g}{\partial y} \frac{\partial y}{\partial p} \right) dt
      + \left. \left( \frac{\partial h}{\partial y} \frac{\partial y}{\partial p} \right) \right|_{t=t_f}
  \right]
+ \left[
    \int_{t_0}^{t_f} \frac{\partial g}{\partial p} dt
    + \left. \left( \frac{\partial h}{\partial p} \right) \right|_{t=t_f}
  \right]
\label{rythmos:apdx:eqn:sens:d_d_hat_d_p-2}
\end{equation}
%
and substituting in (\ref{rythmos:apdx:eqn:sens:weak-direct-c-3}) gives
%
\begin{eqnarray}
\frac{\partial \hat{d}}{\partial p} 
& = & \int_{t_0}^{t_f} \left( \Lambda^H \frac{\partial c}{\partial p} \right) dt
- \left. \left( \Lambda^H \frac{\partial c}{\partial \dot{y}} \frac{\partial y}{\partial p} \right) \right|_{t=t_0}
\nonumber \\
& & + \int_{t_0}^{t_f} \frac{\partial g}{\partial p} dt
    + \left. \left( \frac{\partial h}{\partial p} \right) \right|_{t=t_f}
\nonumber \\
& & - \left. \left[ \left(
    \Lambda^H \frac{\partial c}{\partial \dot{y}}
    +  \frac{\partial h}{\partial y}
  \right)  \frac{\partial y}{\partial p} \right] \right|_{t=t_f}
\label{rythmos:apdx:eqn:sens:d_d_hat_d_p-3}
\end{eqnarray}
%
We are not finished yet since as we stated earlier, the choice for the
weighting functions $\Lambda(t)$ have not yet been uniquely specified.
Looking at the two endpoint conditions in
(\ref{rythmos:apdx:eqn:sens:d_d_hat_d_p-3}) involving $\Lambda^H$, we see that
there is an expression at $t=t_0$ and an expression at $t=t_f$, and we can
choice either one of these as a boundary condition to close the adjoint
equation.  While in theory either of these endpoint conditions could be chosen
for the adjoint equation to specify different weighting functions
$\Lambda(t)$, as stated in [???], only the condition at $t=t_f$ of
%
\begin{equation}
\left. \left(
  \Lambda^H \frac{\partial c}{\partial \dot{y}}
  +  \frac{\partial h}{\partial y}
\right) \right|_{t=t_f}
 = 0,
\label{rythmos:apdx:eqn:sens:adj-trans:fc}
\end{equation}
%
yields a stable DAE which can be integrated backward in time.  Finally,
substituting (\ref{rythmos:apdx:eqn:sens:adj-trans:fc}) and $\partial y_0 /
{}\partial p$ for $\partial y / {}\partial p$ in the second expression at
$t=t_0$ in (\ref{rythmos:apdx:eqn:sens:d_d_hat_d_p-3}) we arrive the final
expression for reduced derivative
%
%
\begin{equation}
\frac{\partial \hat{d}}{\partial p} =
\int_{t_0}^{t_f} \left(
    \frac{\partial g}{\partial p}
    + \Lambda^H \frac{\partial c}{\partial p}
  \right) dt
  + \left. \left( \frac{\partial h}{\partial p} \right) \right|_{t=t_f}
  - \left. \left( \Lambda^H \frac{\partial c}{\partial \dot{y}} \frac{\partial y}{\partial p} \right) \right|_{t=t_0}.
\label{rythmos:apdx:eqn:sens:d_d_hat_d_p_final}
\end{equation}

In conclusion, (\ref{rythmos:apdx:eqn:sens:adj-trans}) and
(\ref{rythmos:apdx:eqn:sens:adj-trans:fc}) define the set of adjoint DAE
equations that solve for $\Lambda(t)$ that are integrated backwards in time
and (\ref{rythmos:apdx:eqn:sens:d_d_hat_d_p_final}) gives the expression for
the computation of $\partial {}\hat{d} / {}\partial p$ in terms of the
computed adjoint $\Lambda(t)$.

\subsubsection{Derivation of adjoint sensitivities for transient parameters}

\subsection{Derivation of adjoint sensitivities using reduced Lagrangian}

In this section we consider the derivation of adjoint approaches for computing
sensitivities of the reduced response function $\hat{d}(p,v(t_0,t_f))$ with
respect to $p$ and $v(t)$ for $t\in[t_0,t_f]$ using a Lagrangian approach.  The
primary devise that that we use here is the Lagrangian which is a special
function which is the weighted composition of the response function and the
DAE constraints.  This derivation follows in a similar manner from the one in
[???].  The advantage of the Lagrangian approach is that the algebraic
manipulations are more straightforward but the disadvantage is that it lacks
the more comfortable approaches used in the above direct sensitivity weak-form
derivations for the adjoint.

The Lagrangian is
%
\begin{eqnarray}
L(y(t_0:t_f),p,v(t_0:t_f),\Lambda(t_0:t_f))
& = & \int_{t_0}^{t_f} g(y(t),p,v(t),t) dt + h(y(t_f),p) \nonumber \\
& & + \int_{t_0}^{t_f} \Lambda(t)^H c(\dot{y}(t),y(t),p,v(t),t) dt,
\label{rythmos:apdx:eqn:sens:L}
\end{eqnarray}
\begin{tabbing}
\hspace{4ex}where\hspace{1ex}\= \\
\>	$\Lambda(t) \in \mathcal{C}|\mathcal{G}$ are the Lagrange multipliers defined on $t\in[t_0,t_f]$,
\end{tabbing}
%
and $\Lambda(t)^H\in\mathcal{G}|\mathcal{C}$ is the adjoint operator
of $\Lambda(t)$.

A few remarks about the Lagrangian function in (\ref{rythmos:apdx:eqn:sens:L})
are in order before we get into the detailed derivations.  While the
Lagrangian function can have a significant theoretical interpretation in a
variety of contexts, it is usually primarily just a convenient means to derive
sensitivities of constrained problems.  For instance, it is well known that
the variations of the Lagrangian set to zero represent the first-order
optimality conditions for the constrained optimization problem of minimizing
(or maximizing) (\ref{rythmos:apdx:eqn:sens:d}) (with $n_g=1$ of course)
subject to
(\ref{rythmos:apdx:eqn:sens:c})--(\ref{rythmos:apdx:eqn:sens:c:ic}).

In the following derivations, we will plug the implicit state function
$y(p,v(t_0:t),t)$ into (\ref{rythmos:apdx:eqn:sens:L}), forming a reduced
Lagrangian $\hat{L}(p,v(t_0:t_f))$ and then take variations with
respect to $p$ and $v(t_0:t_f)$.  From these variations, followed by
some other manipulations, the adjoint equation will fall out as well
as the other derivative computations.

Note: I am a little very uncomfortable with this approach of plugging
$y(p,v(t_0:t),t)$ into the Lagrangian and then taking variations.  However,
this approach works correctly for steady-state problems and therefore should
also work for transient problems.  Therefore, for now, I am just going to
accept that I don't fully appreciate the foundation for this general approach
but I will just accept that it is bullet proof and gives the right result
every time.

\subsubsection{Sensitivities for steady-state parameters and the adjoint equation}

In this section we will derive the expressions for $\partial {}\hat{d}
/ {}\partial p$ and in the process will derive the adjoint equation
and boundary conditions that solves for $\Lambda(t)$.  We start by
taking the variation of (\ref{rythmos:apdx:eqn:sens:L}) (through the
implicit function $y(p,v(t_0:t),t)$) with respect to $p$ and we claim
that this will be equal to $\partial {}\hat{d} / {}\partial p$.  This
variation is
%
\begin{eqnarray}
\frac{\partial \hat{d}}{\partial p}
& = & \frac{\partial \hat{L}}{\partial p}
\nonumber \\
& = & \int_{t_0}^{t_f} \left( \frac{\partial g}{\partial y} \frac{\partial y}{\partial p} +  \frac{\partial g}{\partial p} \right) dt
+ \left. \left(  \frac{\partial h}{\partial y} \frac{\partial y}{\partial p} + \frac{\partial h}{\partial p} \right) \right|_{t=t_f}
\nonumber \\
& & + \int_{t_0}^{t_f} \Lambda^H \left( \frac{\partial c}{\partial \dot{y}} \frac{d}{dt}\left( \frac{\partial y}{\partial p} \right)
   + \frac{\partial c}{\partial y} \frac{\partial y}{\partial p} +  \frac{\partial c}{\partial p} \right) dt.
\label{rythmos:apdx:eqn:sens:d_L_hat_d_p}
\end{eqnarray}
%
Substituting the integration by parts
%
\begin{equation}
\int_{t_0}^{t_f} \left[ \left( \Lambda^H \frac{\partial c}{\partial \dot{y}} \right) \frac{d}{dt}\left( \frac{\partial y}{\partial p} \right) \right] dt
= \left. \left( \Lambda^H \frac{\partial c}{\partial \dot{y}} \frac{\partial y}{\partial p} \right) \right|_{t_0}^{t_f}
- \int_{t_0}^{t_f} \left[ \frac{d}{dt}\left( \Lambda^H \frac{\partial c}{\partial \dot{y}} \right) \frac{\partial y}{\partial p} \right] dt
\end{equation}
%
into (\ref{rythmos:apdx:eqn:sens:d_L_hat_d_p}) and rearranging yields
%
\begin{eqnarray}
\frac{\partial \hat{d}}{\partial p}
& = & \int_{t_0}^{t_f} \left(
    \frac{\partial g}{\partial p}
    + \Lambda^H \frac{\partial c}{\partial p}
  \right) dt
  + \left. \left( \frac{\partial h}{\partial p} \right) \right|_{t=t_f}
  - \left. \left( \Lambda^H \frac{\partial c}{\partial \dot{y}} \frac{\partial y}{\partial p} \right) \right|_{t=t_0}
\nonumber \\
& & + \int_{t_0}^{t_f} \left(
    - \frac{d}{dt}\left( \Lambda^H \frac{\partial c}{\partial \dot{y}} \right)
    +  \Lambda^H \frac{\partial c}{\partial y} +  \frac{\partial g}{\partial y}
  \right) \frac{\partial y}{\partial p} dt
\nonumber \\
& & + \left. \left[ \left(
    \Lambda^H \frac{\partial c}{\partial \dot{y}}
    +  \frac{\partial h}{\partial y}
  \right)  \frac{\partial y}{\partial p} \right] \right|_{t=t_f}
\label{rythmos:apdx:eqn:sens:d_L_hat_d_p_2}.
\end{eqnarray}
%
We will now identify the integrand in the second integral in
(\ref{rythmos:apdx:eqn:sens:d_L_hat_d_p_2}) as the adjoint equation and we
will force it to be satisfied as
%
\begin{equation}
- \frac{d}{dt}\left( \Lambda^H \frac{\partial c}{\partial \dot{y}} \right)
+  \Lambda^H \frac{\partial c}{\partial y} +  \frac{\partial g}{\partial y} = 0, \; t \in \left[ t_0, t_f \right].
\label{rythmos:apdx:eqn:sens:adj-trans-2}
\end{equation}
%
We must also specify boundary conditions for the adjoint equation in
(\ref{rythmos:apdx:eqn:sens:adj-trans-2}).  As stated in [???] for index-0 and
index-1 DAEs, we can simply specify the endpoint condition
%
\begin{equation}
\left. \left(
  \Lambda^H \frac{\partial c}{\partial \dot{y}}
  +  \frac{\partial h}{\partial y}
\right) \right|_{t=t_f}
 = 0,
\label{rythmos:apdx:eqn:sens:adj-trans:fc-2}
\end{equation}
%
which drops out the last term in (\ref{rythmos:apdx:eqn:sens:d_L_hat_d_p_2})
and, together with (\ref{rythmos:apdx:eqn:sens:adj-trans-2}) and substituting
$\partial y_0 / {}\partial p$ for $\partial y / {}\partial p$ in the second
expression at $t=t_0$, leaves
%
\begin{equation}
\frac{\partial \hat{d}}{\partial p} =
\int_{t_0}^{t_f} \left(
    \frac{\partial g}{\partial p}
    + \Lambda^H \frac{\partial c}{\partial p}
  \right) dt
  + \left. \left( \frac{\partial h}{\partial p} \right) \right|_{t=t_f}
  - \left. \left( \Lambda^H \frac{\partial c}{\partial \dot{y}} \frac{\partial y_0}{\partial p} \right) \right|_{t=t_0}
\label{rythmos:apdx:eqn:sens:d_d_hat_d_p_final-2}
\end{equation}
%
which is the same as (\ref{rythmos:apdx:eqn:sens:d_d_hat_d_p_final}) derived
in another way.

\subsubsection{Adjoint sensitivities for transient parameters}

Here we consider the derivation of the sensitivity
%
\begin{equation}
\frac{\partial \hat{d}}{\partial v(t)} \in \mathcal{G}|\mathcal{V}, \; \mbox{for} \; t\in[t_0,t_f].
\end{equation}
%
Our derivation for these sensitivities for the transient parameters
$v(t)$ will be a little different than for the steady-state parameters
$p$.  The primary reason for using a different derivation is that the
infinite-dimensional transient parameters $v(t_0:t_f)$ are
fundamentally different than finite-dimensional steady-state parameters
$p$.  Because of the infinite-dimensional nature of $v(t_0,t_f)$, we
use a different tool of functional analysis.  The tool we use is the
application of the infinite-dimensional operator $\partial \hat{L} /
{}\partial v(t_0:t_f)$ to the infinite-dimensional perturbation
$\delta v(t_0:t_f)$ where we will then identify $\partial {}\hat{d} /
{}\partial v(t) = {}\partial {}\hat{L} / {}\partial v(t)$ in the
integrand of
%
\begin{equation}
\left( \frac{\partial \hat{d}}{\partial v(t_0:t_f)} \right) \delta v(t_0:t_f)
= \left( \frac{\partial \hat{L}}{\partial v(t_0:t_f)} \right) \delta v(t_0:t_f)
= \int_{t_0}^{t_f} \frac{\partial \hat{L}}{\partial v(t)} \delta v(t) dt
= \int_{t_0}^{t_f} \frac{\partial \hat{d}}{\partial v(t)} \delta v(t) dt.
\label{rythmos:apdx:eqn:sens:d_d_hat_d_v_delta_v}
\end{equation}
%
The integral in (\ref{rythmos:apdx:eqn:sens:d_d_hat_d_v_delta_v}) is the
definition of the inner product of two functions in an
infinite-dimensional space.
   
We begin our derivation of $\partial {}\hat{d} / {}\partial v(t)$ by
linearizing $\hat{L}(p,v(t_0:t_f))$ with respect to $\delta
v(t_0:t_f)$ as
%
\begin{eqnarray}
\left( \frac{\partial \hat{d}}{\partial v(t_0:t_f)} \right) \delta v(t_0:t_f)
& = & \left( \frac{\partial \hat{L}}{\partial v(t_0:t_f)} \right) \delta v(t_0:t_f)
\nonumber \\
& = & \int_{t_0}^{t_f} \left( \frac{\partial g}{\partial y} \frac{\partial y}{\partial v} \delta v +  \frac{\partial g}{\partial v} \delta v \right) dt
+ \left. \left(  \frac{\partial h}{\partial y} \frac{\partial y}{\partial v} \delta v \right) \right|_{t=t_f}
\nonumber \\
& & + \int_{t_0}^{t_f} \Lambda^H \left( \frac{\partial c}{\partial \dot{y}} \frac{d}{dt}\left( \frac{\partial y}{\partial v} \right) \delta v
   + \frac{\partial c}{\partial y} \frac{\partial y}{\partial v} \delta v +  \frac{\partial c}{\partial v} \delta v \right) dt.
\label{rythmos:apdx:eqn:sens:d_L_hat_d_v_delta_v}
\end{eqnarray}
%
Substituting the integration by parts
%
\begin{equation}
\int_{t_0}^{t_f} \left[ \left( \Lambda^H \frac{\partial c}{\partial \dot{y}} \right) \frac{d}{dt}\left( \frac{\partial y}{\partial v} \right) \delta v \right] dt
= \left. \left( \Lambda^H \frac{\partial c}{\partial \dot{y}} \frac{\partial y}{\partial v} \delta v \right) \right|_{t_0}^{t_f}
- \int_{t_0}^{t_f} \left[ \frac{d}{dt}\left( \Lambda^H \frac{\partial c}{\partial \dot{y}} \right) \frac{\partial y}{\partial v} \delta v \right] dt
\end{equation}
%
into (\ref{rythmos:apdx:eqn:sens:d_L_hat_d_v_delta_v}) and rearranging
yields
%
\begin{eqnarray}
\left( \frac{\partial \hat{d}}{\partial v(t_0:t_f)} \right) \delta v(t_0:t_f)
& = & \int_{t_0}^{t_f} \left[ \left(
    \frac{\partial g}{\partial v}
    + \Lambda^H \frac{\partial c}{\partial v}
  \right) \delta v \right] dt
  - \left. \left( \Lambda^H \frac{\partial c}{\partial \dot{y}} \frac{\partial y}{\partial v} \delta v \right) \right|_{t=t_0}
\nonumber \\
& & + \int_{t_0}^{t_f} \left[ \left(
    - \frac{d}{dt}\left( \Lambda^H \frac{\partial c}{\partial \dot{y}} \right)
    +  \Lambda^H \frac{\partial c}{\partial y} +  \frac{\partial g}{\partial y}
  \right) \frac{\partial y}{\partial v} \delta v \right] dt
\nonumber \\
& & + \left. \left[ \left(
    \Lambda^H \frac{\partial c}{\partial \dot{y}}
    +  \frac{\partial h}{\partial y}
  \right)  \frac{\partial y}{\partial v} \delta v \right] \right|_{t=t_f}
\label{rythmos:apdx:eqn:sens:d_L_hat_d_v_delta_v_2}.
\end{eqnarray}
%
By applying our choice for the adjoint
(\ref{rythmos:apdx:eqn:sens:adj-trans-2}) and its final condition
(\ref{rythmos:apdx:eqn:sens:adj-trans:fc-2}) and noting that $y(t_0) =
y_0(p)$ is not a function so that of $v$ and therefore $\partial y /
{}\partial v(t) = 0$ at $t=t_0$, then what is left of
(\ref{rythmos:apdx:eqn:sens:d_L_hat_d_v_delta_v_2}) becomes
%
\begin{eqnarray}
\left( \frac{\partial \hat{d}}{\partial v(t_0:t_f)} \right) \delta v(t_0:t_f)
& = & \int_{t_0}^{t_f} \left[ \left(
    \frac{\partial g}{\partial v}
    + \Lambda^H \frac{\partial c}{\partial v}
  \right) \delta v \right] dt
\label{rythmos:apdx:eqn:sens:d_L_hat_d_v_delta_v_3}.
\end{eqnarray}

By comparing (\ref{rythmos:apdx:eqn:sens:d_L_hat_d_v_delta_v_3}) to
(\ref{rythmos:apdx:eqn:sens:d_d_hat_d_v_delta_v}) it is clear that
%
\begin{equation}
\frac{\partial \hat{d}}{\partial v(t)}^H
= \frac{\partial g}{\partial v(t)}^H + \frac{\partial c}{\partial v(t)}^H \Lambda(t)
\label{rythmos:apdx:eqn:sens:d_d_hat_d_v_t}
\end{equation}
%
is the reduced gradient that we are seeking.

\subsection{The adjoint, the reduced gradient and the augmented adjoint}

We now restate the adjoint in (\ref{rythmos:apdx:eqn:sens:adj-trans}) and
(\ref{rythmos:apdx:eqn:sens:adj-trans:fc}), and the reduced sensitivities in
(\ref{rythmos:apdx:eqn:sens:d_d_hat_d_p_final}) in standard column-wise form
for $\Lambda$ which yields the adjoint DAE equations
%
\begin{eqnarray}
- \frac{d}{dt}\left( \frac{\partial c}{\partial \dot{y}}^H \Lambda \right)
+  \frac{\partial c}{\partial y}^H \Lambda + \frac{\partial g}{\partial y}^H
& = & 0, \; t \in \left[ t_0, t_f \right],
\label{rythmos:apdx:eqn:sens:adj} \\
\left.\left( \frac{\partial c}{\partial \dot{y}}^H \Lambda \right)\right|_{t=t_f}
& = & - \left. \frac{\partial h}{\partial y}^H \right|_{y=y(t_f)},
\label{rythmos:apdx:eqn:sens:adj:fc}
\end{eqnarray}
%
and the reduced gradient expression
%
\begin{equation}
\frac{\partial \hat{d}}{\partial p}^H =
\int_{t_0}^{t_f} \left(
    \frac{\partial g}{\partial p}^H
    + \frac{\partial c}{\partial p}^H \Lambda
  \right) dt
  + \left. \frac{\partial h}{\partial p}^H \right|_{t=t_f}
  - \left. \left( \frac{\partial y_0}{\partial p}^H \frac{\partial c}{\partial \dot{y}}^H \Lambda \right) \right|_{t=t_0}.
\label{rythmos:apdx:eqn:sens:d_d_hat_d_p_final-2}
\end{equation}
%

As described in [???], for some classes of up to index-2 DAEs, the form of the
adjoint in
(\ref{rythmos:apdx:eqn:sens:adj})--(\ref{rythmos:apdx:eqn:sens:adj:fc}) may be
unstable while the {}\textit{augmented adjoint system}
%
\begin{eqnarray}
\frac{d}{dt}\left( \bar{\Lambda} \right)
+  \frac{\partial c}{\partial y}^H \Lambda + \frac{\partial g}{\partial y}^H \Lambda
& = & 0, \; t \in \left[ t_0, t_f \right],
\label{rythmos:apdx:eqn:sens:aug-adj-de} \\
\bar{\Lambda} + \frac{\partial c}{\partial \dot{y}}^H
& = & 0, \; t \in \left[ t_0, t_f \right],
\label{rythmos:apdx:eqn:sens:aug-adj-ae} \\
\left. \bar{\Lambda} \right|_{t=t_f}
& = & \left. \frac{\partial h}{\partial y}^H \right|_{y=y(t_f)},
\label{rythmos:apdx:eqn:sens:aug-adj:fc}
\end{eqnarray}
%
generally is stable when the forward DAE is stable.

\begin{SANDdistribution}
% External
% Housekeeping copies necessary for every unclassified report:
\SANDdistInternal{1}{9018}{Central Technical Files}{8945-1}
\SANDdistInternal{2}{0899}{Technical Library}{9610}
\SANDdistInternal{2}{0612}{Review \& Approval Desk}{4916}
% If report has a Patent Caution or Patent Interest, add this:
%\SANDdistInternal{3}{0161}{Patent and Licensing Office}{4916}
\end{SANDdistribution}

\end{document}
