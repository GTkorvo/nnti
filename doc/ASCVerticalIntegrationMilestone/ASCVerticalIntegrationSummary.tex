\documentclass[pdf,ps2pdf,11pt]{SANDreport}
\usepackage{pslatex}

%Local stuff
\usepackage{graphicx}
\usepackage{latexsym}
\input{rab_commands}

\raggedright

% If you want to relax some of the SAND98-0730 requirements, use the "relax"
% option. It adds spaces and boldface in the table of contents, and does not
% force the page layout sizes.
% e.g. \documentclass[relax,12pt]{SANDreport}
%
% You can also use the "strict" option, which applies even more of the
% SAND98-0730 guidelines. It gets rid of section numbers which are often
% useful; e.g. \documentclass[strict]{SANDreport}

% ---------------------------------------------------------------------------- %
%
% Set the title, author, and date
%

\title{\center
ASC Vertical Integration Milestone
}
\author{
Roscoe Bartlett (Technical PI) \\
Scott Collis (Management PI) \\
Todd Coffey \\
David Day \\
Rob Hoekstra \\
Russell Hooper \\
Roger Pawlowski \\
Eric Phipps \\
Denis Ridzal \\
Heidi Thornquist \\
Jim Willenbring \\
More authors ???\\ \\
Sandia National
Laboratories\footnote{ Sandia is a multiprogram laboratory operated by Sandia
Corporation, a Lockheed-Martin Company, for the United States Department of
Energy under Contract DE-AC04-94AL85000.}, Albuquerque NM 87185 USA, \\ \\
\date{}
}

% ---------------------------------------------------------------------------- %
% Set some things we need for SAND reports. These are mandatory
%
\SANDnum{SAND2007-xxx}
\SANDprintDate{??? 2007}
\SANDauthor{
Roscoe Bartlett (Technical PI) \\
Scott Collis (Management PI) \\
Todd Coffey \\
David Day \\
Rob Hoekstra \\
Russell Hooper \\
Roger Pawlowski \\
Eric Phipps \\
Denis Ridzal \\
Heidi Thornquist \\
Jim Willenbring \\
More authors ???\\ \\
\date{}
}

% ---------------------------------------------------------------------------- %
% The following definitions are optional. The values shown are the default
% ones provided by SANDreport.cls
%
\SANDreleaseType{Unlimited Release}
%\SANDreleaseType{Not approved for general release}

% ---------------------------------------------------------------------------- %
% The following definition does not have a default value and will not
% print anything, if not defined
%
%\SANDsupersed{SAND1901-0001}{January 1901}

% ---------------------------------------------------------------------------- %
%
% Start the document
%
\begin{document}

\maketitle

% ------------------------------------------------------------------------ %
% An Abstract is required for SAND reports
%

%
\begin{abstract}
%

??? ToDo: Fill this in ... ???

%
\end{abstract}
%

% ------------------------------------------------------------------------ %
% An Acknowledgement section is optional but important, if someone made
% contributions or helped beyond the normal part of a work assignment.
% Use \section* since we don't want it in the table of context
%
%\clearpage
%\section*{Acknowledgment}
%
%
%The format of this report is based on information found
%in~\cite{Sand98-0730}.

% ------------------------------------------------------------------------ %
% The table of contents and list of figures and tables
% Comment out \listoffigures and \listoftables if there are no
% figures or tables. Make sure this starts on an odd numbered page
%
%\clearpage
%\tableofcontents
%\listoffigures
%\listoftables

% ---------------------------------------------------------------------- %
% An optional preface or Foreword
%\clearpage
%\section{Preface}
%Although muggles usually have only limited experience with
%magic, and many even dispute its existence, it is worthwhile
%to be open minded and explore the possibilities.

% ---------------------------------------------------------------------- %
% An optional executive summary
%\clearpage
%\section{Summary}
%Once a certain level of mistrust and scepticism has
%been overcome, magic finds many uses in todays science
%and engineering. In this report we explain some of the
%fundamental spells and instruments of magic and wizardry. We
%then conclude with a few examples on how they can be used
%in daily activities at national Laboratories.

% ---------------------------------------------------------------------- %
% An optional glossary. We don't want it to be numbered
%\clearpage
%\section*{Nomenclature}
%\addcontentsline{toc}{section}{Nomenclature}
%\begin{itemize}
%\item[alohomora]
%spell to open locked doors and containers
%\end{itemize}

% ---------------------------------------------------------------------- %
% This is where the body of the report begins; usually with an Introduction
%
\SANDmain % Start the main part of the report


%
\section{Introduction}
%

Many challenges exist in using cutting-edge research-driven numerical
algorithms and solvers to impact challenging production-quality applications.
Even more challenges exist in allowing production-quality applications to help
drive the research of numerical algorithms in a significant way.  Further,
the challenges become more daunting when considering new predictive simulation
tools like invasive UQ, sensitivity analysis, and optimization methods.
Addressing these challenges requires dedicated and sustained efforts in
mathematical abstraction and design, algorithms design, application program
structuring and flexibility, and ultimately good object-oriented software
engineering.  The FY2007 ASC Level-2 Vertical Integration Milestone has
addressed many of these challenges by demonstrating the vertical integration
of many different types of numerical solvers in Trilinos [???] and
assimilating these algorithms with the application code Charon [???] in a way
that should serve as a model for other solvers and applications.

On the numerical algorithms side, it is non-trivial to develop and vertically
integrate state-of-the-art massively parallel numerical algorithm solvers
ranging all the way from basic linear algebra data-structures through linear
and nonlinear solvers up to transient solvers and optimization.  Each of these
different numerical algorithm solver components is developed by a different
team of experts to achieve highest level of quality in the implemented
algorithms in a way that a small single team of non-expert developers can not
match.  This effort is further complicated by the stringent parallel
scalability requirements that are part of capability computing [???].

To address the vertical integration problem, a major thrust of this milestone
was to further develop and demonstrate the Thyra interface layer in Trilinos
as a way to seamlessly integrate and plug-and-play different implementations of
various numerical solver components, each developed by different teams of
expert algorithm developers.  In short, the goal of Thyra within Trilinos is
to cleanly support nearly any vertical integration of numerical solvers that
makes sense mathematically in a scalable and maintainable way.

To demonstrate this vertical interoperability and integration, we focused on
the Charon [???] application code with an emphasis on semiconductor problems
related to the QASPR [???] program.  In particular, we targeted sensitivity
analysis and parameter estimation optimization problems [ToDo: Say some more
about this?].  As a bonus, we also demonstrated some of the developed
capabilities on reacting flow problems in Charon and in ??? problems in
Aria/SIERRA.

The bridge between numerical algorithms and application codes is software.
Numerical algorithms must be implemented in software and great effort and care
is required to achieve maximum robustness, speed, and scalability.
Integrating numerical software into application codes is often difficult and
tedious work and keeping the numerical solver software and the application
software up to date with each other is critical to allow the flow of ideas and
capabilities back and forth between algorithm developers and application
developers.  Good software engineering practices are needed to keep the
software ``bridge'' between algorithms and applications alive and productive
for all involved.

During the course of this milestone work, we have shown how to scalably manage
the vertical integration of advanced numerical solver algorithms from basic
linear algebra, linear solvers, nonlinear solvers, stability and bifurcation
methods, transient solvers, and simulation-constrained optimization.  We have
demonstrated various instances of vertically integrated solvers with several
different problems modeled in production applications.  We have also developed
a software integration process and infrastructure between Charon and Trilinos
through which algorithm developers and application developers can more closely
work together the yield faster, more robust, and better tailored numerical
algorithms to meet the specific needs of important application customers while
at the same time stimulating publishable numerical algorithm research.


%
\section{Trilinos algorithms and vertical integration with Thyra}
%

A primary goal of this work was to drive the development and vertical
integration of the algorithms in the packages Belos (block linear solvers),
Anasazi (block eigensolvers), Rythmos (time integrators), and MOOCHO
(optimization) in order to implement advanced solver/analysis/optimization
capabilities.  In addition, we incorporated several other Trilinos packages
including Epetra (linear algebra data structure packages), Ifpack (incomplete
factorization preconditioners), ML (multi-level algebraic preconditioners),
Amesos (direct sparse solvers), NOX (nonlinear equation solvers), LOCA
(stability and bifurcation methods) and more.

{\bsinglespace
\begin{figure}[p]
\begin{center}
%\fbox{
\includegraphics*[angle=270,scale=0.90
]{VerticalPackageIntegration}
%}
\end{center}
\caption{
\label{fig:VerticalPackageIntegration}
Example of six different levels of vertical integration that exist in a
transient ODE/DAE solver in Rythmos.  }
\end{figure}
\esinglespace}

Vertical integration of numerical algorithms referees to the nesting of
numerical solvers within each other to enable the construction of
sophisticated multi-component solver capabilities.  For example, a modularized
ODE/DAE transient solver such as is implemented Rythmos, shown in Figure
{}\ref{fig:VerticalPackageIntegration}, demonstrates at least six different
levels of vertical algorithm integration.  It is critically important that
these types of modular solvers be constructed in a way that allows for
almost any individual algorithm/solver component to be swapped out for a
version better suited to a given problem.  For example, while an implicit
backward Euler time stepper may be appropriate for one particular class of
problems, a higher-order implicit time stepper, such as a variable-order
variable step size BDF method, may be superior for another important class of
problems.  The same need for flexibility and modularity is required for the
selection of preconditioners, linear solvers, and other numerical algorithm
components.  Without this type of flexibility, application developers may be
justified in writing their own suboptimal algorithms due to an inability to
change an inappropriate component of a more general solver (there are many
examples of this in production codes).

Integrating independently developed numerical algorithms and software in an
efficient and manageable way requires the development of standard
object-oriented interfaces.  Standard interfaces break the non-scalable N-to-M
dependencies between different numerical software components into separate
scalable 1-to-N and 1-to-M dependencies.

Standard interfaces for these types of numerical algorithms have been developed
and refined in the Trilinos package {}\textit{Thyra}\footnote{The word
{}\textit{Thyra} means ``interface'', or ``grand entrance'' in Greek.}.  Thyra
is comprised of a layered set of abstract C++ interfaces to linear operators
and vectors, preconditioners and linear solvers, an interface to nonlinear
models called the ModelEvaluator, and a nonlinear equation solver interface.
Layered on top of the Thyra interfaces are Rythmos interfaces defining a
number of higher-level abstractions for time stepping and time integration
algorithms.  All of these interfaces are built on the concept of Abstract
Numerical Algorithms (ANA) where only the essential mathematical properties of
the objects are considered without any implementation specific details.  This
high-level ANA approach allows a for a level of generality, reuse, and
efficiency that is not possible with other types of approaches.

{\bsinglespace
\begin{figure}
\begin{center}
%\fbox{
\includegraphics*[angle=270,scale=0.90
]{ModelEvaluator}
%}
\end{center}
\caption{
\label{fig:ModelEvaluator}
Thyra ModelEvaluator interface, nonlinear abstract numerical algorithms,
nonlinear applications, and linear solver services which demonstrate the
decoupling and scalability of the Thyra standard interface approach. }
\end{figure}
\esinglespace}

Of particular significance to this milestone is the use of the Thyra nonlinear
ModelEvaluator interface and supporting software.  Through a single interface,
a variety of nonlinear problems are presented to NOX, LOCA, Rythmos and MOOCHO
using both Charon and Aria/SIERRA as shown in Figure
{}\ref{fig:ModelEvaluator}.  The Stratimikos component shown in Figure
{}\ref{fig:ModelEvaluator} is a Trilinos package built on the Thyra
preconditioner and linear solver interfaces that provides unified
parameter-driven access to a great number of preconditioner and linear solver
capabilities in Trilinos.  The ModelEvaluator and Stratimikos approach is a
clear demonstration of breaking up N-to-M dependencies which results in
scalable 1-to-N and 1-to-M dependencies.  Adding a new nonlinear ANA solver
only requires a single set of Thyra interfaces and then this solver can be
used by the entire set of application codes that support the needed
functionality.  Likewise, an application just needs to implement the single
ModelEvaluator interface, and then it can access all of the various supported
nonlinear solvers.  Finally, a new preconditioner or linear solver can be
wrapped under Stratimikos and then be available to all applications and
nonlinear ANAs that support the ModelEvaluator interface.  This type of reuse
and interoperability for massively parallel nonlinear applications and
algorithms has never been achieved to this degree before and this type of
reuse and interoperability will increase in the future.

This milestone work clearly highlights the effectiveness of the Thyra interface
layer and ANA approach as demonstrated by the variety of the different types
of vertically integrated solver configurations that were achieved and the
types of numerical problems that were solved.  Specific examples of different
vertical solver integrations and problems solved are given in Section
{}\ref{sec:demonstration}.


%
\section{Charon/QASPR}
%

ToDo: Rob: Please fill in about half a page of text describing this.


%
\section{Application and Trilinos development nightly integration testing}
%

Through the course of the milestone work, we found that in order to keep
moving forward and avoid backslides in capability (which happened Early on),
we where forced to implement the nightly building and testing of the
development versions of Charon and Trilinos.  Every night, we take what is in
the Charon and Trilinos development repositories and build the combined Charon
+ Trilinos application and run a large set of regression tests.  In the time
since we started nightly building and testing, the number of tests in the
Charon test suite has gone up from under 50 to over 100, and 30 of the new
tests are directly related to this milestone work.

In the execution of this nightly building and testing process, we have learned
many things about how to do continuous integration [???] of an application and
Trilinos and we have realized many important unplanned benefits and will reap
many more benefits in the future if this process is maintained and extended.
There are both production-related benefits and research-related benefits that
help both the application developers and the algorithm developers to achieve
their goals.  On the research side, this massively reduces the overhead
required for algorithm developers to try their algorithms out on production
quality problems.  Developing a numerical solver with a production problem
exposes the algorithm developer to a whole host of issues (e.g.\ poor scaling,
ill-conditioning, difficult convergence etc.) that are hard to replicate in
model problems.  On the production side, constant integration insures that the
application and Trilinos are always up to date and satisfying the
application's requirements.  Therefore, when it is time for a release, only a
final set of acceptance tests are required and then the codes can be branched
and released shortly after.  This helps to reduce a whole host of risks such
as slipped schedules and broken features\footnote{Also known as regressions}.

Bottom line, nightly building and testing of an application code and Trilinos
brings algorithm developers and application developers closer together --
exchanging ideas and concerns -- and makes Trilinos a more customer focused
effort while still helping drive publishable numerical algorithm solver
development and reduces barriers for new algorithms to have impact through
production application codes.


%
\section{Demonstration solver vertical integrations and calculations}
\label{sec:demonstration}
%

While many different numerical solver configurations where used to solve a
variety of different numerical problems in Charon and Aria during the
milestone work, some of the more noteworthy examples\footnote{Each of these
examples are noteworthy for different reasons.} are given below.

{}\noindent\textbf{Steady-state semiconductor current-matching parameter
estimation problems} [MOOCHO, Stratimikos, Belos, Ifpack, Epetra]: We solved
both single-point and multi-point inverse optimization problems for the QASPR
semiconductor model 5614.  Here, we used MOOCHO to solve a
simulation-constrained least-squares optimization problem to minimize the
deviation between the simulated current through the device and the target
current by manipulating the poorly know defect reaction parameters.  We showed
significant improvement in accuracy and speedup over non-invasive block-box
methods. [ToDo: RAB: We have not really compared to the non-invasive block-box
methods but we could if we wanted too].  In addition, the forward sensitivity
method using pseudo-block GMRES in Belos showed superior performance to
AztecOO (the current production iterative linear solver in Trilinos) [ToDo:
RAB: We need to modify the setup of this model and fix some problems in Belos
to really show this].

{}\noindent\textbf{Transient QASPR semiconductor forward simulation} [Rythmos,
NOX, Stratimikos, Belos, Ifpack, Epetra]: We used a high-order
accuracy-controlling implicit BDF integrator algorithm in Rythmos (along with
other mentioned vertically integrated packages) to solve forward simulation
problems for the QASPR 2n2222 semiconductor model on ??? processors
demonstrating X, Y, and/or Z [ToDo: RAB: We need actually get the results and
fill in what this demonstrates!].
 
{}\noindent\textbf{Transient QASPR semiconductor sensitivities} [Rythmos, NOX,
Stratimikos, Belos, Ifpack, Epetra]: We used the new forward sensitivity
solver in Rythmos to demonstrate the calculation of current sensitivities with
respect to defect reaction parameters.  The Rythmos/Charon implementation
produced the sensitivities faster with greater accuracy then was previously
possible using non-invasive block-box finite-difference methods [ToDo: RAB: We
don't have these results yet but this is my number one technical goal leading
up to the end of the milestone].
 
{}\noindent\textbf{Transient parameter-estimation problem with QASPR
semiconductor model} [MOOCHO, Rythmos, NOX, Stratimikos, Belos, Ifpack,
Epetra]: We used the new forward sensitivity solver in Rythmos to solve
transient parameter estimation problems using MOOCHO as the optimization
algorithm [ToDo: RAB: There is a small chance that we could have this working
once we have basic forward sensitivities working].

{}\noindent\textbf{Block eigen solve for reacting flow problem} [LOCA, NOX,
Anasazi, Stratimikos, Belos, ML, Epetra]: We solved a ??? problem for the ??? 
model on ??? processors demonstrating ??? [ToDo: RAB: We need to fill in the
details and be sure to highlight what this shows and what is impressive or
noteworthy about this].

{}\noindent\textbf{Design optimization problem with Aria/SIERRA} [MOOCHO,
Stratimikos, Belos, Ifpack, Epetra]: We have demonstrated the minimally
invasive optimization algorithm in MOOCHO on a ??? problem in Aria/SIERRA.
This work demonstrates that the approaches taken are general and applicable to
applications other than Charon. The development of the Aria ModelEvaluator
interface will pave the way for more nonlinear algorithms, such as Rythmos, to
be assimilated into Aria/SIERRA. [ToDo: RAB: We need a better description for
this problem and the final results].


%
\section{Conclusions}
%

There are a number of conclusions that we have drawn as a result of this
milestone work:

\begin{itemize}

{}\item Predictive simulations capable of answering tomorrows questions
requires moving beyond the basic forward solve and requires the incorporation
of invasive technologies for sensitivities, optimization, and other advanced
numerical algorithms.

{}\item Solving complex numerical problems (such as transient sensitivities)
to the highest quality with the greatest efficiency requires the vertical
integration of many different types of advanced numerical algorithms that can
be tailored to the specific problem.

{}\item The vertical integration of a large number of advanced numerical
algorithms requires the development and adoption of standard interfaces.

{}\item Thyra standard interfaces for linear operators and vectors,
preconditioners and linear solvers, nonlinear models, and nonlinear solvers
have allowed the vertical integration of a large variety of numerical solvers
and access to a variety of nonlinear applications.

{}\item Application codes must present themselves as a ModelEvaluator and then
hand over nearly complete control to the numerical solver(s) in order to take
full advantage of advanced nonlinear numerical algorithms.  Monolithic forward
time stepping application codes can not take advantage of these more
sophisticated solution techniques.

{}\item Nightly building and testing of the development versions of the
application and Trilinos:

  \begin{itemize}

  {}\item results in better production results and better research,

  {}\item brings algorithm developers and application developers closer
  together allowing for better exchange of ideas and concerns,

  {}\item makes Trilinos a more customer focused, and

  {}\item helping drive algorithm development and reduces barriers for new
  algorithms to have impact on production applications.

  \end{itemize}

{}\item Other application projects and scientific support software projects
should consider adopting the type of continuous integration that is used with
Charon + Trilinos that was developed as part of this milestone work.

\end{itemize}

[ToDo: RAB: We need to add some references here and fill in the missing
references above ...]


% ---------------------------------------------------------------------- %
% References
%
\clearpage
\bibliographystyle{plain}
\bibliography{references}
\addcontentsline{toc}{section}{References}

% ---------------------------------------------------------------------- %
% Appendices should be stand-alone for SAND reports. If there is only
% one appendix, put \setcounter{secnumdepth}{0} after \appendix
%
%\appendix
%\input{???}

%\begin{SANDdistribution}
%\end{SANDdistribution}

\end{document}
