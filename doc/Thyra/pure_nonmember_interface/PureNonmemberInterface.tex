\documentclass[pdf,ps2pdf,11pt]{SANDreport}
\usepackage{pslatex}

%Local stuff
\usepackage{graphicx}
\usepackage{latexsym}
%
% Command to print 1/2 in math mode real nice
%
\newcommand{\myonehalf}{{}^1 \!\!  /  \! {}_2}

%
% Command to print over/under left aligned in math mode
%
\newcommand{\myoverunderleft}[2]{ \begin{array}{l} #1 \\ \scriptstyle #2 \end{array} }

%
% Command to number equations 1.a, 1.b etc.
%
\newcounter{saveeqn}
\newcommand{\alpheqn}{\setcounter{saveeqn}{\value{equation}}
\stepcounter{saveeqn}\setcounter{equation}{0}
\renewcommand{\theequation}
	{\mbox{\arabic{saveeqn}.\alph{equation}}}}
\newcommand{\reseteqn}{\setcounter{equation}{\value{saveeqn}}
\renewcommand{\theequation}{\arabic{equation}}}

%
% Shorthand macros for setting up a matrix or vector
%
\newcommand{\bmat}[1]{\left[ \begin{array}{#1}}
\newcommand{\emat}{\end{array} \right]}

%
% Command for a good looking \Re in math enviornment
%
\newcommand{\RE}{\mbox{\textbf{I}}\hspace{-0.6ex}\mbox{\textbf{R}}}

%
% Commands for Jacobians
%
\newcommand{\Jac}[2]{\displaystyle{\frac{\partial #1}{\partial #2}}}
\newcommand{\jac}[2]{\partial #1 / \partial #2}

%
% Commands for Hessians
%
\newcommand{\Hess}[2]{\displaystyle{\frac{\partial^2 #1}{\partial #2^2}}}
\newcommand{\hess}[2]{\partial^2 #1 / \partial #2^2}
\newcommand{\HessTwo}[3]{\displaystyle{\frac{\partial^2 #1}{\partial #2 \partial #3}}}
\newcommand{\hessTwo}[3]{\partial^2 #1 / (\partial #2 \partial #3)}



%\newcommand{\Hess2}[3]{\displaystyle{\frac{\partial^2 #1}{\partial #2 \partial #3}}}
%\newcommand{\myHess2}{\frac{a}{b}}
%\newcommand{\hess2}[3]{\partial^2 #1 / (\partial #2 \partial #3)}

%
% Shorthand macros for setting up a single tab indent
%
\newcommand{\bifthen}{\begin{tabbing} xxxx\=xxxx\=xxxx\=xxxx\=xxxx\=xxxx\= \kill}
\newcommand{\eifthen}{\end{tabbing}}

%
% Shorthand for inserting four spaces
%
\newcommand{\tb}{\hspace{4ex}}

%
% Commands for beginning and ending single spacing
%
\newcommand{\bsinglespace}{\renewcommand{\baselinestretch}{1.2}\small\normalsize}
\newcommand{\esinglespace}{}



% If you want to relax some of the SAND98-0730 requirements, use the "relax"
% option. It adds spaces and boldface in the table of contents, and does not
% force the page layout sizes.
% e.g. \documentclass[relax,12pt]{SANDreport}
%
% You can also use the "strict" option, which applies even more of the
% SAND98-0730 guidelines. It gets rid of section numbers which are often
% useful; e.g. \documentclass[strict]{SANDreport}

% ---------------------------------------------------------------------------- %
%
% Set the title, author, and date
%

\title{\center
Pure Nonmember Function Interfaces to Abstract C++ Classes}
\author{
Roscoe A. Bartlett \\ Optimization/Uncertainty Estim \\ \\ Sandia National
Laboratories\footnote{ Sandia is a multiprogram laboratory operated by Sandia
Corporation, a Lockheed-Martin Company, for the United States Department of
Energy under Contract DE-AC04-94AL85000.}, Albuquerque NM 87185 USA, \\ }
\date{}

% ---------------------------------------------------------------------------- %
% Set some things we need for SAND reports. These are mandatory
%
\SANDnum{SAND2007-xxx}
\SANDprintDate{??? 2007}
\SANDauthor{
Roscoe A. Bartlett \\ Optimization/Uncertainty Estim \\ \\
}

% ---------------------------------------------------------------------------- %
% The following definitions are optional. The values shown are the default
% ones provided by SANDreport.cls
%
\SANDreleaseType{Unlimited Release}
%\SANDreleaseType{Not approved for general release}

% ---------------------------------------------------------------------------- %
% The following definition does not have a default value and will not
% print anything, if not defined
%
%\SANDsupersed{SAND1901-0001}{January 1901}

% ---------------------------------------------------------------------------- %
%
% Start the document
%
\begin{document}

\maketitle

% ------------------------------------------------------------------------ %
% An Abstract is required for SAND reports
%

%
\begin{abstract}
%
A pure nonmember function interface to an abstract C++ class might provide the
best approach for keeping clean minimal interfaces, insulating client code
from changes to the interface, and providing a uniform interface when other
nonmember nonfriend functions are used.
%
\end{abstract}
%

% ------------------------------------------------------------------------ %
% An Acknowledgement section is optional but important, if someone made
% contributions or helped beyond the normal part of a work assignment.
% Use \section* since we don't want it in the table of context
%
%\clearpage
%\section*{Acknowledgment}
%
%
%The format of this report is based on information found
%in~\cite{Sand98-0730}.

% ------------------------------------------------------------------------ %
% The table of contents and list of figures and tables
% Comment out \listoffigures and \listoftables if there are no
% figures or tables. Make sure this starts on an odd numbered page
%
\clearpage
\tableofcontents
%\listoffigures
%\listoftables

% ---------------------------------------------------------------------- %
% An optional preface or Foreword
%\clearpage
%\section{Preface}
%Although muggles usually have only limited experience with
%magic, and many even dispute its existence, it is worthwhile
%to be open minded and explore the possibilities.

% ---------------------------------------------------------------------- %
% An optional executive summary
%\clearpage
%\section{Summary}
%Once a certain level of mistrust and scepticism has
%been overcome, magic finds many uses in todays science
%and engineering. In this report we explain some of the
%fundamental spells and instruments of magic and wizardry. We
%then conclude with a few examples on how they can be used
%in daily activities at national Laboratories.

% ---------------------------------------------------------------------- %
% An optional glossary. We don't want it to be numbered
%\clearpage
%\section*{Nomenclature}
%\addcontentsline{toc}{section}{Nomenclature}
%\begin{itemize}
%\item[alohomora]
%spell to open locked doors and containers
%\end{itemize}

% ---------------------------------------------------------------------- %
% This is where the body of the report begins; usually with an Introduction
%
\SANDmain % Start the main part of the report

%
\section{Introduction}
%

Object-oriented programming has been used and refined for many decades in a
variety of programming languages.  Some of the most basic descriptions of
object orientation speak of {}\textit{programming with objects} and refer to
such concepts as {}\textit{object methods}, {}\textit{polymorphism}, and
{}\textit{encapsulation}.  These concepts can be expressed in different ways
in different programming languages and each language lends itself to different
idioms for how object orientation is to be used to its fullest.  Here we focus
on the C++ language and combine some of the more modern idioms being advocated
in C++ to propose an extension involving the consistent use of nonmember
functions to encapsulate external clients of an abstract interfaces from the
details and changes to the interface.  Here we will draw on the advice of
several respected authors on C++ programming
{}\cite{C++CodingStandards05,EffectiveC++3rd??}.

The focus here is on issues related to object-oriented programming in the C++
language and specifically the interaction between an abstract interface
(consisting of pure virtual functions), clients that use objects through the
interface, and subclasses of the interface that provide concrete
implementations of all of the member functions.

Our main goal here is to describe an approach for developing C++ interfaces
and encapsulation mechanisms which protect clients of an abstract C++
interface from changes to the interface's virtual functions.  This is
especially critical when the interface represents an important
interoperability mechanism and is part of a library which may have many
diverse and unaccessible external clients which the library developers can not
directly change.

What we want is to have an approach to developing, maintaining, and using
abstract C++ interfaces that:

\begin{itemize}

{}\item\textit{Provides for the absolute minimal abstract C++ interfaces}: An
abstract C++ interface is the critical specification of the capabilities of an
object which must be able to cover the needs of a large set of potential
clients and allow great flexibility and efficiency in the implemenation of
subclasses.  The more minimal an interface is, the more likely it will be
adopted for a larger community and the easier it will be to develop powerful
``Decorator'', ``Composite'' and other such general subclasses.  Minimal and
efficient interfaces are especially critical for interoperability.  The
integrity and the quality of the interface should suffer from having been
developed and used for many different clients which resulted in poor quality
maintaince.

{}\item\textit{Maintains a uniform, consistent, and convenient interface for
the clients of the abstract interface}: We want clients to be able to access
the capabilities of the object in a clean way that is robust to changes in the
interface.

{}\item\textit{Avoids many of the Gotchas associated with object oriented
programming in C++}: In particualar, we want to avoid problems associated with
overloaded virtual functions {}\cite[Item ???]{C++Gotchas???}, virtual
functions with default arguments {}\cite[Item ???]{C++Gotchas???}, and other
such problems.

{}\item\textit{Allows for changes to an abstract interface's virtual function
set in such as way consistent with the above goals}: As requirements become
more clear or change over the life cycle of a piece of software, changes to
the specification of the virtual function set will be inevitable in order to
satisfy the new requirements in an efficient an safe way, and to maintain a
minimal interface.  We want to avoid a sub-standard abstract interface that is
cluttered with backward-compatiable functions for older clients.

\end{itemize}

In particular, two idoms have been advocated that are designed to address many
of the above issues: The Nonvirtual Interface (NVI) idiom {}\cite[Item
39]{C++CodingStandards05}, and the non-member function idiom {}\cite[Item
44]{C++CodingStandards05}.  Other guidelines that are pertinent to our
discussion are ``perfer minimal classes to monolithic classes'' {}\cite[Item
33]{C++CodingStandards05}, ``perfer providing abstract interfaces''
{}\cite[Item 36]{C++CodingStandards05}, ``practice safe overridding''
{}\cite[Item 38]{C++CodingStandards05} {}\cite[Gotcha 74]{C++Gotchas03}, and
``avoid overloading virtual functions'' {}\cite[Gotcha 73]{C++Gotchas03}.

The Nonvirtual Interface (NIV) idiom [???] advocates making all virtual
functions non-public (i.e.\ either private or protected) and making all public
functions nonvirtual.  For example, we might have an interface that looks like:

{\small\begin{verbatim}
  class BlobBase {
  public:
    // Non-virtual public interface
    void foo(int a=0) { implNonconstFoo(a); }
    void foo(int a=0) const { implFoo(a); }
  protected: // or private:
    // Pure virtual non-public functions to be overridden
    virtual void implNonconstFoo(int a) = 0;
    virtual void implFoo(int a) const = 0;
  };
\end{verbatim}}

The details of the NVI idiom are given in {}\cite[Item
???]{C++CodingStandards05} and {}\cite[Item ???]{EffectiveC++3rd??} but
basically the idiom allows clients to call regular member functions and
overloaded member function on the object without the problems associated with
overloaded virtual functions and it also avoids problems with default function
arguments since the default values are only defined in the non-public,
nonvirtual function interface.

Another idiom that is advocated in {}\cite[Item 44]{C++CodingStandards05} and
{}\cite[Item ???]{EffectiveC++3rd??} is to perfer writing a function as a
nonmember function unless it needs access to private data.  This increases
encapsulation and improves modularity.  Typically, this idiom is described in
the context of concrete classes which actually have private data, but it is
applicable for abstract interfaces as well.  If some capability can be
performed just using the existing public interface, then that capability
should be implemented as a nonmember function.  Adding another nonvirtual
function to the interface (or worse making the new function virtual with a
default implementation) mostly just clutters up the abstract interface and
complicates maintenance.  For example, some function {}\texttt{goo(...)} could
be implemented in terms of {}\texttt{BlobBase\-::foo(int)} as:

{\small\begin{verbatim}
  void goo( BlobBase &obj ) {
    obj.foo(0);
    obj.foo(1);
  }
\end{verbatim}}

The NVI idiom and ``nonmember function'' idiom, can and should be used
together, but they are also somewhat at odds with each other.  The NVI idiom
means that all operations that are directly implemented as a virtual function
on the abstract interface would be accessed using the corresponding public
nonvirtual member function.  The ``nonmember function'' idiom means that all
other functions would be implemented as nonmember nonfriend functions.
However, the straigtforward combination of these two idiom has several
disadvantages:

\begin{itemize}

{}\item\textit{Client interface is a mix of member and nonmember functions}:
The most obvious disadvantage of having an interface composed of both
nonmember and member functions is that it can be hard for the developers of
client code to remember how to call an operation.  Is it
{}\texttt{obj.foo(i)}, or it is {}\texttt{foo(obj,i)}?

{}\item\textit{Changes the virtual function structure in the abstract
interface are difficult for clients to handle}: A change to the virtual
function structure requires that either clients be changed or that the
interface be polluted with public functions that no longer need direct access
to the nonpublic virtual functions.  For example, what if requirements for the
abstract interface change such that it would be beneficial, from a design
point of view, to change the specification of a virtual function.  The change
might involve a modification to the signiture and/or the behavior of the
function.  Such a change in the virtual function would naturally involve a
similar change in the corresponding public nonvirtual member function that
calls the virtual function.  Let's also assume that the current capability of
the function in question can be maintained through a simple function that
calls the newly updated function.  Now the problem; how do we implement this
change without impacting the current clients of the interface?  There are one
of two possibilities: a) move the function from a member function to a
nonmember function, or b) leave the current public nonvirtual member function
in the abstract interface and make it call the newly updated member function.
Both of these chocies are fraught with problems.

\end{itemize}

Let's examine the two possibilities for handling changes to the virtual
function structure of the abstract interface mentioned above.

First, if we want to keep the abstract interface minimal, then we want to
choose option 'a' which involves moving the old public nonvirtual member
function out of the interface nad making it a nonmember nonfriend function.
However, that would require changing the syntax by which all clients that
currently call the function.  This change is simple to make since one just
needs to replace {}\texttt{obj.foo(...)} with {}\texttt{foo(obj,...)} and one
could almost write a script to perform the refactoring.  However, this type of
automated refactoring could never be performed 100\% correctly and preexisting
clients outside of our control (i.e.\ clients of our libraries) could not be
changed easily.

Second, if we want to minimize the impact on existing clients (i.e.\ if our
library is widely used by external clients out of our control), then we wnat
to choose optioion 'b' to leave the old public nonvirtual function in the
abstract interface and to augment the interface with the new public nonvirtual
function corresponding to the refactored nonpublic virtual function.  The
problem with this approch of course is that it no longer maintains a minimal
clean interface.  Over time, such refactorings will result in a bloat of the
abstract interface which is discouraged by many experts in object-oriented
programming in C++ {}\cite[Item 33]{C++CodingStandards05}.

In the next section, an approach for addressing the problems of combining
these two idioms is presented which involves the adoption of a pure nonmember
function interface.

%
\section{The Pure Nonmember Function Interface Idiom}
%

Here we present an variation of the NVI idiom that is more complementary with
the ``nonmember function'' idiom.  The idea is the replace the pubic
nonvirtual member functions in the abstract interface with nonmember friend
functions.  Therefore, a simple interface using the ``pure nonmember function
interface'' idiom would look like:

{\small\begin{verbatim}
  namespace SomeNamespace {

  class class BlobBase;

  // Forward prototypes for nonmember functions that
  // will directly call virtual functions
  void foo( BlobBase & obj, int a=0);
  void foo( const BlobBase & obj, int a=0);

  class BlobBase {
    friend void foo( BlobBase & obj, int a);
    friend void foo( const BlobBase & obj, int a);
  protected: // or private:
    // Pure virtual non-public functions to be overridden
    virtual void implNonconstFoo(int a) = 0;
    virtual void implFoo(int a) const = 0;
  };

  } // namespace SomeNamespace
\end{verbatim}}

{}\noindent{}and the nonmember friend functions would be implemented as:

{\small\begin{verbatim}
  void SomeNamespace::foo( BlobBase & obj, int a)
  {
    foo.implNonconstFoo(a);
  }

  void SomeNamespace::foo( const BlobBase & obj, int a)
  {
    foo.implFoo(a);
  }
\end{verbatim}}

Other functions that can be implemented in terms of the existing capabilities
on the object without requiring special access would be implemented as
nonmember nonfriend functions such as:

{\small\begin{verbatim}
  void goo( BlobBase &obj ) {
    foo(obj,0);
    foo(obj,1);
  }
\end{verbatim}}

This appraoch has all of the same advantages of the NIV idiom with respect to
allowing for overloading without problems and for allowing for a single
defintion of default parameter values.  Note that it is critical that the
virtual functions themselves must remain non-public since we can't allow
clients to be calling these directly (for lots of reasons) and therefore,
these special nonmember functions must be friends.  Even through at first
sight replacing the public nonvirtual member functions with corresponding
nonmember friend functions looks to be more completed, there are several
advantages to doing this:

\begin{itemize}

{}\item\textit{The client accesses capabilities in a more consistent way}: For
example, a client would invoke every operation on an object using a nonmember
function, independent of how that function was treated.  For example, the
client would call {}\texttt{foo(obj)} or {}\texttt{goo(obj)} consistently as
nonmember functions without having to worry how these are implemented now or
in the future.

{}\item\textit{Changes to the structure of the virtual function set can be
handed without affecting clients and without cluttering the abstract
interface}: If a virtual function needs to be modified in some significant
way, the nonmember function that calls that virtual function can be changed,
and the old nonmember friend function can be turned into a plain nonmember
nonfriend function and removed from the abstract interface.

\end{itemize}

To see how changes to the virtual function structure can be handled without
impacting clients (other than require that they be recompiled), consider a new
set of requirements where the {}\texttt{foo()} functions need to be changed to
accept a {}\texttt{Bar} object (represented through its own abstract
interface) instead of just an integer, and also needs to accept an extra
boolean argument.  The update to the virtual functions and the direct
nonmember friend functions would look something like:

{\small\begin{verbatim}
  namespace SomeNamespace {

  class Bar;
  class BlobBase;

  // Forward prototypes for nonmember functions that
  // will directly call virtual functions
  void foo( BlobBase& obj, const Bar &bar, bool useDog = false );
  void foo( const BlobBase& obj, const Bar &bar, bool useDog = false );

  class BlobBase {
    friend void implNonconstFoo( BlobBase& obj, const Bar &bar, bool useDog );
    friend void implFoo( const BlobBase& obj, const Bar &bar, bool useDog );
  protected: // or private:
    // Pure virtual non-public functions to be overridden
    virtual void implNonconstFoo( const Bar &bar, bool useDog ) = 0;
    virtual void implFoo( const Bar &bar, bool useDog ) const = 0;
  };

  } // namespace SomeNamespace
\end{verbatim}}

{}\noindent{}where the direct nonmember friend functions now have the
implementation:

{\small\begin{verbatim}
  void SomeNamespace::foo( BlobBase & obj, const Bar &bar, bool useDog )
  {
    foo.implNonconstFoo(bar,useDog);
  }

  void SomeNamespace::foo( const BlobBase & obj,  const Bar &bar, bool useDog )
  {
    foo.implFoo(bar,useDog);
  }
\end{verbatim}}

Now, what about all of the clients that relied on the old definition of the
{}\texttt{foo()} functions?  In this case, let's assume that the old meaning
and behavior of the {}\texttt{foo()} functions can be retained by using a
default implementation of {}\texttt{Bar} called {}\texttt{DefaultBar} and a
value of {}\texttt{useDog=false} which gives the following nonmember nonfriend
functions:

{\small\begin{verbatim}
  void SomeNamespace::foo( BlobBase & obj, int a)
  {
    foo(obj,DefaultBar(a),false);
  }

  void SomeNamespace::foo( const BlobBase & obj, int a)
  {
    foo(obj,DefaultBar(a),false);
  }
\end{verbatim}}

The above new nonmember nonfriend {}\texttt{foo()} function overloads could
then be included in the same file as the other ``standard'' nonmember
functions where {}\texttt{goo()}, for instance, is also declared and defined.
After this refactoring, clients that currently use expressions like
{}\texttt{foo(obj,0)} now just need to be recompiled and that is it!

%
\section{The Full Impact of Changing Virtual Functions}
%

As mentioned eariler, there are clarifications, changes, and augmentations to
requirements for software that beg for changes in the structure and behavior
of the virtual funtions on an abstract interface.  The ``pure nonmember
function interface'' idiom described above takes care of insulating clients
from most types of changes to the virtual function set, but how do these
changes affect subclasses of the abstract C++ interface that override virtual
functions?  There are two main categories of subclasses of abstract C++
interfaces that are most important: a) those that are owned and controled by
the library, and b) those that are developed but external users out of the
control of the library.  Subclasses can also be classified as i) those that
are direct subclasses of the base abstract interface (e.g.\ such as
``Decorator'' and ``Composite'' subclasses), and ii) those that are indirect
subclasses of the base abstract interface.  Any subclasses that are owned by
or accessible by the library developers can be easily changed is most cases.
Also, indirect subclasses can also largely be insulated from changes to the
base abstract interface in many cases if the intermediate subclasses are
designed well.  The key then is to create a set of appropriate intermediate
subclasses, tailored to specific types of use cases, owned by the library,
that most external subclass will derive from.  The remaining stumbling block
are those inaccessible external subclasses that directly derive from the base
abstract interface (i.e.\ type 'b.i' subclasses as defined above).

Graphically, the impact of changing the virtual functions in a base interface
class are shown in Figure ???

ToDo: Give some good examples of this!

ToDo: Show the class digram mentioned above!

% ---------------------------------------------------------------------- %
% References
%
\clearpage
\bibliographystyle{plain}
\bibliography{references}
\addcontentsline{toc}{section}{References}

% ---------------------------------------------------------------------- %
% Appendices should be stand-alone for SAND reports. If there is only
% one appendix, put \setcounter{secnumdepth}{0} after \appendix
%
%\appendix
%%
\section{Guidelines for reformatting of source code}
\label{sec:reformatting-guidelines}
%

When a sufficiently common coding style is not being used by all developers in
a project and no recommendations for a common coding style exists, then some
guidelines are needed for the situations where code written by one individual
is modified by another individual that uses a different coding style.  These
guidelines address how developers should conduct themselves when modifying
source files written largely by someone else.

\begin{enumerate}

{}\item First and foremost, each developer should respect the other
developers' formatting styles when modifying thier code.  If a
developer has a preferred Emacs style, then that style should be
listed explicitly at the top of each source file that is modified.
This will help other developers that use Emacs to stay consistent with
the file's style.

{}\item When only small changes are needed, a developer should abide by the
formatting style already in use in the file.  This helps to respect other
developers and helps to avoid needless changes for the version control system
to have to track.  Again, when user-defined file-specific Emacs styles are
specified, then it is easy to maintain a file's style when editing files
through Emacs.

{}\item Reformatting a file written by someone else and checking it in
is only justified if significant changes are made.  However, if a
developer needs to understand a complicated piece of code in order to
make perhaps even a small change in the end, then that developer may
also be justified in reformatting the file.  When a reformatting is
done, the new Emacs formatting style should be added to the top of the
source file in order to make it easier for the original owner of the
file and other developers to maintain the new style.

{}\item Multiple re-formats of the same file should not be checked in over and
over again as this will result in massive increases the the amount of
information that the version control system needs to keep track of and makes
diffs more difficult to perform.

\end{enumerate}

The above guidelines ensure that individuals are given maximal freedom to
format code to their liking but also helps to foster the shared ownership and
development of code.  In addition, the use of user-defined file-specific
formats makes it easy for developers to accommodate formatting styles
different from their own.


%\begin{SANDdistribution}
%\end{SANDdistribution}

\end{document}
