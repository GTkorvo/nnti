\documentclass[acmtoms,acmnow]{acmtrans2m}
\usepackage{graphicx}
%\usepackage{hyperref}

\newtheorem{theorem}{Theorem}[section]
\newtheorem{conjecture}[theorem]{Conjecture}
\newtheorem{corollary}[theorem]{Corollary}
\newtheorem{proposition}[theorem]{Proposition}
\newtheorem{lemma}[theorem]{Lemma}
\newdef{definition}[theorem]{Definition}
\newdef{remark}[theorem]{Remark}

\newcommand{\mychapter}[1]{\section{#1}}
\newcommand{\mysection}[1]{\subsection{#1}}
\newcommand{\mysubsection}[1]{\subsubsection{#1}}
\newcommand{\mysubsubsection}[1]{\subsubsection*{#1}}

%
% Command to print 1/2 in math mode real nice
%
\newcommand{\myonehalf}{{}^1 \!\!  /  \! {}_2}

%
% Command to print over/under left aligned in math mode
%
\newcommand{\myoverunderleft}[2]{ \begin{array}{l} #1 \\ \scriptstyle #2 \end{array} }

%
% Command to number equations 1.a, 1.b etc.
%
\newcounter{saveeqn}
\newcommand{\alpheqn}{\setcounter{saveeqn}{\value{equation}}
\stepcounter{saveeqn}\setcounter{equation}{0}
\renewcommand{\theequation}
	{\mbox{\arabic{saveeqn}.\alph{equation}}}}
\newcommand{\reseteqn}{\setcounter{equation}{\value{saveeqn}}
\renewcommand{\theequation}{\arabic{equation}}}

%
% Shorthand macros for setting up a matrix or vector
%
\newcommand{\bmat}[1]{\left[ \begin{array}{#1}}
\newcommand{\emat}{\end{array} \right]}

%
% Command for a good looking \Re in math enviornment
%
\newcommand{\RE}{\mbox{\textbf{I}}\hspace{-0.6ex}\mbox{\textbf{R}}}

%
% Commands for Jacobians
%
\newcommand{\Jac}[2]{\displaystyle{\frac{\partial #1}{\partial #2}}}
\newcommand{\jac}[2]{\partial #1 / \partial #2}

%
% Commands for Hessians
%
\newcommand{\Hess}[2]{\displaystyle{\frac{\partial^2 #1}{\partial #2^2}}}
\newcommand{\hess}[2]{\partial^2 #1 / \partial #2^2}
\newcommand{\HessTwo}[3]{\displaystyle{\frac{\partial^2 #1}{\partial #2 \partial #3}}}
\newcommand{\hessTwo}[3]{\partial^2 #1 / (\partial #2 \partial #3)}



%\newcommand{\Hess2}[3]{\displaystyle{\frac{\partial^2 #1}{\partial #2 \partial #3}}}
%\newcommand{\myHess2}{\frac{a}{b}}
%\newcommand{\hess2}[3]{\partial^2 #1 / (\partial #2 \partial #3)}

%
% Shorthand macros for setting up a single tab indent
%
\newcommand{\bifthen}{\begin{tabbing} xxxx\=xxxx\=xxxx\=xxxx\=xxxx\=xxxx\= \kill}
\newcommand{\eifthen}{\end{tabbing}}

%
% Shorthand for inserting four spaces
%
\newcommand{\tb}{\hspace{4ex}}

%
% Commands for beginning and ending single spacing
%
\newcommand{\bsinglespace}{\renewcommand{\baselinestretch}{1.2}\small\normalsize}
\newcommand{\esinglespace}{}


           
\markboth{Roscoe Bartlett et al.}{Linear Algebra Function Prototyes in C++}

\title{
A Simple Convention for the Specification of Linear Algebra Function
Prototypes in C++ }
            
\author{
ROSCOE A. BARTLETT \\
Sandia National Laboratories, Albuquerque NM 87185 USA}

%            
\begin{abstract} 
%
This short note describes a simple convention for the specification of C++
function prototypes for linear algebra operations with vectors and matrices
(or general linear operators).  This convention leads to function prototypes
that are derived directly from the mathematical expressions themselves (and
are therefore easy to remember), allow for highly optimized implementations
(through inlining in C++), and do not rely on any sophisticated C++ techniques
so that even novice C++ programmers can understand and step through the code in
a debugger.
%
\end{abstract}
%
            
\category{...}{...}{...}
            
\terms{Algorithms, Design, Performance, Standardization} 
            
\keywords{Object-Orientation, Vectors, Interfaces, C++, ...}
            
\begin{document}
            
\begin{bottomstuff} 
Sandia is a multiprogram laboratory operated by Sandia Corporation, a
Lockheed-Martin Company, for the United States Department of Energy
under Contract DE-AC04-94AL85000.
\end{bottomstuff}
            
\maketitle

\section{Introduction}

Linear algebra computations such as linear operation applications and
the solution of linear systems serve as the building blocks for
numerical algorithms and consume the majority of the runtime of
numerical codes.  These linear algebra abstractions transcend details
such as matrix storage formats (of which there are many) and linear
system solver codes (sparse or dense, direct or iterative).  Primary
linear algebra abstractions include vectors and matrices and the
operations that can be performed with them.  C++ abstractions for
vectors and matrices abound.

Once convenient vector and linear operator classes {}\texttt{Vec} and
{}\texttt{LinearOp} are created, for instance, there is a need to
implement BLAS-like linear algebra operations.  Given that C++ has
operator overloading, it would seem reasonable to implement these
operations using a MATLAB${}^{\mbox{\copyright}}$ like notation.  For
example, the linear operator applications $y = y + A^T x$ might be
represented in C++ with the statement {}\texttt{y = y + trans(A)*x}
(the character {}\texttt{'} can not be used for transpose since it is
not a C++ operator).  MATLAB is seen by many in the numerical
computational community to be the ideal for the representation of
linear algebra operations using only ASCII characters
{}\cite{ref:demmel_1997}.  The advantages of such an interface are
obvious; It is almost the same as standard mathematical notation,
which makes it very easy to match the implementation with the
operation for the application programmer, and makes the code much
easier to understand.  The primary disadvantage for this approach in
C++ is that the straightforward implementation requires a lot of
overhead because operators are implemented in a pair-wise fashion
involving temporaries.  For example, for the operation {}\texttt{y = y
+ trans(A)*x}, a temporary matrix ($n^2$ overhead) and two temporary
vectors ($2n$ overhead) would be created by the compiler.
Specifically, the compiler would perform the following operations:
{}\texttt{LinearOp t1 = trans(A); Vec t2 = t1*x; Vec t3 = y + t2; y =
t3;}.  Attempts have been made to come up with a strategy in C++ to
implement operations like {}\texttt{y = y + trans(A)*x} in a way where
little overhead is required beyond a direct BLAS call
{}\cite{ref:parker_1997}.  It is relatively easy to implement these
operator functions with only a little constant-time overhead for a
small set of linear algebra operations {}\cite[pages
675-677]{ref:stroustrup_1997}.  However, for more elaborate
expressions, a compile-time expression parsing method is needed.  Some
have advocated preprocessing tools, while others have looked at using
C++'s template mechanisms {}\cite{ref:veldhuizen_et_al_1998},
{}\cite{ref:parker_1997}.  In any case, these methods are complex and
not trivial to implement.  Methods based on runtime parsing are also
possible but add more of a runtime penalty.  Aliasing is also another
big problem.  For example, suppose we allow users to write expressions
like
%
\[
y = x + v + \alpha M^{T} + \beta y.
\]
%
An efficient parser that tries to minimize temporaries will have to scan the
entire expression and realize that $y = \beta y$ must be performed first and
then no temporaries are needed.  A naive parser may perform $y = x$ first and
then result in an incorrect evaluation.  The problem is that the more
efficient the parser the more complicated it is and the harder it will be for
inexperienced users to debug through this code.

Another problem is that operator-overloading implementations in C++ can never
generate error messages of the same quality as MATLAB.  Consider an expression
of the form {}\texttt{r = A*x + B*y + C*z} where the matrix {}\texttt{B} and
the vector {}\texttt{y} are mismatched.  Since MATLAB is an interpretive
language, the MATLAB interpreter can give a very good error message that gives
the file name, the line number and even the statement that caused the error.
Generating this type of error using operator overloading in C++ is generally
not possible.  One would have at the very least to open a debugger, set a
breakpoint, and then step back up the call stack in order to get the same
information.  Therefore, operator overloading using the current C++ standard
will never achieve the same level of usability as operator syntax in MATLAB.

Without using operator overloading to allow application code to use syntax
like {}\texttt{y = y + trans(A)*x}, how can linear algebra operations be
implemented efficiently?  The simple answer is to use regular functions
(member or non-member) inlined to call BLAS-like implementations.  For
example, for the operation $y = y + A^T x$, one might provide a function like
{}\texttt{add\_to\_multiply\_transpose(A,x,\&y);}.  It is trivial to implement
such a function to call the BLAS, for instance, with no overhead if a good
inlining C++ compiler is used.  The problem with using functions is that it is
difficult to come up with good names that users can remember.  For example,
the above operation has been called {}\texttt{Blas\_Mat\_Vec\_Mult(...)} in
LAPACK++ {}\cite{ref:pozo_1996}, {}\texttt{vm\_multadd(...)} in Meschach++
{}\cite{ref:roberts_et_al_1996}, and {}\texttt{mult(...)} in MTL
{}\cite{ref:lumsdaine_and_siek_1998}.  Even knowing the names of these
functions is not enough.  You must also know the order the arguments go in and
how are they passed.

{}\textbf{ToDo:} Mention recent ACM TOMS article 

\section{A convention for specifying function prototypes}

Here we consider a convention for constructing C++ function prototypes.
Function prototypes are constructed according to this convention where the
name of the function and the order of the arguments is easily composed from
the mathematical expression itself.  To illustrate the convention, consider
the operation $y = y + \alpha A^T x$.  First, rewrite the operation in the
form $y +\!= \alpha A^T x$ (this is well understood by C, C++ and Perl
programmers).  Next, translate into MATLAB-like notation as {}\texttt{y +=
alpha*A'*x} (except MATLAB does not have the operator {}\texttt{+=}).
Finally, for {}\texttt{Vec} objects {}\texttt{y} and {}\texttt{x} and a
{}\texttt{LinearOp} object {}\texttt{A}, the function call and its prototype are
shown in Figure {}\ref{rsqppp:fig:example_linalgpack_convention}.  In this
function prototype, type {}\texttt{Transp} is a simple C++ {}\texttt{enum}
with the values {}\texttt{TRANS} or {}\texttt{NOTRANS}, and the type
{}\texttt{Scalar} can be a simple typedef to {}\texttt{double} or might be a
template argument.

{\bsinglespace
\begin{figure}
\fbox{\parbox{\textwidth}{

{}\textbf{Function Call}\\

\begin{tabular}{c*{13}{@{\hspace{1ex}}c}}
\texttt{y}
	& \texttt{+=}
		& \texttt{alpha}
			& \texttt{*}
				& \texttt{A'}
					& \texttt{*}
						& \texttt{x}
							&
								& $y$
									& $\alpha$
										& $A^T$
											& $x$
												&
													& \\
$\downarrow$
	& $\downarrow$
		& $\downarrow$
			& $\downarrow$
				& $\downarrow$
					& $\downarrow$
						& $\downarrow$
							&
								& $\downarrow$
									& $\downarrow$
										& $\downarrow$
											& $\downarrow$
												&
													& \\
\texttt{V}
	& $\overbrace{\mbox{\texttt{p\_}}}$
		& \texttt{S}
			& \texttt{t}
				& \texttt{M}
					& \texttt{t}
						& \texttt{V}
							& \texttt{(}
								& \texttt{\&y,}
									& \texttt{alpha,}
										& $\overbrace{\mbox{\texttt{A, TRANS,}}}$
											& \texttt{x}
												& \texttt{)}
													& $\Longrightarrow$	\texttt{Vp\_StMtV(\&y,alpha,A,TRANS,x)}
\end{tabular}
\begin{tabbing}xxx\= \kill
\textbf{Function Prototype} \\
\texttt{void Vp\_StMtV( const Ptr<Vec>\& y, Scalar alpha, const LinearOp\& A, Transp A\_trans,}\\
\texttt{\hspace{4ex}const Vec\& x);}\\
\end{tabbing}

}}
\caption[Example of naming convention for functions for linear algebra function]{
\label{rsqppp:fig:example_linalgpack_convention}
Example of the linear algebra naming convention for $y +\!\!= \alpha A^T x$
}
\end{figure}
\esinglespace}


{\bsinglespace
\begin{figure}
\fbox{\parbox{\textwidth}{
\begin{tabular}{*{3}{l}}
\underline{\bf Operation}		& \underline{\bf Character (Lower Case)}	&	\\
\texttt{=}(assignment,equals)	& \texttt{\_}(underscore)					&	\\
\texttt{+=}(plus equals)		& \texttt{p\_}								&	\\
\texttt{-=}(minus equals)		& \texttt{m\_}								&	\\
\texttt{*=}(times equals)		& \texttt{t\_}								&	\\
\texttt{+}(addition,plus)		& \texttt{p}								&	\\
\texttt{-}(subtraction,minus)	& \texttt{m}								&	\\
\texttt{*}(multiplication,times)& \texttt{t}								&	\\
								&											&	\\
\underline{\bf Operand Type}	& \underline{\bf Character (Upper Case)}	& \underline{\bf Argument(s)}	\\
Scalar							& \texttt{S}								& \texttt{Scalar} \\
Vector							& \texttt{V}								& (rhs) \texttt{const Vec\&} \\
								&											& (lhs) \texttt{const Ptr<Vec>\&} \\
LinearOp							& \texttt{M}								& (rhs) \texttt{const LinearOp\&, Transp} \\
\end{tabular}
}}
\caption[Naming convention summary table for linear algebra functions]{
\label{rsqppp:tbl:linalgpack_naming_convention}
Naming convention for linear algebra functions in C++
}
\end{figure}
\esinglespace}


A summary of this convention is shown in Figure
{}\ref{rsqppp:tbl:linalgpack_naming_convention}.  Given this
convention, it is easy to go back and forth between the mathematical
notation and the function prototype. For example, consider the
function call and its mathematical expression\\[1.0ex]

{\bsinglespace
\hspace*{4ex}\parbox{\textwidth}{
{}\texttt{Vp\_StMtV( \&y, alpha, A, NOTRANS, x )}\\
$\Longrightarrow$\\
$y +\!= \alpha A x$.
}
\esinglespace}\\[1ex]

One difficulty with this convention is dealing with Level-2 and Level-3 BLAS
that have expressions such as\\[1ex]

\hspace*{4ex}$y = \alpha \: \mbox{op}(A) \: x + \underbrace{\beta}_{
	\mbox{?} } y$ \hspace{4ex} (xGEMM).\\[1ex]

Given $\beta \neq 1$ we can not simply rewrite the above BLAS operation using
+=.  To deal with this problem, $\beta$ is moved to the end of the argument
list and has a default value of 1.0 as\\[1.0ex]

\hspace*{4ex}\texttt{Vp\_StMtV( \&y, alpha, A, A\_trans, x,
  $\underbrace{\mbox{\texttt{beta}}}_{\mbox{default to 1.0}}$ )}.\\[1.0ex]

Only exact equivalents to the Level-2 and Level-3 BLAS need be
explicitly implemented (e.g. {}\texttt{Vp\_StMtV(...)}).  Functions
for simpler expressions can be generated automatically using template
functions.  As an example, consider the linear algebra operation and
its function call\\[1.0ex]

{\bsinglespace
\hspace*{4ex}\parbox{\textwidth}{
$y = A x$ \hspace{4ex} (xGEMV $\rightarrow y = \alpha \: \mbox{op}(A) x + \beta y$)\\
$\Longrightarrow$\\
\texttt{V\_MtV( \&y, A, NOTRANS, x )}.
}
\esinglespace}\\[1.0ex]

In the above example, the template function {}\texttt{V\_MtV(...)} can be
inlined to call {}\texttt{Vp\_StMtV(...)} which in turn can be inlined to call
the BLAS function {}\texttt{DGEMV(...)}, for instance.  The use of these
automatically generated functions makes the application code more readable and
also allows for specialization of these simpler operations later if desired.
The implementation of the above template function {}\texttt{V\_MtV(...)} is
trivial and is

{\bsinglespace\small
\begin{verbatim}
    template<class M, class V>
    inline void V_MtV( const Ptr<V>\& y, const M& A, Transp A_trans, const V& x )
    {
      Vp_StMtV( y, 1.0, A, A_trans, x, 0.0 );
    }
\end{verbatim}
\esinglespace}

Similar templated functions can also be generated for the partial
simplifications {}\texttt{Vp\_MtV(...)} and {}\texttt{V\_StMtV(...)}.

Longer expressions such as $y = \alpha A^T x + B z$ are easily handled using multiple
function calls such as\\[1ex]

{\bsinglespace
\hspace*{4ex}\parbox{\textwidth}{
$y = \alpha A^T x + B z$ \\
$\Longrightarrow$\\
\texttt{V\_StMtV( \&y, alpha, A, TRANS, x );}\\
\texttt{Vp\_MtV( \&y, B, NOTRANS, z );}
}
\esinglespace}

As stated above, only the base BLAS operation
{}\texttt{Vp\_StMtV(...)} (e.g. {}\texttt{xGEMV(...)}) must be
implemented for the specific vector and linear-operator/matrix types
{}\texttt{Vec} and {}\texttt{LinearOp}.  For example, if these are
simple encapsulations of BLAS compatible serial vectors and matrices
(e.g. TNT style) then the call the the BLAS functions can be written
as template functions for all serial dense vector and matrix (column
oriented) classes.  For example, an inlined call to
{}\texttt{DGEMV(...)} can be performed as\\[1ex]

{\bsinglespace\small
\begin{minipage}{\textwidth}
\begin{verbatim}
    template<class M, class V>
    inline void Vp_StMtV( const Ptr<V>& y, Scalar alpha, const M& A,
      Transp A_trans, const V& x, Scalar beta = 1.0 )
    {
      DGEMV( A_trans == NOTRANS ? 'N' : 'T', rows(A), cols(A), alpha
        ,&A(0,0), &A(0,1) - &A(0,0), &x(0), &x(1) - &x(0), beta
        ,&(*y)(0), &(*y)(1) - &(*y)(0) );
    }
\end{verbatim}
\end{minipage}
\esinglespace}

Of course the above function would also have to handle the cases where
{}\texttt{rows(A)} and/or {}\texttt{cols(A)} was 1 but the basic idea should
be clear.  By calling {}\texttt{rows(...)} and {}\texttt{cols(...)} as
nonmember functions, they can be overloaded to call the appropriate member
functions on the matrix object since there is no standard names for these.

When {}\texttt{Vec} and {}\texttt{LinearOp} are polymorphic types, one
can use a common trick to implement {}\texttt{Vp\_StMtV(...)} using
member functions.  For example, {}\texttt{Vp\_StMtV(...)} can be
implemented as\\[1ex]

{\bsinglespace\small
\begin{minipage}{\textwidth}
\begin{verbatim}
    class Vec { ... };

    class LinearOp {
    public:
      virtual void apply( const Ptr<Vec>& y, Scalar alpha, Transp A_trans,
        const Vec& x, Scalar beta ) const = 0;
      ...
    };

    inline void Vp_StMtV( const Ptr<Vec>& y, Scalar alpha, const LinearOp& A,
      Transp A_trans, const Vec& x, Scalar beta = 1.0 )
    {
      A.apply(y, alpha, A_trans, x, beta);
    }
\end{verbatim}
\end{minipage}
\esinglespace}\\[1ex]

Using these inlined non-member functions there is no extra overhead
beyond the unavoidable virtual function calls.  There is onsistent
calling of linear algebra operations irregardless whether the vector
and matrix objects are concrete or abstract.

This convention can, of course, also be used for level-1 vector-vector
operations like {}\texttt{Vt\_S(\&y,alpha)} for $y = \alpha y$,
{}\texttt{V\_V(\&y,x)} for $y = x$, {}\texttt{Vp\_StV(...)} for $y = y +
\alpha x$, {}\texttt{V\_VpV(\&z,x,y)} for $z = x + y$, and
{}\texttt{V\_VmV(\&z,x,y)} for $z = x - y$.  One can write as many of these
convenience wrapper functions as desired.

\section{Conclusions}

In summary, this convention makes it easy to write out correct calls to linear
algebra operations without having to resort to complex operator-overloading
techniques.  After all, the main appeal for operator overloading is to make it
easy for users to remember how the call linear algebra operations and to make
written code easier to read and interpret.  The convention described in this
note meets both of these goals and also results in code that is easy for
novice C++ developers to understand and debug through.  Debugging code can
easily take longer than writing it in the first place.  When concrete
abstractions of dense linear algebra types are used, it was shown that these
functions do not have to impose any overhead beyond direct BLAS calls if
inlining is used.  When polymorphic vector and matrix types are used, inlining
to call the virtual functions also results in no extra overhead.  This
convention has been used in the development of MOOCHO {}\cite{ref:moocho} and
in Thyra {}\cite{ref:thyra}.

{\bsinglespace
\bibliographystyle{acmtrans}
\nopagebreak
\scriptsize
\bibliography{bartlett_roscoe}
\esinglespace}

\begin{received}
Received: ???; revised: ???; accepted: ???
\end{received}
\end{document}


