\documentclass[pdf,ps2pdf,11pt]{SANDreport}
\usepackage{pslatex}

%Local stuff
\usepackage{graphicx}
\usepackage{latexsym}
%
% Command to print 1/2 in math mode real nice
%
\newcommand{\myonehalf}{{}^1 \!\!  /  \! {}_2}

%
% Command to print over/under left aligned in math mode
%
\newcommand{\myoverunderleft}[2]{ \begin{array}{l} #1 \\ \scriptstyle #2 \end{array} }

%
% Command to number equations 1.a, 1.b etc.
%
\newcounter{saveeqn}
\newcommand{\alpheqn}{\setcounter{saveeqn}{\value{equation}}
\stepcounter{saveeqn}\setcounter{equation}{0}
\renewcommand{\theequation}
	{\mbox{\arabic{saveeqn}.\alph{equation}}}}
\newcommand{\reseteqn}{\setcounter{equation}{\value{saveeqn}}
\renewcommand{\theequation}{\arabic{equation}}}

%
% Shorthand macros for setting up a matrix or vector
%
\newcommand{\bmat}[1]{\left[ \begin{array}{#1}}
\newcommand{\emat}{\end{array} \right]}

%
% Command for a good looking \Re in math enviornment
%
\newcommand{\RE}{\mbox{\textbf{I}}\hspace{-0.6ex}\mbox{\textbf{R}}}

%
% Commands for Jacobians
%
\newcommand{\Jac}[2]{\displaystyle{\frac{\partial #1}{\partial #2}}}
\newcommand{\jac}[2]{\partial #1 / \partial #2}

%
% Commands for Hessians
%
\newcommand{\Hess}[2]{\displaystyle{\frac{\partial^2 #1}{\partial #2^2}}}
\newcommand{\hess}[2]{\partial^2 #1 / \partial #2^2}
\newcommand{\HessTwo}[3]{\displaystyle{\frac{\partial^2 #1}{\partial #2 \partial #3}}}
\newcommand{\hessTwo}[3]{\partial^2 #1 / (\partial #2 \partial #3)}



%\newcommand{\Hess2}[3]{\displaystyle{\frac{\partial^2 #1}{\partial #2 \partial #3}}}
%\newcommand{\myHess2}{\frac{a}{b}}
%\newcommand{\hess2}[3]{\partial^2 #1 / (\partial #2 \partial #3)}

%
% Shorthand macros for setting up a single tab indent
%
\newcommand{\bifthen}{\begin{tabbing} xxxx\=xxxx\=xxxx\=xxxx\=xxxx\=xxxx\= \kill}
\newcommand{\eifthen}{\end{tabbing}}

%
% Shorthand for inserting four spaces
%
\newcommand{\tb}{\hspace{4ex}}

%
% Commands for beginning and ending single spacing
%
\newcommand{\bsinglespace}{\renewcommand{\baselinestretch}{1.2}\small\normalsize}
\newcommand{\esinglespace}{}


\newtheorem{dumb_fact}{Dumb Fact}[section]

% If you want to relax some of the SAND98-0730 requirements, use the "relax"
% option. It adds spaces and boldface in the table of contents, and does not
% force the page layout sizes.
% e.g. \documentclass[relax,12pt]{SANDreport}
%
% You can also use the "strict" option, which applies even more of the
% SAND98-0730 guidelines. It gets rid of section numbers which are often
% useful; e.g. \documentclass[strict]{SANDreport}

% ---------------------------------------------------------------------------- %
%
% Set the title, author, and date
%

\title{\center
A Linear Algebra Interpretation of Non-Euclidean Scalar Products and Vector
Spaces}
\author{
Roscoe A. Bartlett \\ Optimization/Uncertainty Estim \\ \\ Sandia National
Laboratories\footnote{ Sandia is a multiprogram laboratory operated by Sandia
Corporation, a Lockheed-Martin Company, for the United States Department of
Energy under Contract DE-AC04-94AL85000.}, Albuquerque NM 87185 USA, \\ }
\date{}

% ---------------------------------------------------------------------------- %
% Set some things we need for SAND reports. These are mandatory
%
\SANDnum{SAND2006-xxx}
\SANDprintDate{??? 2005}
\SANDauthor{
Roscoe A. Bartlett \\ Optimization/Uncertainty Estim \\ \\
}

% ---------------------------------------------------------------------------- %
% The following definitions are optional. The values shown are the default
% ones provided by SANDreport.cls
%
\SANDreleaseType{Unlimited Release}
%\SANDreleaseType{Not approved for general release}

% ---------------------------------------------------------------------------- %
% The following definition does not have a default value and will not
% print anything, if not defined
%
%\SANDsupersed{SAND1901-0001}{January 1901}

% ---------------------------------------------------------------------------- %
%
% Start the document
%
\begin{document}

\maketitle

% ------------------------------------------------------------------------ %
% An Abstract is required for SAND reports
%

%
\begin{abstract}
%
Many numerical algorithms are derived, analyzed and expressed with respect to
Euclidean vector spaces.  However, many applied mathematicians have shown the
utility of explessing and implementing many different types of numerical
algorithms with respect to non-Eucliean vector spaces.  Coming from a
functional analysis background, it natural to express many types of numerical
algorithms in non-Eucliean form by introducing the notion of an scalar (or
inner) product.  The introduction of a non-Eucliean vector space an scalar
product fundamentally changes the meaning of linear operators and other
constructs commonly used the express numerical algorithms.  The purpose of
this paper is to provide a foundation for understanding the meaning and
implications of expressing numerical algorithms in non-Eucliean form.  This
discussion requires no background in functional analysis and is based purely
on basic finite-dimensional linear algebra.  The goal is to provide the reader
with a level of confidence in expressing numerical algorithms in non-Eucliean
form.  A simple procedure is presented for taking any numerical algorithm
expressed using Euclidean vector spaces and translating it to non-Euclidean
form in the most general way possible.  Examples and analysis of the issues
involved are demonstrated for different types of numerical algorithms such as
Newton methods, quasi-Newton methods, optimization globalization methods, and
inequality constraints.
%
\end{abstract}
%

% ------------------------------------------------------------------------ %
% An Acknowledgement section is optional but important, if someone made
% contributions or helped beyond the normal part of a work assignment.
% Use \section* since we don't want it in the table of context
%
\clearpage
\section*{Acknowledgment}
The authors would like to thank ...

The format of this report is based on information found
in~\cite{Sand98-0730}.

% ------------------------------------------------------------------------ %
% The table of contents and list of figures and tables
% Comment out \listoffigures and \listoftables if there are no
% figures or tables. Make sure this starts on an odd numbered page
%
\clearpage
\tableofcontents
\listoffigures
%\listoftables

% ---------------------------------------------------------------------- %
% An optional preface or Foreword
%\clearpage
%\section{Preface}
%Although muggles usually have only limited experience with
%magic, and many even dispute its existence, it is worthwhile
%to be open minded and explore the possibilities.

% ---------------------------------------------------------------------- %
% An optional executive summary
%\clearpage
%\section{Summary}
%Once a certain level of mistrust and scepticism has
%been overcome, magic finds many uses in todays science
%and engineering. In this report we explain some of the
%fundamental spells and instruments of magic and wizardry. We
%then conclude with a few examples on how they can be used
%in daily activities at national Laboratories.

% ---------------------------------------------------------------------- %
% An optional glossary. We don't want it to be numbered
%\clearpage
%\section*{Nomenclature}
%\addcontentsline{toc}{section}{Nomenclature}
%\begin{itemize}
%\item[alohomora]
%spell to open locked doors and containers
%\end{itemize}

% ---------------------------------------------------------------------- %
% This is where the body of the report begins; usually with an Introduction
%
\SANDmain % Start the main part of the report

written in terms of Euclidean vector spaces where dot products are used for
the scalar inner product.  For example, the inner loop of a linear conjugate
gradient (CG) method $k=0\ldots{}$ for solving $A x = b$, initialized using $r
= r_0 = b - A x_0$, is often written in Euclidean (or dot-product) form as
%
\begin{eqnarray*}
\rho_k & = & r^H r, \\
p & = & r + \frac{\rho_k}{\rho_{k-1}} p, \\
q & = & A p, \\
\alpha & = & \frac{\rho_k}{p^H q}, \\
x & = & x + \alpha p, \\
r & = & r - \alpha q.
\end{eqnarray*}
%
An experienced mathematician will look at the above algorithm an immediately
write down the generalized form by replacing the dot products $r^H r$ and $p^H
q$ with the scalar products $<r,r>$ and $<p,q>$ and restate the inner loop of
the above CG algorithm as
%
\begin{eqnarray*}
\rho_k & = & <r,r>, \\
p & = & r + \frac{\rho_k}{\rho_{k-1}} p, \\
q & = & A p, \\
\alpha & = & \frac{\rho_k}{<p,q>}, \\
x & = & x + \alpha p, \\
r & = & r - \alpha q.
\end{eqnarray*}
%

Just as with linear CG, many numerical algorithms expressed in Euclidean form
with dot products and Euclidean norms $||.||_2$ (such as various optimization
algorithms, stability analysis methods, time integration methods, etc.) have
straightforward extensions to non-Euclidean vectors and vector spaces.  What we
would like is to have a straightforward process by which we can analyze many
different types of existing numerical algorithms expressed in Euclidean form
and then write out, if possible, the more general non-Euclidean form of these
algorithms.  We also want to do this in such a way that we do not have to
revisit all of the mathematical assumptions and theorems that went into the
development of the algorithm.

One might ask the following questions.  What's the big deal in replacing dot
products with scalar products?  What is this scalar product $<.,.>$ and what
does this mean?  What is the relationship between vectors $p$ and $q$ for
algorithms stated in Euclidean form and in non-Euclidean form?  By what
justification can one just replace dot products like $p^H q$ with scalar
products $<p,q>$?  What other changes do we need to make due to this subtle
change of replacing dot products with scalar products?  How is the definition
of linear operators and other objects affected by the introduction of
non-Euclidean scalar products?  What does all of this buy you?  Here we seek
to answer all of these questions in a way that a person without knowledge of
functional analysis or other advanced mathematics can understand and
appreciate.  All that we assume is that the reader has a basic understanding
of linear algebra and a familiarity with multi-variable numerical algorithms
like Newton's method [???] for nonlinear equations.

Here we present a linear algebra interpretation of finite dimensional
non-Euclidean inner product spaces and how they influence numerical algorithms
and applications.  The goal of this treatment is to present this topic in a
way that non-mathematicians can understand and appreciate.  The basic approach
will be to show the relationship between typical Euclidean-based vectors and
vector spaces (i.e.\ where the dot product is used for the inner product and a
linear operator is equivalent to a matrix) and non-Euclidean basis
representations, vectors, and vector spaces (i.e.\ where the inner product is
defined by a positive-definite matrix and a linear operator is not necessarily
equivalent to a matrix).  What we will show is a straightforward way to take
many different types of numerical algorithms that are expressed in Euclidean
form and then to analyze them for non-Euclidean vectors and spaces and see if
they can be transformed for use with non-Euclidean spaces.  What we will show
is that the expression of a numerical algorithm in a non-Euclidean space is
essentially equivalent to performing a linear transformation of variables and
model functions except that we do not need to actually perform the
transformation at the model level which has many different advantages.

\section{Introduction to vector spaces, basis representations, scalar products, and natural norms}

In this section, we provide a quick overview of the concepts of
finite-dimensional vector spaces, vector basis and coefficient
representations, scalar products, and norms.  The mathematical system
described here is that of finite-dimensional Hilbert spaces [???].  Here we
show straightforward connections between Euclidean and non-Euclidean
representations of vectors.  In this introductory material, we deal with
general vectors in a complex space $\CPLX^n$ with complex scalar elements.

\subsection{Basis and coefficient representations of vectors and vector spaces}

Consider a complex-valued vector space $\mathcal{S} {}\subseteq {}\CPLX^n$ with
the basis vectors $e_i\in\CPLX^m$, for $i=1\ldots{}m$, such that any vector
$x\in\CPLX^n$ can be represented as the linear combination
%
\begin{equation}
x = \sum_{i=1}^{m} \tilde{x}_i e_i
\label{eqn:x_basis_sum}
\end{equation}
%
where $\tilde{x}\in\mathcal{S}$ is known as the {}\textit{coefficient vector}
for $x$ in the space $\mathcal{S}$.  In order for the set of vectors $\{e_i\}$
to form a valid basis, they must minimally be linearly independent and $m
{}\le n$ must be true.  In a finite dimensional setting, when we say that some
vector is in some space $\mathcal{S}$ what we means is that it can be composed
out of a linear combination of the space's basis vectors as shown in
(\ref{eqn:x_basis_sum}).

Another way to represent (\ref{eqn:x_basis_sum}) is in the matrix form
%
\begin{equation}
x = E \tilde{x}
\label{eqn:x_basis_matrix}
\end{equation}
%
where $E\in\CPLX^{n\times{}m}$ is called the {}\textit{Basis Matrix} who's
columns are the basis vectors for the space $\mathcal{S}$; in other words
%
\begin{equation}
E = {\bmat{cccc} e_1 & e_2 & {}\ldots & e_m {}\emat}.
\end{equation}
%

The basis matrix form (\ref{eqn:x_basis_matrix}) will allow us to use standard
linear algebra notation later in various types of derivations and
manipulations.

The choice of which of the two different representations of a vector $x$ or
$\tilde{x}$ has a dramatic impact on the interpretation of the operations in
a numerical algorithm.

\subsection{Standard {}\textit{vector} operations}

A few different types of operations can be performed on just the coefficients
for a set of vectors which have the same meaning for the vectors themselves.
These are the set of classic {}\textit{vector} operations of assignment to
zero, vector scaling, and vector addition which are stated as

\begin{itemize}

{}\item $x = 0$:

$x = E \tilde{x} = 0 \; \Rightarrow \; \tilde{x} = 0$ 

{}\item $z = \alpha x$:

$z = E \tilde{z} = \alpha x = \alpha E \tilde{x} = E ( \alpha
\tilde{x} ) \; \Rightarrow \; \tilde{z} = \alpha \tilde{x}$

{}\item $z = x + y$:

$z = E \tilde{z} = x + y = E \tilde{x} + E \tilde{y} = E ( \tilde{x} +
\tilde{y} ) \; \Rightarrow \; \tilde{z} = \tilde{x} + \tilde{y}$.

\end{itemize}

Note that other types of element-wise operations on the coefficients like
element-wise products and divisions are not equivalent to the corresponding
operations on the vectors themselves and are hence not {}\textit{vector}
operations.

\subsection{Square, invertible basis representations}

Up to this point, the vector space $\mathcal{S}$ can be a strict subspace of
$\CPLX^n$ since $m < n$ may be true.  We will now focus on the case where $m =
n$ which gives a nonsingular basis matrix $E\in\CPLX^{n\times{}n}$ that can be
used to represent any vector $x\in\CPLX^n$.  As a result, $E^{-1}$ is well
defined and can be used in our expressions and derivations.

\subsection{Definition of the scalar (or inner) product for a vector space}

Now consider the dot inner product of any two vectors $x,y\in\CPLX^n$ which
takes the well known form
%
\begin{equation}
x^H y = \sum_{i=1}^n \mbox{conjugate}(x_i) y_i.
\label{eqn:inner_x_y_sum}
\end{equation}
%
Using the substitution $x = E {}\tilde{x}$ and $y = E {}\tilde{y}$, the inner
product in (\ref{eqn:inner_x_y_sum}) can be represented as
%
\begin{equation}
x^H y = ( \tilde{x}^H E^H ) ( E \tilde{y} ) = \tilde{x}^H Q \tilde{y},
\label{eqn:inner_x_y_E_prod}
\end{equation}
%
where $Q = E^H E$ is a symmetric positive-definite matrix.  It is this matrix
$Q$ that is said to define the scalar (or inner) product of two coefficient
vectors $\tilde{x},\tilde{y}\in\mathcal{S}$ as
%
\begin{equation}
x^H y = <\tilde{x},\tilde{y}>_{\mathcal{S}} = \tilde{x}^H Q \tilde{y}.
\label{eqn:inner_prod_defined}
\end{equation}
%

\subsection{Definition of the natural norm for a vector space}

The natural norm $||.||_{\mathcal{S}}$ of a vector space is defined as
%
\begin{equation}
||x|| = \sqrt{x^H x} = \sqrt{<\tilde{x},\tilde{x}>_{\mathcal{S}}} = ||\tilde{x}||_{\mathcal{S}}
\label{eqn:natural_norm_defined}
\end{equation}
%
where $<\tilde{x},\tilde{x}>_{\mathcal{S}}$ is defined in
(\ref{eqn:inner_prod_defined}) in terms of the scalar product matrix $Q$.

\subsection{Orthonormal and orthogonal basis representations}

Note that all orthonormal sets of basis vectors, i.e.
\[
e_i^H e_j = \left\{ \begin{array}{ll} 1 & \mbox{if} \; i = j \\ 0 & \mbox{if} \; i \ne j \end{array}  \right.
\]
result in an \textit{orthogonal matrix}\footnote{
%
In most linear algebra text books and
literature, the term {}\textit{orthogonal matrix} is used to denote a matrix
who's columns are orthonormal.  This means that a matrix with just orthogonal
columns (i.e.\ $e_i^H e_j = \delta \ne 1$ when $i=j$) is not an orthogonal
matrix.  It would seem to make more sense that a matrix with orthogonal
columns should be called an ``orthogonal matrix'' and a matrix with
orthonormal columns should be called an ``orthonormal matrix'' but this is not
the standard use.
%
} $E$ that gives identity for the scalar product matrix $Q = E^H E = I$.
Therefore, all orthonormal sets of basis vectors result in a Euclidean scalar
product, even if the basis vectors are not Cartesian (i.e.\ $e_i^T {}\ne
{\bmat{ccccc} 0 & {}\ldots & 1 & {}\ldots & 0 {}\emat}$). Also note that all
orthogonal sets of basis vectors give a scalar product matrix $Q = E^H E$ that
is diagonal.

When the scalar product matrix $Q$ is diagonal, it is trivial to compute a
diagonal scaling matrix $Q^{\myonehalf}$ and then use this scaling matrix to
scale all vectors and operators before the numerical algorithm even sees them.
In these cases, it is questionable whether the more general concepts of scalar
products is worth the effort in expressing and implementing numerical
algorithms, which is our ultimate goal here.  Therefore, we are primarily
focused on problem that require more than simply diagonal scaling.

\subsection{Equivalence of basis representations and the scalar product}

One important detail to mention is that given a particular vector space
$\mathcal{S}$ with its corresponding scalar product defined using $Q$ in
(\ref{eqn:inner_prod_defined}), there are infinitely many different selections
for the basis $E$ that give the same scalar product.  To see this, let
$F\in\CPLX^{n\times{}n}$ be any orthogonal matrix (i.e.\ $F^H F = I$).  We can
the use a particular choice for $F$ to transform the scalar product as
%
\begin{equation}
x^H y = \tilde{x}^H Q \tilde{y}
= \tilde{x}^H ( E^H E ) \tilde{y}
= \tilde{x}^H E^H ( F^H F ) E \tilde{y}
= \tilde{x}^H ( F E )^H ( F E ) \tilde{y}
= ( \tilde{x}^H \bar{E}^H ) ( \bar{E} \tilde{y} )
= \bar{x}^H \bar{y}
\label{eqn:inner_x_y_F_E_prod}
\end{equation}
%
where $\bar{x} = {}\bar{E} {}\tilde{x}$, $\bar{x} = {}\bar{E} {}\tilde{x}$,
and $\bar{E} = F E$.  We see that $\bar{E}\in\CPLX^{n\times{}n}$ actually
forms a different vector space $\bar{\mathcal{S}}$, but, for the same
coefficient vectors, its scalar product is exactly the same as for
$\mathcal{S}$.  Therefore, when we define a vector space by its scalar
product, we are really defining a whole collection of vector spaces instead of
just one.  This is because there are infinitely many different sets of basis
vectors that give infinitely many different vector representations for a
particular set of coefficients but all have the same scalar product.

\subsection{Linear operators}

A Euclidean linear operator $A\in\CPLX^m|\CPLX^n$ is a object that maps
vectors from the spaces $\CPLX^n$ to $\CPLX^m$ as
%
\begin{equation}
y = A x,
\label{eqn:fwd_op_apply}
\end{equation}
%
where $x\in\CPLX^n$ and $y\in\CPLX^m$, and also obeys the linear properties
%
\begin{equation}
z = A(\alpha u + \beta v) = \alpha A u + \beta A v
\label{eqn:linear_op_properties}
\end{equation}
%
for all $\alpha,\beta\in\CPLX$ and $u,v\in\CPLX^n$.

For every Euclidean linear operator $A$ it is possible to define another
linear operator object associated with it called the {}\textit{adjoint}
Euclidean linear operator, denoted $A^H\in\CPLX^n|\CPLX^m$, which maps vectors
from the spaces $\CPLX^m$ to $\CPLX^n$ as
%
\begin{equation}
y = A^H x
\label{eqn:adjoint_op_apply}
\end{equation}
%
where $x\in\CPLX^m$ and $y\in\CPLX^n$.

For Euclidean linear operators $A$, the adjoint Euclidean linear operator
$A^H$ is equal to the matrix element-wise conjugate transpose, or $A^H =
(A)^H$ where we use the notation $(A)^H$ to denote the matrix element-wise
conjugate transpose.

There are also other forms of a linear operator that correspond to different
basis representations of the vectors.  Given a particular vector basis
representation $x = E {}\tilde{x}$, we will also define another form of a
linear operator which we refer to as the {}\textit{non-Euclidean coefficient}
form.

According to our notation, the Euclidean operator application is shown in
(\ref{eqn:fwd_op_apply}).  The {}\textit{non-Euclidean coefficient linear
operator} application is written as
%
\begin{equation}
\tilde{y} = \tilde{A} \tilde{x},
\label{eqn:fwd_op_apply_euclidean}
\end{equation}
%
where $\tilde{x}\in\mathcal{D}$, $\tilde{y}\in\mathcal{R}$, and
$\mathcal{D}\subseteq\CPLX^n$ and $\mathcal{R}\subseteq\CPLX^m$.  The
Euclidean $A$ and non-Euclidean coefficient $\tilde{A}$ linear operators are
related as $\tilde{A} = E_{\mathcal{R}}^{-1} A E_{\mathcal{D}}$ and $A =
E_{\mathcal{R}} {}\tilde{A} E_{\mathcal{D}}^{-1}$.  The relationship between
these two linear operators is explored in Section ???.

{}\begin{dumb_fact} An adjoint linear operator is not the same as the matrix
Hermitian transpose $(A)^H$ when dealing with non-Euclidean vector spaces.
{}\end{dumb_fact}

In other words, while the adjoint of the Euclidean linear operator $A^H$ is
equal to the Hermitian transpose of the forward Euclidean operator $A$, this is
not generally true for non-Euclidean linear operators.

The forward and adjoint non-Euclidean coefficient linear operators
$\tilde{A}$ and $\tilde{A}^H$, respectively, are related to each other with
respect to the scalar products through the adjoint relationship
%
\begin{equation}
v^H (A u)
= <\tilde{v},\tilde{A} \tilde{u}>_{\mathcal{R}}
= (A^H v)^H u
= <\tilde{A}^H \tilde{v},\tilde{u}>_{\mathcal{D}}
\label{eqn:adjoint_relation}
\end{equation}
%
for all $\tilde{u}\in\mathcal{D}$ and $\tilde{v}\in\mathcal{R}$.  In
(\ref{eqn:adjoint_relation}) we see the relationship between a linear
operator, its adjoint, and the scalar products associated with its range and
domain spaces.

A linear operator is refereed to as {}\textit{invertible} if another unique
linear operator $A^{-1}\in\mathcal{D}|\mathcal{R}$ exists such that
%
\[
A^{-1} A = A A^{-1} = I.
\]
%
Likewise, the {}\textit{inverse} linear operator $A^{-1}$ also has an
{}\textit{adjoint inverse} linear operator $A^{-H}\in\mathcal{R}|\mathcal{D}$
associated with it which satisfies
%
\[
A^{-H} A^H = A^H A^{-H} = I.
\]
%

While the adjoint of a non-Euclidean coefficient linear operator is not
generally equal to the Hermitian transpose of the forward linear operator, the
inverse does have the same relationship as in the Euclidean case.  In other
words, the inverse of a non-Euclidean coefficient linear operator
$\tilde{A}^{-1}$ is in fact equal to the matrix inverse of the forward
non-Euclidean linear operator $(\tilde{A})^{-1}$.  This will be shown out in
Section ???

Linear operators are used to represent a variety of different types of objects
in a numerical algorithm.  Even vectors $x\in\CPLX^n$ can be viewed as a
linear operator $x\in\CPLX^n|\CPLX$ where the domain space for the forward
operator is simply $\CPLX$ which gives the forward operator $y = x v$ (where
$v\in\CPLX$ and $y\in\CPLX^n$) and the adjoint operator $y = x^H v$ (where
$v\in\mathcal{S}$ and $y\in\CPLX$ ).  Here we see that a vector $x$ or
$\tilde{x}$ can take on a dual role.  In one role, $\tilde{x}$ is just an
array of coefficients with a scalar product attached to it.  In another role,
it can represent a linear operator where the action of $x^H y$ invokes the
scalar product $<\tilde{x},\tilde{y}>$.

\subsection{Dealing only with scalar products and vector coefficients in algorithm construction}

It is important to recognize that both a vector $x$ and its corresponding
coefficient vector $\tilde{x}$ (where $x = E\tilde{x}$) can be represented as
arrays of scalars in a computer program.  However, our goal is to go about
formulating and implementing numerical algorithms and applications so as to
only manipulate arrays of the natural coefficient vectors $\tilde{x}$ and
never manipulate the coefficients of the Euclidean representation of the
vectors $x$ themselves.  The reason that one would only want to deal with the
natural coefficients of the vectors in a vector space and the scalar product
is that it may be inconvenient and/or very expensive to build a set of basis
vectors so that the Euclidean form of the vectors themselves can be formed and
manipulated directly.  This is the case, for example, in many different
finite-element discretization methods for PDEs [???].

\section{Impact of non-Euclidean scalar products on matrix representations of linear operators}
\label{sec:matrix_representations_of_linear_operators}

As stated above, for every linear operator $A$ there is a corresponding
{}\textit{non-Euclidean coefficient linear operator} $\tilde{A}$.  In
addition, every finite-dimensional linear operator has one of several
potential matrix representations $\hat{A}$.  The different representations
depend on how the domain and range spaces relate to the matrix representation.

Here we consider the impact that non-Euclidean vector spaces and scalar
products have on Euclidean linear operators $A\in\mathcal{R}|\mathcal{D}$,
their non-Euclidean coefficient forms $\tilde{A}$, and their different
possible matrix representations $\hat{A}$.  We consider two such matrix
representations in the following two subsections, the ``natrual'' matrix
representation and the ``Euclidean'' matrix representation.

\subsection{The ``natural'' matrix representation of a linear operator}

First, lets consider the ``natural'' matrix representation $\hat{A}$ of a
linear operator $A$ in terms of the basis vectors for the spaces $\mathcal{D}$
and $\mathcal{R}$ which takes the form
%
\begin{equation}
A = E_{\mathcal{R}} \hat{A} E_{\mathcal{D}}^H.
\label{eqn:A_coeff_natural}
\end{equation}
%
Given this matrix form of $A$, the non-Euclidean coefficient form of the
linear operator is
%
\begin{eqnarray}
y
& = & E_{\mathcal{R}} \tilde{y} \nonumber \\
& = & A x \nonumber \\
& = & ( E_{\mathcal{R}} \hat{A} E_{\mathcal{D}}^H ) ( E_{\mathcal{D}} \tilde{x} ) \nonumber \\
& = & E_{\mathcal{R}} ( \hat{A} Q_{\mathcal{D}} \tilde{x} ) \nonumber \\
& = & E_{\mathcal{R}} ( \hat{A} Q_{\mathcal{D}} ) \tilde{x} \nonumber \\
&  & \Rightarrow \nonumber \\
\tilde{A} & = & \hat{A} Q_{\mathcal{D}}
\label{eqn:A_natural_apply}
\end{eqnarray}
%
where $Q_{\mathcal{D}} = E_{\mathcal{D}}^H E_{\mathcal{D}}$ is the scalar
product matrix for the space $\mathcal{D}$.  Hence, we see that applying the
operator $A$ using (\ref{eqn:A_coeff_natural}) to transform the vector
coefficients $\tilde{x}$ to $\tilde{y}$ involves injecting the scalar product
matrix $Q_{\mathcal{D}}$ before multiplying by the ``natural'' coefficient
matrix $\hat{A}$.  Using this notation, we differentiate between the adjoint
operator denoted $\tilde{A}^H$ and the Hermitian transpose of the forward
operator denoted $(\tilde{A})^H$.

Now lets consider the definition of the adjoint using (\ref{eqn:A_coeff_natural}) which is
%
\begin{eqnarray}
v
& = & E_{\mathcal{D}} \tilde{v} \nonumber \\
& = & A^H u \nonumber \\
& = & ( E_{\mathcal{D}} \hat{A}^H E_{\mathcal{R}}^H ) ( E_{\mathcal{R}} \tilde{u} ) \nonumber \\
& = & E_{\mathcal{D}} ( \hat{A}^H Q_{\mathcal{R}} \tilde{u} ) \nonumber \\
& = & E_{\mathcal{D}} ( \hat{A}^H Q_{\mathcal{R}} ) \tilde{u} \nonumber \\
&  & \Rightarrow \nonumber \\
\tilde{A}^H & = & \hat{A}^H Q_{\mathcal{R}}
\label{eqn:A_natural_apply_adjoint}
\end{eqnarray}
%
where $Q_{\mathcal{R}} = E_{\mathcal{R}}^H E_{\mathcal{R}}$ is the scalar
product matrix for the space $\mathcal{R}$.  This time, the application of the
adjoint requires the injection of the scalar product matrix
$Q_{\mathcal{R}}$.

Here we see the definition of the adjoint non-Euclidean coefficient linear
operator $\tilde{A}^H=\hat{A}^H Q_{\mathcal{R}}$ is not equal to the Hermitian
transpose of the forward non-Euclidean coefficient linear operator
$(\tilde{A})^H = Q_{\mathcal{D}}^H {}\hat{A}^H$.  Here we now see the critical
difference between a linear operator and a matrix when dealing with linear
operators that operate on the vector coefficients of vectors with
non-Euclidean basis representations.

\begin{dumb_fact}
When writing algorithms in vector coefficient form with non-Euclidean scalar
products, the adjoint non-Euclidean coefficient linear operator $\tilde{A}^H$
is not the same as the matrix conjugate transpose of the forward non-Euclidean
coefficient linear operator $\tilde{A}$.  In other words, using our notation,
$\tilde{A}^H {}\ne (\tilde{A})^H$ in general.
\end{dumb_fact}

It is easy to show that (\ref{eqn:A_natural_apply}) and
(\ref{eqn:A_natural_apply_adjoint}) satisfy the adjoint relationship
(\ref{eqn:adjoint_relation}) as
%
\begin{eqnarray}
<\tilde{A} \tilde{u}, \tilde{v}>_{\mathcal{R}}
& = & (\hat{A} Q_{\mathcal{D}} \tilde{u} )^H Q_{\mathcal{R}} (\tilde{v}) \nonumber \\
& = & ( \tilde{u}^H Q_{\mathcal{D}} \hat{A}^H ) Q_{\mathcal{R}} (\tilde{v}) \nonumber \\
& = & (\tilde{u}^H ) Q_{\mathcal{D}} ( \hat{A}^H Q_{\mathcal{R}} \tilde{v}) \nonumber \\
& = & <\tilde{u}, \tilde{A}^H \tilde{v}>_{\mathcal{D}} \Box
\label{eqn:adjoint_relation_proved}
\end{eqnarray}
%

If the linear operator $A$ is invertible such that $A^{-1}$ exists, then the
inverse non-Euclidean coefficient linear operation $\tilde{A}^{-1}$ is given
by
%
\begin{eqnarray}
y
& = & E_{\mathcal{D}} \tilde{y} \nonumber \\
& = & A^{-1} x \nonumber \\
& = & (E_{\mathcal{R}} \hat{A} E_{\mathcal{D}}^T)^{-1} (E_{\mathcal{R}} \tilde{x}) \nonumber \\
& = & E_{\mathcal{D}}^{-H} \hat{A}^{-1} (E_{\mathcal{R}}^{-1} E_{\mathcal{R}}) \tilde{x}  \nonumber \\
& = & ( E_{\mathcal{D}} E_{\mathcal{D}}^{-1} ) E_{\mathcal{D}}^{-H} \hat{A}^{-1} \tilde{x}  \nonumber \\
& = & E_{\mathcal{D}} (E_{\mathcal{D}}^{-1} E_{\mathcal{D}}^{-H}) \hat{A}^{-1} \tilde{x}  \nonumber \\
& = & E_{\mathcal{D}} ( Q_{\mathcal{D}}^{-1} \hat{A}^{-1} \tilde{x} ) \nonumber \\
& = & E_{\mathcal{D}} ( Q_{\mathcal{D}}^{-1} \hat{A}^{-1} ) \tilde{x} \nonumber \\
& \Rightarrow &  \nonumber \\
\tilde{A}^{-1} & = & Q_{\mathcal{D}}^{-1} \hat{A}^{-1}.
\label{eqn:A_natural_matrix_inverse_apply}
\end{eqnarray}
%
Therefore, applying the inverse of the natural coefficient representation of
linear operator to the vector coefficients involves applying the inverse of
the scalar product matrix $Q_{\mathcal{D}}^{-1}$.

The adjoint inverse non-Euclidean coefficient linear operator
$\tilde{A}^{-H}\in\mathcal{R}|\mathcal{D}$ is also easy to derive and is given
by
%
\begin{eqnarray}
y
& = & E_{\mathcal{R}} \tilde{y} \nonumber \\
& = & A^{-H} x \nonumber \\
& = & (E_{\mathcal{R}} \hat{A} E_{\mathcal{D}}^H)^{-H} (E_{\mathcal{D}} \tilde{x}) \nonumber \\
& = & E_{\mathcal{R}}^{-H} \hat{A}^{-H} E_{\mathcal{D}}^{-1} E_{\mathcal{D}} \tilde{x}  \nonumber \\
& = & ( E_{\mathcal{R}} E_{\mathcal{R}}^{-1} ) E_{\mathcal{R}}^{-H} \hat{A}^{-H} \tilde{x}  \nonumber \\
& = & E_{\mathcal{R}} (E_{\mathcal{R}}^{-1} E_{\mathcal{R}}^{-H}) \hat{A}^{-H} \tilde{x}  \nonumber \\
& = & E_{\mathcal{R}} ( Q_{\mathcal{R}}^{-1} \hat{A}^{-H} \tilde{x} ) \nonumber \\
& = & E_{\mathcal{R}} ( Q_{\mathcal{R}}^{-1} \hat{A}^{-H} ) \tilde{x} \nonumber \\
& \Rightarrow &  \nonumber \\
\tilde{A}^{-H} & = & Q_{\mathcal{R}}^{-1} \hat{A}^{-H}.
\label{eqn:A_natural_matrix_adjoint_inverse_apply}
\end{eqnarray}
%
Therefore, applying the adjoint inverse of the natural coefficient
representation of linear operator to the vector coefficients involves applying
the inverse of the scalar product matrix $Q_{\mathcal{R}}^{-1}$.

Here we see that the inverse non-Euclidean coefficient forward and adjoint
linear operators $\tilde{A}^{-1} $ and $\tilde{A}^{-H}$, respectively,
actually are to the simple matrix inverses of the non-Euclidean coefficient
forward and adjoint linear operators $\tilde{A} $ and $\tilde{A}^H$,
respectively.

\begin{dumb_fact}
The inverse non-Euclidean coefficient linear operator $\tilde{A}^{-1}$
actually is the same as the matrix inverse of the forward non-Euclidean
coefficient linear operator $\tilde{A}$.  In other words, using our notation,
$\tilde{A}^{-1} = (\tilde{A})^{-1}$.
\end{dumb_fact}

\subsection{The ``Euclidean'' matrix representation of a linear operator}
\label{sec:Euclidean_matrix_rep_linear_operator}

Now consider another matrix representation of a linear operator where the
forward operator application (\ref{eqn:fwd_op_apply}) is implemented as
%
\begin{equation}
\tilde{y} = \tilde{A} \tilde{x} = \hat{A} \tilde{x}
\label{eqn:Euclidean_forward_op_apply}
\end{equation}
%
where $x = E_{\mathcal{D}}\tilde{x}$ and $y = E_{\mathcal{R}}\tilde{y}$.  This
representation is quite common in many different codes and makes good sense in
many cases.

Given the matrix representation of the forward operator application in
(\ref{eqn:Euclidean_forward_op_apply}) one can derive the adjoint operator
from the adjoint relationship as
%
\begin{eqnarray}
<\tilde{A} \tilde{u}, \tilde{v}>_{\mathcal{R}}
& = & ( \hat{A} \tilde{u} )^H Q_{\mathcal{R}} (\tilde{v}) \nonumber \\
& = & ( \tilde{u}^H \hat{A}^H ) Q_{\mathcal{R}} (\tilde{v}) \nonumber \\
& = & \tilde{u}^H ( Q_{\mathcal{D}} Q_{\mathcal{D}}^{-1} ) \hat{A}^H Q_{\mathcal{R}} \tilde{v} \nonumber \\
& = & ( \tilde{u}^H ) Q_{\mathcal{D}} ( Q_{\mathcal{D}}^{-1} \hat{A}^H Q_{\mathcal{R}} \tilde{v} ) \nonumber \\
& = & <\tilde{u}, \tilde{A}^H \tilde{v}>_{\mathcal{D}} \nonumber \\
&  & \Rightarrow \nonumber \\
\tilde{A}^H & = & Q_{\mathcal{D}}^{-1} \hat{A}^H Q_{\mathcal{R}}
\label{eqn:A_adjoint_Euclidean_defined}
\end{eqnarray}
%
From (\ref{eqn:A_adjoint_Euclidean_defined}) we can see that applying the
adjoint in this case requires that the inverse of the scalar product matrix
$Q_{\mathcal{D}}^{-1}$ be applied.

From (\ref{eqn:Euclidean_forward_op_apply}) or
(\ref{eqn:A_adjoint_Euclidean_defined}), one can derive the exact
representation of the operator $A$ that is consistent with this matrix
representation.

First, from (\ref{eqn:Euclidean_forward_op_apply}) we see that
%
\begin{eqnarray}
y
& = & E_{\mathcal{R}} \tilde{y} \nonumber \\
& = & E_{\mathcal{R}} ( \hat{A} \tilde{x} ) \nonumber \\
& = & E_{\mathcal{R}} \hat{A} ( E_{\mathcal{D}}^{-1} E_{\mathcal{D}} ) \tilde{x} \nonumber \\
& = & ( E_{\mathcal{R}} \hat{A} E_{\mathcal{D}}^{-1} ) ( E_{\mathcal{D}} \tilde{x} ) \nonumber \\
& = & A x \nonumber \\
&  & \Rightarrow \nonumber \\
A & = & E_{\mathcal{R}} \hat{A} E_{\mathcal{D}}^{-1}
\label{eqn:A_Euclidean}
\end{eqnarray}
%

We can also derive the representation of $A$ from
(\ref{eqn:A_adjoint_Euclidean_defined}) as
%
\begin{eqnarray}
y
& = & E_{\mathcal{D}} \tilde{y} \nonumber \\
& = & E_{\mathcal{D}} ( Q_{\mathcal{D}}^{-1} \hat{A}^H Q_{\mathcal{R}} \tilde{x} ) \nonumber \\
& = & E_{\mathcal{D}} ( E_{\mathcal{D}}^H E_{\mathcal{D}} )^{-1} \hat{A}^H ( E_{\mathcal{R}}^H E_{\mathcal{R}} ) \tilde{x} \nonumber \\
& = & E_{\mathcal{D}} ( E_{\mathcal{D}}^{-1} E_{\mathcal{D}} ^{-H} ) \hat{A}^H ( E_{\mathcal{R}}^H E_{\mathcal{R}} ) \tilde{x} \nonumber \\
& = & ( E_{\mathcal{D}} E_{\mathcal{D}}^{-1} ) ( E_{\mathcal{D}} ^{-H} \hat{A}^H E_{\mathcal{R}}^H ) ( E_{\mathcal{R}} \tilde{x} ) \nonumber \\
& = & A^H x \nonumber \\
&  & \Rightarrow \nonumber \\
A^H & = & E_{\mathcal{D}}^{-H} \hat{A}^H E_{\mathcal{R}}^H  \nonumber \\
\label{eqn:A_Euclidean_adjoint} 
&  & \Rightarrow \nonumber \\
A & = & E_{\mathcal{R}} \hat{A} E_{\mathcal{D}}^{-1}
\label{eqn:A_Euclidean_again}
\end{eqnarray}
%
Note that we already know that $A$ in (\ref{eqn:A_Euclidean}) and
(\ref{eqn:A_Euclidean_again}) satisfies the adjoint relationship, since
(\ref{eqn:A_adjoint_Euclidean_defined}) was derived from the adjoint
relationship.

Given the ``Euclidean'' form of $A$ in (\ref{eqn:A_Euclidean}), the action of
the inverse linear operator $A^{-1}$ (should it exist) in the operation $y =
A^{-1} x$ is given by
%
\begin{eqnarray}
y
& = & E_{\mathcal{D}} \tilde{y} \nonumber \\
& = & A^{-1} x \nonumber \\
& = & (E_{\mathcal{R}} \hat{A} E_{\mathcal{D}}^{-1})^{-1} (E_{\mathcal{R}} \tilde{x}) \nonumber \\
& = & E_{\mathcal{D}} \hat{A}^{-1} (E_{\mathcal{R}}^{-1} E_{\mathcal{R}}) \tilde{x}  \nonumber \\
& = & E_{\mathcal{D}} ( \hat{A}^{-1} \tilde{x} ) \nonumber \\
& \Rightarrow &  \nonumber \\
\tilde{y} & = & \hat{A}^{-1} \tilde{x}.
\label{eqn:A_Euclidean_matrix_inverse_apply}
\end{eqnarray}
%

Likewise, the action of the adjoint inverse linear operator $A^{-H}$ of the
form $y = A^{-H} x$ is also easy to derive and is given by
%
\begin{eqnarray}
y
& = & E_{\mathcal{R}} \tilde{y} \nonumber \\
& = & A^{-H} x \nonumber \\
& = & (E_{\mathcal{R}} \hat{A} E_{\mathcal{D}}^{-1})^{-H} (E_{\mathcal{D}} \tilde{x}) \nonumber \\
& = & E_{\mathcal{R}}^{-H} \hat{A}^{-H} E_{\mathcal{D}}^H E_{\mathcal{D}} \tilde{x}  \nonumber \\
& = & ( E_{\mathcal{R}} E_{\mathcal{R}}^{-1} ) E_{\mathcal{R}}^{-H} \hat{A}^{-H} ( E_{\mathcal{D}}^H E_{\mathcal{D}} ) \tilde{x}  \nonumber \\
& = & E_{\mathcal{R}} ( Q_{\mathcal{R}}^{-1} \hat{A}^{-H} Q_{\mathcal{D}} \tilde{x} ) \nonumber \\
& \Rightarrow &  \nonumber \\
\tilde{y} & = & Q_{\mathcal{R}}^{-1} \hat{A}^{-H} Q_{\mathcal{D}} \tilde{x}.
\label{eqn:A_Euclidean_matrix_adjoint_inverse_apply}
\end{eqnarray}
%

\section{Impact of non-Euclidean scalar products on derivative representations}

Here we describe how to correctly compute and/or apply the derivative of a
multi-variable (vector) function so as to be consistent with the function's
domain and range spaces.  We will see that these issues are closely related to
the discussion of different matrix representations in
Section~\ref{sec:matrix_representations_of_linear_operators}.

Here, we will deal with real-valued vector spaces denoted with $\RE^n$.  The
reason we do this is that while derivatives for complex-valued functions are
well defined, their use in optimization and other types of numerical
algorithms can be a little tricky and therefore we stick with real-valued
functions here to avoid trouble.

We now consider multi-variable scalar functions and multi-variable vector
functions in the next two subsections.

{}\textbf{WARNING:} In the derivative discussion below, the usage the space
notation $\mathcal{X}$ is incorrect and should be replaced with $\CPLX^n$ in
many cases and visa versa.

\subsection{Derivatives of multi-variable scalar functions}
\label{sec:scalar_function_derivatives}
 
Consider the Euclidean-vector, scalar-valued function $f(x)$
\[
x\in\RE^n \rightarrow f\in\RE.
\]
The definition of the first derivative of this function comes from the
first-order variation
\[
\delta f = \Jac{f}{x} \delta x.
\]
Therefore, the derivative $\jac{f}{x}$ first and foremost is a linear operator
that when applied to some variation in $x$ of $\delta x$ gives the resulting
variation $\delta f$, to first order, in the function $f$.  For scalar-valued
functions, it is common to define the {}\textit{gradient} of the function
which is defined as $\nabla f = \jac{f}{x}^T {}\in \RE^n$ and is usually
represented as a vector in the space $\RE^n$ and this gives
\[
\delta f = {\nabla f}^T \delta x.
\]
Let the coefficients of the gradient vector be denoted as $\tilde{\nabla f}$
such that $\nabla f = E {}\tilde{\nabla f}$, where $E$ is the basis for the
space $\mathcal{X}$.

Now consider an implementation of the function $f(x)$ that takes in the coefficients
$\tilde{x}$ and returns $f$ as
%
\[
\tilde{x}\in\RE^n \rightarrow g\in\RE.
\]
%
where
%
\[
f(x) = g(\tilde{x}).
\]
%
The function $g$ is what would be directly implemented in a computer code in
many cases.  Since $\tilde{x} = E^{-1} x$, we see that
%
\[
\Jac{f}{x} = \Jac{g}{\tilde{x}} \Jac{\tilde{x}}{x}  = \Jac{g}{\tilde{x}} E^{-1}
\]
%
which gives
%
\begin{equation}
\nabla f = E^{-T} \nabla g.
\label{eqn:grad_f_E_grad_f_bar}
\end{equation}
%
Equating (\ref{eqn:grad_f_E_grad_f_bar}) to $\nabla f = E {}\tilde{\nabla f}$
and performing some manipulation we see that
%
\begin{eqnarray}
\nabla f
& = & E \tilde{\nabla f} \nonumber \\
& = & E^{-T} \nabla g \nonumber \\
& & \nonumber \\
& \Rightarrow & \nonumber \\
& & \nonumber \\
\tilde{\nabla f}
& = & E^{-1} E^{-T} \nabla g \nonumber \\
& = & (E^T E)^{-1} \nabla g \nonumber \\
& = & {Q}^{-1} \nabla g
\label{eqn:grad_f_tilde}
\end{eqnarray}
%
where $Q = E^T E$.  Therefore, to compute the coefficients
$\tilde{\nabla f}$ for the gradient vector $\nabla f$ given the gradient
$\nabla g$ for the function $g(\tilde{x})$, one must apply the inverse of the
scalar product matrix $Q^{-1}$ as shown in
(\ref{eqn:grad_f_tilde}).  Note that this results in the inner product
%
\[
{\nabla f}^T \delta x = (\tilde{\nabla f})^T Q (\tilde{\delta x})
= ({Q}^{-1} {\nabla g})^T Q (\tilde{\delta x})
= {\nabla g}^T ( {Q}^{-1} Q ) \tilde{\delta x}
= {\nabla g}^T \tilde{\delta x}
\]
%
which is nothing more than the simple dot product involving arrays of data
that are directly stored and manipulated in the computer.  This is the first
case that we will see of a {}\textit{scaling invariant} computation where the
gradient's scalar product value is independent of the choice of the basis.  In
this case, it would be more efficient to implement the gradient ${\nabla f}^T$
as a linear operator ${\nabla f}^T = {}\jac{f}{x}$ instead of as a vector in
order to avoid having to apply the inverse ${Q}^{-1}$ just to remove its
effect later using $Q$ in the scalar product.  The vector form of the gradient
$\nabla f\in\mathcal{X}$, however, is critical in many types of numerical
algorithms since it gets assigned to other vector objects and gets passed to
linear operators (i.e. it becomes the right-hand side for a linear system).

Note that representation ${\nabla f}^T = {}\jac{f}{x}$ as a linear operator
stored as ${\nabla g}^T = {}\jac{g}{\tilde{x}}$ is equivalent to the
``Euclidean'' form of the linear operator described in
Section~\ref{sec:Euclidean_matrix_rep_linear_operator} where the range space
is simply $\mathcal{R}=\RE^1$.  However, the vector representation $\nabla f =
E {}\tilde{\nabla f}$, where $\tilde{\nabla f} = {Q}^{-1}
{}\nabla g$, is equivalent to the ``natural'' matrix representation of the
linear operator $\nabla f\in\mathcal{X}|\RE$.

{}\textbf{ToDo:} Derive and describe the impact of the scalar product on the
Hessian matrix for $f(x)$.  I do not know what this is exactly but I need to
derive this so that I can determine that the Newton step for the minimization
algorithm is not effected!  In think the Hessian operator is $\nabla^2 f =
Q^{-1} {}\, {}\nabla^2 g {}\, Q^{-1}$ but In need to verify this for sure.

\subsection{Derivatives of multi-variable vector functions}
\label{sec:vector_function_derivatives}

We now consider the extension of the above discussion of scalar-valued
functions to vector-valued function $f(x)$ of the form
\[
x\in\mathcal{X} \rightarrow f\in\mathcal{F}.
\]
Again, many different algorithms consider the first-order variation
\[
\delta f = \Jac{f}{x} \delta x.
\]
In this notation, $\jac{f}{x}$ is a linear operator that maps vectors from
$\delta x\in\mathcal{X}$ to $\delta f\in\mathcal{F}$.

The vectors take the form $x = E_{\mathcal{X}} {}\tilde{x}$, $f =
E_{\mathcal{R}} {}\tilde{f}$, $\delta x = E_{\mathcal{X}} {}\tilde{\delta x}$
and $\delta f = E_{\mathcal{R}} {}\tilde{\delta f}$ where ${}\tilde{x}$,
${}\tilde{f}$, ${}\tilde{\delta x}$, and ${}\tilde{\delta f}$ are the
coefficient vectors that would typically be directly stored and manipulated in
a computer program.

Now consider the case where function $f(x)$ is implemented in coefficient form
through the function
%
\[
\tilde{x}\in\RE^n \rightarrow g\in\RE^m.
\]
%
where
%
\[
f(x) = E_{\mathcal{F}} g(\tilde{x}).
\]
%
The function $g(\tilde{x})$ is what would typically be implemented in a
computer code and $\jac{g}{\tilde{x}}$ could be efficiently and simply
computed using automatic differentiation (AD) [???] for example.  The full
forward linear operator would then be
%
\begin{equation}
\Jac{f}{x} = E_{\mathcal{F}} \Jac{g}{\tilde{x}} \Jac{\tilde{x}}{x}  = E_{\mathcal{F}} \Jac{g}{\tilde{x}} E_{\mathcal{X}}^{-1}
\label{eqn:d_f_d_x_Euclidean}
\end{equation}
%
which takes the same form as the ``Euclidean'' representation of the linear
operator described in Section~\ref{sec:Euclidean_matrix_rep_linear_operator}.
This operator $A = \jac{f}{x}$ can either be formed and stored using some
matrix representation or can be applied implicitly.

One has two choices how to actually implement the operator $A = \jac{f}{x}$
using a matrix representation.  The first option is to just explicitly store
the matrix $\jac{g}{\tilde{x}}$ that would be directly computed from the
function $g(\tilde{x})$ using AD for instance.  The forward operation
application $y = (\jac{f}{x}) x$ would then be applied in coefficient form as
%
\[
\tilde{y} = \Jac{g}{\tilde{x}} \tilde{x}.
\]
%
This ``Euclidean'' form, however, would then require that the adjoint be
implemented as
%
\begin{equation}
y = {\Jac{f}{x}}^T x \; \Rightarrow \;
\tilde{y} = {Q_{\mathcal{X}}}^{-1} {\Jac{g}{\tilde{x}}}^T Q_{\mathcal{F}} \tilde{x}
\label{eqn:Euclidean_adjoint_jac_apply}
\end{equation}
%
as shown in (\ref{eqn:A_adjoint_Euclidean_defined}), which requires the
application of the inverse of the scalar product matrix
${Q_{\mathcal{X}}}^{-1}$ with each application of the adjoint.

The other option for a matrix representation is to compute and store
$\hat{A} = (\jac{g}{\tilde{x}}) {Q_{\mathcal{X}}}^{-1}$ and this gives the ``natural''
representation
%
\begin{equation}
\Jac{f}{x} = E_{\mathcal{F}} \Jac{g}{\tilde{x}} E_{\mathcal{X}}^{-1}
= E_{\mathcal{F}} \Jac{g}{\tilde{x}} E_{\mathcal{X}}^{-1} ( E_{\mathcal{X}}^{-T} E_{\mathcal{X}}^T ) 
= E_{\mathcal{F}} \Jac{g}{\tilde{x}} ( E_{\mathcal{X}}^T E_{\mathcal{X}} )^{-1} E_{\mathcal{X}}^T 
= E_{\mathcal{F}} \hat{A} E_{\mathcal{X}}^T
\label{eqn:d_f_d_x_natural_rep}
\end{equation}
%
Note that forming the product $(\jac{g}{\tilde{x}}) {Q_{\mathcal{X}}}^{-1}$
may be very expensive to do in practice and can destroy the sparsity of
$\jac{g}{\tilde{x}}$.  Note that this is equivalent to the vector
representation of $\nabla f$ described in
Section~\ref{sec:scalar_function_derivatives}.

\section{Impact of non-Euclidean scalar products on various numerical algorithms}

Here we discuss the bread and butter of the impact of scalar products in how
they affect numerical algorithms that we develop and implement.  The approach
taken here is to first start with the algorithms stated in Euclidean form
without regard to issues of scalar products.  This is fine as long as we
recognize that the vectors, $x$ for instance, that we are dealing with will
eventually be substituted for there basis and coefficient form $x =
E\tilde{x}$ from which we do manipulations.  What we will try to do is to see
how the expressions in the algorithm change and we will try to perform the
manipulations so that we are left with the only the vector coefficients (i.e.\
$\tilde{x}$), scalar product matrices (i.e.\ $Q_{\mathcal{X}}$), and linear
operators.  We will also try to remove any explicit dependence on the exact
form of the basis representation (i.e.\ the basis $E_{\mathcal{X}}$ should not
appear in any final form of the coefficient expressions).

The general approach is summarized as:
%
\begin{enumerate}
%
{}\item State the algorithm in Euclidean form using vectors with respect to a
Euclidean basis (e.g.\ $x$) with simple dot products (e.g.\ $x^H y$) etc.
%
{}\item Substitute the basis representations for all vectors (e.g.\ $x = E
{}\tilde{x}$) in all expressions.
%
{}\item Manipulate the expressions and try to decompose all operations into
coefficient form involving only the vector coefficients (e.g.\ $\tilde{x}$),
scalar product matrices (e.g.\ $Q_{\mathcal{X}}$), and other model-defined
linear operators if needed.
%
\end{enumerate}
%
To demonstrate the process, consider the Euclidean form of the inner CG
iteration
%
\begin{eqnarray*}
\rho_k & = & r^H r, \\
p & = & r + \frac{\rho_k}{\rho_{k-1}} p, \\
q & = & A p, \\
\alpha & = & \frac{\rho_k}{p^H q}, \\
x & = & x + \alpha p, \\
r & = & r - \alpha q.
\end{eqnarray*}
%
In this algorithm, the linear operator $A\in\mathcal{S}|\mathcal{S}$ is
symmetric so we are dealing with just one vector space $\mathcal{S}$ with
scalar product $Q$.  Let $E\in\CPLX^{n\times{}n}$ be any basis representation
such that $Q=E^H E$.  Substituting $r=E\tilde{r}$, $p=E\tilde{p}$,
$q=E\tilde{q}$, and $x=E\tilde{x}$ in the above inner loop expressions yields
%
\begin{eqnarray*}
\rho_k & = & \tilde{r}^H E^H E \tilde{r}, \\
E \tilde{p} & = & E \tilde{r} + \frac{\rho_k}{\rho_{k-1}} E \tilde{p}, \\
E \tilde{q} & = & ( E \tilde{A} E^{-1} ) E \tilde{p}, \\
\alpha & = & \frac{\rho_k}{\tilde{p}^H E^H E \tilde{q}}, \\
E \tilde{x} & = & E \tilde{x} + \alpha E \tilde{p}, \\
E \tilde{r} & = & E \tilde{r} - \alpha E \tilde{q}, \\
\\
\Rightarrow \\
\\
\rho_k & = & \tilde{r}^H Q \tilde{r}, \\
E \tilde{p} & = & E ( \tilde{r} + \frac{\rho_k}{\rho_{k-1}} \tilde{p} ), \\
E \tilde{q} & = & E ( \tilde{A} \tilde{p} ), \\
\alpha & = & \frac{\rho_k}{\tilde{p}^H Q \tilde{q}}, \\
E \tilde{x} & = & E ( \tilde{x} + \alpha \tilde{p}), \\
E \tilde{r} & = & E ( \tilde{r} - \alpha \tilde{q}), \\
\\
\Rightarrow \\
\\
\rho_k & = & <\tilde{r},\tilde{r}>, \\
\tilde{p} & = & \tilde{r} + \frac{\rho_k}{\rho_{k-1}} \tilde{p}, \\
\tilde{q} & = & \tilde{A} \tilde{p}, \\
\alpha & = & \frac{\rho_k}{<\tilde{p},\tilde{q}>}, \\
\tilde{x} & = & \tilde{x} + \alpha \tilde{p}, \\
\tilde{r} & = & \tilde{r} - \alpha \tilde{q}.
\end{eqnarray*}
%
As seen in the above example, if after this transformation we can manipulate
the expressions such that the coefficient forms do not explicitly involve the
basis matrix $E$ but instead only involve the scalar product matrix $Q = E^H
E$ and the non-Euclidean coefficient forms of the linear operators, then we
have succeeded in deriving a general form of the algorithm that will work for
all non-Euclidean vector spaces.

It is critical to note that when the selection of the scalar products affects
an algorithm then a good selection for the scalar products can positively
impact the performance of the algorithm.  The dramatic improvement in the
performance of various numerical algorithms that is possible with the proper
selection of scalar products is documented in [???] and [???].  Many numerical
algorithms applied to applications that are based on discretizations of PDEs
can show mesh-independent scaling when using the proper scalar products for
instance [???].

\subsection{Newton methods}

The first set of methods that we will consider are Newton methods [???].  In
their most basic form, a Newton method seeks to solve a set of multi-variable
nonlinear equations
%
\[
f(x) = 0
\]
%
where $x\in\RE^n$ and
%
\[
x\in\RE^n \rightarrow f\in\RE^n
\]
%
is a vector function of the form described in
Section~\ref{sec:vector_function_derivatives} where $f(x) = E_{\mathcal{F}}
g(\tilde{x})$ and $g(\tilde{x})$ is what is implemented in the computer.  The
undamped Newton method seeks to improve the estimate of the solution $x_k$ by
solving the linear system
%
\begin{equation}
\Jac{f}{x} d = - f(x_k)
\label{eqn:newton_system}
\end{equation}
%
and then update the estimate using
%
\begin{equation}
x_{k+1} = x_k + d.
\label{eqn:newton_update}
\end{equation}
%
It can be shown than when $x_0$ is sufficiently close to a solution $x^*$ such
that $f(x^*)=0$, and if $\jac{f}{x}$ is nonsingular, then the iterates $x_1,
x_2, {}\ldots, x_k, x_{k+1}$ converge quadratically with
%
\[
||x_{k+1}-x^*|| < C ||x_k-x^*||^2
\]
for some constant $C\in\RE$.  In a real Newton method, some type of
modification is generally applied to the step computation in
(\ref{eqn:newton_system}) and/or the update in (\ref{eqn:newton_update}) in
order to insure convergence from remote starting points $x_0$.

We now consider the impact that non-Euclidean basis representations and scalar
products have on two forms of the Newton step computation: exact and inexact.

\subsubsection{Exact Newton methods}

In an exact Newton method, the Newton system in (\ref{eqn:newton_system}) is
solved to a high precision.  Now let's consider the impact that substituting
non-Euclidean basis representation has on the Newton method.  The basis
representations are $x = E_{\mathcal{X}}\tilde{x}$ and $f =
E_{\mathcal{F}}\tilde{f}$ for the spaces $\mathcal{X}\in\RE^n$ and
$\mathcal{F}\in\RE^n$.  Now, let us assume the ``Euclidean'' representation
for $\jac{f}{x}$ which gives the coefficient form of (\ref{eqn:newton_system})
as
%
\begin{equation}
\Jac{g}{\tilde{x}} \tilde{d} =  - g.
\label{eqn:newton_system_exact_Euclidean}
\end{equation}
%
We then substitute $\tilde d$ into the update in (\ref{eqn:newton_update})
which is
%
\begin{equation}
\tilde{x}_{k+1} = \tilde{x}_k + \tilde{d}.
\label{eqn:newton_update_exact_Euclidean}
\end{equation}
%
Comparing (\ref{eqn:newton_system})--(\ref{eqn:newton_update}) with
(\ref{eqn:newton_system_exact_Euclidean})--(\ref{eqn:newton_update_exact_Euclidean}),
it is clear that the choice of the basis functions for the spaces
$\mathcal{X}$ or $\mathcal{F}$ has no impact on the Newton steps that are
generated.  This {}\textit{invariance} property of Newton's method is one of
its greatest strengths.  However, solving the Newton system exactly can be
very expensive and taking full steps can cause the algorithm to diverge and
modifications to handle these issues are considered later.  First, however,
the inexact computation of the Newton step is discussed in the next
subsection.

\subsubsection{Inexact Newton methods}

In an inexact Newton method, the linear system in (\ref{eqn:newton_system}) is
not solved exactly, but instead is only solved to a tolerance of
%
\begin{equation}
\frac{||\Jac{f}{x} d + f_k||_{\mathcal{F}}}{||f_k||_{\mathcal{F}}} \le \eta
\label{eqn:newton_system_inexact_Euclidean}
\end{equation}
%
where $\eta\in\RE$ is known as the forcing term and typically is selected such
that $\eta {}\propto ||f_k||_{\mathcal{F}}$ in order to ensure quadratic
convergence.  The coefficient representation of
(\ref{eqn:newton_system_inexact_Euclidean}) takes the form
%
\begin{equation}
\frac{\left(\Jac{g}{\tilde{x}} \tilde{d} + g_k\right)^T Q_{\mathcal{F}} \left(\Jac{g}{\tilde{x}} \tilde{d} + g_k\right)}
{g_k^T Q_{\mathcal{F}} g_k} \le \eta^2
\label{eqn:newton_system_inexact_Euclidean_coeff}
\end{equation}
%
From (\ref{eqn:newton_system_inexact_Euclidean_coeff}) we see that the
selection of the scalar product matrix $Q_{\mathcal{F}}$ that defines the norm
$||.||_{\mathcal{F}}$ (as defined in (\ref{eqn:natural_norm_defined})) can
have a large impact on the newton step computation.  However, assuming the
``Euclidean'' form of the forward operator is used as in
(\ref{eqn:newton_system_exact_Euclidean}), then the selection of the scalar
product for the space $\mathcal{X}$ has no impact on the computed Newton step.
Such a computation is said to be {}\textit{affine invariant} [???].

\subsection{Minimization, merit functions and globalization methods}

Let's consider the minimization of a multi-variable scalar function
%
\begin{eqnarray}
\mbox{min} & & f(x)
\end{eqnarray}
%
where $f(x)$ of the form described in
Section~\ref{sec:scalar_function_derivatives} where $f(x) = g(\tilde{x})$ and
$g(\tilde{x})$ is what is actually implemented in a computer program.

As stated in Section~\ref{sec:scalar_function_derivatives}, the coefficient
vector for the gradient $\nabla f$, which takes the form $\tilde{\nabla f} =
{Q_{\mathcal{X}}}^{-1} {}\nabla g$, is affected by the definition of the basis
$E_{\mathcal{X}}$ but the scalar product
%
\begin{equation}
{\nabla f}^T d
= ({Q_{\mathcal{X}}}^{-1} {}\nabla g)^T Q_{\mathcal{X}} (\tilde{d})
 = {\nabla g}^T (\tilde{d})
\label{eqn:descent_inner_prod}
\end{equation}
%
is not affected, where $d=E_{\mathcal{X}}\tilde{d}\in\mathcal{X}$ is some
search direction.

One of the most basic requirements for many minimization algorithms is the
descent requirement which can be stated as
%
\begin{equation}
{\nabla f}^T d < 0
\label{eqn:descent_condition}
\end{equation}
%
for $\nabla f {}\ne 0$.

Consider the steepest-descent direction $d = -{}\gamma\nabla f$ where $\gamma
>0$ is some constant.  With a Euclidean basis, the coefficient vector for this
direction takes the form $\tilde{d} = -{}\gamma\nabla g$.  However, when a
non-Euclidean basis is used, the coefficient vector for the the
steepest-descent direction is
%
\[
\tilde{d} = - \gamma \, {Q_{\mathcal{X}}}^{-1} \nabla g. 
\]
%
Therefore, the choice of the scalar product can have a dramatic impact on the
steepest-descent direction.  The descent property for the steepest-descent direction then
becomes
%
\[
{\nabla f}^T d
= ( {\nabla g}^T {Q_{\mathcal{X}}}^{-1}) Q_{\mathcal{X}} (-\gamma\,{Q_{\mathcal{X}}}^{-1} \nabla g)
= -\gamma\,{\nabla g}^T {Q_{\mathcal{X}}}^{-1} \nabla g < 0.
\]
%
for $\nabla g {}\ne 0$.  Therefore, the descent property for the
steepest-descent direction is changed even though the scalar product
definition itself is not.

Another selection for the step direction takes the form $d = - B^{-1} {}\nabla
f$ where $B$ is some approximation for the Hessian of $f(x)$.  Since ${}\nabla
f$ changes with a non-Euclidean basis, so will this search direction.  The
choice of $B$ for variable metric methods will be addressed in
Section~\ref{sec:variable_metric_quasi_Newton_methods}.

Descent alone is not sufficient to guarantee convergence.  Instead, more
stringent conditions must be met.  One such set of conditions include a
sufficient decrease condition
%
\begin{equation}
f(x_k + \alpha d) \le f_k + c_1 \alpha (\nabla f_k)^T d
\label{eqn:sufficent_decrease_condition}
\end{equation}
%
(often know as the {}\textit{Armijo condition}), and a curvature condition
%
\begin{equation}
(\nabla f(x_k + \alpha d))^T d \le c_2 (\nabla f_k)^T d
\label{eqn:curvature_condition}
\end{equation}
%
where $0 < c_1 < c_2 < 1$.  Together,
(\ref{eqn:sufficent_decrease_condition})--(\ref{eqn:curvature_condition}) are
known as the {}\textit{Wolfe conditions} [???].

Now let's consider the coefficient form of the conditions in
(\ref{eqn:sufficent_decrease_condition})--(\ref{eqn:curvature_condition}) for
non-Euclidean basis' which from (\ref{eqn:descent_inner_prod}) become
%
\begin{equation}
g(\tilde{x}_k + \alpha \tilde{d}) \le g_k + c_1 \alpha (\nabla g_k)^T \tilde{d}
\label{eqn:sufficent_decrease_condition_scaled}
\end{equation}
%
and
%
\begin{equation}
(\nabla g(\tilde{x}_k + \alpha \tilde{d}))^T \tilde{d} \le c_2 (\nabla g_k)^T \tilde{d}.
\label{eqn:curvature_condition_scaled}
\end{equation}
%
It is clear from
(\ref{eqn:sufficent_decrease_condition_scaled})--(\ref{eqn:curvature_condition_scaled})
that even through the selection of the scalar product defined by
$Q_{\mathcal{X}}$ affects the steepest-descent direction, for instance, it
does not actually affect the Wolf conditions for a general direction
$\tilde{d}$.  The computation of the direction $\tilde{d}$ can, however, be
impact by the choice of the scalar product as described above.  What this
means is that the Wolfe conditions are invariant to the selection of the basis
for the space $\mathcal{X}$ but the search direction.  Again, invariance with
respect to the selection of the basis is consider a very attractive property
for numerical algorithms.

\subsection{Least-squares merit functions}

Here we consider the impact that non-Euclidean scalar products have on
standard least-square merit functions of the form
%
\begin{equation}
m(x) = f(x)^T f(x)
\label{eqn:least_squares_merit_function}
\end{equation}
%
where $f(x)$ is a multi-variable vector-valued function of the form described
in Section~\ref{sec:vector_function_derivatives} which is implemented in terms
of $g(\tilde{x})$ where $f(x) = E_{\mathcal{F}} g(\tilde{x})$.  The
least-squares function defined in (\ref{eqn:least_squares_merit_function}) is
used in a variety of contexts from globalization methods for nonlinear
equations $f(x)=0$ [???] to data fitting optimization methods [???].

The gradient $\nabla m\in\mathcal{X}$ of $m(x)$ defined in
(\ref{eqn:least_squares_merit_function}) is given by
%
\begin{equation}
\nabla m = \Jac{f}{x}^T f.
\label{eqn:least_squares_merit_function_gradient}
\end{equation}
%
When $\jac{f}{x}$ is represented in ``Euclidean'' form as shown in
(\ref{eqn:d_f_d_x_Euclidean}), the coefficient form of the adjoint
Jacobian-vector product in (\ref{eqn:least_squares_merit_function_gradient}),
shown in (\ref{eqn:Euclidean_adjoint_jac_apply}), is given by
%
\begin{equation}
\tilde{\nabla m} = {Q_{\mathcal{X}}}^{-1} {\Jac{g}{\tilde{x}}}^T Q_{\mathcal{F}} g.
\label{eqn:least_squares_merit_function_gradient_coeff}
\end{equation}
%
In (\ref{eqn:least_squares_merit_function_gradient_coeff}) we see that the
gradient direction for the least-squares merit function in
(\ref{eqn:least_squares_merit_function}) is impacted by both the scalar
product matrices $Q_{\mathcal{X}}$ and $Q_{\mathcal{F}}$.

\subsection{Variable metric quasi-Newton methods}
\label{sec:variable_metric_quasi_Newton_methods}

Non-Euclidean scalar products can dramatically improve the performance of
optimization methods that use variable-metric quasi-Newton methods [???].
Here we will consider a popular form of variable-metric approximation called
the BFGS formula [???] which is defined as
%
\[
B_+ = B - \frac{(B s) (B s)^T}{s^T B s} + \frac{y y^T}{y^T s}
\label{eqn:bfgs_update}
\]
%
where $B$ is the current approximation to the Hessian $\nabla^2 f$ and $B_+$
is the updated approximation.

Generally, the update vectors are defined as $y = {}\nabla f_k - {}\nabla
f_{k-1}$ and $s = x_k - x_{k-1}$ but the analysis here is independent of the
actual choices for these vectors.  What will be made clear here is the impact
that the non-Euclidean scalar products have on the various implementations of
this method.

We will consider two forms of the above approximation.  First, we consider an
explicit implementation that directly stores the coefficients of the matrix in
the ``natural'' form.  Second, we consider an implicit implementation that
only stores pairs of update vectors and applies the inverse implicitly.  The
implicit representation then leads naturally to a limited-memory
implementation.

\subsubsection{Explicit BFGS matrix representation}

For the explicit matrix representation we will assume that $B$ and $B_+$ are
being stored in the ``natural'' coefficient forms of $B = E {}\hat{B} E^T$
and $B_+ = E {}\hat{B}_+ E^T$.  Note that the basis matrix $E$ is generally
not given explicitly and a unique choice is not known; only the scalar product
matrix $Q = E^T E$ is known.  By substituting in the coefficient forms of $B =
E {}\hat{B} E^T$, $B_+ = E {}\hat{B}_+ E^T$, $y = E {}\tilde{y}$, and $s =
E {}\tilde{s}$ into (\ref{eqn:bfgs_update}) and performing some manipulation
we obtain
%
\begin{eqnarray}
E \hat{B}_+ E^T
& = & E \hat{B} E^T
  - \frac{[(E \hat{B} E^T)(E \tilde{s})][(E \hat{B} E^T)(E \tilde{s})]^T}{(E \tilde{s})^T(E \hat{B} E^T)(E \tilde{s})}
  + \frac{(E \tilde{y})(E \tilde{y})^T}{(E \tilde{y})^T (E \tilde{s})} \nonumber \\
& = & E \hat{B} E^T
  - \frac{E(\hat{B} Q \tilde{s})(\hat{B} Q \tilde{s})^T E^T}{\tilde{s}^T Q (\hat{B} Q \tilde{s})}
  + \frac{E \tilde{y} \tilde{y}^T E^T}{\tilde{y}^T Q \tilde{s}} \nonumber \\
& = & E \left[
  \hat{B}
  - \frac{(\hat{B} Q \tilde{s})(\hat{B} Q \tilde{s})^T}{\tilde{s}^T Q (\hat{B} Q \tilde{s})}
  + \frac{\tilde{y} \tilde{y}^T}{\tilde{y}^T Q \tilde{s}}
  \right] E^T \nonumber \\
& \Rightarrow & \nonumber \\
\hat{B}_+
& = & \hat{B}
  - \frac{(\hat{B} Q \tilde{s})(\hat{B} Q \tilde{s})^T}{\tilde{s}^T Q (\hat{B} Q \tilde{s})}
  + \frac{\tilde{y} \tilde{y}^T}{\tilde{y}^T Q \tilde{s}}.
\label{eqn:explicit_bfgs_update}
\end{eqnarray}
%
What (\ref{eqn:explicit_bfgs_update}) shows is that the ``natural'' matrix
representation of $B$ can be updated to $B_+$ by using the coefficients of the
vectors $\tilde{s}$ and $\tilde{y}$, the matrix coefficients $\hat{B}$
themselves, and the action of the scalar product matrix $Q$.  Note that the
final expressions for the update do not contain the basis matrix $E$ itself
since this matrix is not known in general.  Also note that $\tilde{q} =
{}\hat{B}Q\tilde{s}$ is just the coefficient vector from the output of the
action of $q = B s$ and the remaining operations involving $Q$ which are
$\tilde{s}^T Q\tilde{q}$ and $\tilde{s}^T Q\tilde{q}$ are simply applications
of the scalar products $<s,q>$ and $<y,y>$ and therefore no direct access the
the $Q$ operator is needed here.  However, note that applying the ``natural''
representation of $B$ does require the ability apply $Q$ as a linear operator
and not just a scalar product.

What all this means is that code that currently implements an explicit BFGS
update assuming for a Euclidean basis should only need minor modifications in
order to work correctly for non-Euclidean scalar products.

Note that applying the inverse of $B = E {}\hat{B} E^T$ for $v = B^{-1} u$
is simply a special case of (\ref{eqn:A_natural_matrix_adjoint_inverse_apply})
and is given as
%
\begin{eqnarray}
v
& = & E \tilde{v} \nonumber \\
& = & B^{-1} u \nonumber \\
& = & (E \hat{B} E^T)^{-1} (E \tilde{u}) \nonumber \\
& = & E ( Q^{-1} \hat{B}^{-1} \tilde{u}) \nonumber \\
& \Rightarrow &  \nonumber \\
\tilde{v} & = & Q^{-1} \hat{B}^{-1} \tilde{u}.
\label{eqn:B_inverse}
\end{eqnarray}
%
Therefore, applying the inverse of the natural coefficient representation of
$B$ involves applying the inverse of the scalar product matrix $Q^{-1}$.

\subsubsection{Implicit BFGS matrix representation}

For the implicit representation of a BFGS approximation we will consider the
approximation of the inverse $H = B^{-1}$ and the update $s = H_+^{-1} y$
using the update vectors $s$ and $y$ which is given by the formula
%
\begin{equation}
H_+ = V^T H V + \rho s s^T
\end{equation}
%
where
\begin{eqnarray}
\rho & = & \frac{1}{y^T s}, \\
V & = & I - \rho y s^T. \\
\end{eqnarray}
Here we consider a so-called limited-memory implementation (L-BFGS) where $m$
sets of update quantities $\{s_i,y_i,\rho_i\}$ are stored for the iterations
$i = k-1, k-2, {}\ldots, k-m$ which are used to update from the initial matrix
inverse approximation $H_0 = B_0^{-1}$ to give $H$ after the $m$ updates (see
[???] for details).  The implementation of the inverse Hessian-vector product
$v = H u$ is provided by a simple two-loop algorithm involving only simple
vector operations like dot products, vector scalings, vector additions, and
the application of the linear operator $H_0$.  Therefore, we will go and skip
ahead and write the general non-Euclidean coefficient form of this algorithm.
This simple algorithm is called the two-loop recursion [???] which is stated
as

\bifthen
\textbf{L-BFGS two-loop recursion for computing $\tilde{v} = \tilde{H} \tilde{u}$} \\
\\
$\tilde{q} = \tilde{u}$ \\
\textbf{for} $i = k-1, \ldots , k-m$ \\
\> $\alpha_i = \rho_i <\tilde{s}_i,\tilde{q}>$ \\
\> $\tilde{q} = \tilde{q} - \alpha_i \tilde{y}_i$ \\
\textbf{end} \\
$\tilde{r} = \tilde{H}_0 \tilde{q}$ \\
\textbf{for} $i = k-m, \ldots , k-1$ \\
\> $\beta = \rho_i <\tilde{y}_i,\tilde{r}>$ \\
\> $\tilde{r} = \tilde{r} + (\alpha_i - \beta) \tilde{s}_i$ \\
\textbf{end} \\
$\tilde{v} = \tilde{r}$
\eifthen

While it is subtle, the insertion of the general scalar products
$<\tilde{s}_i,\tilde{q}>$ and $<\tilde{y}_i,\tilde{r}>$ can result in a
dramatic improvement in the performance of minimization methods that use it
and it has been shown to have mesh-independent convergence properties (i.e.\
the number of iterations does not increase as the mesh is refined) for some
classes of PDE-constrained optimization problems [???].

\subsection{Inequality constraints}

Consider a simple set of bound inequality constraints of the form
%
\begin{equation}
a \le x
\label{eqn:bound_inequlity}
\end{equation}
%
where $x,a\in\mathcal{S}$ with basis representations $x = E\tilde{x}$ and $a =
E\tilde{a}$.  Inequality constraints of this form present a difficult problem
for numerical algorithms using non-Euclidean basis matrices $E$ since the
inequality constraint in (\ref{eqn:bound_inequlity}) is really a set of
element-wise constraints
%
\begin{equation}
a_i \le x_i , \; \mbox{for} \; i=1 \ldots n.
\label{eqn:bound_inequlity_elementwise}
\end{equation}
%
The element-wise nature of (\ref{eqn:bound_inequlity_elementwise}) means that
we can not simply substitute the coefficient vector components $\tilde{x}_i$
and $\tilde{a}_i$ in for $x_i$ and $a_i$.  One could, however, simply
substitute in the coefficient vector components and have the algorithm enforce
%
\begin{equation}
\tilde{a}_i \le \tilde{x}_i , \; \mbox{for} \; i=1 \ldots n,
\label{eqn:bound_inequlity_coefficients_elementwise}
\end{equation}
%
but then that may fundamentally change the meaning of these constraints and
may destroy the physical utility of these constraints for the application.
Although, note that in some types of applications this type of substitution
may be very reasonable.  For example, in standard finite-element
discretizations of PDEs, the vector coefficients directly correspond to
physical quantities such as temperature, stress, and velocity at the mesh
nodes.  Therefore bounding these types of coefficients may be very reasonable
even through a non-Euclidean scalar product is desirable in order to introduce
mesh-dependent scaling into other parts of the algorithm.  In other types of
discretizations, such as those that use a spectral basis, there is no physical
meaning to the coefficients so inequalities involving these are meaningless.
Note that imposing the inequality constraints in non-Euclidean coefficient
form as in (\ref{eqn:bound_inequlity_coefficients_elementwise}) is equivalent
to imposing the inequalities in Euclidean form as
%
\begin{equation}
E^{-1} a \le E^{-1} x
\label{eqn:bound_inequlity_with_basis_inverses}
\end{equation}
%
which is important when performing the initial transformation from the
Euclidean form (i.e.\ using dot products $x^H y$) to the non-Euclidean
coefficient form (i.e.\ using scalar products $<\tilde{x},\tilde{y}>$).  Here,
we hope that in doing the transformation of the entire algorithm that we can
remove any explicit mention of the basis matrix $E$ itself.

In cases where component-wise inequalities on vector coefficients is not
useful, one has no choice but to form an explicit basis and to pose these
constraints as general linear inequality constraints of the form
%
\[
\tilde{b} \le E \tilde{x},
\]
%
where $\tilde{b} = E\tilde{a}$.  Even if an explicit basis must be formed in
order to preserve the meaning of the inequality constraints, there is still
utility in expressing an algorithm in general non-Euclidean coefficient form
since it avoids having to convert all vectors back and forth using the basis
representation or having to invert the basis matrix.

Therefore, if it is reasonable to impose inequality constraints on the
coefficient vectors themselves, then ANAs involving inequalities with
non-Euclidean scalar products can be very reasonable and straightforward to
implement.  When replacing the Euclidean inequalities with the vector
coefficients is not reasonable, then the an explicit basis representation is
required to express the constraints.

\section{Vector Coefficient Forms of Numerical Algorithms}

Here we finally come to reality.  Up to this point in the discussion we have
been very careful to differentiate the vector $x$ from the vector coefficients
$\tilde{x}$ related by the equation $x = E\tilde{x}$.  We have viewed
algorithms in Euclidean form using the vectors $x$ and $y$ and simple
Euclidean dot products $x^H y$ and then in non-Euclidean coefficient form
using coefficient vectors $\tilde{x}$ and $\tilde{y}$ and scalar products
$<\tilde{x},\tilde{y}>$.  When mathematicians write numerical algorithms in
coefficient form, however, they do not typically use math accents like
$\tilde{x}$ and $\tilde{A}$ or acknowledge the related Euclidean forms.
Instead, they use non-accented identifiers and often the only clue that we
are dealing with non-Euclidean vectors, vector spaces, and linear operators
expressed in vector coefficient form is that simple dot products like $x^H y$
are replaced with $<x,y>$.  As we have show above, expressing algorithms in
vector coefficient form with non-Euclidean scalar products has a dramatic
impact on the definition linear operators, derivative computations, and the
meaning of certain types constructs line inequality constraints.  For example,
we showed in Section ??? that the adjoint non-Euclidean coefficient linear
operator $\tilde{A}^H$ is not the same thing as the matrix conjugate transpose
of the forward non-Euclidean coefficient linear operator $\tilde{A}$.

\begin{dumb_fact}
When most mathematicians write a numerical algorithm using the scalar product
notation $<x,y>$, the vectors $x$ and $y$ are the {}\textit{coefficients} of
the vectors and all of the linear operators become non-Euclidean coefficient
operators which are {}\textbf{not} equivalent to matrices in general!
\end{dumb_fact}

However, using the approach outlined above, one can comfortably go between the
Euclidean dot product form (i.e.\ $x^H y$) and the non-Euclidean scalar
product form (i.e.\ $<x,y>$).

\section{Summary}

Here we have presented an approach to looking at non-Euclidean scalar product
spaces that deals in very straightforward terms using simple concepts of
linear algebra.  The idea is to first look at all algorithms assuming
Euclidean vector spaces and explicit Euclidean coefficient vectors and then to
substitute in the basis representation for non-Euclidean vector spaces.  After
this substitution, one then tries to manipulate the expressions to come up
with the building blocks of scalar products and linear operators and only
considers the explicit representation and manipulation of the coefficient
vectors and never the Euclidean coefficients of the vectors themselves.

\section{ToDo}

\begin{itemize}

{}\item To make this type of discussion more helpful, it would be nice to have
a concrete application and numerical algorithm example to work through to show
the impact of all of this.  This could, in fact, make a nice journal paper to
show off Thyra if done well.

\end{itemize}

% ---------------------------------------------------------------------- %
% References
%
\clearpage
\bibliographystyle{plain}
\bibliography{references}
\addcontentsline{toc}{section}{References}

% ---------------------------------------------------------------------- %
% Appendices should be stand-alone for SAND reports. If there is only
% one appendix, put \setcounter{secnumdepth}{0} after \appendix
%
\appendix
%%
\section{Thyra Operator/Vector Interfaces C++ Declarations}
\label{app:tsfcore_classes}
%

{\scriptsize\begin{verbatim}
namespace Thyra {

using RangePack::Range1D;
template<class Scalar> class VectorSpaceFactoryBase;
template<class Scalar> class VectorSpaceBase;
template<class Scalar> class VectorBase;
template<class Scalar> class OpBase;
template<class Scalar> class LinearOpBase;
template<class Scalar> class MultiVectorBase;
template<class Scalar>

class VectorSpaceFactoryBase {
public:
  virtual Teuchos::RefCountPtr< const VectorSpaceBase<Scalar> > createVecSpc(int dim) const = 0;
};

template<class Scalar>
class VectorSpaceBase {
public:
  virtual ~VectorSpaceBase() {}
  virtual Index dim() const = 0;
  virtual bool isCompatible( const VectorSpaceBase<Scalar>& vecSpc ) const = 0;
  virtual Teuchos::RefCountPtr< VectorBase<Scalar> > createMember() const = 0;
  virtual bool isInCore() const;
  virtual Teuchos::RefCountPtr< MultiVectorBase<Scalar> > createMembers(int numMembers) const;
  virtual Teuchos::RefCountPtr<VectorBase<Scalar> > createMemberView(
            const RTOpPack::MutableSubVectorBaseT<Scalar> &raw_v ) const;
  virtual Teuchos::RefCountPtr<const VectorBase<Scalar> > createMemberView(
            const RTOpPack::SubVectorBaseT<Scalar> &raw_v ) const;
  virtual Teuchos::RefCountPtr<MultiVectorBase<Scalar> > createMembersView(
            const RTOpPack::MutableSubMultiVectorBaseT<Scalar> &raw_mv ) const;
  virtual Teuchos::RefCountPtr<const MultiVectorBase<Scalar> > createMembersView(
            const RTOpPack::SubMultiVectorBaseT<Scalar> &raw_mv ) const;
  virtual Teuchos::RefCountPtr< const VectorSpaceFactoryBase<Scalar> > smallVecSpcFcty() const;
  virtual Scalar scalarProd( const VectorBase<Scalar>& x, const VectorBase<Scalar>& y ) const;
  virtual void scalarProds( const MultiVectorBase<Scalar>& X, const MultiVectorBase<Scalar>& Y
    ,Scalar scalar_prods[] ) const;
  virtual Teuchos::RefCountPtr< const VectorSpaceBase<Scalar> > clone() const;
};

template<class Scalar>
class OpBase {
public:
  virtual ~OpBase();
  virtual Teuchos::RefCountPtr< const VectorSpaceBase<Scalar> > domain() const = 0;
  virtual Teuchos::RefCountPtr< const VectorSpaceBase<Scalar> > range() const = 0;
  virtual bool opSupported(ETransp M_trans) const;
};

template<class Scalar>
class LinearOpBase : virtual public OpBase<Scalar> {
public:
  virtual void apply( const ETransp M_trans, const VectorBase<Scalar> &x
    ,VectorBase<Scalar> *y ,const Scalar alpha=1.0 ,const Scalar beta=0.0 ) const = 0;
  virtual Teuchos::RefCountPtr<const LinearOpBase<Scalar> > clone() const;
  virtual void apply( const ETransp M_trans ,const MultiVectorBase<Scalar> &X
    ,MultiVectorBase<Scalar> *Y ,const Scalar alpha=1.0 ,const Scalar beta=0.0 ) const;
};

template<class Scalar>
class MultiVectorBase : virtual public LinearOpBase<Scalar> {
public:
  virtual Teuchos::RefCountPtr<const VectorBase<Scalar> > col(Index j) const;
  virtual Teuchos::RefCountPtr<VectorBase<Scalar> > col(Index j) = 0;
  virtual Teuchos::RefCountPtr<const MultiVectorBase<Scalar> > clone_mv() const;
  virtual Teuchos::RefCountPtr<MultiVectorBase<Scalar> > clone_mv();
  virtual Teuchos::RefCountPtr<const MultiVectorBase<Scalar> > subView( const Range1D& col_rng ) const;
  virtual Teuchos::RefCountPtr<MultiVectorBase<Scalar> > subView( const Range1D& col_rng );
  virtual Teuchos::RefCountPtr<const MultiVectorBase<Scalar> > subView( const int numCols, const int cols[] ) const;
  virtual Teuchos::RefCountPtr<MultiVectorBase<Scalar> > subView( const int numCols, const int cols[] );
  virtual void applyOp( const RTOpPack::RTOpT<Scalar> &primary_op, const size_t num_multi_vecs
    ,const MultiVectorBase<Scalar>* multi_vecs[], const size_t num_targ_multi_vecs
    ,MultiVectorBase<Scalar>* targ_multi_vecs[], RTOpPack::ReductTarget* reduct_objs[], const Index primary_first_ele
    ,const Index primary_sub_dim,const Index primary_global_offset, const Index secondary_first_ele
    ,const Index secondary_sub_dim ) const;
  virtual void applyOp( const RTOpPack::RTOpT<Scalar> &primary_op, const RTOpPack::RTOpT<Scalar> &secondary_op
    ,const size_t num_multi_vecs, const MultiVectorBase<Scalar>* multi_vecs[], const size_t num_targ_multi_vecs
    ,MultiVectorBase<Scalar>* targ_multi_vecs[], RTOpPack::ReductTarget* reduct_obj, const Index primary_first_ele
    ,const Index primary_sub_dim, const Index primary_global_offset, const Index secondary_first_ele
    ,const Index secondary_sub_dim ) const;
  virtual void getSubMultiVectorBase( const Range1D &rowRng, const Range1D &colRng
    ,RTOpPack::SubMultiVectorBaseT<Scalar> *sub_mv ) const;
  virtual void freeSubMultiVectorBase( RTOpPack::SubMultiVectorBaseT<Scalar>* sub_mv ) const;
  virtual void getSubMultiVectorBase( const Range1D &rowRng, const Range1D &colRng
    ,RTOpPack::MutableSubMultiVectorBaseT<Scalar> *sub_mv );
  virtual void commitSubMultiVectorBase( RTOpPack::MutableSubMultiVectorBaseT<Scalar>* sub_mv );
  void apply( const ETransp M_trans, const VectorBase<Scalar> &x, VectorBase<Scalar> *y, const Scalar alpha
    ,const Scalar beta ) const;
  Teuchos::RefCountPtr<const LinearOpBase<Scalar> > clone() const;
};

template<class Scalar>
class VectorBase : virtual public MultiVectorBase<Scalar> {
public:
  virtual ~VectorBase() {}
  virtual Teuchos::RefCountPtr< const VectorSpaceBase<Scalar> > space() const = 0;
  virtual void applyOp( const RTOpPack::RTOpT<Scalar> &op, const size_t num_vecs
    ,const VectorBase<Scalar>* vecs[], const size_t num_targ_vecs ,VectorBase<Scalar>* targ_vecs[]
    ,RTOpPack::ReductTarget *reduct_obj ,const Index first_ele ,const Index sub_dim
    ,const Index global_offset ) const = 0;
  virtual void getSubVectorBase( const Range1D& rng, RTOpPack::SubVectorBaseT<Scalar>* sub_vec ) const;
  virtual void freeSubVectorBase( RTOpPack::SubVectorBaseT<Scalar>* sub_vec ) const;
  virtual void getSubVectorBase( const Range1D& rng, RTOpPack::MutableSubVectorBaseT<Scalar>* sub_vec );
  virtual void commitSubVectorBase( RTOpPack::MutableSubVectorBaseT<Scalar>* sub_vec );
  virtual void setSubVectorBase( const RTOpPack::SparseSubVectorBaseT<Scalar>& sub_vec );
  // Multi-vector overrides
  ...
};

template<class Scalar>
void applyOp( const RTOpPack::RTOpT<Scalar> &op, const size_t num_vecs
    ,const VectorBase<Scalar>* vecs[], const size_t num_targ_vecs ,VectorBase<Scalar>* targ_vecs[]
    ,RTOpPack::ReductTarget *reduct_obj ,const Index first_ele=1 ,const Index sub_dim=0
    ,const Index global_offset=0 );

template<class Scalar>
void applyOp( const RTOpPack::RTOpT<Scalar> &primary_op, const size_t num_multi_vecs
  ,const MultiVectorBase<Scalar>* multi_vecs[], const size_t num_targ_multi_vecs
  ,MultiVectorBase<Scalar>* targ_multi_vecs[], RTOpPack::ReductTarget* reduct_objs[], const Index primary_first_ele=1
  ,const Index primary_sub_dim=1, const Index primary_global_offset=0, const Index secondary_first_ele=1
  ,const Index secondary_sub_dim=0 ) const;

template<class Scalar>
void applyOp( const RTOpPack::RTOpT<Scalar> &primary_op, const RTOpPack::RTOpT<Scalar> &secondary_op
  ,const size_t num_multi_vecs, const MultiVectorBase<Scalar>* multi_vecs[], const size_t num_targ_multi_vecs
  ,MultiVectorBase<Scalar>* targ_multi_vecs[], RTOpPack::ReductTarget *reduct_obj, const Index primary_first_ele=1
  ,const Index primary_sub_dim=0, const Index primary_global_offset=0, const Index secondary_first_ele=1
  ,const Index secondary_sub_dim=0 ) const;

} // namespace Thyra
\end{verbatim}}


%\begin{SANDdistribution}
%\end{SANDdistribution}

\end{document}
