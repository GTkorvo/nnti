%
\section{Guidelines for Reformatting of Source Code}
%

When a sufficiently common coding style is not being used by all developers,
some guidelines are needed when code written by one individual is modified by
another individual.  These guidelines address how developers should conduct
themselves when modifying source files written largely by someone else.

\begin{enumerate}

{}\item First and foremost, each developer should respect other developer's
formatting styles when modifying code that other developers have written.

{}\item When only small changes are needed, a developer should abide by the
formatting style already in use in the file.  This helps to respect other
developers and helps to avoid needless changes for the version control system
to have to track.

{}\item A developer should only reformat an file written by someone else and
check it in if more significant changes are made.  Also, if a developer needs
to understand a complicated piece of code in order to make even perhaps a
small change in the end, that developer can also reformat the code and check
it in.  In all cases, such a reformatting must use the suggested format
outlined in Section {}\ref{thyracodingguidelines:formatting:sec}.

{}\item Once a source file has been checked in that uses the suggested format
style outlined in Section {}\ref{thyracodingguidelines:formatting:sec} no-one
else should reformat the file to a different format and check-in the file.
Repeated changes to the formatting of a source code file can case check-in
conflicts and will result in massive increases the the amount of information
that the version control system needs to keep track of.

\end{enumerate}

The above guidelines ensure that individuals are given maximal freedom to
format code to their liking but also helps to foster the shared ownership and
development of code.  In addition, the specification of a ``suggested'' format
helps to decide tie breakers and helps to avoid problems with the version
control system.
