%
\section{Guidelines for Reformatting of Source Code}
\label{sec:reformatting-guidelines}
%

When a sufficiently common coding style is not being used by all developers,
some guidelines are needed for when code written by one individual is modified
by another individual.  These guidelines address how developers should conduct
themselves when modifying source files written largely by someone else.

\begin{enumerate}

{}\item First and foremost, each developer should respect other developer's
formatting styles when modifying code that other developers have written.  If
a developer has a prefered Emacs style, then that style should be listed
explicitly at the top of each source file that is modifed.

{}\item When only small changes are needed, a developer should abide by the
formatting style already in use in the file.  This helps to respect other
developers and helps to avoid needless changes for the version control system
to have to track.  Again, if user-defined Emacs styles are specified, then
this is easy to do when editing files through Emacs.

{}\item Reformatting a file written by someone else and checking it in is only
justificed if significant changes are made.  Also, if a developer needs to
understand a complicated piece of code in order to make even perhaps a small
change in the end, that developer may also be justificed in reformatting the
file.  When a reformatting is done, the new Emacs formatting style should be
added to the top of the source file in order to make it easier for the orginal
owner of the file to maintain it as well.

{}\item Multiple reformats of the same file should not be checked in as this
will result in massive increases the the amount of information that the
version control system needs to keep track of.

\end{enumerate}

The above guidelines ensure that individuals are given maximal freedom to
format code to their liking but also helps to foster the shared ownership and
development of code.  In addition, the use of user-defined, file-specific
formats makes it easy for developers to accomidate the different formatting
styles of other developers.
