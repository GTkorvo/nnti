%
\section{Miscellaneous amendments to ``C++ Coding Standards''}
\label{sec:misc_amendments}
%

In this appendix, we provide amendments mentioned in Appendix
{}\ref{sec:cpp_coding_standards} to some of the items in
{}\cite{C++CodingStandards05} that we feel are inappropriate for our domain.

%
\subsection{Amendments to items related to compiler/linker incompatibilities}
%

There are three items in {}\cite{C++CodingStandards05} that relate to
portability problems associated with mixing and matching code in different
binary libraries compiled with different C++ compilers or with different
compiler options.  In this context, the authors use the term ``module'' to
mean a single library or a set of libraries containing binary object code that
defines the ``module''.

In general, one can not assume that object code compiled by two or more
different C++ compilers will work together since the name-mangling needed for
type-safe linkage is not even specified by the ISO C++ standard.  A more
typical problem is when the same compiler is used, but different compiler
and/or linker options are used.  For example, some compilers allow you to turn
support for exception handling on and off and if an exception is thrown by one
module it will not be handled correctly by another module that has exception
handling support turned off.  A similar problem can happen when mixing static
and shared libraries, in Linux for example, where RTTI is handled differently
and can result in dynamic casting failures in cases where it would otherwise
succeed.

In our model of software deployment, we distribute source code and a build
process that users can manipulate in order to set the exact compiler and
linker options to match what is used by other libraries and the application
code that uses the libraries.  Because we develop class libraries, it is
simply not realistic to isolate this type of code into libraries with small
``Facade''type interfaces.

The specific items that we consider inappropriate are:

\begin{itemize}

{}\item\textit{Item 60: Avoid allocating and deallocating memory in different
modules}:

{}\item\textit{Item 62: Don't allow exceptions to propagate across module
boundaries}:

{}\item\textit{Item 63: Use sufficiently portable types in a module's
interface}:

\end{itemize}

All three of these items are related to the problem of mixing code created by
different compiler and/or linker options.  However, they may also be related
to mixed language programming.  For example, in order to ensure that your
module is the most reusable, you might create a C-compatible interface that
allows clients coding in C (and even Fortran 77 in some cases) to call and be
called by your module.  If mixed language programming is the issue, then a
special {}\texttt{extern `"C"} interface should be created for the module
which will automatically satisfy Items 60, 62, and 63.

%
\subsection{Amendments for 'using' declarations and directives}
%

In {}\cite[Item 59]{C++CodingStandards05}, the authors say to never put
'using' declarations into header files or before {}\texttt{\#include}s and
that 'using namespace SomeNamespace' directives are perfectly safe for code in
source files after all {}\texttt{\#include}s.  However, we will argue that:

\begin{itemize}

{}\item{}employing {}\texttt{using} declarations to inject names of C++
classes or enums from one namespace into another is fairly safe (this is more
lax than what is suggesed in {}\cite[Item 59]{C++CodingStandards05})

{}\item{}employing a {}\texttt{using namespace ...} directive in any context
is harmful and should be avoided (this is more strict than what is suggesed in
{}\cite[Item 59]{C++CodingStandards05}).

\end{itemize}

{}\noindent{}However, we agree that employing {}\texttt{using} declarations
for nonmember functions is dangerous and is to be avoided because of problems
related to overloading and when overload are declared.

Are all using declarations employed in header files dangerous?  In
{}\cite[Item 59]{C++CodingStandards05}, the authors clearly show that
employing 'using' declarations for nonmember functions is dangerous because of
overloading.  But what about empolying 'using' declarations for C++ classes?

To investigate the issues involved, consider the following toy C++ program (in
the file NamespaceClassUsingIssues.cpp):

{\small\begin{verbatim}
    //
    // Header-like declarations
    //
    
    #include <iostream>
    
    namespace NamespaceA {
    
    template<class T>
    class A {
    public:
      explicit A( const T& a ) : a_(a) {}
      void print( std::ostream &os ) const
        { os << "\na="<<a_<<"\n"; }
    private:
      T a_;
    };
    
    } // namespace NamespaceA
    
    // Add a using declaration to inject 'A' into another namespace
    namespace NamespaceB {
    
    using NamespaceA::A;
    
    } // namespace NamespaceB
    
    // Now use the A class without the namespace qualification
    namespace NamespaceB {
    
    A<double> foo( std::ostream &os, const A<int> &aa );
    // NOTE: Above, we do not need namespace qualification for 'A'
    
    } // namespace NamespaceB
    
    
    //
    // Implementations
    //
    
    // Create another A class in the global namespace.  With care, we should not
    // have any problems with this and our code should not be affected by the
    // presence of this class.
    template<class T>
    class A {
    public:
      explicit A( const T& a ) : a_(a) 
        { std::cerr << "\nOh no, called ::A::A(...)!\n"; exit(1); }
      void print( std::ostream &os ) { os << "\na="<<a_<<"\n"; }
    private:
      T a_;
    };
    
    // See what happens when you define another class A in NamespaceB which
    // conflicts with the using declaration!  This should not be allowed and
    // should be caught by the compiler!
    
    #ifdef SHOW_DUPLICATE_CLASS_A
    
    namespace NamespaceB {
    
    template<class T>
    class A {
    public:
      explicit A( const T& a ) : a_(a) 
        { std::cerr << "\nOh no, called ::A::A(...)!\n"; exit(1); }
      void print( std::ostream &os ) { os << "\na="<<a_<<"\n"; }
    private:
      T a_;
    };
    
    } // namespace NamespaceB
    
    #endif // SHOW_DUPLICATE_CLASS_A
    
    
    // Define function in NamespaceB without namespace qualification for class A
    NamespaceB::A<double>
    NamespaceB::foo( std::ostream &os, const A<int> &aa )
    {
      A<double> ab(2.0);
      aa.print(std::cout);
      ab.print(std::cout);
      return ab;
    }
    // NOTE: Above, we need explicit namespace qualification for the return type
    // 'NamespaceB::A<double>' since we use namespace qualification to define
    // nonmember functions (see Thyra coding guidelines).  Without this namespace
    // qualification, the global class '::A' would be assumed and you would get a
    // compilation error.  However, within the function, which is in the scope of
    // NamespaceB, we don't need namespace qualifications!
    
    
    //
    // User's code.  This code does not typically live in a namespace (or is in
    // another unrelated namespace).  Here, some explicit namespace qualification
    // and using declarations will be required to avoid ambiguities.
    //
    
    int main()
    {
    
    #if defined(SHOW_MISSING_USING_DECL)
      // Here, no using declaration is provided.  This will result in the global
      // class '::A' being used below which will result in a compiler error when
      // the NamespaceB::foo(...) function is called.  This is a feature!
    #elif defined(SHOW_ERRONEOUS_USING_DIRECTIVE)
      // Here we try to just inject all of the names from NamespaceA into the
      // local scope.  However, this will result in the names 'NamespaceA::A' and
      // '::A' being equally visible which will result in a compiler error when
      // the first unqualified 'A' object is created below!
      using namespace NamespaceA;
    #else
      // Inject the class name 'A' into the local scope and will override any
      // (sloppy) names polluting the global namespace.  This will cause the global
      // '::A' class to be sort of hidden (which is good!).
      using NamespaceA::A;
    #endif
    
      A<int> aa(5);
      A<double> ab = NamespaceB::foo(std::cout,aa);
      ab.print(std::cout);
      
      return 0;
      
    }
\end{verbatim}}

The above program defines a templated class {}\texttt{A} in namespace
{}\texttt{NamespaceA} and then does a {}\texttt{using NamespaceA::A} to inject
this class name into {}\texttt{NamespaceB}.

When the program is compiled and run, one gets:

{\small\begin{verbatim}
    $ g++ -ansi -pedantic -Wall -o NamespaceClassUsingIssues.exe
    NamespaceClassUsingIssues.cpp
    
    $ ./NamespaceClassUsingIssues.exe 
    
    a=5
    
    a=2
    
    a=2
\end{verbatim}}

This program has a few different ifdefs to show different types of errors that
a compiler will detect.

\begin{enumerate}

{}\item\textit{What happens if you try to define another class ~A~ in
namespace ~NamespaceB~?}

In the case of nonmember functions, overloads of a function exhibit strange
and non-intuitive behavior when one employs 'using' declarations.  However,
what happens with classes?

In the above program, when ones defines the macro
{}\texttt{SHOW\-\_DUPLICATE\-\_CLASS\-\_A} when compiling, one will get:

{\small\begin{verbatim}
    $ g++ -ansi -pedantic -Wall -DSHOW_DUPLICATE_CLASS_A \
       -o NamespaceClassUsingIssues.exe NamespaceClassUsingIssues.cpp             

    NamespaceClassUsingIssues.cpp:63: error: declaration of `class 
       NamespaceA::A<T>' in `NamespaceB' which does not enclose `NamespaceA'
    NamespaceClassUsingIssues.cpp:63: confused by earlier errors, bailing out
\end{verbatim}}

Above, the error message generated by g++ is not very good but at least the
compiler will not allow this code to compile.  This is in stark contrast to
what happens when you have overloaded member functions which {}\cite[Item
59]{C++CodingStandards05} complains about.

{}\textbf{Takehome Message}: Employing {}\texttt{using
SomeNamespace::SomeClass} declarations to inject names from one namespace into
another seems to be safe and does not suffer from the gotchas associated with
{}\texttt{using} declarations for (overloaded) nonmember functions.

{}\item\textit{What happens when the user's code does not have an
appropriate using declaration?}

While the {}\texttt{using NamespaceA::A} declaration in {}\texttt{NamespaceB}
allows the code in {}\texttt{NamespaceB} to avoid having to explicitly qualify
{}\texttt{NamespaceA::A} all the time, this does not automatically mean that
user code that does not live in {}\texttt{NamespaceB} will not have to do
something to get at the name {}\texttt{A}.  The user can either do explicit
qualification {}\texttt{Namespace::A} or can put a {}\texttt{using
NamespaceA::A} declaration at the top of their namespace or in each function
that they have (as is done in the {}\texttt{main()} function above).

In the above program, if you define the macro
{}\texttt{SHOW\-\_MISSING\-\_USING\-\_DECL}, the {}\texttt{using Namespace::A}
declaration will be missing in {}\texttt{main()} and this will result in the
compiler finding the global {}\texttt{::A} class which will cause a compiler
error when {}\texttt{NamespaceB::foo(...)} gets called.  Here is what one gets
when compiling:

{\small\begin{verbatim}
    $ g++ -ansi -pedantic -Wall -DSHOW_MISSING_USING_DECL \
      -o NamespaceClassUsingIssues.exe NamespaceClassUsingIssues.cpp 

    NamespaceClassUsingIssues.cpp: In function `int main()':
    NamespaceClassUsingIssues.cpp:121: error: invalid initialization of 
       reference of type 'const NamespaceA::A<int>&' from expression of type '
       A<int>'
    NamespaceClassUsingIssues.cpp:80: error: in passing argument 2 of `
       NamespaceA::A<double> NamespaceB::foo(std::ostream&, const 
       NamespaceA::A<int>&)'
\end{verbatim}}

While the above error message generated by g++ here is not all that great, at
least the compiler catches the mistake.

{}\textbf{Takehome Message}: Always do {}\texttt{using
SomeNamespace::SomeClass} to inject the names from other namespaces that you
want to use to protect yourself from others who pollute the global namespace.

{}\item\textit{What happens when the user code employs a {}\texttt{using
namespace NamespaceA} directive when there are conflicting global names?}

Since there is a global class {}\texttt{::A}, the user can not simply employ a
{}\texttt{using namespace NamespaceA} directive or the compiler will complain
that it does not know which class to use.

In the above program, when one defines the macro
{}\texttt{SHOW\-\_ERRONEOUS\-\_USING\-\_DIRECTIVE} when compiling one gets:

{\small\begin{verbatim}
    $ g++ -ansi -pedantic -Wall -DSHOW_ERRONEOUS_USING_DIRECTIVE \
      -o NamespaceClassUsingIssues.exe NamespaceClassUsingIssues.cpp 

    NamespaceClassUsingIssues.cpp: In function `int main()':
    NamespaceClassUsingIssues.cpp:120: error: use of `A' is ambiguous
    NamespaceClassUsingIssues.cpp:45: error:   first declared as `
       template<class T> class A' here
    NamespaceClassUsingIssues.cpp:10: error:   also declared as `
       template<class T> class NamespaceA::A' here
    NamespaceClassUsingIssues.cpp:120: error: parse error before `>' token
    NamespaceClassUsingIssues.cpp:121: error: use of `A' is ambiguous
    NamespaceClassUsingIssues.cpp:45: error:   first declared as `
       template<class T> class A' here
    NamespaceClassUsingIssues.cpp:10: error:   also declared as `
       template<class T> class NamespaceA::A' here
    NamespaceClassUsingIssues.cpp:121: error: parse error before `>' token
    NamespaceClassUsingIssues.cpp:122: error: `ab' undeclared (first use 
       this function)
    NamespaceClassUsingIssues.cpp:122: error: (Each undeclared identifier 
       is reported only once for each function it appears in.)
\end{verbatim}}

Note that this type of example goes against the advise in {}\cite[Item
59]{C++CodingStandards05} where they say that it is safe to employ
{}\texttt{using namespace SomeNamespace} directives in {}\texttt{*.cpp} source
files.  This example shows that this does not protect you from others that
pollute the global namespace.  Note that code that is written this way might
compile one day and not the next as it is fragile and can be broken by other
people that pollute the global namespace.

{}\textbf{Takehome Message}: Never employ {}\texttt{using namespace
AnyNamespace} as you cannot guarantee the integrity of your code since people
outside of your namespace can cause your code to not compile.

\end{enumerate}
