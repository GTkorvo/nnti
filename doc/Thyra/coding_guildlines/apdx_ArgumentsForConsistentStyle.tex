%
\section{Arguments for consistent code formatting style}
\label{sec:arguments-for-consistent-style}
%

While there are reasonable ways to handle different code formatting styles
within a projects, there are arguments for preferring a more consistent code
formatting styles.

Probably the laxest opinion on using a consistent coding sytle on a project
comes from {}\cite[Item 0]{C++CodingStandards05} where the authors state:

\begin{quote}

Do use consistent formatting within each source file or even each project,
because it's jarring to jump around among several styles in the same piece of
code.

\end{quote}

Much stronger arguments for working toward a consistent code formatting style
within a project are made by other influential and respected individuals and
organizations who represent very different views of software development.
These organizations and persons vary from open source projects (i.e.\ GNU) to
newere Agile methods (e.g.\ Extreme Programming) to software giants like
Microsoft.  As different as these different people and organizations view the
nature of software (e.g.\ GNU vs.\ Microsoft) and how it should be developed
(e.g.\ Extreme Programming vs.\ Microsoft), they all agree that some degree of
consistency in coding standards is a good idea.

%
\subsection{Open source software (the GNU project)}
%

On one end of the spectrum is open source software that one can think of as
the freest form of software.  A GNU package is usually not even developed by a
cohesive set of developers but yet the official GNU Coding
Standard\footnote{{}\texttt{http://www.gnu.org/prep/standards/standards.html}}
states:

\begin{quote}

The rest of this section gives our recommendations for other aspects of C
formatting style ... We don't think of these recommendations as requirements
... But whatever style you use, please use it consistently, since a mixture of
styles within one program tends to look ugly. If you are contributing changes
to an existing program, please follow the style of that program.

\end{quote}

While the above passage does not mandate a consistent coding style within a
package (because it can't, its free software), it does recommend a coding
style\footnote{The official GNU formatting style is one of the built-in styles
in Emacs called the ``gnu'' style} and it asks that each project please use a
consistent coding style throught.

%
\subsection{Extreme Programming}
%

While the Extreme Programming and GNU movements are miles apart in terms of
how it expects coders to work together to create code, they both agree that
using a consistent coding style within a project is important.

In his landmark 2000 book ``Extreme Programming Explained''
{}\cite{ExtremeProgrammingExplained}, Kent Beck explicitly listed ``Coding
Standards'' as one of XP's twelve recommended practices.  As a result, many XP
projects have insisted on requiring every member of the team to code in the
same way.  So much to the point that one should not be able to tell who wrote
a piece of code just in how it is layed out.  As of this writting, almost
every source of information on XP on the internet takes a very strong opinion
on the adoption of a consistent coding style by an XP group.  The particular
style of coding style is not important, what is important is that everyone on
the team helps to formulate and agrees to use the same coding style.

In the 2005 second edition ``Extreme Programming Explained: Second Edition'',
Kent Beck has restructured XP and now the ``Coding Standards'' practice is no
longer listed as a practice.  Does this mean that consistent code formatting
is not longer important in XP?  Actually, in her article ``The New
XP''\footnote{{}\texttt{http://www.agilexp.org/downloads/TheNewXP.pdf}} which
outlines the second edition of Beck's book and compares to the first edition,
Michele Marchesi states:

\begin{quote}

You must note that in the new XP we cannot find original practices of
{}\textit{coding standards}, that is considered obvious ... and
{}\textit{metaphor} ....

\end{quote}

And to put to rest any doubt how Beck himself feels about consistent coding
styles he states in the second edition:

\begin{quote}

For example, people get passionate about coding style.  While there are
undoubedly better styles and worse styles, the most important style issues is
that the team chooses to work towards a common style.  Idiosyncratic coding
styles and the values revealed by them, individual freedom at all costs, don't
help the team suceed.

\end{quote}

Therefore, it is clear that the flagship of the Agile programming movement,
XP, clearly advocates that a team of developers work towards a consistent code
formatting style.

%
\subsection{Code Complete}
%

ToDo: Fill this in!

%
\subsection{How common coding styles should be chosen}
%

While the above varied sources have different levels of opinions on the
importance on code formatting style standards, they all agree that it is the
developers themselves that should comeup with the standards, and not
non-technical managers.  They also all seem to agree that a coding standard
that is too ridged about code formatting will do more harm than good.

Therefore, a team of software developers should get together and collectively
decide on a set of guidelines for code formatting and each member should try
to follow the spirit of the agreeded upon style as much as is reasonble while
bending the guidelines where appropriate.
