%
\section{Miscellaneous amendments to ``C++ Coding Standards''}
\label{sec:misc_amendments}
%

In this appendix, some of the amendments mentioned in Appendix
{}\ref{sec:cpp_coding_standards} to some of the items in
{}\cite{C++CodingStandards05} are discussed in more detail.


%
\subsection{Amendments to items related to compiler/linker incompatibilities}
%

There are three items in {}\cite{C++CodingStandards05} that relate to
portability problems associated with mixing and matching code in
different binary libraries compiled with different C++ compilers or
with different compiler options.  In this context, the authors use the
term ``module'' to mean a single library or a set of libraries
containing simiarly compiled binary object code.

In general, one can not assume that object code compiled by two or more
different C++ compilers will work together since the name-mangling needed for
type-safe linkage is not even specified by the ISO C++ standard.  A more
typical problem is when the same compiler is used, but different compiler
and/or linker options are used.  For example, some compilers allow you to turn
support for exception handling on and off and if an exception is thrown by one
module it will not be handled correctly by another module that has exception
handling support turned off.  A similar problem can happen when mixing static
and shared libraries, in Linux for example, where RTTI is handled differently
and can result in dynamic casting failures in cases where it would otherwise
succeed.

In our model of software deployment, we distribute source code and a
build process that users can manipulate in order to set the exact
compiler and linker options to match what is used by other libraries
and the application code that uses the libraries.  Because we develop
class libraries, it is simply not realistic to isolate this type of
code into libraries with small ``Facade'' type interfaces that are
advocated in {}\cite{C++CodingStandards05}.

The specific items that we consider inappropriate are:

\begin{itemize}

{}\item\textit{Item 60: Avoid allocating and deallocating memory in different
modules}:

{}\item\textit{Item 62: Don't allow exceptions to propagate across module
boundaries}:

{}\item\textit{Item 63: Use sufficiently portable types in a module's
interface}:

\end{itemize}

All three of these items are related to the problem of mixing code
created by different compiler and/or linker options.  However, they
may also be related to mixed language programming.  For example, in
order to ensure that your module is the most reusable, you might
create a C-compatible interface that allows clients coding in C (and
even Fortran 77 in some cases) to call and be called by your module.
If mixed language programming is the issue, then a special
{}\texttt{extern "C"} interface should be created for the module which
will automatically satisfy Items 60, 62, and 63.  Note that reference
counting machinery in the {}\ttt{RCP} and {}\ttt{ArrayRCP} classes
actually solves the problem of calling {}\ttt{new} and {}\ttt{delete}
in different modules that is described in Item 60 because the
deallocator object that calls {}\ttt{delete} is create and assigned in
the same module where {}\ttt{new} is called which guarantees that they
are consistent.


%
\subsection{Amendments for 'using' declarations and directives}
\label{sec:using_amendments}
%

In {}\cite[Item 59]{C++CodingStandards05}, the authors say to never put
'using' declarations into header files or before {}\texttt{\#include}s and
that 'using namespace SomeNamespace' directives are perfectly safe for code in
source files after all {}\texttt{\#include}s.  However, we will argue that:

\begin{itemize}

{}\item{}employing {}\texttt{using} declarations to inject names of
C++ classes or enums from one namespace into another is perfectly safe
(this is more lax than what is suggested in {}\cite[Item
59]{C++CodingStandards05})

{}\item{}employing a {}\texttt{using namespace SomeNameSpace}
directive in any context is harmful and should be avoided (this is
more restrictive than what is suggested in {}\cite[Item
59]{C++CodingStandards05}).

\end{itemize}

{}\noindent{}However, we agree that employing {}\texttt{using}
declarations for nonmember functions is dangerous and is to be avoided
because of problems related to overloading and in what order overloads
are declared and used (as described in {}\cite[Item
59]{C++CodingStandards05}) .

Are all using declarations employed in header files dangerous?  In
{}\cite[Item 59]{C++CodingStandards05}, the authors clearly show that
employing 'using' declarations for nonmember functions is dangerous
because of overloading.  But what about employing 'using' declarations
for C++ classes and other types?

To investigate the issues involved, consider the following toy C++ program (in
the file NamespaceClassUsingIssues.cpp):

{\small\lstinputlisting{NamespaceClassUsingIssues.cpp}}

The above program defines a templated class {}\texttt{A} in namespace
{}\texttt{NamespaceA} and then does a {}\texttt{using NamespaceA::A} to inject
this class name into {}\texttt{NamespaceB}.

When the program is compiled and run with g++ (version 4.3.4), one
gets:

{\small\begin{verbatim}
    $ g++ -ansi -pedantic -Wall -o NamespaceClassUsingIssues.exe
    NamespaceClassUsingIssues.cpp
    
    $ ./NamespaceClassUsingIssues.exe 
    
    a=5
    
    a=2
    
    a=2
\end{verbatim}}

This program has a few different ifdefs to show different types of errors that
a compiler will detect.

\begin{enumerate}

{}\item\textit{What happens if one tries to define another class
{}\ttt{A} in namespace {}\ttt{NamespaceB}?}

In the case of nonmember functions, overloads of a function exhibit strange
and non-intuitive behavior when one employs 'using' declarations.  However,
what happens with classes?

In the above program, when one defines the macro
{}\texttt{SHOW\-\_DUPLICATE\-\_CLASS\-\_A} when compiling, one will get the
following compile-time error:

{\small\begin{verbatim}
  $ g++ -ansi -pedantic -Wall -DSHOW_DUPLICATE_CLASS_A \
     -o NamespaceClassUsingIssues.exe NamespaceClassUsingIssues.cpp             

  NamespaceClassUsingIssues.cpp:53: error: redefinition of
    'class NamespaceA::A<T>'
  NamespaceClassUsingIssues.cpp:11: error: previous definition of
    'class NamespaceA::A<T>'
\end{verbatim}}

Above, the error message generated by g++ 4.3.4 is very good and
pinpoints the problem exactly.  This is in stark contrast to what
happens when you have overloaded member functions which {}\cite[Item
59]{C++CodingStandards05} explains.

{}\textbf{Take-home Message}: Employing {}\texttt{using
SomeNamespace::SomeClass} declarations to inject names from one
namespace into another seems to be safe and does not suffer from the
gotchas associated with {}\texttt{using} declarations for (overloaded)
nonmember functions.

{}\item\textit{What happens when the user's code does not have an
appropriate using declaration?}

While the {}\texttt{using NamespaceA::A} declaration in {}\texttt{NamespaceB}
allows the code in {}\texttt{NamespaceB} to avoid having to explicitly qualify
{}\texttt{NamespaceA::A} all the time, this does not automatically mean that
user code that does not live in {}\texttt{NamespaceB} will not have to do
something to get at the name {}\texttt{A}.  The user can either do explicit
qualification {}\texttt{Namespace::A} or can put a {}\texttt{using
NamespaceA::A} declaration at the top of their namespace or in each function
that they have (as is done in the {}\texttt{main()} function above).

In the above program, if one defines the macro
{}\texttt{SHOW\-\_MISSING\-\_USING\-\_DECL}, the {}\texttt{using
Namespace::A} declaration will be missing in {}\texttt{main()} and
this will result in the compiler finding the global {}\texttt{::A}
class which will cause a compiler error when
{}\texttt{NamespaceB::foo(...)} gets called.  Here is the error
message that one gets when compiling with this macro defined:

{\small\begin{verbatim}
    $ g++ -ansi -pedantic -Wall -DSHOW_MISSING_USING_DECL \
      -o NamespaceClassUsingIssues.exe NamespaceClassUsingIssues.cpp 

    NamespaceClassUsingIssues.cpp: In function `int main()':
    NamespaceClassUsingIssues.cpp:121: error: invalid initialization of 
       reference of type 'const NamespaceA::A<int>&' from expression of type '
       A<int>'
    NamespaceClassUsingIssues.cpp:80: error: in passing argument 2 of `
       NamespaceA::A<double> NamespaceB::foo(std::ostream&, const 
       NamespaceA::A<int>&)'
\end{verbatim}}

The above error message generated by g++ 4.3.4 here is not all bad as
the compiler catches the mistake and states the types involved.

{}\textbf{Take-home Message}: Always employ {}\texttt{using
SomeNamespace::SomeClass} to inject type names from other namespaces that you
want to use in your namespace to protect your code from others who pollute the
global namespace.

{}\item\textit{What happens when the user code employs a {}\texttt{using
namespace NamespaceA} directive when there are conflicting names?}

Since there is a global class {}\texttt{::A}, the user can not simply employ a
{}\texttt{using namespace NamespaceA} directive or the compiler will complain
that it does not know which class to use.

In the above program, when one defines the macro
{}\texttt{SHOW\-\_ERRONEOUS\-\_USING\-\_DIRECTIVE} when compiling one
gets the following very good compile error message:

{\small\begin{verbatim}
    $ g++ -ansi -pedantic -Wall -DSHOW_ERRONEOUS_USING_DIRECTIVE \
      -o NamespaceClassUsingIssues.exe NamespaceClassUsingIssues.cpp 

    NamespaceClassUsingIssues.cpp: In function `int main()':
    NamespaceClassUsingIssues.cpp:120: error: use of `A' is ambiguous
    NamespaceClassUsingIssues.cpp:45: error:   first declared as `
       template<class T> class A' here
    NamespaceClassUsingIssues.cpp:10: error:   also declared as `
       template<class T> class NamespaceA::A' here
    NamespaceClassUsingIssues.cpp:120: error: parse error before `>' token
    NamespaceClassUsingIssues.cpp:121: error: use of `A' is ambiguous
    NamespaceClassUsingIssues.cpp:45: error:   first declared as `
       template<class T> class A' here
    NamespaceClassUsingIssues.cpp:10: error:   also declared as `
       template<class T> class NamespaceA::A' here
    NamespaceClassUsingIssues.cpp:121: error: parse error before `>' token
    NamespaceClassUsingIssues.cpp:122: error: `ab' undeclared (first use 
       this function)
    NamespaceClassUsingIssues.cpp:122: error: (Each undeclared identifier 
       is reported only once for each function it appears in.)
\end{verbatim}}

Note that this type of example goes against the advise in {}\cite[Item
59]{C++CodingStandards05} where the authors state that it is safe to
employ {}\texttt{using namespace SomeNamespace} directives in
{}\texttt{*.cpp} source files.  This example shows that this does not
protect the code from others that pollute the global namespace.  Note
that code that is written this way might compile one day and not the
next as it is fragile and can be broken by other people that pollute
the global namespace.

{}\textbf{Take-home Message}: Never employ {}\texttt{using namespace
AnyNamespace} in any context as you cannot guarantee the integrity of
your code since people outside of your namespace can cause your code
to not compile.

\end{enumerate}
