%
\section{Introduction}
%

{}\textbf{ToDo:} Get consensus from a critical mass of Trilinos
developers.

This document describes C++ coding guidelines for Trilinos and related
software projects.  It builds upon the guidelines in \emph{C++ Coding
Standards} by Sutter and Alexandrescu {}\cite{C++CodingStandards05}
(the 101 items are outlined in
Appendix~\ref{sec:cpp_coding_standards}) and the Teuchos-based memory
management approach described in
{}\cite{TeuchosMemoryManagementGuide}.  The guidelines in this
document specifically address the development of object-oriented
numerical C++ libraries, using the tools in the Trilinos package
Teuchos.  While this document's main purpose is to define guidelines
to improve the common look \& feel, interoperability, safety,
maintainability, and quality of Trilinos software (for both interfaces
and support software), it is also general enough to be applied to many
other C++ projects that, for instance, might interact with Trilinos.

The book \emph{C++ Coding Standards} {}\cite{C++CodingStandards05}
distills the C++ community's best practices\footnote{The book was
  written in 2004 for the previous C++ standard, but many best
  practices still apply, and all of them apply to software that does
  not yet use the the C++11 standard.} from a number of authors. It
also covers many design topics that apply more generally beyond
software written in C++.  As a result, {}\cite{C++CodingStandards05}
provides a comprehensive foundation for creating well-designed,
high-quality C++ software.  The goal of this document is not to
restate the contents of that book, but instead to provide amendments
to specific items in {}\cite{C++CodingStandards05} and tailor them for
numerical libraries.  

The zeroth item\footnote{First item, zero based.} ``Don't sweat the
small stuff'' intentionally avoids specific recommendations on issues
such as the arbitrary conventions for naming identifiers and the
formatting of code.  While issues related to coding style are less
important than other issues, there are arguments for adopting a more
consistent code formatting style.  Some of these arguments are
outlined in Appendix~\ref{sec:arguments-for-consistent-style}.
Therefore, one of the purposes of this document is to suggest
reasonable and minimal guidelines for naming conventions and code
formatting that provide for enough code uniformity to facilitate
collaborative code development, code reviews, and maintenance in Agile
software development processes
{}\cite{ExtremeProgrammingExplained2nd04}.

More important than code formatting, a consistent set of naming
conventions for C++ classes, functions, variables, and other entities
also helps to improve common look \& and feel, collaborative software
development, and overall quality.  Since clients of the software must
interact with these names, it is even more important that a set of
naming conventions be used as consistently as possible in the client
interfaces to provide common look \& feel.

Finally, important general C++ coding guidelines are covered that
complete and amend those in {}\cite{C++CodingStandards05}.  While
formatting and naming recommendations do not affect the meaning of C++
code, other coding guidelines do, and should therefore receive more
attention.  Unlike naming conventions and code formatting, these
guidelines are difficult to change after a significant amount of code
has been written.

Coding and documentation guidelines, such as defined here, are
critical in order to perform code reviews that have been proven to
improve software quality {}\cite{CodeComplete2nd04}.  Without coding
and documentation guidelines, code reviews collapse into arguments
about basic coding constructs (e.g., why not make a class's variables
public?  Or, why not use raw pointers and raw calls to new and delete
everywhere?) and are much less productive
{}\cite{ImplementingLeanSoftwareDevelopment}.

The rest of this document is organized as follows.
Section~\ref{sec:guideline_structure} describes how the guidelines are
organized and categorized.  General naming conventions are presented
in Section~\ref{sec:NamingConventions}, which they help provide a
context for later code examples.  This is followed in
Section~\ref{sec:NOSF} with guidelines for naming and organizing
source files.  Next, important general C++ coding guidelines are
described in Section~\ref{tcdg:codingguidelines:sec} that affect
software quality in critical ways.  Following this, reasonable and
minimal formatting guidelines are covered in
Section~\ref{tcdg:formatting:sec}.  Finally, guidelines for Doxygen
documentation are provided in Section~\ref{sec:DocumentionGuidelines}.

The appendices include both abbreviated references and detailed
discussions.  Appendix~\ref{sec:cpp_coding_standards} lists the 101
guidelines from {}\cite{C++CodingStandards05}, and identifies which
items are amended or invalidated by the Trilinos guidelines described
in this document.  Appendix~\ref{sec:summary_guidelines} concisely
lists the coding guidelines presented in this document.
Appendix~\ref{apdx:summary_of_idioms} summarizes the idioms and
conventions surrounding the use of the Teuchos memory management
classes.  Appendix~\ref{sec:misc_amendments} contains discussions of
items from {}\cite{C++CodingStandards05} that are amended or
invalidated in this set of coding guidelines for Trilinos.  Most
importantly, it clarifies our recommendations for {}\texttt{using}
declarations that differ from {}\cite[Item 59]{C++CodingStandards05}.
Appendix~\ref{sec:arguments-for-consistent-style} argues for adopting
a consistent code formatting style in a single development team or
single project (which current Agile development methods require).
Finally, Appendix~\ref{sec:reformatting-guidelines} gives guidelines
for when one developer may legitimately reformat a source file written
by another developer, if the developers have not agreed upon a more
consistent code formatting style.


%
\section{Structure of the guidelines}
\label{sec:guideline_structure}
%

Before getting into the specific guidelines, the structure of the
guidelines are described.  The different categories of guidelines
are presented and then alpha-numeric designations for the guidelines
are discussed.


%
\subsection{Categories of guidelines}
%

The guidelines are organized into different categories by subsection
such as naming conventions, naming and organization of source files,
general C++ coding guidelines, formatting guidelines, and finally
Doxygen documentation guidelines.  However, there are other categories
of guidelines to consider.  These different categories are managed
using keywords associated with each of the guidelines.  These
different categorizations and keywords are given below.

\begin{itemize}

{}\item Strength of the recommendation:

  \begin{itemize}

  {}\item\textbf{\StrongRecGC}: The given guideline is strongly
  recommended.  Any code that does not follow the guideline will be
  considered to either be bad code and/or to be strongly
  non-compliant.

  {}\item\textbf{\RecGC}: The given guideline is recommended.  Code
  that does not follow the guideline will be questioned but will not
  necessarily considered bad code.  However, code that does not follow
  the guideline will be considered non-complaint.

  {}\item\textbf{\SuggestGC}: The given guideline would be suggested
  as good alternative to choose over other alternatives.  However, code
  that does not follow the guideline will not be considered bad code
  or to necessarily be non-complaint code.

   \end{itemize}

{}\item\textbf{\InteropGC}: The guideline plays a primary role in the
interoperability with other compliant software.  Code that violates a
{}\StrongRecGC{} guideline in this category will be considered not
directly interoperable with other compliant software.  (However, it
may be possible to develop adapters that provide an interoperable 
interface.)

{}\item\textbf{\ComLookFeelGC}: The guideline plays a primary role in
the common look and feel of compliant software.  Common look and feel
is important for users in helping them to reuse knowledge gained from
one set of compliant software in other sets of compliant software.
Common look and feel helps avoid user programming errors and improve
uniformity of user code.

{}\item\textbf{\SafetyGC}: The guideline is primarily concerned with
safety where if ignored would result in software that is more
difficult for users to use correctly, more difficult to maintain, or
more likely to contain undefined behavior (e.g., memory access
errors).  For example, the use of {}\ttt{const} is considered to be a
safety issue.  Maintainability, portability, and code clarity are also
closely related to safety so {}\SafetyGC{} will be used to also cover
guidelines that address these attributes as well.

{}\item\textbf{\TeamGC}: The guideline mainly affects the ability of a
team to work together effectively.  Almost every convention that
increases uniformity in the code helps to improve team interaction and
productivity.  Therefore, every one of these guidelines could be
labeled {}\TeamGC{}.  This would not be informative, though.  Instead,
we will only label guidelines with {}\TeamGC{} when that is the
primary motivation for the guideline.  For example, code formatting
generally does not impact end users at all (unless they look at raw
code in header files) yet can greatly affects shared ownership of
code.

\end{itemize}

Every guideline is given a recommendation strength and at least one
other guideline category.  However, a guideline may be assigned more
than one non-strength category (e.g., \InteropGC{} and \SafetyGC).


%
\subsection{Alphanumeric item designations}
\label{sec:alpha_numeric_designations}
%

We refer to specific items in this document using numbered acronyms
starting with \textbf{TCDG} and the version number (e.g., 1.0).  For
example, we label the first naming convention guideline as
\textbf{TCDG 1.0 NC 1}.  These abbreviations may be used in code
reviews as short-hand references to specific guidelines.  The version
number of the coding standard is important in order to allow changes
in future coding guidelines and allow the numbers to change from
version to version.  For example, \textbf{NC 1} in \textbf{TCDG
  1.0} might become \textbf{NC 3} in \textbf{TCDG X.Y}.  We refer
to guidelines in \cite{C++CodingStandards05} using a numbered
acronym such as \textbf{CPPCS Item 15} (``Use const proactively'').

Whenever a new guideline is added or the ID of an existing guideline
is changed, the version number must be incremented.


%
\section{Naming conventions (NC)}
\label{sec:NamingConventions}
%

C++ classes, functions, variables, data members etc. should be named
and used consistently.  The following guidelines are consistent with
common practice as exemplified in {}\cite{EffectiveC++3rd05} for
example and are also largely consistent with the Java naming
standard.\footnote{
  {}\ttt{http://java.sun.com/docs/codeconv/html/CodeConvTOC.doc.html}}
Naming conventions affecting external interfaces are given in
Section~\ref{sec:ExternalNamingConventions} while naming convention
suggestions for internal software entities are given in
Section~\ref{sec:InternalNamingConventions}.  We give naming
conventions first, before coding guidelines (in Section
\ref{tcdg:codingguidelines:sec}), not because they are more important,
but because they are used in code examples throughout this text.


%
\subsection{External naming conventions}
\label{sec:ExternalNamingConventions}
%

This section gives naming conventions for external interfaces that
affect the user experience and common look and feel.

\begin{itemize}

{}\item\NCNamespaceAllNames: For C++ class names, functions, and
variables, use C++ namespaces which include the package name (e.g.,
{}\ttt{Thyra::VectorBase}, {}\ttt{Tpetra::Map}, etc.).  For other
global identifiers such as C and Fortran functions and macros, use the
package name as a prefix (e.g., {}\ttt{MPI\_Comm()},
{}\ttt{TEUCHOS\_ASSERT()}).  Under no circumstance should any globally
visible symbol be used (including at link time) that is not namespaced
(e.g., no global struct with a name like {}\ttt{Vec}).

{}\textit{Justification}: Any software that does not namespace its
global identifiers is unsuitable to be integrated with other software.
A single software project's sloppiness can pollute the global
namespace.

{}\item\NCClassNames: Names for C++ classes and structs should
generally be capitalized.  Separate words should be concatenated and
capitalized (this convention is called ``CamelCase'').  For example:

{\small\begin{verbatim}
  class SomeClass {...};
\end{verbatim}}


{}\item\NCNamespaceNames: C++ namespaces should follow the same naming
convention as C++ classes.  Namespace names should not contain too many
acronyms and should not be too short or too common.  For example:

{\small\begin{verbatim}
  namespace MyNameSpace {
  ...
  } // namespace MyNameSpace 
\end{verbatim}}


{}\item\NCEnumNames: Enumeration type names should follow the same
convention as for class and struct names, but they should also begin
with the capital letter 'E' (signifying an enum type).  Enumeration
values should be all uppercase with underscores between words and
should use a common prefix for scoping within the enum type.  Also,
enum values should in general use the default value assignment defined
by the compiler, as this aids their use as indexes into zero-based
arrays.  For example:

{\small\begin{verbatim}
  enum ESolveStatus {
    SOLVE_STATUS_CONVERGED,
    SOLVE_STATUS_UNCONVERGED,
    SOLVE_STATUS_UNKNOWN
  };
\end{verbatim}}


{}\textit{Justification}: Using a capital 'E' for {}\texttt{enum}s allows the
definition of other types with the same basic name that contain other data.
For example, {}\texttt{ESolveStatus} in an {}\texttt{enum} enumerating the
different types of solve status and {}\texttt{SolveStatus} is a C++ struct
that contains an {}\texttt{ESolveStatus} member along with some other data.
The use of the scoping prefix (i.e.\ {}\texttt{SOLVE\-\_STATUS\_} above) is
also recommended in {}\cite[Section 11.4]{CodeComplete2nd04}.


{}\item\NCObjectNames: Formal function arguments and other object identifiers
should, in general, start with a lower-case letter and then use capitalization
for following words with no underscores between words in general.  For
example:

{\small\begin{verbatim}
  ClassType1 obj;
  ClassType2 objectForMyThing;
  ClassType3 objectForYourThing;
\end{verbatim}}


{}\textit{Exception:} Identifiers that have mathematical symbols in
them such as {}\texttt{x}, {}\texttt{J}, and {}\texttt{alpha} should
use lower case names separated by underscores {}\texttt{\_}.  For
example:

{\small\begin{verbatim}
  Vector curr_x;
  Matrix curr_J;
  Scalar curr_alpha;

  class SomeClass {
  public:
    const Vector& get_x() const;
    const Matrix& get_J() const;
    Scalar get_alpha() const;
  };
\end{verbatim}}


\textit{Justification:} The Java convention
\texttt{objectIdentifierName} using capitalization with no underscores
produces shorter readable identifiers for English names but does not
work well for identifiers with mathematical symbols.  It is important
to maintain the case of mathematical symbols, since \texttt{x} and
\texttt{X} in mathematical notation may denote different
objects.\footnote{For example, the ``Householder convention'' commonly
  used by numerical analysts prefers to reserve Roman capitals for
  matrices, and Roman lower-case letters for vectors.}  Omitting the
underscore would result in confusing ambiguity regarding the symbol's
case, as in for example {}\texttt{currx} or {}\texttt{currX}.  In
these cases, it is far better to use underscores and write
\texttt{curr\_x} as shown above.  While many programmers consider it
bad practice in general to differentiate variable names by case alone
(see ``Don't differentiate variable names solely by capitalization''
in \cite[Section 11.7]{CodeComplete2nd04}), mathematical software must
support the mathematical convention that case matters.


\item\NCFunctionNames: Names for functions should use the same naming
  convention as for object identifier.  For example:

{\small\begin{verbatim}
  class SomeClass {
  public:
    void someMemberFunction(...);
  };

  void someOtherFunction(...);
\end{verbatim}}


\item\NCBaseDefaultClassNames: In general, the top-level C++ base
  class for some abstraction should use an unadorned name (e.g.,
  \texttt{Vector}, rather than \texttt{VectorBase}).  The base class
  should contain (almost) no implementations and certainly no object
  data (see \cite[Item 36]{C++CodingStandards05}).  If a default
  implementation of some of the aspects of the base class are desired
  (to make it easier to define concrete subclasses), then they should
  be put in a derived node subclass with the suffix
  \texttt{DefaultBase} (e.g., \texttt{Vector\-Default\-Base}).  Any
  default concrete implementation of an abstraction should generally
  use the prefix \texttt{Default} along with any other important
  prefixes (e.g., \texttt{DefaultSpmdVector}).  For example:

{\small\begin{verbatim}
  // Pure virtual base class 
  class Vector
    : ... // Other base classes
  {
  public:
    virtual void applyOp(...) const = 0;
    ...
  };

  // Node base class with some default implementations
  class VectorDefaultBase : virtual public Vector
  {
  public:
    void applyOp(...) const; // default implementation
    ...
  };

  // A general default implementation for SPMD vectors
  class DefaultSpmdVector
    : virtual public VectorDefaultBase // use some default implementations
  {
  public:
    void applyOp(...) const; // Specialized overrides
    ...
  private:
    ...
  };
\end{verbatim}}

  \textit{Alternate}: An alternative naming convention is to state the
  noun before the adjective.  This results in the base class and all derived classes starting with the same noun.  The preceding example would now be:

{\small\begin{verbatim}
  // Pure virtual base class 
  class Vector
    : ... // Other base classes
  {
  public:
    virtual void applyOp(...) const = 0;
    ...
  };

  // Node base class with some default implementations
  class VectorDefaultBase : virtual public Vector
  {
  public:
    void applyOp(...) const; // default implementation
    ...
  };

  // A general default implementation for SPMD vectors
  class VectorDefaultSpmd
    : virtual public VectorDefaultBase // use some default implementations
  {
  public:
    void applyOp(...) const; // Specialized overrides
    ...
  private:
    ...
  };
\end{verbatim}}

\textit{Justification}: The alternative naming convention helps to group base class and derived classes together when the objects are listed alphabetically.  This is most helpful when browsing Doxygen and for listing file names in a directory.  

\item\NCConstNonconstAccessFunctionName: In general, functions that
  return objects that are contained within a wrapper object should
  have the prefix \texttt{Nonconst} added to the function that returns
  the non-const reference (or pointer) to the contained object.  For
  example,

{\small\begin{verbatim}
  class SomeClass {
  public:
    RCP<Part> getNonconstPart();
    RCP<const Part> getPart() const;
    ...
  };
\end{verbatim}}

  \textit{Justification}: The choice to name the access functions
  \texttt{getNonconstPart()} and \texttt{getPart()} as opposed to
  \texttt{getPart()} and \texttt{getConstPart()} is somewhat
  arbitrary.  However, using \texttt{nonconst} should be preferred in
  order to make it more explicit that a non-const object reference is
  being requested.  Also, a constant view of a part of an object is
  always cheaper that returning a non-constant view of the part (see
  the discussion of the ``generalized view'' design pattern in
  {}\cite{TeuchosMemoryManagementGuide}).  Therefore, to be safe and
  err on the side of efficiency, the non-constant access function
  should be harder to call (because it requires more characters to
  type) than the constant access function.

\end{itemize}


%
\subsection{Internal naming conventions}
\label{sec:InternalNamingConventions}

This section gives naming conventions that affect internal software
entities.  Such naming conventions do not directly affect users,
unless they decide to inspect the code or run a debugger on it.
However, internal naming conventions do strongly influence
collaborative software development and the shared ownership of code.
In fact, such internal naming conventions might actually influence
collaborative development more than external naming conventions due to
the amount of code involved.  Thus, they are an important part of
forming a cohesive productive team.  The conventions given below are
used in the Thyra package as well as much of Teuchos.


\begin{itemize}
\item\NCDataMemberNames: Names for data members within a class should
  use the same naming convention as for other object identifier names,
  but should end with an underscore.  For example:

{\small\begin{verbatim}
  class SomeClass {
  public:
    ...
  private:
    int someDataMember_;
  };
\end{verbatim}}

  \textit{Justification:} Using an underscore after a data variable
  name helps class maintainers identify the variable's scope, and
  differentiate it from a local variable or a member function.  This
  immediate visual identification avoids ``shadowing'' declarations,
  which cause portability problems on some compilers, especially when
  warnings are elevated to errors.

  \textit{Exception}: Public data members in simple C++ structs (i.e.\
  where no invariants need to be maintained) should not contain
  underscores.  For example:

{\small\begin{verbatim}
  struct SolveStatus {
    ESolveStatus solveStatus;
    double achievedTol;
    std::string message;
    ...
  };
\end{verbatim}}

  \textit{Exception:} Names for data members within a class that have
  mathematical symbols in them, such as {}\texttt{x}, {}\texttt{J},
  and {}\texttt{alpha}, should use lower-case names separated by
  underscores {}\texttt{\_}.  For example:

{\small\begin{verbatim}
  Vector curr_x_;
  Matrix curr_J_;
  Scalar curr_alpha_;
\end{verbatim}}

\textit{Justification:} See {}\textbf{NC 4} above.

\textit{Alternate:} Instead of using an underscore at the end of the name, a prefix {}\texttt{m\_} can be used to distinguish class data members.  For example:

{\small\begin{verbatim}
  class SomeClass {
  public:
    ...
  private:
    int m_someDataMember;
  };
\end{verbatim}}

\textit{Justification:}  This can sometimes be easier to read, especially when dereferencing pointers.

\end{itemize}


%
\section{Naming and organization of source files (NOSF)}
\label{sec:NOSF}
%

Since most C++ code is organized around classes, the file structure
should also primarily be organized around classes and the nonmember
functions that interact with these classes.  The primary goal of these
file naming guidelines is to create file names that are globally
unique and will therefore facilitate {}\texttt{\#include}s without
need for directory paths in the {}\texttt{\#include} statement.  The
basic idea is that a source file should be named based on what it has,
not where it is.  The following guidelines help to define how to
organize code into source files and how to name those source files.
The directory structure of source files is beyond the scope of this
document.


\begin{itemize}

\item\NOSFFileExtension: These file names avoid common problems with
  portability to various Unix and Windows platforms and enable better
  tools support (like language-specific formatting in Emacs).

\item\NOSFClassFiles: As a general rule, assign the source code for
  any major C++ class and supporting code to a single set of header
  and source files.  The file name should be composed from the
  namespace names enclosing the classes and other code, along with the
  class name itself.  For instance, for the class
  \texttt{NameSpaceA\-::InnerNamespace\-::SomeClass}, the header and
  source files would be named
  \texttt{NameSpaceA\-\_InnerNamespace\-\_SomeClass.[hpp,cpp]}.  This
  convention assures global uniqueness of file names.  In addition,
  having a single set of files for each class helps keep a single
  encapsulation unit of code together, which makes it easier to search
  the encapsulation unit.

\item\NOSFIncludeGuards: All header files, without exception, should
  use include guards {}\cite[Item 24]{C++CodingStandards05}.  For
  example, the file
  {}\texttt{NameSpaceA\-\_InnerNamespace\-\_SomeClass.hpp} would have
  the basic structure:

  {\small\begin{verbatim}
  #ifndef NAMESPACEA_INNERNAMESPACE_SOMECLASS_HPP
  #define NAMESPACEA_INNERNAMESPACE_SOMECLASS_HPP

  #include "SomeFile.hpp"

  ...

  #endif // NAMESPACEA_INNERNAMESPACE_SOMECLASS_HPP
  \end{verbatim}}

  Above, the comment {}\texttt{//
  NAMESPACEA\-\_INNERNAMESPACE\-\_SOMECLASS\-\_HPP} after the final
  \texttt{\#endif} helps to show the preprocessor structure in the
  file.  It is helpful in cases where other \texttt{\#ifdef} or
  \texttt{\#if} structures are used.  This is a very minor amendment
  to \cite[Item 24]{C++CodingStandards05}.

\item\NOSFCopyrightHeaders: All copyrightable software source file should
include a copyright headers.  For example:

  {\small\begin{verbatim}
  // @HEADER
  // ...
  // @HEADER
  ...
  \end{verbatim}}

  \textit{Justification}: The tags {}\ttt{@HEADER} are looked for in
  automated tools to update the copyright header for large groups of
  files.  Without some standard for setting off copyright text, it is
  much more difficult to maintain correct and up-to-date copyright
  info.

\end{itemize}

%
\section{Coding guidelines}
\label{tcdg:codingguidelines:sec}
%

Coding guidelines have much more influence on the meaning and
correctness of C++ than formatting guidelines, and thus deserve more
attention.  We assume by default all of the coding guidelines in the
book \emph{C++ Coding Standards} \cite{C++CodingStandards05}.  In this
document, we provide additional coding guidelines and, in some cases,
amend items in {}\cite{C++CodingStandards05}.  Where this document is
silent, {}\cite{C++CodingStandards05} is to be considered the
authoritative source.  Appendices \ref{sec:cpp_coding_standards} and
\ref{sec:misc_amendments} amend some items in
\cite{C++CodingStandards05}.

%
\subsection{General coding guidelines (GCG)}
%

This section discusses general coding guidelines.  These guidelines
affect software quality in a major way and are not just a matter of
personal preference or style.

%
\subsubsection{Error handling}
%

\begin{itemize}

\item\GCGTestForException: For developer errors, prefer to throw
  exceptions derived from {}\texttt{std\-::logic\-\_error} instead of
  using the {}\texttt{assert(...)}  macro as recommended in
  {}\cite[Item 68]{C++CodingStandards05}.  A ``logic error'' would be
  treated differently from a real run-time error and would therefore
  come with different assumptions about the state of the object after
  the exception was thrown.  In particular, a ``real error'' (i.e.,
  not just an internal developer error) should always provide the
  basic guarantee to leave the object in a valid state \cite[Item
  71]{C++CodingStandards05}, while code that throws a ``logic error''
  cannot make any such guarantees in general.  Therefore, if an
  object's method throws an exception derived from
  \texttt{std\-::logic\-\_error}, the object should in general be
  viewed as unusable and should be deleted immediately.  To debug
  exceptions, a break point may be placed on the function
  {}\texttt{Teuchos::Test\-For\-Exception\-\_break()}\footnote{In
  {}\texttt{gdb}, a break-point would be set as {}\texttt{b
  Teuchos::Test\-For\-Exception\-\_break()}.} which will be called
  just before an exception is thrown through these macros.  In the
  future, more sophisticated features like automatically attaching a
  debugger or printing the call stack may be added for some systems.
  Therefore throwing an exception derived from
  \texttt{std\-::logic\-\_error} using these macros should be
  preferred to using the {}\texttt{assert(...)} macro as it gives us
  more control over what happens when one of these types of
  programming errors occurs.  Also, these exception macros make it
  much easier to generate better error messages than what you would
  get from a simple use of the \texttt{assert(...)} macro.

{}\item\GCGStandardCatchStatements: Catch all exceptions before they
exit {}\ttt{main()} by writing it as:

{\small\begin{verbatim}
  #include "Teuchos_VerboseObject.hpp"
  #include "Teuchos_StandardCatchMacros.hpp"
  ...
  int main( int argc, char* argv[] )
  {
    const Teuchos::RCP<Teuchos::FancyOStream> out =
      Teuchos::VerboseObjectBase::getDefaultOStream();
    ...
    bool success = true;
    try {
      ...
    } 
    TEUCHOS_STANDARD_CATCH_STATEMENTS(true, *out, success);
    return (success ? EXIT_SUCCESS : EXIT_FAILURE);
  }
\end{verbatim}}

By using {}\ttt{TEUCHOS\-\_STANDARD\-\_CATCH\-\_STATEMENTS(...)}, the
process number will be printed and the stack-trace from any exceptions
thrown using the standard Teuchos macros will be printed.

\end{itemize}


%
\subsubsection{Memory management}
%


\begin{itemize}


{}\item\GCGAvoidRawPointers: The Teuchos memory management approach
described in {}\cite{TeuchosMemoryManagementGuide} mentioned below
which include all of the standard C++ container classes (when using a
checked STL implementation), {}\texttt{Teuchos::\-Ptr},
{}\texttt{Teuchos::\-RCP}, {}\texttt{Teuchos::\-Array},
{}\texttt{Teuchos::\-Array\-RCP}, and
{}\texttt{Teuchos::\-Array\-View} allow all code to be written without
any explicit raw C++ pointers and eliminate undefined behavior (i.e.\
instead of segfaults) from C++ programs related to the usage of
memory.  In debug mode, these classes eliminate undefined behavior by
providing full run-time checking that results in exceptions being
thrown and excellent error messages.  When a checked C++ standard
library is used (e.g., when {}\texttt{\_GXXLIB\_DEBUG} is defined with
g++), then all of the standard C++ library classes are checked as
well.


{}\item\GCGUseStdString: While {}\texttt{std\-::string} is not debug checked
in a typical implementation, indexing and other unchecked operations with
{}\texttt{std\-::string} objects is much less common in numerical code and
therefore is less likely to result in memory-usage errors inside of numerical
code.  However, when a checked C++ library implementation is used (e.g., when
{}\texttt{\_GXXLIB\_DEBUG} is defined with g++), then {}\texttt{std\-::string}
is very safe.


{}\item\GCGTeuchosRCP: (see Tables~\ref{fig:func_args_value_type} and
{}\ref{fig:func_args_ref_type}): {}\ttt{Teuchos\-::RCP} is critical to
use for persisting associations\footnote{\textit{Persisting
associations} are associations that exist between two or more objects
that extend past a single function call for formal function arguments,
or a single statement for function return objects
{}\cite{RefCountPtrBeginnersGuide, TeuchosMemoryManagementGuide}.} 
because it ensures that objects remain valid until they are no longer
needed yet ensures that objects are properly destroyed.  Replace all
mentions to the class {}\texttt{boost::\-shared\_ptr} with
{}\texttt{Teuchos::\-RCP} in all items in
{}\cite{C++CodingStandards05}.


{}\item\GCGTeuchosPtr: (see Tables~\ref{fig:func_args_value_type} and
{}\ref{fig:func_args_ref_type}): The class {}\texttt{Teuchos\-::Ptr}
simply takes the place of a raw pointer to a single object but is
always default initialized to NULL.  In debug mode, it throws
exceptions when trying to dereference a null pointer.  For
non-persisting associations\footnote{\textit{Non-Persisting
associations} are associations between two or more objects that exist
only within a single function call for formal function arguments, or a
single statement for function return objects, where no memory of any
of the objects is retained as a side effect after the function returns
or the statement ends {}\cite{RefCountPtrBeginnersGuide,
TeuchosMemoryManagementGuide}.} {}\ttt{Teuchos\-::Ptr} is important
because it ensures that valid objects are returned on dereferencing.
The use of {}\ttt{Teuchos\-::Ptr} for semi-persisting
associations\footnote{\textit{Semi-persisting associations} are
associations that (like persisting associations) exist between two or
more objects that extend past a single function call for formal
function arguments, or a single statement for function return objects
except where the use of the objects and the lifetime of the
association have more rigid constraints requiring greater care in use
{}\cite{RefCountPtrBeginnersGuide, TeuchosMemoryManagementGuide}.} is
important because it can detect dangling references where the pointer
is dereferenced after the underlying object has been destroyed.


{}\item\GCGNonmemberConstructors: Using non-member constructors gives
greater flexibility in how a class object is initialized, simplifies
the maintenance of the class, and makes the debug-mode node tracing
checking bullet-proof {}\cite{TeuchosMemoryManagementGuide}.

Non-member constructors take the form:

{\small\begin{verbatim}
  class SomeClass {
  public:
    // No public constructors!
    ..
  };

  // Non-member constructor
  RCP<SomeClass> someClass(...);
\end{verbatim}}


{}\item\GCGGeneralizedViews: Using ``generalized view'' semantics
leads to the greatest implementation freedom and the best performance
in all cases; albeit with more strict usage patterns (see the
``generalized view'' design pattern in
{}\cite{TeuchosMemoryManagementGuide}).

If {}\ttt{SomeBaseClass} provides a view if itself as {}\ttt{Part}
objects, then applying the generalized view design pattern results in
the interface functions:

{\small\begin{verbatim}
  class SomeBaseClass {
  public:
    virtual RCP<Part> getNonconstPart() = 0;
    virtual RCP<const Part> getPart() const = 0;
    ...
  };
\end{verbatim}}

The ``generalized view'' design pattern along with a concrete example
from Thyra is described in great detail in
{}\cite{TeuchosMemoryManagementGuide}.  Note that views of concrete
classes do not have to use ``generalized view'' semantics and can
instead use ``direct view'' semantics where appropriate.  See all the
details about the ``non-member constructor'' idiom in
{}\cite{TeuchosMemoryManagementGuide}.


{}\item\GCGTeuchosArrayView: (see Tables~\ref{fig:func_args_value_type} and
{}\ref{fig:func_args_ref_type}): This class allows all of the useful
capabilities of a {}\texttt{std::vector} which do not include adding or
removing entries.  In debug mode, all of the access functions (including
iterators) are fully checked.  In optimized mode, unchecked raw pointers are
used and the only overhead is a size argument (which is usually passed with
raw pointers anyway).


{}\item\GCGTeuchosArray: (see Tables~\ref{fig:func_args_value_type}
and {}\ref{fig:func_args_ref_type}): The primary reason to use
{}\texttt{Teuchos::\-Array} instead of {}\texttt{std::\-vector} is
that {}\texttt{Teuchos::\-Array} is part of the Teuchos system of
memory management types and results in stronger run-time checking.
While {}\texttt{Teuchos::\-Array} gets all of its real functionality
from {}\texttt{std::\-vector}, prefer to use
{}\texttt{Teuchos::\-Array} as it provides more capabilities and
portable debug checking.  For instance
{}\texttt{Teuchos::\-Array::\-operator[]} is range checked in debug
mode regardless whether there is an underling checked STL
implementation or not (see {}\cite[Item 83]{C++CodingStandards05}).
In debug mode, the iterator is also run-time checked.  In addition,
{}\texttt{Teuchos\-::Array} will automatically convert into an
{}\texttt{Teuchos\-::ArrayView} object safely when used in function
calls and in debug mode, will catch dangling references
{}\cite{TeuchosMemoryManagementGuide}.


{}\item\GCGTeuchosArrayRCP: (see Tables~\ref{fig:func_args_value_type} and
{}\ref{fig:func_args_ref_type}): Note that {}\texttt{Teuchos::ArrayRCP} does
{}\underline{not} take the place of a contiguous container class such as
{}\texttt{Teuchos::\-Array}.  A {}\texttt{Teuchos::ArrayRCP} object cannot
change the size of the array, it can only provide for reference-counted
sharing of an array of data of fixed size and provide sub-views of contiguous
parts of the managed array.  All access to data (both through
{}\texttt{Teuchos::\-ArrayRCP::\-operator[]} and iterators) is run-time checked
in a debug build.


{}\item\GCGReturnSmartPointersByValue: (see
Tables~\ref{fig:func_return_value_type} and
{}\ref{fig:func_return_reference_type}): Returning smart pointer
objects by value is critical for properly setting up the machinery for
persisting and semi-persisting associations and to fully enabled
debug-mode checking {}\cite{TeuchosMemoryManagementGuide}.


{}\item\GCGRawReferencesNonPersisting: (see Tables
{}\ref{fig:func_args_value_type} and {}\ref{fig:func_args_ref_type}):
Raw C++ references cannot be used to detect dangling references in a
debug-mode build and therefore should only be used for non-persisting
associations {}\cite{TeuchosMemoryManagementGuide}.


{}\item\GCGSemiPersisting: (see Tables
{}\ref{fig:func_args_value_type}, {}\ref{fig:func_args_ref_type},
{}\ref{fig:func_return_value_type}, and
{}\ref{fig:func_return_reference_type}): Objects of type {}\ttt{Ptr}
and {}\ttt{ArrayView} are light-weight and efficient in a non-debug
mode build but are fully checked in a debug-mode build and therefore
lead to safe efficient code {}\cite{TeuchosMemoryManagementGuide}.


{}\item\GCGPostponeRawPointers: When raw C++ pointers must be exposed to
communicate with other code that uses raw C++ pointers, encapsulate the raw
C++ pointer as fast as possible and then only give up a raw pointer at the
last possible moment.  For example,

{\small\begin{verbatim}
  SomeForeignClass* get_raw_foreign_obj_ptr();
  do_some_foreign_stuff(SomeForeignClass* foreign_obj_ptr);

  void foo()
  {
    // Get the raw pointer into a proper encapsulated class object right away!
    Ptr<SomeForeignClass> foreignObj(get_raw_foreign_obj_ptr());

    // Lots of code ...

    // Only expose the raw pointer directly in the foreign function call! 
    do_some_foreign_stuff(&*foreignObj);
  }
\end{verbatim}}


\end{itemize}


%
\subsubsection{Object control}
%


\begin{itemize}


{}\item\GCGTeuchosParamterListAcceptor: The
{}\texttt{Teuchos\-::Parameter\-List} class provides many useful features that
make it easy to accept user options in a flexible and fully validated way (see
Teuchos documentation for more details).  The
{}\texttt{Teuchos\-::Parameter\-List\-Acceptor} interface defines a consistent
flexible protocol for setting and managing a parameter list.


{}\item\GCGTeuchosParamterListValidation: All user parameters and
sub-lists passed in through a
{}\texttt{Teuchos\-::Parameter\-List\-Acceptor} should be fully
validated.  The main tool for this is the member function
{}\texttt{validateParameters(...)}.  Using this function and other
other approaches, when a user misspells a parameter or sub-list, uses
the wrong type for a parameter, or provides an invalid parameter
value, they will get an exception thrown with a helpful error message.
Also, objects are only responsible for validating their own parameters
and sub-lists, and not those of other objects that they hold sub-lists
for.


\end{itemize}


%
\subsubsection{Object introspection}
%


\begin{itemize}
\item\GCGStdStream: Sending output directly to \texttt{std\-::cout} or
  \texttt{std\-::cerr} destroys the flexibility of numerical software
  and does not perform well in SPMD programs.  Instead, produce output
  using one of the following approaches.

  \begin{itemize}
  \item\GCGTeuchosFancyOStream: A \texttt{Teuchos\-::Fancy\-OStream}
    class object can wrap any \texttt{std\-::ostream} object and helps
    to produce structured indented output and to create more readable
    output in an SPMD program (even when every processor produces
    output).
  
  \item\GCGTeuchosDescribable: The \texttt{Teuchos\-::Describable}
    interface is the appropriate way to allow clients to print the
    current state of an object in a flexible way.  The verbosity of
    the output is controlled by an input enum parameter.
  
  \item\GCGTeuchosVerboseObject: Clients can set output streams and
    verbosity levels in code directly through the
    {}\texttt{VerboseObject} interface or (more limited) through the
    parameter list (see the
    {}\texttt{Teuchos\-::ParameterListAcceptor} interface described
    above).  Any {}\texttt{FancyOStream} object can be set in C++ code
    (not through the parameter list) or an output file can be given in
    the parameter list to accept output on an object-by-object basis.
    The verbosity level can be set in code or more generally
    specialized on an object-by-object based in the parameter list.
    If no object-specific output stream is set, then
    {}\texttt{Teuchos\-::Verbose\-Object\-Base\-::get\-Default\-OStream()}
    will be used.
  
  \item\GCGTeuchosVerboseObjectBaseDefaultOStream: The stream provided
    by
    \texttt{*Teuchos\-::Verbose\-Object\-Base\-::get\-Default\-OStream()}
    is set up by default to do clean printing in an SPMD program and
    can also be set up through a
    \texttt{Teuchos\-::Command\-Line\-Processor} object to control how
    output is produced and formatted.

  \end{itemize}

{}\item\GCGTeuchosTimers: To properly support timers in Trilinos, do
the following (where {}\ttt{<UCPACAKGENAME>} is the upper-case package
name such as {}\ttt{THYRA}, {}\ttt{RYTHMOS}, etc.):

  \begin{itemize}

  {}\item Add
  {}\ttt{PACKAGE\_ADD\_ENABLE\_TEUCHOS\_TIME\_MONITOR\_OPTION()} to
  the package's top-level CMakeLists.txt file.

  {}\item Add
  {}\ttt{@<UCPACAKGENAME>\_TEUCHOS\_TIME\_MONITOR\_DECLARATIONS@} to
  the packages {}\ttt{\$\{PACKAGE\_NAME\}\_config.h.in} file.

  {}\item To add timers to specific C++ functions, in the body of the
  functions add calls to
  {}\ttt{<UCPACAKGENAME>\_FUNC\_TIME\_MONITOR("<PACKAGENAME>:
  Somefunction()")} and
  {}\ttt{<UCPACAKGENAME>\_FUNC\_TIME\_MONITOR\_DIFF("<PACKAGENAME>:
  Somefunction(): some part", SomePart)} (the latter when there is
  more than one timer in a function).

  \end{itemize}

The macros {}\ttt{<UCPACAKGENAME>\_FUNC\_TIME\_MONITOR[\_DIFF]} have
empty definitions when timers are disabled.  See
TrilinosCMakeQuickstart.txt for user-level details.

\end{itemize}


%
\subsubsection{Miscellaneous coding guidelines}
%

\begin{itemize}

\item\GCGSafeNamespaceUsage: This is an amendment to Item 59 in
  {}\cite{C++CodingStandards05} which is explained in
  Appendix~\ref{sec:using_amendments}.

\item\GCGExplicitTemplateArguments: Some C++ compilers have trouble
  inferring the template arguments in a template function call.
  Explicitly stating the template arguments in that case can improve
  portability to those compilers that do not correctly implement
  template argument inference.  For example, in Thyra, every
  non-member function is templated on the \texttt{Scalar} type such
  as:

{\small\begin{verbatim}
  template<class Scalar>
  sum(const VectorBase<Scalar> &x);
\end{verbatim}}

When portability is a concern or when implicit conversions in the input
arguments are needed, then prefer to call such functions by specifying the
template argument(s) as:

{\small\begin{verbatim}
  Scalar mySum = sum<Scalar>(myVec);
\end{verbatim}}


\item\GCGTeuchosAs: The templated C++ function
  \texttt{Teuchos::as<T\_to>(T\_from)} and the class specializations
  that it calls invoke run-time tests, in debug mode, to ensure
  correctness of the results of a conversion.  This includes the
  conversion of strings into numbers (that is, replacing
  \texttt{atof()} and \texttt{atoi()}) as well as conversions that may
  result in loss of precision or meaning (such as \texttt{double} to
  \texttt{int}, \texttt{long int} to \texttt{int}, \texttt{int} to
  \texttt{char}, or \texttt{unsigned int} to \texttt{int}).  The
  optimized implementations of these conversion functions are
  typically unchecked for speed.  A version of this function which
  always does run-time checking, called
  \texttt{Teuchos\-::asSafe\-<\-T\_to\->\-(\-T\_from\-)}, is also
  available in order to validate user data.

  \textit{Justification}: Unchecked conversions result in many
  different kinds of errors.  A safe program needs to be able to check
  all potentially unsafe conversions at run time.  The implicit
  conversion rules allowed in C which were carried over to C++ may
  result in unsafe code.

\item\GCGNamespaceClassFuncDefs: The member functions of a class
  should be defined in the same order as their declarations and should
  generally be defined within a namespace enclosure.  For example,
  given the declaration of

{\small\begin{verbatim}
  ----------------------------------
  // SomeNamespace_SomeClass.hpp

  namespace SomeNamespace {

  class SomeClass {
  public:
    void someFunc();
    ...
  };

  } // namespace SomeNamespace
  ----------------------------------
\end{verbatim}}

the safest and tersest ways to define the member functions in the
source file is

{\small\begin{verbatim}
  ----------------------------------
  // SomeNamespace_SomeClass.cpp

  namespace SomeNamespace {

  void SomeClass::someFunc()
  {
    ...
  }

  } // namespace SomeNamespace
  ----------------------------------
\end{verbatim}}

\textit{Justification}: Using the namespace enclosure instead of a
\texttt{using namespace SomeNamespace} directive insures that you can
never accidentally provide another definition for some other class
member function in another namespace.  Explicit namespace
qualification is not needed since misspelling any part of the
prototype will result in a compile-time error.


\item\GCGNamespaceFuncDefs: For example, for the nonmember function
  prototype

{\small\begin{verbatim}
  ----------------------------------
  // SomeNamespace_someFunc.hpp

  namespace SomeNamespace {

  void someFunc( const int data );

  } // namespace SomeNamespace
  ----------------------------------
\end{verbatim}}

the safest way to define the nonmember function is

{\small\begin{verbatim}
  ----------------------------------
  // SomeNamespace_someFunc.cpp

  void SomeNamespace::someFunc( const int data )
  {
    ...
  }

  ----------------------------------
\end{verbatim}}

\textit{Justification}: Using explicit namespace qualification avoids
spelling and other mistakes that can accidentally result in the
definition of a new function \cite[Section 8.2]{stroustrup97}.
While the linker catches such mistakes, it can be very hard to
diagnose the problem when this happens.


\item\GCGFuncArgsOrder: For example:

{\small\begin{verbatim}
  void someFunc(
    const T1 &arg1,         // Input
    const Ptr<T2> &arg2,    // Input/Output
    const Ptr<T3> &arg3,    // Output
    const int arg4 = 0      // Optional input argument with default value
    );
\end{verbatim}}

This ordering of arguments is only a general suggestion as a different
ordering of arguments may be chosen based on other criteria.  See
Section~\ref{sec:func_arguments} for a description of the use of the
\texttt{Ptr} class.


\item\GCGNomemberObjectFuncs: For example:

{\small\begin{verbatim}
  void someModifyingFunc(
    SomeClass &obj,
    const int arg1,
    ...
    );

  void someNonModifyingFunc(
    const SomeClass &obj,
    const int arg1,
    ...
    );
\end{verbatim}}

Note that in the case of \texttt{someModifyingFunc(...)}, the output
argument is listed first instead of after the input argument(s) which breaks
typical convention of having input/output arguments (which all objects that
are modified are) come after input arguments.  However, this is more
consistent with established convention such as in Python and other languages
where the \texttt{self} argument is always the first explicit (or implicit)
argument.  Note that this is also a situation where a non-const reference
argument makes the most sense.


\item\GCGEnumFuncArgs: While the built-in type \texttt{bool} is very
  convenient to use as a formal function argument, it also allows for
  conversions from every built-in type and every pointer type.  While
  using an enumeration type and its values is more verbose, it is also
  self documenting and is safer.  For example, what does the third
  argument mean in the following example?

{\small\begin{verbatim}
  apply( A, 2.0, true, x, y );\end{verbatim}}

When the \texttt{bool} argument is changed to an enum, the function call
becomes:

{\small\begin{verbatim}
  apply( A, 2.0, USE_TRANSPOSE, x, y );
\end{verbatim}}

and the meaning is much more clear.  Therefore, when self
documentation and compile-time safety are important, prefer to define
and use \texttt{enum}s over \texttt{bool}s as formal function
arguments (see \cite[Section 12.6]{CodeComplete2nd04}).

\item\GCGOverloadedVirtualFunctions: Overloaded virtual functions
  cause severe portability problems with many compilers and result in
  shadowing warnings that are elevated to errors in may systems
  \cite[Item 33]{EffectiveC++3rd05}.

\item\GCGAvoidSmartPtrFuncOverloads: Overloading functions on
  different smart pointer types, such as \texttt{RCP} or \texttt{Ptr},
  can create ambiguous function calls that will not happen when using
  raw C++ pointers or references \cite{TeuchosMemoryManagementGuide}.
  Therefore, keep the names of the functions different such as shown
  below.

{\small\begin{verbatim}
  void nonconstFoo(const RCP<A> &a);
  void foo(const RCP<const A> &a);
\end{verbatim}}


\item\GCGStandardCppHeaders: Only include the C++ \texttt{<cX>}
  versions of the standard C \texttt{<X.h>} headers.  For example,
  include \texttt{<cmath>}, \texttt{<cstdlib>}, and \texttt{<cassert>}
  instead of \texttt{<math.h>}, \texttt{<stdlib.h>}, and
  \texttt{<assert.h>}.  Avoid all uses of \texttt{using namespace std}
  directives and instead prefer explicit namespace qualification such
  as \texttt{std::sqrt} or using declarations such as \texttt{using
    std::sqrt} only within function definitions.  See \cite[Section
  16.1.2]{stroustrup97} for a complete list of the standard C++
  versions of the standard C headers.

  \textit{Justification}: See Appendix~\ref{sec:misc_amendments} for a
  clarification of Item 59 in \cite{C++CodingStandards05} dealing
  with the issue of \texttt{using} declarations and directives.


\item\GCGTemplateHeaders\label{gcg:template-instantation-files}:
Breaking up templated C++ code into the four files
\texttt{SomeClass[\_decl,\_def].[hpp,cpp]} (as described below)
allows for a portable and bullet-proof solution to handing templated
C++ code which allows for a) controlled explicit or implicit template
instantiation, b) minimization of first-time compilation, c)
minimization of recompilations, and d) handling of any and all types
of circular dependency in declarations and definitions (same as are
allowed with non-templated C++ code).

As an example, consider three classes \ttt{A}, \ttt{B}, and \ttt{C}
where \ttt{A} and \ttt{B} refer to each other and where \ttt{C} has no
chance of being involved in a circular reference involving \ttt{A} and
\ttt{B}.  The four files \texttt{A[\_decl,\_def].[hpp,cpp]} for class
\ttt{A} as well as the file \texttt{B\_decl.hpp} are shown below (the
other files for class \ttt{B} are similar):


{\small\begin{verbatim}
  ----------------------------------
  // A.hpp

  #include "A_decl.hpp"
  // If using explicit instantiation, include A_def.hpp.  Otherwise not.
  #include "A_def.hpp"
  // NOTE: This file gets configured by CMake to include A_def.hpp or not!


  ----------------------------------
  // A_decl.hpp

  #ifndef A_DECL_HPP
  #define A_DECL_HPP

  #include "B_decl.hpp"  // Only include decl in case of circular ref
  #include "C.hpp"       // No chance of cicular ref

  namespace Thyra {

  template<class Scalar>
  class A {
  pubic:
    void doSomething(const B<T> &b) const;
    ...
  private:
    RCP<C<T> > c_;
  };

  } // namespace Thyra

  #endif // A_DECL_HPP


  ----------------------------------
  // B_decl.hpp

  #ifndef B_DECL_HPP
  #define B_DECL_HPP

  namespace Thyra {

  template<class Scalar> class A; // Forward only due to circular ref!

  template<class Scalar>
  class B {
  pubic:
    void doSomething(const A<T> &a) const;
  };

  } // namespace Thyra

  #endif // B_DECL_HPP


  ----------------------------------
  // A_def.hpp

  #ifndef A_DEF_HPP
  #define A_DEF_HPP

  #include "B.hpp" // Must include for implicit instant to work!

  namespace Thyra {

  template<class Scalar>
  void A::doSomething(const B<T>& b)
  {
     b.doSomething(*this);
  }

  } // namespace Thyra

  #endif // A_DEF_HPP


  ----------------------------------
  // A.cpp

  #include "A_decl.hpp"  // Helps test header sufficiency

  #ifdef HAVE_THYRA_EXPLICIT_INSTANTIATION
  #include "A_def.hpp"
  #include "Teuchos_ExplicitInstantiationHelpers.hpp"
  namespace Thyra {TEUCHOS_CLASS_TEMPLATE_INSTANT_SCALAR_TYPES(A)}
  #endif // HAVE_THYRA_EXPLICIT_INSTANTIATION
\end{verbatim}}


General client code \underline{always} includes the \ttt{A.hpp} form
of the file without regard for whether implicit or explicit
instantiation is enabled (i.e.,\ whether or not
\ttt{HAVE\_THYRA\_EXPLICIT\_INSTANTIATION} is defined).

The 100\% bullet-proof rules for breaking up template code like this are:

\begin{itemize}

\item All header-like declarations that would go into an ordinary
  non-template \ttt{*.hpp} header file go into
  \ttt{SomeClass\_decl.hpp} including class declarations and inline
  function definitions.

\item All implementation code that would go into an ordinary
  non-template \ttt{*.cpp} source file go into
  \ttt{SomeClass\_def.hpp} including class member definitions and
  non-member function definitions.

\item Always include \ttt{SomeOtherClasss\_decl.hpp} in the
  \ttt{SomeClass\_decl.hpp} file if there is any chance that a
  circular dependency may exist between the two types
  \ttt{SomeOtherClasss} and \ttt{SomeClass}.  Otherwise, if there is
  no chance of a circular dependence then the header
  \ttt{SomeOtherClasss.hpp} should be included (instead of the
  possible \ttt{\_decl.hpp} form).  If the two classes are in
  different libraries then there is no chance of a circular type
  dependency (because well designed software does not allow this; see
  \cite{AgileSoftwareDevelopment}).

\item If \ttt{SomeClass\_decl.hpp} includes
  \ttt{SomeOtherClass\_decl.hpp}, then \ttt{SomeClass\_def.hpp} must
  include \ttt{SomeOtherClass.hpp}.  This is needed in order for
  implicit instantiation to work correctly.

\item The header file \ttt{SomeClass.hpp} is designed to be included
  by general clients and either includes only
  \ttt{SomeClass\_decl.hpp} or also includes \ttt{SomeClass\_def.hpp}
  depending on if implicit or explicit instantiation is being used.
  Actually, this file must be auto-generated during the CMake configure
  process to include or not include \ttt{SomeClass\_def.hpp} because
  CMake does not consider preprocessor defines and therefore the form
  of \ttt{SomeClass.hpp} cannot contain even a protected include of
  \ttt{SomeClass\_def.hpp} or CMake will recompile client code when
  \ttt{SomeClass\_def.hpp} changes.  See examples of how to
  auto-generate \ttt{SomeClass.hpp} by looking at the Thyra package.

\item All required instantiations must be provided in the file
  \ttt{SomeClass.cpp}.  For standard scalar types (e.g., \ttt{double},
  \ttt{float}, \ttt{std::complex<double>}, \ttt{std::complex<float>},
  etc.) the standard macro
  \ttt{TEUCHOS\_CLASS\_TEMPLATE\_INSTANT\_SCALAR\_TYPES(...)} is
  provided which is set at configure time to determine the
  desired/required explicit instantiations.  More general
  instantiations can also be performed by defining a macro in the file
  \ttt{SomeClass\_def.hpp} file and then instantiating this macro
  using the helper macro
  \ttt{TEUCHOS\_MACRO\_TEMPLATE\_INSTANT\_SCALAR\_TYPES(...)} (See
  examples from real Thyra source code).

\end{itemize}

If one follows the above guidelines, one will never have dependency
ordering problems with templated code.  The partitioning the template
code into the four files \texttt{SomeClass[\_decl,\_def].[hpp,cpp]}
gives template code all the desirable compilation properties of
non-template code.  That is, changes to the implementation of
\ttt{SomeClass} only require the recompilation of the source file
\texttt{SomeClass.cpp} and not any other source files.  Also, the
amount of code that a C++ compiler has to see to compile any single
\ttt{*.cpp} file is much less when explicit instantiation is enabled
and this can massively speed up first-time compilation.  Overall,
explicit instantiation can massively speed up first-time compilation
and later recompilations as code is modified.


\end{itemize}


%
\subsection{Specification of data members and passing and returning
  objects from functions}
\label{sec:func_arguments}
%

The guidelines for specifying local variables and data members,
passing objects to and from functions, and returning objects from
functions given in \cite{TeuchosMemoryManagementGuide} are summarized
in Tables
\ref{fig:data_member_value_type}--\ref{fig:func_return_reference_type}.
In general, it is assumed that arguments passed through the smart
pointer types \ttt{Ptr}, \ttt{RCP}, \ttt{ArrayView}, and
\ttt{ArrayRCP} are non-null by default.  If the argument is allowed to
be null, then that must be documented in the Doxygen
$\backslash$\ttt{param} field for that argument.


\begin{table}
%
%\fbox{
\begin{center}
%
%\begin{minipage}{{}\textwidth}
%
\begin{tabular}{|l|l|}
%
\multicolumn{2}{c}{\textbf{Class Data Members for Value-Type Objects}} \\
%
\hline
Data member purpose
& Data member delcaration \\
\hline
\hline
non-shared, single, const object
& {}\texttt{const S s\_;} \\
\hline
non-shared, single, non-const object
& {}\texttt{S s\_;} \\
\hline
non-shared array of non-const objects
& {}\texttt{Array<S> as\_;} \\
\hline
shared array of non-const objects
& {}\texttt{RCP<Array<S> > as\_;} \\
\hline
non-shared statically sized array of non-const objects
& {}\texttt{Tuple<S,N> as\_;} \\
\hline
shared statically sized array of non-const objects
& {}\texttt{RCP<Tuple<S,N> > as\_;} \\
\hline
shared fixed-sized array of const objects
& {}\texttt{ArrayRCP<const S> as\_;} \\
\hline
shared fixed-sized array of non-const objects
& {}\texttt{ArrayRCP<S> as\_;} \\
\hline
\end{tabular}

%
%\end{minipage}
%
\end{center}
%} % end fbox
\caption{\label{fig:data_member_value_type}
Idioms for class data member declarations for value-type objects.}
%
\end{table}


\begin{table}
%
%\fbox{
\begin{center}
%
%\begin{minipage}{{}\textwidth}
%
\begin{tabular}{|l|l|}
%
\multicolumn{2}{c}{\textbf{Class Data Members for Reference-Type Objects}} \\
%
\hline
Data member purpose
& Data member delcaration \\
\hline
\hline
non-shared or shared, single, const object
& {}\texttt{RCP<const A> a\_;} \\
\hline
non-shared or shared, single, non-const object
& {}\texttt{RCP<A> a\_;} \\
\hline
non-shared array of shared const objects
& {}\texttt{Array<RCP<const A> > aa\_;} \\
\hline
non-shared array of shared non-const objects
& {}\texttt{Array<RCP<A> > aa\_;} \\
\hline
shared fixed-sized array of shared const objects
& {}\texttt{ArrayRCP<RCP<const A> > aa\_;} \\
\hline
``...'' (const ptr)
& {}\texttt{ArrayRCP<const RCP<const A> > aa\_;} \\
\hline
shared fixed-sized array of shared non-const objects
& {}\texttt{ArrayRCP<RCP<const A> > aa\_;} \\
\hline
``...'' (const ptr)
& {}\texttt{ArrayRCP<const RCP<const A> > aa\_;} \\
\hline
\end{tabular}

%
%\end{minipage}
%
\end{center}
%} % end fbox
\caption{\label{fig:data_member_reference_type}
Idioms for class data member declarations for reference-types
objects.}
%
\end{table}


\begin{table}[p]
%
%\fbox{
\begin{center}
%
%\begin{minipage}{{}\textwidth}
%
\begin{tabular}{|l|l|}
%
\multicolumn{2}{c}{\textbf{Passing IN Non-Persisting Associations to Value Objects as Func Args}} \\
%
\hline
Argument Purpose
& Formal Argument Declaration \\
\hline
\hline
single, non-changeable object (required)
& {}\texttt{S s} $\;\;\;\;$\textbf{or}$\;\;\;\;$ {}\texttt{const S s} $\;\;\;\;$\textbf{or}$\;\;\;\;$ {}\texttt{const S \&s} \\
%& \begin{minipage}{12ex}
%    {\tiny .}\\
%    {}\texttt{S s} \\
%    or \\
%    {}\texttt{const S s} \\
%    or \\
%    {}\texttt{const S \&s}
%    {\tiny .}
%  \end{minipage} \\
\hline
single, non-changeable object (optional)
& {}\texttt{const Ptr<const S> \&s} \\
\hline
single, changeable object (required)
& {}\texttt{const Ptr<S> \&s}  $\;\;\;\;$\textbf{or}$\;\;\;\;$ {}\texttt{S \&s} \\
%& \begin{minipage}{20ex}
%    {\tiny .}\\
%    {}\texttt{const Ptr<S> \&s} \\
%    or \\
%    {}\texttt{S \&s} \\
%    {\tiny .}
%  \end{minipage} \\
\hline
single, changeable object (optional)
& \texttt{const Ptr<S> \&s} \\
\hline
array of non-changeable objects
& {}\texttt{const ArrayView<const S> \&as} \\
\hline
array of changeable objects
& {}\texttt{const ArrayView<S> \&as} \\
\hline
%
\end{tabular} \\[3ex]
%
\begin{tabular}{|l|l|}
%
\multicolumn{2}{c}{\textbf{Passing IN Persisting Associations to Value Objects as Func Args}} \\
%
\hline
Argument Purpose
& Formal Argument Declaration \\
\hline
\hline
array of non-changeable objects
& {}\texttt{const ArrayRCP<const S> \&as} \\
\hline
array of changeable objects
& {}\texttt{const ArrayRCP<S> \&ss} \\
\hline
\end{tabular} \\[3ex]
%
\begin{tabular}{|l|l|}
%
\multicolumn{2}{c}{\textbf{Passing OUT Persisting Associations for Value Objects as Func Args}} \\
%
\hline
Argument Purpose
& Formal Argument Declaration \\
\hline
\hline
array of non-changeable objects
& {}\texttt{const Ptr<ArrayRCP<const S> > \&as} \\
\hline
array of changeable objects
& {}\texttt{const Ptr<ArrayRCP<S> > \&as} \\
\hline
\end{tabular} \\[3ex]
%
\begin{tabular}{|l|l|}
%
\multicolumn{2}{c}{\textbf{Passing OUT Semi-Persisting Associations for Value Objects as Func Args}} \\
%
\hline
Argument Purpose
& Formal Argument Declaration \\
\hline
\hline
array of non-changeable objects
& {}\texttt{const Ptr<ArrayView<const S> > \&as} \\
\hline
array of changeable objects
& {}\texttt{const Ptr<ArrayView<S> > \&as} \\
\hline
\end{tabular}

%
%
%\end{minipage}
%
\end{center}
%} % end fbox
\caption{\label{fig:func_args_value_type}
Idioms for passing value-type objects to C++ functions.}
%
\end{table}


\begin{table}[p]
%
%\fbox{
\begin{center}
%
\begin{tabular}{|l|l|}
%
\multicolumn{2}{c}{\textbf{Passing IN Non-Persisting Associations to Reference (or Value) Objects as Func Args}} \\
%
\hline
Argument Purpose
& Formal Argument Declaration \\
\hline
\hline
single, non-changeable object (required)
& {}\texttt{const A \&a} \\
\hline
single, non-changeable object (optional)
& {}\texttt{const Ptr<const A> \&a} \\
\hline
single, changeable object (required)
& {}\texttt{const Ptr<A> \&a}  $\;\;\;\;$\textbf{or}$\;\;\;\;$ {}\texttt{A \&a} \\
%& \begin{minipage}{20ex}
%    {\tiny .}\\
%    {}\texttt{const Ptr<A> \&a} \\
%    or \\
%    {}\texttt{A \&a} \\
%    {\tiny .}
%  \end{minipage} \\
\hline
single, changeable object (optional)
& \texttt{const Ptr<A> \&a} \\
\hline
array of non-changeable objects
& {}\texttt{const ArrayView<const Ptr<const A> > \&aa} \\
\hline
array of changeable objects
& {}\texttt{const ArrayView<const Ptr<A> > \&aa} \\
\hline
\end{tabular} \\[3ex]
%
\begin{tabular}{|l|l|}
%
\multicolumn{2}{c}{\textbf{Passing IN Persisting Associations to Reference (or Value) Objects as Func Args}} \\
%
\hline
Argument Purpose & Formal Argument Declaration \\
\hline
\hline
single, non-changeable object
& {}\texttt{const RCP<const A> \&a} \\
\hline
single, changeable object
& {}\texttt{const RCP<A> \&a} \\
\hline
array of non-changeable objects
& {}\texttt{const ArrayView<const RCP<const A> > \&aa} \\
\hline
array of changeable objects
& {}\texttt{const ArrayView<const RCP<A> > \&aa} \\
\hline
\end{tabular} \\[3ex]
%
\begin{tabular}{|l|l|}
%
\multicolumn{2}{c}{\textbf{Passing OUT Persisting Associations for Reference (or Value) Objects as Func Args}} \\
%
\hline
Argument Purpose
& Formal Argument Declaration \\
\hline
\hline
single, non-changeable object
& {}\texttt{const Ptr<RCP<const A> > \&a} \\
\hline
single, changeable object
& {}\texttt{const Ptr<RCP<A> > \&a} \\
\hline
array of non-changeable objects
& {}\texttt{const ArrayView<RCP<const A> > \&aa} \\
\hline
array of changeable objects
& {}\texttt{const ArrayView<RCP<A> > \&aa} \\
\hline
\end{tabular} \\[3ex]
%
\begin{tabular}{|l|l|}
%
\multicolumn{2}{c}{\textbf{Passing OUT Semi-Persisting Associations for Reference (or Value) Objects as Func Args}} \\
%
\hline
Argument Purpose
& Formal Argument Declaration \\
\hline
\hline
single, non-changeable object
& {}\texttt{const Ptr<Ptr<const A> > \&a} \\
\hline
single, changeable object
& {}\texttt{const Ptr<Ptr<A> > \&a} \\
\hline
array of non-changeable objects
& {}\texttt{const ArrayView<Ptr<const A> > \&aa} \\
\hline
array of changeable objects
& {}\texttt{const ArrayView<Ptr<A> > \&aa} \\
\hline
\end{tabular}

\end{center}
\caption{\label{fig:func_args_ref_type}
Idioms for passing reference-type objects to C++ functions.}
%} % end fbox
\end{table}




\begin{table}
%
%\fbox{
\begin{center}
%
%\begin{minipage}{{}\textwidth}
%
\begin{tabular}{|l|l|}
%
\multicolumn{2}{c}{\textbf{Returning Non-Persisting Associations to Value Objects}} \\
%
\hline
Purpose
& Return Type Declaration \\
\hline
\hline
Single copied object (return by value)
& {}\texttt{S} \\
\hline
Single non-changeable object (required)
& {}\texttt{const S\&} \\
\hline
Single non-changeable object (optional)
& {}\texttt{Ptr<const S>} \\
\hline
Single changeable object (required)
& {}\texttt{S\&} \\
\hline
Single changeable object (optional)
& {}\texttt{Ptr<S>} \\
\hline
Array of non-changeable objects
& {}\texttt{ArrayView<const S>} \\
\hline
Array of changeable objects
& {}\texttt{ArrayView<S>} \\
\hline
%
\end{tabular} \\[3ex]
%
\begin{tabular}{|l|l|}
%
\multicolumn{2}{c}{\textbf{Returning Persisting Associations to Value Objects}} \\
%
\hline
Purpose
& Return Type Declaration \\
\hline
\hline
Array of non-changeable objects
& {}\texttt{ArrayRCP<const S>} \\
\hline
Array of changeable objects
& {}\texttt{ArrayRCP<S>} \\
\hline
\end{tabular} \\[3ex]
%
\begin{tabular}{|l|l|}
%
\multicolumn{2}{c}{\textbf{Returning Semi-Persisting Associations to Value Objects}} \\
%
\hline
Purpose
& Return Type Declaration \\
\hline
\hline
Array of non-changeable objects
& {}\texttt{ArrayView<const S>} \\
\hline
Array of changeable objects
& {}\texttt{ArrayView<S>} \\
\hline
\end{tabular}

%
%
%\end{minipage}
%
\end{center}
%} % end fbox
\caption{\label{fig:func_return_value_type}
Idioms for returning value-type objects from C++ functions.}
%
\end{table}


\begin{table}
%
%\fbox{
\begin{center}
%
%\begin{minipage}{{}\textwidth}
%
\begin{tabular}{|l|l|}
%
\multicolumn{2}{c}{\textbf{Returning Non-Persisting Associations to Reference (or Value) Objects }} \\
%
\hline
Purpose
& Return Type Declaration \\
\hline
\hline
Single cloned object
& {}\texttt{RCP<A>} \\
\hline
Single non-changeable object (required)
& {}\texttt{const A\&} \\
\hline
Single non-changeable object (optional)
& {}\texttt{Ptr<const A>} \\
\hline
Single changeable object (required)
& {}\texttt{A\&} \\
\hline
Single changeable object (optional)
& {}\texttt{Ptr<A>} \\
\hline
Array of non-changeable objects
& {}\texttt{ArrayView<const Ptr<const A> >} \\
\hline
Array of changeable objects
& {}\texttt{ArrayView<const Ptr<A> >} \\
\hline
%
\end{tabular} \\[3ex]
%
\begin{tabular}{|l|l|}
%
\multicolumn{2}{c}{\textbf{Returning Persisting Associations to Reference (or Value) Objects}} \\
%
\hline
Purpose
& Return Type Declaration \\
\hline
\hline
Single non-changeable object
& {}\texttt{RCP<const A>} \\
\hline
Single changeable object
& {}\texttt{RCP<A>} \\
\hline
Array of non-changeable objects
& {}\texttt{ArrayView<const RCP<const A> >} \\
\hline
Array of changeable objects
& {}\texttt{ArrayView<const RCP<A> >} \\
\hline
\end{tabular} \\[3ex]
%
\begin{tabular}{|l|l|}
%
\multicolumn{2}{c}{\textbf{Returning Semi-Persisting Associations to Reference (or Value) Objects}} \\
%
\hline
Purpose
& Return Type Declaration \\
\hline
\hline
Single non-changeable object
& {}\texttt{Ptr<const A>} \\
\hline
Single changeable object
& {}\texttt{Ptr<A>} \\
\hline
Array of non-changeable objects
& {}\texttt{ArrayView<const Ptr<const A> >} \\
\hline
Array of changeable objects
& {}\texttt{ArrayView<const Ptr<A> >} \\
\hline
\end{tabular}

%
%
%\end{minipage}
%
\end{center}
%} % end fbox
\caption{\label{fig:func_return_reference_type}
Idioms for returning reference-type objects from C++ functions.}
%
\end{table}


%
\section{Formatting of source code}
\label{tcdg:formatting:sec}
%

We mean by ``formatting'' the use of white space or other optional
syntax elements (such as optional curly braces in C++), or changes to
the ordering of code elements, that change the appearance of code to a
human programmer, without changing the meaning of the program to the
compiler.\footnote{While changing the name of a class, function or
  variable may change the meaning of a program, if name changes are
  done in such a way as to avoid name collisions, then naming
  conventions do not affect the meaning of the program.  They are
  therefore related to other formatting issues such as the treatment
  of white space.}  C++, unlike some languages such as Fortran 77 or
Python, allows considerable formatting freedom.  With freedom comes
responsibility, in this case to make code easier for both its author
and other programmers to read, understand, extend, and maintain.

Our main goal in this section is to try to provide reasonable
recommendations for those formatting issues that, while largely a
matter of style and personal preference, also affect overall code
readability.  Consistent, well thought out code formatting helps to
promote pair programming and joint ownership of code
{}\cite{ExtremeProgrammingExplained99}.  The formatting and
indentation guidelines presented here are largely consistent with the
recommendations in {}\cite[Chapter 31]{CodeComplete2nd04} and try to
reduce the amount of ``right drift'' that can occur with some common
formatting and indentation styles.

Source code should always be formatted consistently within a single
file or a set of tightly coupled files \cite[Item
0]{C++CodingStandards05}.  Ideally, source code should be formatted
with sufficient consistency across a code project so as not to cause
undue difficulty in shared maintenance and in performing code reviews
\cite{CodeComplete2nd04}.  Some consistency in formatting helps
facilitate multiple ownership and shared development of a collection
of software, especially when using techniques such as Extreme
Programming (XP) \cite{ExtremeProgrammingExplained99}.  See
Appendix~\ref{sec:arguments-for-consistent-style} for an outline of
the arguments for adopting a consistent code formatting style.

Many source code editors can handle most of the work of
formatting.\footnote{This includes popular text editors such as Emacs
  and Vim (a derivative of the \texttt{vi} text editor), as well as
  integrated development environments such as Microsoft Visual Studio
  and Eclipse.  Emacs and Vim support multiple, custom, file-specific
  formatting styles for C++, for example.  Tools like Artistic Style
  \cite{ArtisticStyle} can format source files from the command line.}
This allows individual developers to work with any style they would
like for files that they create, and also makes it easier for
developers to edit files created by other developers without changing
their styles.  Appendix~\ref{sec:reformatting-guidelines} gives some
guidelines for how individuals should conduct themselves when multiple
code formatting styles are in use within a given project.

The indentation guidelines outlined below are automatically supported
by the Emacs text editor when using the custom style ``thyra'' defined
in the Emacs package file \texttt{cc-thyra-styles.el}.\footnote{See
  \texttt{Trilinos/packages/thyra/emacs/README} for a description of
  the ``thyra'' Emacs style} Other custom styles may be added to this
file and used as well.  Any of these styles can be listed in each
source file, so that anyone using Emacs can automatically use a
particular indentation style without having to fight the editor to
manually reformat code to abide by a foreign style.


%
\subsection{General formatting source code principles (FSCP)}
%

Some general principles of good formatting, based on the discussion in
\cite[Section 31.1]{CodeComplete2nd04}, are:

\begin{itemize}

\item\FSCPShowLogicalStructure: It is somewhat subjective what
  formatting styles ``show the logical structure'' of code but
  McConnell makes some good arguments for some styles over others.
  However, it is up the group of programmers to decide as a group what
  style items ``show the logical structure''.

\item\FSCPImproveReadability: There are specific studies cited in
  \cite[Chapter 31]{CodeComplete2nd04} that provide good evidence to
  prefer some styles over others.

\item\FSCPRetainFormatting: Changing one line of code should not
  require changes to other lines of code to maintain the formatting
  style.

\item\FSCPFollowCommonIdioms: When there is no good technical argument
  for one formatting style choice over another, then the style choice
  that is the most common should be used.\footnote{The measure of the
    commonality of a particular style choice can be determined
    according to a local software development community or the larger
    developer community.}  This is not advocated explicitly in
  \cite[Chapter 31]{CodeComplete2nd04} but it is a good idea in
  general to follow popular idioms when there are several equally good
  choices and therefore the decision is arbitrary.  However, not
  selecting a single style choice can create artificial complexity in
  the code, due to irregularity in formatting.

\end{itemize}


%
\subsection{Specific guidelines for formatting source code (FSC)}
%

Below, specific recommendations are spelled out that try to conform to common
practices but also try to avoid excessive ``right drift'':


\begin{itemize}

\item\FSCConsistentInFile: Consistent formatting includes the
  placement of braces, the number spaces to indent etc.
  {}\textit{Justification}: This is recommended in \cite[Item
  0]{C++CodingStandards05}.

\item\FSCEightyCharWide: Keeping most of the source code within the
  first 80 character columns helps to make the code more readable and
  helps to facilitate side-by-side two-column editing and comparisons
  of source code.  Most of the style and indentation guidelines
  described below help to avoid code that extends beyond the 80th
  column too rapidly. \textit{Justification}: ``Studies show that up
  to ten-word text widths are optimal for eye tracking'' \cite[Item
  0]{C++CodingStandards05}.  Also, certain situations (such as
  emergency debugging) require editing code using a terminal rather
  than one's favorite development environment, and that terminal may
  have width limited to 80 columns.

\item\FSCIndentSpaces: The amount of spaces to use per indentation
  level is up to the individual developer, but an indentation of only
  \textit{two spaces} is recommended (and is set in the `Emacs
  `thyra'' indentation style).  A study showed that an indentation
  offset of two to four spaces was optimal for code reading
  comprehension \cite[Section 31.2]{CodeComplete2nd04}.  Whatever
  indentation amount is used, it should be consistent in at least each
  source and header file \cite[Item 0]{C++CodingStandards05} (which
  can be enforced using a custom Emacs indentation style).  Emacs by
  default will put in a tab when the tab width is equal to the number
  of indentation spaces.  Emacs may be told always to use spaces
  instead of tabs by setting:

{\small\begin{verbatim}
  (setq indent-tabs-mode nil)
\end{verbatim}}

  in the indentation style (as is done in the ``thyra'' style).
  However, it is easy to support different preferences for the amount
  of spaces to indent by using a user-defined indentation style for
  Emacs.  Similar customization is possible in other text editors such
  as Vim.

  \textit{Justification}: ``Some teams legitimately choose to ban tabs
  \dots when misused, [tabs] turn indenting into out-denting and
  non-denting.''  \cite[Item 0]{C++CodingStandards05}.


\item\FSCTwoVertialSpaces.

  \textit{Justification}: Using two blank vertical spaces is
  preferable to long lines with some filler like '-' or '=' or other
  separators.  They clearly separate the entities and are easier to
  maintain (see \cite[Section 31.8]{CodeComplete2nd04}).


{}\item\FSCNoNamespaceIndent: Indenting for namespace enclosures results in
unnecessary, and in some cases excessive, indentation.  Instead, for example,
use:

{\small\begin{verbatim}
  namespace MyNameSpace {


  namespace MyInnerNamespace {


  class SomeClass {..};


  void someFunc(...) {...}


  } // namespace MyInnerNamespace


  } // namespace MyNameSpace
\end{verbatim}}

Above, note that two vertical blank lines are used between each of the
major entities (see above item).

\textit{Justification}: While indentation within namespaces is helpful
in small example code fragments, it provides little help in showing
namespace structure in more realistic code.  The use of commented end
braces is generally sufficient to show namespace structure and will
not result in excessively indented code.  In addition, typically, each
file will only contain code from one (or more nested) namespace and
therefore indenting for namespaces provides no useful information.
Not indenting for namespace enclosures is also consistent with the
``ansi'', the ``kr'', and the ``linux'' styles as defined by Artistic
Style \cite{ArtisticStyle}.


\item\FSCClassDeclarationLayout.

\begin{figure}
%
{\small\begin{verbatim}
  class SomeClass {

    // Friends

    friend void foo();

    friend class SomeOtherClass;

  public:

    // Public types

    typedef int integral_type;

    // Public member functions

    void func1();

  protected:

    // Protected member functions

    void func2();

  private:

    // Private types

    typedef std::vector<int> int_array_t;

    // Private data members

    int data1_;
    int_array_t array1_;

    // Private member functions

    void func3();

  };
\end{verbatim}}
%
\caption{\label{fig:class_decl_layout}
Example of suggested layout of a C++ class declaration complete with ordering
of sections, indentation, and line spacing.}
%
\end{figure}

\textit{Justification}: This ordering of sections and data members is quite
common \cite[Section 31.8]{CodeComplete2nd04}.  Above, we show private
member functions after private data members, since private data members are
more prominent and more common in the class implementations than are private
member functions.  Also, private types (where typedefs are most common) must
be listed before they are used in the declaration of the private data members.
Note that public types used in public member functions must be listed above
(or at least forward declared) before the public member functions that use
them.


\item\FSCFunctionProtypes: Below, guidelines for formatting short
  function prototypes and long prototypes are given.  These guidelines
  seek to produce function prototypes that are tight (i.e., minimize
  white space), are robust to modifications, and help make code fit
  within the 80 columns per line limit.  This indentation style should
  also be applied to function definitions and function calls.

  \begin{itemize}

  \item\FSCShortFunctionPrototypesOneLine: For example,

  {\small\begin{verbatim}
  ReturnType someFunction( int arg = 0 );
  \end{verbatim}}

  or 

  {\small\begin{verbatim}
  ReturnType someFunction(int arg=0);
  \end{verbatim}}

  or some other style for white-space within '(...)' but the opening '(' should
  come directly after the function name in all cases.

  \item\FSCLongFunctionPrototypesIndent: Function prototypes that cannot
  approximately fit on a single line in the first 80 character columns should
  have the function arguments listed starting on the second line with one unit
  of indentation (e.g., two spaces) from the function return type and function
  name line.  For example, several different valid formats for a longer
  function prototype are:

  {\small\begin{verbatim}
  ReturnType someFunction(
    int arg1,
    bool arg2,
    const ArrayView<double> &arg3,
    const std::string &arg4  = ""
    );
  \end{verbatim}}
  
  or
  
  {\small\begin{verbatim}
  ReturnType someFunction(
    int arg1, bool arg2, const ArrayView<double> &arg3,
    const std::string &arg4  = ""
    );
  \end{verbatim}}
  
  or
  
  {\small\begin{verbatim}
  ReturnType someFunction(
    int arg1, bool arg2, const ArrayView<double> &arg3,
    const std::string &arg4 = "" );
  \end{verbatim}}
  
  or
  
  {\small\begin{verbatim}
  ReturnType someFunction( int arg1, bool arg2,
    const ArrayView<double> &arg3, const std::string &arg4 = "" );
  \end{verbatim}}

  As shown above, the function arguments can be listed separately on different
  lines, or in groups on sets of lines.  The arguments can begin on the same
  line as the type + function name line or can start on the next line.  The
  ending parenthesis ')' can appear on the same line as the last line of
  arguments or can appear alone on the last line.  Other formats are possible
  also and can be appropriate in different situations.

  \textit{Justification}: See \cite[Section 31.1]{CodeComplete2nd04}.
  
  \item\FSCFunctionReturn: A function prototype's return type should appear
  on the same line as the function name unless it is excessively long and
  would result in the return type + function name line to extend past the 80th
  character column.  When the return type + function name is too long, then it
  can be listed on separate lines with no indent, for example, as:
  
  {\small\begin{verbatim}
  Teuchos::RCP<ReturnType>
  someVeryLongAndVeryImportantFunction(
    int arg1, bool arg2, const ArrayView<double> &arg3,
    const std::string &arg4  = ""
    );
  \end{verbatim}}

  However, listing the function return type on a separate line even in cases
  of shorter prototypes is also acceptable.

  \end{itemize}


\item\FSCOrderFunctionDefinitions: For example, one should order the
  definitions of a set of functions the same as the ordering of the
  declarations.  Maintaining the ordering of definitions and
  declarations makes the code more readable and more maintainable.
  For example, if the function definitions are ordered the same as the
  declarations, it can be easy to spot that a function definition is
  missing.  This could help diagnose an otherwise puzzling link error
  caused by the missing definition.

\item\FSCBracePlacement: Two basic styles of brace placement and
  indentation in control structures are recommended here.  The first
  general style is a modification of the K\&R style
  \cite{ArtisticStyle} where the brace comes immediately after the
  control statement on the same line shown as:

{\small\begin{verbatim}
  // Modified K&R Style (recommended)
  if (someCondition) {
    ...
  }
  else {
    ...
  }
\end{verbatim}}

Note that the pure K\&R style (for example, as defined by Artistic Style
\cite{ArtisticStyle}) shown as:

{\small\begin{verbatim}
  // Pure K&R Style (*NOT* recommended)
  if (someCondition) {
    ...
  } else {
    ...
  }
\end{verbatim}}

is not recommended.  Even through pure K\&R style meets McConnell's strict
pictorial definition of ``emulation of pure block style'' (i.e.\ the
equivalent to pure block format such as in Visual Basic) which he says is
good, he actually recommends the above modified K\&R style (as do we since we
feel it is more readable).

The second general style that is recommended is the ``ANSI'' style
\cite{ArtisticStyle} where the opening brace begins flush on the next
line from the control statement shown as:

{\small\begin{verbatim}
  // ANSI Style (recommended)
  if (someCondition)
  {
    ...
  }
  else
  {
    ...
  }
\end{verbatim}}

Both the modified K\&R and the ANSI styles help to avoid right drift.  The
modified K\&R style creates tighter code vertically and seems to be preferred
by many communities and authors but variations of the ANSI style are also very
common.  Note that the ANSI style seems to have a distinct advantage in cases
where the control statement is continued over multiple lines.  For example,
the modified K\&R style with line continuations looks like:

{\small\begin{verbatim}
  // Modified K&R Style with line continuations (*NOT* recommended)
  if ( someLongCondition &&
    anotherVeryLongCondition &&
    theLongestConditionThatWillFitOnOneLine ) {
    // Statements
    ...
  }
\end{verbatim}}

{}\noindent{}and it is hard to argue that this shows the logical structure of
code.  One could argue that the ANSI style which looks like:

{\small\begin{verbatim}
  // ANSI Style with line continuations (recommended)
  if ( someLongCondition &&
    anotherVeryLongCondition &&
    theLongestConditionThatWillFitOnOneLine )
  {
    // Statements
    ...
  }
\end{verbatim}}

better shows the logical structure of the code in clearly separating the
control structure logic from the inner block of code.

Note that while the modified K\&R style meets McConnell's blessing of
``showing the logical structure of code'' where he refers to it as
``emulating pure block'' format that he cites the ANSI styles as
violating this principle \cite[Section 31.1]{CodeComplete2nd04}.
However, it is somewhat subjective what styles ``show the logical
structure'' and McConnell himself seems to contradict himself at times
(see the formatting of if/else statements below).

When choosing between one of these styles, try to be consistent at least
within a single file.  However, for control statements that extend over a
single line, prefer the ``ANSI'' style.

Below, the application of the modified K\&R style and the ANSI styles are
shown in the context of several different types of C++ loop and control
structures.

  \begin{itemize}

  {}\item\textit{Formatting if/else if/else
  statements}: When applied to if statements, the two recommended styles are:

  {\small\begin{verbatim}
  // Modified K&R Style (recommended)
  if (someCondition) {
    ...
  }
  else if (someOtherCondition) {
    ...
  }
  else {
    ...
  }
  \end{verbatim}}
  
  {}\noindent{}and:
  
  {\small\begin{verbatim}
  // ANSI Style (recommended)
  if (someCondition)
  {
    ...
  }
  else if (someOtherCondition)
  {
    ...
  }
  else
  {
    ...
  }
  \end{verbatim}}

  {}\item\textit{Formatting switch/case statements}:
  The two recommended formats for switch/case statements are:

  {\small\begin{verbatim}
  // Modified K&R Style (recommended)
  switch (someEnumValue) {
    case ENUM_VALUE1:
      ...
      break;
    case ENUM_VALUE2:
      ...
      break;
    default:
      TEUCHOS_TEST_FOR_EXCEPT(true, "Should never get there!");
  }
  \end{verbatim}}

  {}\noindent{}and

  {\small\begin{verbatim}
  // ANSI Style (recommended)
  switch (someEnumValue)
  {
    case ENUM_VALUE1:
      ...
      break;
    case ENUM_VALUE2:
      ...
      break;
    default:
      TEUCHOS_TEST_FOR_EXCEPT(true, "Should never get there!");
  }
  \end{verbatim}}

  As shown above, every switch structure should have a {}\texttt{default} case
  that throws an exception (see ``use the default clause to detect errors''
  in {}\cite[Section 15.1]{CodeComplete2nd04}).

  Also, if needed, the case blocks can be wrapped in braces as:

  {\small\begin{verbatim}
  // Modified K&R Style (recommended)
  switch (someEnumValue) {
    case ENUM_VALUE1: {
      ...
      break;
    }
    case ENUM_VALUE2: {
      ...
      break;
    }    
    default: {
      TEUCHOS_TEST_FOR_EXCEPT(true, "Should never get there!");
    }
  }
  \end{verbatim}}

  {}\noindent{}and

  {\small\begin{verbatim}
  // ANSI Style (recommended)
  switch (someEnumValue)
  {
    case ENUM_VALUE1:
    {
      ...
      break;
    }
    case ENUM_VALUE2:
    {
      ...
      break;
    }
    default:
    {
      TEUCHOS_TEST_FOR_EXCEPT(true, "Should never get there!");
    }
  }
  \end{verbatim}}

  {}\item\textit{Formatting for and while loops}: The two recommended styles
  for formatting for loops are:

  {\small\begin{verbatim}
  // Modified K&R Style (recommended)
  for ( int i = 0; i < size; ++i ) {
    ...
  }
  \end{verbatim}}

  {}\noindent{}and:

  {\small\begin{verbatim}
  // ANSI Style (recommended)
  for ( int i = 0; i < size; ++i )
  {
    ...
  }
  \end{verbatim}}

  Note that line continuations are often needed for a for loops control
  structure, especially if long type names or variable names are used.  In
  these cases, the ANSI style is more highly recommended as:

  {\small\begin{verbatim}
  // ANSI Style (recommended)
  for (
    std::vector<SomeVeryLongClassName>::const_iterator itr = longVarName.begin();
    itr != someLongVariableName.end();
    ++itr )
  {
    ...
  }
  \end{verbatim}}

  Similarly, while loops should be formatted as:

  {\small\begin{verbatim}
  // Modified K&R Style (recommended)
  while ( someCondition ) {
    ...
  }
  \end{verbatim}}

  {}\noindent{}or:

  {\small\begin{verbatim}
  // ANSI Style (recommended)
  while ( someCondition )
  {
    ...
  }
  \end{verbatim}}

  \end{itemize}

\end{itemize}


%
\section{Doxygen documentation guidelines}
\label{sec:DocumentionGuidelines}
%

In this section, a set of reasonable guidelines are stated for writing Doxygen
(and plain old) documentation for classes, functions, etc.\ that makes the
specification clear but is not too verbose or hard to maintain.  While other
types of higher-level documentation are also needed such as design documents
and tutorials, guidelines for these other types of higher-level documentation
are not covered here.


%
\subsection{General principles for function and class level documentation (DOXP)}
%


\begin{itemize}


\item\DOXPVariableLevelOfDetail: Important interfaces or widely
  disseminated concrete classes or functions require an appropriate
  level of precise documentation.  Concrete implementations that are
  less widely disseminated can provide less (or none in some cases)
  Doxygen documentation if the implementation code itself is
  sufficiently easy to understand.  However, major parts of an
  implementation should include at least some plain old (i.e.\
  non-Doxygen) documentation to describe what it is doing.

\item\DOXPFullyDocumentInterfaces: In the case of important abstract
  interfaces, the full specification of behavior (e.g., invariants,
  preconditions, and postconditions) for the compliant objects must be
  stated \cite[Item 69]{C++CodingStandards05}.  In some cases, this
  must be done completely within the Doxygen documentation for the
  interface.  In other cases, standard unit testing code can be used
  to help specify the behavior of the interface.  In fact, compiled
  and verified unit testing code may be superior to standard Doxygen
  documentation since it cannot be ignored or become invalid.  On the
  other hand, it may be difficult for readers to wade through unit
  testing code to find the specification of behavior and therefore
  both Doxygen documentation and unit testing code should be used to
  provide the fullest benefit.  Also, Doxygen documentation can
  automatically include bits and pieces of compiled and tested code
  using the \texttt{$\backslash$dontinclude} and related Doxygen
  commands.

\item\DOXPFullyDocumentUserIntefaces: This item is an amendment to the
  above item as a special case for ``user'' interfaces.  A ``user''
  could be one who simply writes client code to the interface, one who
  provides implementations of the interface, or both.  Users should
  not be expected to study unit testing code to figure out the
  preconditions and/or postconditions for a function call.

\item\DOXPWrongDocumentationBad: Documentation must be maintained as
  code is changed.  Therefore, excessive or unnecessary documentation
  that is not rigorously maintained degrades the overall quality of
  code.  However, documentation with small errors is generally better
  than no documentation at all.

\item\DOXPDoNotRepeat: Strive for a single source for documentation
  for an entity.  Repeated documentation is hard to maintain.

\item\DOXPDocumentationMaintainItself: Any significant fragments of
  code that are shown in the Doxygen-generated HTML documentation
  should come from compiled and tested code.  This can be accomplished
  by using the \texttt{$\backslash$dontinclude} or related Doxygen
  command to read in code fragments automatically.  In this way, the
  compiler and our test suite can be used to help verify the code
  fragments in our Doxygen documentation.

\end{itemize}


%
\subsection{Specific Doxygen documentation principles (DOX)}
%

Now that some of the general goals for our Doxygen documentation have
been presented, more detailed guidelines are given below.


\begin{figure}
%
{\small\begin{verbatim}
/** \brief Apply the linear operator to a multi-vector : <tt>Y =
 * alpha*op(M)*X + beta*Y</tt>.
 *
 * \param M_trans [in] Determines whether the operator or its adjoint 
 * is applied for <tt>op(M)</tt>.
 *
 * \param X [in] The right hand side multi-vector.
 *
 * \param Y [in/out] The target multi-vector being transformed.  When
 * <tt>beta==0.0</tt>, this multi-vector can have uninitialized elements.
 *
 * \param alpha [in] Scalar multiplying <tt>M</tt>, where <tt>M==*this</tt>.
 * The default value of <tt>alpha</tt> is </tt>1.0</tt>.
 *
 * \param beta [in] The multiplier for the target multi-vector <tt>Y</tt>.
 * The default value of <tt>beta</tt> is <tt>0.0</tt>.
 * 
 * <b>Preconditions:</b><ul>
 *
 * <li> <tt>nonnull(this->domain()) && nonnull(this->range())</tt>
 *
 * <li> <tt>this->opSupported(M_trans)==true</tt> (throw
 * <tt>Exceptions::OpNotSupported</tt>)
 *
 * <li> <tt>X.range()->isCompatible(*op(this)->domain()) == true</tt> (throw
 * <tt>Exceptions::IncompatibleVectorSpaces</tt>)
 *
 * <li> <tt>Y->range()->isCompatible(*op(this)->range()) == true</tt> (throw
 * <tt>Exceptions::IncompatibleVectorSpaces</tt>)
 *
 * <li> <tt>Y->domain()->isCompatible(*X.domain()) == true</tt> (throw
 * <tt>Exceptions::IncompatibleVectorSpaces</tt>)
 *
 * <li> <tt>Y</tt> may <i>not</i> alias <tt>X</tt>.  It is up to the client to
 * ensure that <tt>Y</tt> and <tt>X</tt> are distinct since in general this
 * can not be verified by the implementation until, perhaps, it is too late.
 * If possible, an exception will be thrown if aliasing is detected.
 *
 * </ul>
 *
 * <b>Postconditions:</b><ul>
 * <li> After the function returns the multi-vector <tt>Y</tt>
 *      is transformed as indicated above.
 * </ul>
 */
void apply(
  const EOpTransp M_trans,
  const MultiVectorBase<Scalar> &X,
  const Ptr<MultiVectorBase<Scalar> > &Y,
  const Scalar alpha,
  const Scalar beta
  ) const;
\end{verbatim}}
%
\caption{\label{fig:complete_func_doc}
Example of more complete doxygen documentation for a function.}
%
\end{figure}


\begin{itemize}


\item\DOXWriteInHeaders: Writing Doxygen documentation comments
  directly attached to the classes, functions and other entities helps
  make the documentation as tightly tied to the code as possible.  See
  ``Keep comments close to the code they describe'' in \cite[Section
  32.5]{CodeComplete2nd04}.  This has the unfortunate side-effect of
  requiring complete recompilations whenever documentation is
  modified, but the overall benefits are usually worth the
  disadvantages.  Note that the Doxygen documentation can be stripped
  out of Doxygen-generated hyper-linked versions of the code, leaving
  clean C++ code without the clutter of detailed documentation.
  Therefore, developers should browse Doxygen-generated source code
  instead of the source code directly when looking at the code and
  performing code reviews.


\item\DOXUseCentralizedDefintions: Use clear and consistent naming of
  arguments in multiple functions (within the same class and across as
  many classes and functions as makes sense).  Provide a centralized
  definition of these arguments if possible to avoid repeating
  detailed descriptions in each individual function's documentation.
  This helps to avoid duplicate documentation that is likely not to be
  maintained correctly.  In the case of classes, this means providing
  some common definitions in the main ``detailed'' documentation
  section for the class.  In the case of nonmember functions, this
  might involve a common Doxygen group or module (i.e., using the
  \texttt{$\backslash$defgroup} command) for the set of functions.
  In the case of collections of nonmember functions, it may be
  difficult to expect readers to find the common definitions, but
  links to the common documentation are possible using a variety of
  approaches.


\item\DOXUseCentralizedPrePostConditions: For common arguments that
  are shared among many functions, define the most common
  preconditions for them in a central place.  Avoid listing them on a
  function-by-function basis unless they change for an individual
  function.  For a C++ class, place descriptions for these common
  arguments in the main class documentation under a
  {}\texttt{$\backslash$section} named ``Common Function Arguments and
  Pre/Postconditions''.  Only include preconditions for these
  arguments in specific function documentation sections if they differ
  from the most common preconditions.


\item\DOXAddBrief: The {}\texttt{$\backslash$brief} field is used to
  provide the short one-line documentation string that is included in
  the function summary section of classes, groups, namespaces etc.
  Even if no text documentation is needed or wanted, add an empty
%
{\small\begin{verbatim}
  /** \brief . */
  void someFunction();
\end{verbatim}}
%
comment so that Doxygen will include the class, function, or other entity in
the HTML documentation.  Note that this is important when the Doxygen
configuration option {}\texttt{EXTRACT\-\_ALL} is set to {}\texttt{NO}.


\item\DOXAllOrNoneParam: All arguments should be listed in
  \texttt{$\backslash$param} fields with at least the [in], [out], or
  [in/out] specifications.  These should have at least a very short
  description.  Alternately, if the function arguments are
  self-documenting or have already been defined in the common
  documentation section, then no \texttt{$\backslash$param} fields for
  any of the arguments should be included at all.  If any of the
  arguments in a function's documentation are listed in
  \texttt{$\backslash$param} fields, then all arguments should be
  listed in \texttt{$\backslash$param} fields.


\item\DOXReturnsOnlyIfNeeded: Don't add a \texttt{$\backslash$returns}
  description of the return value if it is already clearly specified
  in the \texttt{$\backslash$brief} description of the function.
  However, if the nature of the return value is at all complex, then
  include a \texttt{$\backslash$returns} field to describe it.  When
  referring to the return object, refer to it as \texttt{returnVal}.
  By consistently using the identifier \texttt{returnVal} for the
  return value, users will immediately know what this refers to.


\item\DOXPreferPostconditionsInParam: The postconditions for output
  arguments can be listed directly in the \texttt{$\backslash$param}
  field for the argument if they only involve just that argument in a
  simple way.  Otherwise, if the postconditions are more complex or
  involve multiple arguments in order to specify, then they can be
  listed in the postconditions list.  It may be difficult to
  objectively determine the best place to list the postconditions for
  an output argument.


\item\DOXOrderOfDocFieds: A consistent ordering of sections of
  documentation for a function makes it easier for readers to find
  what they are looking for.


\item\DOXUseRelatesForNonmemberFunctions: If a nonmember function is
  most closely related to a single class, then use the
  \texttt{$\backslash$relates} field to cause the documentation for
  the function to be listed with the classes documentation.  This
  makes it easier for readers to find out everything that they can do
  with a class object (or set of class objects) just by looking at a
  single HTML page and a single summary list of functions (which
  includes member and nonmember related functions).


\item\DOXDocumentInitialVirtualOnly: Only provide detailed
  documentation of the initial declaration of a virtual function in
  the class where it is first defined as \texttt{virtual}.  In
  general, documentation should not be included for the overrides of
  virtual functions in derived classes.  Doxygen automatically puts in
  a link to the original virtual function in the base class so readers
  are just one click away for seeing the detailed documentation.
  Always add an empty
%
{\small\begin{verbatim}
  /** \brief . */
  void someFunction();
\end{verbatim}}
%
comment for every class and every function that should be included in the HTML
documentation but where no text documentation is wanted or needed.


\item\DOXAggregateOverrides: For example, the overrides of the virtual
  functions for the \texttt{Teuchos\-::Parameter\-List\-Acceptor}
  interface would look like:

{\small\begin{verbatim}

  class SomeClass : public Teuchos::ParameterListAcceptor {
  public:

    ...

    /** \name Overriden from Teuchos::ParameterListAccpetor */
    //@{
    /** \brief . */
    void setParameterList(
      Teuchos::RCP<Teuchos::ParameterList> const& paramList);
    /** \brief . */
    Teuchos::RCP<Teuchos::ParameterList> getParameterList();
    /** \brief . */
    Teuchos::RCP<Teuchos::ParameterList> unsetParameterList();
    /** \brief . */
    Teuchos::RCP<const Teuchos::ParameterList> getParameterList() const;
    /** \brief . */
    Teuchos::RCP<const Teuchos::ParameterList> getValidParameters() const;
    //@}

    ...

  };

\end{verbatim}}


\item\DOXAutogenerateExampleCode: Any significant fragment of example
  code that is shown in Doxygen HTML documentation or a latex document
  needs to come from compiled and tested code that can be updated
  automatically.  These C++ code fragments can be selectively inserted
  automatically into Doxygen documentation using the
  \texttt{$\backslash$dontinclude} Doxygen command.


\item\DOXAutogenerateSampleOutput: Sample output included in Doxygen
  documentation should be generated automatically by the test harness
  code and should be written to files that are included in the source
  directory.  The sample output in these files can then be inserted
  into the Doxygen HTML documentation automatically using the
  \texttt{$\backslash$verbinclude} Doxygen command.  Similar
  approaches can also be used for \LaTeX{} documentation.


\end{itemize}
