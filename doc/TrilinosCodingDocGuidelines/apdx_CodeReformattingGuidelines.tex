%
\section{Guidelines for reformatting of source code}
\label{sec:reformatting-guidelines}
%

When a sufficiently common coding style is not being used by all developers in
a project and no recommendations for a common coding style exists, then some
guidelines are needed for the situations where code written by one individual
is modified by another individual that uses a different coding style.  These
guidelines address how developers should conduct themselves when modifying
source files written largely by someone else.

\begin{enumerate}

{}\item First and foremost, each developer should respect the other
developers' formatting styles when modifying their code.  If a
developer has a preferred Emacs style, then that style should be
listed explicitly at the top of each source file that is modified.
This will help other developers that use Emacs to stay consistent with
the file's style.

{}\item When only small changes are needed, a developer should abide by the
formatting style already in use in the file.  This helps to respect other
developers and helps to avoid needless changes for the version control system
to have to track.  Again, when user-defined file-specific Emacs styles are
specified, then it is easy to maintain a file's style when editing files
through Emacs.

{}\item Reformatting a file written by someone else and checking it in
is only justified if significant changes are made.  However, if a
developer needs to understand a complicated piece of code in order to
make perhaps even a small change in the end, then that developer may
also be justified in reformatting the file.  When a reformatting is
done, the new Emacs formatting style should be added to the top of the
source file in order to make it easier for the original owner of the
file and other developers to maintain the new style.

{}\item Multiple re-formats of the same file should not be checked in over and
over again as this will result in massive increases the the amount of
information that the version control system needs to keep track of and makes
diffs more difficult to perform.

\end{enumerate}

The above guidelines ensure that individuals are given maximal freedom to
format code to their liking but also helps to foster the shared ownership and
development of code.  In addition, the use of user-defined file-specific
formats makes it easy for developers to accommodate formatting styles
different from their own.
