\documentclass[pdf,ps2pdf,11pt]{SANDreport}
\usepackage{pslatex}
%Local stuff
\usepackage{graphicx}
\usepackage{latexsym}
%\usepackage{color}
\usepackage[all]{draftcopy}
%
% Command to print 1/2 in math mode real nice
%
\newcommand{\myonehalf}{{}^1 \!\!  /  \! {}_2}

%
% Command to print over/under left aligned in math mode
%
\newcommand{\myoverunderleft}[2]{ \begin{array}{l} #1 \\ \scriptstyle #2 \end{array} }

%
% Command to number equations 1.a, 1.b etc.
%
\newcounter{saveeqn}
\newcommand{\alpheqn}{\setcounter{saveeqn}{\value{equation}}
\stepcounter{saveeqn}\setcounter{equation}{0}
\renewcommand{\theequation}
	{\mbox{\arabic{saveeqn}.\alph{equation}}}}
\newcommand{\reseteqn}{\setcounter{equation}{\value{saveeqn}}
\renewcommand{\theequation}{\arabic{equation}}}

%
% Shorthand macros for setting up a matrix or vector
%
\newcommand{\bmat}[1]{\left[ \begin{array}{#1}}
\newcommand{\emat}{\end{array} \right]}

%
% Command for a good looking \Re in math enviornment
%
\newcommand{\RE}{\mbox{\textbf{I}}\hspace{-0.6ex}\mbox{\textbf{R}}}

%
% Commands for Jacobians
%
\newcommand{\Jac}[2]{\displaystyle{\frac{\partial #1}{\partial #2}}}
\newcommand{\jac}[2]{\partial #1 / \partial #2}

%
% Commands for Hessians
%
\newcommand{\Hess}[2]{\displaystyle{\frac{\partial^2 #1}{\partial #2^2}}}
\newcommand{\hess}[2]{\partial^2 #1 / \partial #2^2}
\newcommand{\HessTwo}[3]{\displaystyle{\frac{\partial^2 #1}{\partial #2 \partial #3}}}
\newcommand{\hessTwo}[3]{\partial^2 #1 / (\partial #2 \partial #3)}



%\newcommand{\Hess2}[3]{\displaystyle{\frac{\partial^2 #1}{\partial #2 \partial #3}}}
%\newcommand{\myHess2}{\frac{a}{b}}
%\newcommand{\hess2}[3]{\partial^2 #1 / (\partial #2 \partial #3)}

%
% Shorthand macros for setting up a single tab indent
%
\newcommand{\bifthen}{\begin{tabbing} xxxx\=xxxx\=xxxx\=xxxx\=xxxx\=xxxx\= \kill}
\newcommand{\eifthen}{\end{tabbing}}

%
% Shorthand for inserting four spaces
%
\newcommand{\tb}{\hspace{4ex}}

%
% Commands for beginning and ending single spacing
%
\newcommand{\bsinglespace}{\renewcommand{\baselinestretch}{1.2}\small\normalsize}
\newcommand{\esinglespace}{}



\raggedright

% If you want to relax some of the SAND98-0730 requirements, use the "relax"
% option. It adds spaces and boldface in the table of contents, and does not
% force the page layout sizes.
% e.g. \documentclass[relax,12pt]{SANDreport}
%
% You can also use the "strict" option, which applies even more of the
% SAND98-0730 guidelines. It gets rid of section numbers which are often
% useful; e.g. \documentclass[strict]{SANDreport}

% ---------------------------------------------------------------------------- %
%
% Set the title, author, and date
%

\title{\center
Nightly Building and Testing of Development Versions of Applications and
Trilinos\\[2ex] {\Large A new foundation for enhanced collaboration in
application and algorithm research and development}}
\author{Roscoe A. Bartlett}

% ---------------------------------------------------------------------------- %
% Set some things we need for SAND reports. These are mandatory
%
\SANDnum{SAND2007-xxx}
\SANDprintDate{??? 2007}
\SANDauthor{Roscoe A. Bartlett}

% ---------------------------------------------------------------------------- %
% The following definitions are optional. The values shown are the default
% ones provided by SANDreport.cls
%
\SANDreleaseType{Unlimited Release}
%\SANDreleaseType{Not approved for general release}

% ---------------------------------------------------------------------------- %
% The following definition does not have a default value and will not
% print anything, if not defined
%
%\SANDsupersed{SAND1901-0001}{January 1901}

% ---------------------------------------------------------------------------- %
%
% Start the document
%
\begin{document}

\maketitle

% ------------------------------------------------------------------------ %
% An Abstract is required for SAND reports
%

%\clearpage

%
%\begin{abstract}
%

%ToDo: Fill this in!

%
%\end{abstract}
%

% ------------------------------------------------------------------------ %
% An Acknowledgement section is optional but important, if someone made
% contributions or helped beyond the normal part of a work assignment.
% Use \section* since we don't want it in the table of context
%
\clearpage
\section*{Acknowledgments}

I would like to thank Scott Collis, Mike Heroux, Russell Hooper, Roger
Pawlowski, Andy Salinger, Jim Willenbring and many others for helpful
conversations that helped shape this idea.

ToDo: Add more individuals that need to be acknowledged.

%
%The format of this report is based on information found
%in~\cite{Sand98-0730}.

% ------------------------------------------------------------------------ %
% The table of contents and list of figures and tables
% Comment out \listoffigures and \listoftables if there are no
% figures or tables. Make sure this starts on an odd numbered page
%
\clearpage
\tableofcontents
%\listoffigures
%\listoftables

% ---------------------------------------------------------------------- %
% An optional preface or Foreword
%\clearpage
%\section{Preface}
%Although muggles usually have only limited experience with
%magic, and many even dispute its existence, it is worthwhile
%to be open minded and explore the possibilities.

% ---------------------------------------------------------------------- %
% An optional executive summary

%\clearpage

%\section{Executive Summary}

% ---------------------------------------------------------------------- %
% An optional glossary. We don't want it to be numbered
%\clearpage
%\section*{Nomenclature}
%\addcontentsline{toc}{section}{Nomenclature}
%\begin{itemize}
%\item[alohomora]
%spell to open locked doors and containers
%\end{itemize}

% ---------------------------------------------------------------------- %
% This is where the body of the report begins; usually with an Introduction
%


\SANDmain % Start the main part of the report


%
\section{Introduction}
%

During the course of the ASC Level-2 Vertical Integration Milestone work
{}\cite{ref:asc-vertical-integration-milestone}, we found that in order to
keep moving forward and avoid backslides in capability (which happened early
on), we needed to implement nightly building and testing\footnote{Charon Dev
and Trilinos Dev has been nightly built and tested separately for many years.
It was the combined Dev versions of Charon + Trilinos that was not nightly
built and tested until the milestone work.} of the development versions of
Charon and Trilinos {}\cite{ref:trilinos}.  Every night, we take what is in
the Charon and Trilinos development repositories and build the combined Charon
+ Trilinos application and run a large set of regression tests.  This work had
broad implications on the nature of the interaction of applications (APPs) and
Trilinos solvers.

Currently, Trilinos is only being released once a year and then APPs can take
several months after a Trilinos release to upgrade Trilinos versions.  What is
being done right now with Trilinos is the maintenance version of the so called
``Big Bang'' integration approach which has all sorts of bad implications
associated with it {}\cite{book:code-complete-2}.  Iterations of one year or
even six months are too long to call ``incremental'' integration if any
significant develop is being done in the APP or in Trilinos.  Modern trends in
software engineering point to much more frequent integrations as a way to
reduce risk and make the development and release process much more predictable
[??? put in lots of references here ???].

In the execution of this nightly building and testing process, we have learned
many things about how to do quasi-continuous integration {}\cite[Section
29.4]{book:code-complete-2}\footnote{By quasi-continuous integration, we mean
that we are suggesting integrating and testing once a day and not every few
hours as some have advocated {}\cite{continuous-integration}.} of application
and algorithms software and we have realized many important unplanned
benefits.  This will allow us to reap many more benefits in the future if this
process is maintained and extended.  There are both production-related
benefits and research-related benefits that help both the application
developers and the algorithm developers to achieve their goals.  On the
research side, this significantly reduces the overhead required for algorithm
developers to try their algorithms out on production quality problems.
Developing a numerical solver with a production problem exposes the algorithm
developer to a whole host of issues (e.g.\ poor scaling, ill-conditioning,
convergence difficulties, etc.) that are hard to replicate with model
problems.  On the production side, constant integration insures that the
application and Trilinos are always up to date and satisfying the
application's requirements.  Therefore, when it is time for a release, only a
final set of acceptance tests are required and then the codes can be branched
and released shortly after.  This helps to reduce a whole host of risks such
as slipped schedules and broken features\footnote{Also known as regressions}.

I have seen several different scenarios over the years where this nightly
building and testing infrastructure would have facilitated collaboration
between application and algorithm developers for the advantage of both.  For
example, suppose an application developer is running a new problem and
discovers some strange behavior from a numerical solver.  The algorithm
developer may look at the results and speculate what the cause of the behavior
might be or if a different variation of the algorithm might help.  However, if
the algorithm developer is stuck having to use a released version of Trilinos,
it will be more difficult to make any major changes in order to investigate
the behavior.  Also, there may already be improvements made to the algorithm
in the development branch of Trilinos that may be able to address the problem.
Without the infrastructure of nightly building and testing, it may not be cost
effective or practical to bring the development versions of the application
and Trilinos up to date in order to try the updated algorithm.  The algorithm
and application developers may have to wait until the next major release of
Trilinos before the new algorithms can be accessed.  This time delay works to
no one's advantage and can kill the collaboration.  With nightly building and
testing in place, an easy path for collaboration is maintained where the
Trilinos algorithm developer can try their latest and greatest algorithms on
these types of challenging production problems and the information learned
from trying to solve these production problems can feed back into algorithm
development.  To some extent, this back and forth development and integration
already happens, but without a foundational process in place to streamline it,
the bidirectional flow is greatly restricted.

Another example where nightly building and testing of the development versions
of the application and Trilinos would ease the path for collaboration is when
an application developers wants to try a new capability in Trilinos without a
lot of work\footnote{This example really happened and it was the inability to
readily access the development version of Trilinos within the application that
killed the collaboration.}.  When the up-to-date development version of
Trilinos is available from within an application, a new algorithm or
capability from Trilinos can be accessed much more easily.  Typically, even a
small increase in the overhead needed to try out a new Trilinos capability in
an application can be enough to kill a potentially fruitful collaboration
between an application and algorithm developer.  Nightly building and testing
of the development versions of the application and Trilinos removes a
unexciting (and therefore demoralizing) but critical obstacle to collaboration
and impact.

Nightly building and testing of an application code and Trilinos
brings algorithm developers and application developers closer together --
exchanging ideas and concerns -- and refocuses Trilinos developers on customer 
efforts while still helping drive publishable numerical algorithm solver
research and reduces barriers for new algorithms to have impact through
production application codes.

For the reminder of this paper, I will refer to the combined development
versions of the application (e.g.\ Charon, Xyce, Alegra etc.) and Trilinos as
APP + Trilinos Dev.


%
{}\section{Outline of proposed APP + Trilinos Dev development and nightly
building and testing}
%

The basic idea of this new approach goes like this (which is mostly what we
already do with Charon + Trilinos Dev right now):

{}\textit{\textbf{APP developers mainly work with APP + Trilinos Release}}:
The developers of APP (e.g. Charon, Alegra, SIERRA) would only do their
production development and run their production test suite against a stable
released version of Trilinos.

{}\textit{\textbf{Trilinos developers mainly work with Trilinos Dev}}: Most of
the developers of Trilinos would do their development work mostly just within
Trilinos and run the Trilinos test suite.  If radical changes are make, then
there should be a way for Trilinos developers to build and run APP + Trilinos
Dev and it's test suite to see if anything will break (before a check in).
This will require some effort to set up and maintain for each APP (see
practices in Section {}\ref{sec:suggested-practices}).

{}\textit{\textbf{Hide APP + Trilinos Dev ``research'' work in APP behind
ifdefs}}: Any ``research'' development done in APP against Trilinos Dev must
be hidden behind {}\#ifdefs so as not to impact the other production-focused
development work being done with APP that relies on the more stable release of
Trilinos.

{}\textit{\textbf{Perform nightly building and testing of APP + Trilinos
Release and APP + Trilinos Dev}}: In performing this process, it is our goal
to avoid unnecessary interactions between APP and Trilinos developers.  If a
Trilinos developer breaks some software in Trilinos, then we want to avoid
bothering an APP developer that has nothing to do with this and no reason to
know about this.  Likewise, if an APP developer breaks the APP code in some
way that does not expose a Trilinos defect, we want to avoid bothering
Trilinos developers with this\footnote{We did not have unnecessary
interactions totally under control for Charon + Trilinos Dev during the
milestone work and it resulted in some waisted time.  See the practices for
how this can be addressed.}.  The following builds are designed to help avoid
unnecessary communication between APP and Trilinos developers while still
catching APP + Trilinos Dev integration problems as efficiently as possible:

  \begin{itemize}

  {}\item\textit{\textbf{APP + Trilinos Release tested against ``production''
  test suite}}: The Dev version of APP would be built and tested against the
  static/stable current release of Trilinos.

    \begin{itemize}

    {}\item\textit{\textbf{Send ``production'' failures only to APP
    developers}}: Test failures for APP + Trilinos Release are only forwarded
    to APP developers, not Trilinos developers since this these are most
    likely caused be a defect added by an APP developer.  It is possible that
    latent defects in the current release of Trilinos may be the cause of the
    failure, but this should be unlikely.

    \end{itemize}

  {}\item\textit{\textbf{APP + Trilinos Dev tested against ``research'' and
  ``production'' test suites}}: The combined APP + Trilinos Dev code with
  enabled {}\#ifdefs and extended ``research'' test suite would be built and
  tested (with both ``production'' and ``research'' tests).

    \begin{itemize}

    {}\item\textit{\textbf{Only send ``production'' failures that did not also
    fail in APP + Trilinos Release to Trilinos developers (representative)}}:
    Only ``production'' tests that failed in this version that did not fail in
    the production APP version would be forwarded to Trilinos developers
    (e.g. to the dedicated APP + Trilinos Representative, see below).

    {}\item\textit{\textbf{Send ``research'' failures only to Trilinos
    developers (representative)}}: By default, average APP developers would
    not see these ``research'' test failures (unless they wanted to in which
    case they could be added to an e-mail list).

    \end{itemize}
                
  \end{itemize}

{}\textit{\textbf{Release APP + Trilinos together or staged }}: A release of
the APP + Trilinos would go one of two routes:

    \begin{itemize}

    {}\item\textit{\textbf{a) Combined tagging and release of APP +
    Trilinos}}: Right before a release of APP, the APP and Trilinos would be
    tagged and branched at the same time to make sure both are as current as
    possible.  This could be very challenging to pull off since it would
    require that Trilinos be releasable at almost any time.

    {}\item\textit{\textbf{b) Staggered releases of APP + Trilinos}}: APP
    developers make a decision to target a release of APP against a very
    recent release of Trilinos.  At the point of the release (or branch for
    the release) of Trilinos, the Dev version of APP drops support for older
    release of Trilinos, moves code from within the protected {}\#ifdefs into
    main Dev build and then any new work with Trilinos Dev is hidden behind
    new {}\#ifdefs.  As a variation, the APP developers may not be in a
    position to quickly transition over to the new Trilinos release so in this
    case the APP should be temporary built and tested against all three
    versions (the previous release, the current release, and the Dev
    versions).  As a variation of this, the APP may not be in a position to
    upgrade to the most recent release of Trilinos right away.  In this case,
    the nightly test harness can be setup to build the APP against three
    versions of Trilinos: i) the old Trilinos release, ii) the new Trilinos
    release, and iii) the development version of Trilinos.  This requires a
    little more effort but it guarantees that when the APP developers decide
    to transition the APP code base to the new Trilinos release, that this will
    happen smoothly without any problems.  At this point, support for the old
    Trilinos release can be dropped, and the APP developers can do a release
    of APP at any time that is convenient.

    \end{itemize}

{}\textit{\textbf{Continue APP + Trilinos Dev nightly building and testing
after APP upgrades to new Trilinos release}}: After APP is upgraded to the next
release of Trilinos, using any of the approach approaches described above, the
nightly building of APP + Trilinos Dev continues where future
incompatibilities between Dev versions of APP and Trilinos are hidden behind
new {}\#ifdefs.


%
{}\section{Advantages and disadvantages to nightly building and testing of APP
+ Trilinos Dev}
%

There are many research and production advantages to maintaining the nightly
building and testing of the development versions of production application
(APP) and Trilinos.  Some of the more significant advantages are described in
the following sections.


%
{}\subsection{Research advantages to nightly building and testing of APP +
Trilinos Dev}
%

Here are some of the research advantages to maintaining the nightly building
and testing APP + Trilinos Dev:

{}\textit{\textbf{Reduces overhead for initial algorithm integration}}: This
massively reduces the overhead needed for a Trilinos algorithms developer to
try out a new algorithm on a production quality problem implemented in APP
because the codes will build right away.  One of the most difficult aspects of
doing serious algorithm research is having access to serious (possibly messy)
production quality problems.

{}\textit{\textbf{Improves chances that new algorithms will have impact}}:
Reducing overhead of the initial software integration effort improves the
chances that a new algorithm or package will benefit a real application and
therefore show real ``impact'' which is always a criticism for research-driven
algorithm developers.  This is related to the previous point but is worth
mentioning by itself.

{}\textit{\textbf{Preserves interesting/challenging problems}}: This preserves
really interesting problems that will be constant drivers for future algorithm
research.  A really challenging numerical test problem that is encountered
after an initial algorithm integration and experimentation effort can be
preserved so that algorithm researchers can come back to it later again and
again to try out new versions of their algorithms.  Spending a lot of effort
to get an algorithm integrated with some production code and then having that
connection and the working examples lost happens again and again with our
current environment.  Even more important is the ability to show ``progress''
as our algorithms improve by running them on the same basic production physics
problems.  Note that preserving interesting test problems is more of an issue
for higher-level algorithms like sensitivity solvers and optimization than it
is for more basic algorithms like linear solvers (and the associated
preconditioners).  The ability to solve a linear system is a prerequisite for
any (semi)implicit forward simulation solver and therefore challenging linear
systems are ubiquitous in scientific computing.  Since higher-level analysis
problems are not part of low-level modeling and forward solver work, they are
often overlooked by APP developers and when the time comes to perform these
higher-level analysis, the APP infrastructure to support such methods is gone.
Nightly building and testing for these higher-level analysis problems preserves
them for future research and future use on critical problems.


%
{}\subsection{Production advantages to nightly building and testing of APP +
Trilinos Dev}
%

Here are some of the production advantages to maintaining the nightly building
and testing of the development versions of a production application and
Trilinos:

{}\textit{\textbf{Expands testing for Trilinos}}: The tests in the APP's test
suite represents an extended test suite for Trilinos.  In our work with
Charon, we have already seen cases where the Charon test suite caught an error
that the Trilinos test suite did not.  Most APPs also build against an
installed version of Trilinos so nightly building and testing of APP +
Trilinos Dev helps to test the installation of Trilinos which is very hard to
do for Trilinos by itself.

{}\textit{\textbf{Enables better scalability testing for Trilinos}}:
Production APPs provide ready access to large-scale parallel problems that can
serve as a vehicle for testing the parallel scalability of Trilinos
algorithms.  Teuchos timers can be used which make it easy to isolate timings
for specific algorithm features and will show load imbalances if they exist.
Automated scripts can query these timings (from the output) and can catch any
problems with scalability that are seen.

{}\textit{\textbf{Reduces time to detection of defects}}: This is directly related
to expanded testing described in the above items but defects introduced in
Trilinos code between releases that break customer functionality (but may not
be caught by native Trilinos tests) will be caught right away.  Even if a test
failure is not fixed right away, knowing exactly when a test failed is an
extremely useful piece of information in being able to track down and fix the
defect later.  The cost of fixing defects increases the longer the time
between when the defect is introduced and the time it is first detected
{}\cite{book:code-complete-2}.  Daily testing reduces the risk that a release
of Trilinos will regress and/or helps to contain costs and the schedule if a
broken feature must be preserved in the next combined release of APP +
Trilinos.

{}\textit{\textbf{Reduces release time and effort}}: It reduces (or eliminates)
the time needed to create a bundled release of APP + Trilinos.  This removes
uncertainty about how long it will take to put out a release of Trilinos
and/or APP. 

{}\textit{\textbf{Allows for more aggressive refactorings and code
improvements}}: It allows for more aggressive refactorings of Trilinos code
since the impact of the refactorings on important APP customers can be
ascertained and fixed right away.  This will allow the architecture to evolve
as needed in a safer and less demanding way.  The ability to refactor code is
the number one issue in making sure a code does not become ``legacy code''.
Without the ability to refactor a code, you automatically insure that the code
will be thrown away or relegated to ``legacy code'' at some point
{}\cite{book:code-complete-2}.

{}\textit{\textbf{Better address customer needs}}: This will bring Trilinos
algorithm developers closer to important Trilinos APP customers so that
customer needs can be addressed more effectively in a timely manner.

{}\textit{\textbf{Reduces all kinds of risk}}: Overall, this simply reduces all
kinds of risks, increases predictability of the development and release
process, and makes Trilinos more responsive to important customers.


%
{}\subsection{Potential disadvantages to nightly building and testing of APP +
Trilinos Dev}
%

Here are some of the potential disadvantages to maintaining the nightly
building and testing of the development versions of a production application
and Trilinos:

{}\textit{\textbf{Will slow down day-to-day development to varying degrees}}:
It will slow down the day-to-day development of Trilinos and the APP to some
extent in that problems are dealt with as they are uncovered by daily building
and testing.  The amount of extra overhead will depend on how aggressively
failing tests are addressed (see the practices in Section
{}\ref{sec:suggested-practices}).  Just keeping APP + Trilinos Dev building
should not impart much overhead at all is most cases.

{}\textit{\textbf{Will require better, more coordinated management practices}}:
This will require some more sophisticated management practices and tools to
keep all of this running smoothly.  This will require some additional effort
over what is done now with more up-front effort to set up.

{}\textit{\textbf{Will impose greater responsibility to meet customer needs}}:
Trilinos developers will have a greater responsibility to meet important APP
customer needs.  A lot of algorithm researchers don't want to sign up for that
kind of responsibility.  However, such individuals don't have to write or
maintain production algorithmic capabilities.  There will always be a place
for more pure algorithms research and more theoretical (i.e.\ less applied)
algorithm researches.

{}\textit{\textbf{Could increase overall development effort}}: It may increase the
overall development time for Trilinos if not managed well.  However,
experience by other projects and organizations suggests that the overall
development time and effort to create and maintain production capabilities
should actually decrease! {}\cite{book:code-complete-2}.


%
{}\section{Suggested practices to support proposed APP + Trilinos Dev
development and nightly building and testing}
\label{sec:suggested-practices}
%

Successfully implementing APP + Trilinos Dev nightly building and testing will
require, or will be made much more effective, by the adopting several
practices.  Some of the more important practices and issues consider are
described below (and are summarized in Appendix {}\ref{sec:checkist}).

{}\textit{\textbf{Separate ``production'' and ``research'' tests}}: Any new
test or example added to APP's test suite based on new features in Trilinos
Dev must be added as new ``research'' tests as not to affect existing
``production'' tests.  In this way, we can easily differentiate between
``production'' regression tests and new ``research'' or ``pre-production''
tests.  This is not full-proof in differentiating defects in APP code verses
Trilinos code, but this will go a long way in helping to suggest where the
problem is.

{}\textit{\textbf{Maintain a dedicated machine for building and testing APP +
Trilinos Dev}}: Having a dedicated, powerful machine for supporting APP +
Trilinos Dev would make it easy to maintain an environment to build and test
APP + Trilinos Dev.  It would also enable Trilinos developers a quick and easy
way to access, build and test APP + Trilinos Dev.  Issues like keeping third
party libraries up to date would only need to be handled on this one machine.
Accounts would be granted as needed and would only require SRN access.  This
would allow any Trilinos or APP developer with an SRN account to quickly log
onto the dedicated machine, and then be able to quickly build APP + Trilinos
Dev and run the ``research'' and ``production'' test suites.  Some helper
scripts and examples also need to be in place to show how to do this.

{}\textit{\textbf{Appoint a dedicated APP + Trilinos Representative}}: One
member of the Trilinos or APP development teams should be designated as the
point person for the APP + Trilinos Dev effort.  This person would be
responsible for filtering test failures and forwarding issues to APP or
Trilinos developers.  This person must be familiar with the APP and Trilinos
for this to be effective.  Ideally, this person will be a co-developer of APP
and Trilinos, so there would be little-to-no learning curve.  Having only one
person be responsible (with perhaps a backup person in their absences) will
make it clear who is accountable for making sure issues are dealt with in a
timely manner.  This person would take the major responsibility of maintaining
the dedicated APP + Trilinos machine described above. The goal is that this
job should not take too much effort if everyone else is doing their jobs well.
However, this job could be a nightmare if this effort is not taken seriously
by everyone involved (including management).  In addition, this responsibility
should come with a specific project/task (P/T).  Having a specific P/T serves
several purposes.  First, it lets the APP + Trilinos Representative know that
this task is ordained and supported from management.  Second, it allows us to
track how much time and expense is going into keeping APP + Trilinos Dev
working.

{}\textit{\textbf{Provide easy access for any Trilinos or APP developer to
build, test, and develop APP + Trilinos Dev}}: Trilinos and APP developers
need a quick and painless way to build APP + Trilinos Dev in order to diagnose
and fix failures.  This may include the ability to checkout and change APP
code to fix the problem.  Of course, modified APP code would need to go
through a code review by APP developers before it was checked into APP's
repository.  Likewise, an APP developer should be able to change and fix
Trilinos code if they are so motivated.  Again, a code review by Trilinos
developers should be done before an APP developer checks in any code changes
to Trilinos.  Having a dedicated machine maintained by the APP + Trilinos
Representative as described above would make this easy to support.  Obviously,
another large benefit to providing easy access to APP + Trilinos Dev is that
it makes it easier for APP or Trilinos developers to try out new Trilinos
algorithms at any time based on the most recent Trilinos code.

{}\textit{\textbf{Fix failed builds of APP + Trilinos Dev ASAP}}: It is
critical that fixing broken builds of APP + Trilinos Dev be given a high
priority and be addressed immediately.  Without the software at least building
and linking to run the tests, we can have no feedback at all about the state
of our software and the entire nightly building and testing process falls
apart.  There can be no exception to this.

{}\textit{\textbf{Address failing ``research'' and ``production'' tests on a
schedule appropriate for the APP + Trilinos collaboration}}: While fixing
failed builds of APP + Trilinos Dev must always be given a high priority and
fixed immediately, addressing failing ``research'' and ``production'' runtime
tests can be done on a variety of different schedules, depending on the nature
of the APP and/or the APP + Trilinos collaboration at any time.  We can
imagine two extremes in how and when failing tests are addressed.

One one extreme, every day, each and every failing test is given a high
priority and fixed ASAP\footnote{This was mostly our policy on the ASC
Vertical Integration Milestone but we did not completely keep up with this.}.
While this approach results in the least risk of experiencing a regression, it
can significantly harm overall productivity (especially for the APP + Trilinos
Representative).

On the other extreme, we might not address any failing tests at all between
releases and wait until the next upcoming release before any of the failed
tests are addressed.  While this other extreme has more risk associated with
it as opposed to fixing failing tests instantly, it still offers significant
advantages to not performing any daily building or testing at all.  First, by
keeping APP + Trilinos Dev building, we can address any of the failures at any
time we wish.  Second, knowing the exact 24 hour period when code changes were
made that caused a failing test is a huge piece of information to help find
the cause of a test failure.  Letting failing tests fail for long periods of
time and only requiring that APP + Trilinos Dev keeps building should only
impart a very minimal overhead to day-to-day development activities.

In between these two extremes, every morning one or more failed tests were
reported, the APP + Trilinos Representative would spend five to ten minutes
every morning looking over the new failing tests and try to diagnose them.  If
the problem can be quickly diagnosed\footnote{Approximately 80\% or so of the
failing Charon + Trilinos Dev tests were diagnosed in just ten minutes or
less.}, then an e-mail can be sent (or a bug report can be filed) to the
parties that can fix the bug.  If the problem can not be easily diagnosed in
five to ten minutes, then the APP + Trilinos Representative might just make a
note of this (e.g.\ file a general bug report, send a general e-mail, etc.) and
then move on with the day's other activities.  Then when time becomes
available, the root causes of the failing tests can be diagnosed when it will
not disturb the flow of other work.  Again, knowing the exact 24 hour period
when a test first failed is a huge piece of information in finding the root
cause of the problem.

In summary, depending on the nature of APP and the relationship between APP
and Trilinos, any level of urgency between these two extremes may be
acceptable and this will still be much better than not doing any daily
building or testing at all.  The approach taken to addressing failing tests
for APP + Trilinos Dev, for any specific APP, will change as the collaboration
goes through periods of greater intensity and lesser intensity.  During
periods of more intense APP + Trilinos collaboration, we will be more
aggressive about addressing failing tests.  During periods of less intense (or
nonexistent) APP + Trilinos collaboration, we can let tests fail for longer
periods of time.

{}\textit{\textbf{Archive test results for sufficiently long periods of
time}}: Test results from APP + Trilinos Release and APP + Trilinos Dev should
be archived for long periods of time.  Typically, only smaller output files
are needed to diagnose most problems and therefore the largest of output files
should typically be excluded from the test results archive.  However, all test
output files should be archived between 24 hour periods.  Having ready access
these test results, and being able to compare the outputs from a passing and a
failing version of a test (separated by 24 hours) is critical in helping to
diagnose failing tests.  Past test results should be pruned and thinned as
needed to conserve disk space.  For example, test results from three months
ago could be deleted except for tests on Friday (or some other day) for each
week.  One exception would be that test results for consecutive days where a
test went from passing to failing should be preserved for a long period of
time (perhaps a year or more) since this is critical evidence in tracking down
failing tests (especially in the extreme where tests are allowed to fail for
very long periods of time).  Easy access to test results can be provided
through a website that anyone with SRN access can access\footnote{We have
provided web access to the test results archive for the Charon + Trilinos Dev
test suite.} (or limit access to those individuals with Need-to-Know in the
case of more sensitive APPs).

{}\textit{\textbf{Transition ``research'' to ``production'' appropriately
after each Trilinos release}}: If a ``research'' algorithm or feature becomes
stable enough and the software implementation is of high enough quality (i.e.\
a ``phase 2'' package in Trilinos), then after the next Trilinos release the
``research'' APP code and tests for that algorithm/feature should be moved to
the APP's ``production'' code and tests.  In this way, if the test fails
later, the APP developers will be the first to face the bug since it is most
likely an APP developer that broke something.

{}\textit{\textbf{Perform APP + Trilinos Release and APP + Trilinos Dev
nightly testing on the same set of platforms}}: The nightly building and
testing of APP + Trilinos Release and APP + Trilinos Dev should be performed
on the same set of platforms.  In this way, if a ``production'' test fails
with APP + Trilinos Dev but not with APP + Trilinos Release on the same
platform, then we have some assurance that something has been broken by in the
``research'' work and not the ``production'' work being done by APP developer.
If testing is done on different platforms, then a ``production'' test failure
with APP + Trilinos Dev may just be due to small difference is rounding or
other small issues that result from using different platforms\footnote{In the
ASC Vertical Integration Milestone work with Charon + Trilinos Dev, we did not
have this in place.  As a result, there were several occasions that new
``production'' tests failed when run with Charon + Trilinos Dev that passed
with Charon + Trilinos Release on a different platform.  Most of these
``production'' test failures were just due to minor rounding or other porting
issues.  The time waisted tracking down these ``production'' test failures
could have been avoided if we had this policy in place.}.

{}\textit{\textbf{Enable more communication between APP and Trilinos
developers}}: All of this will require and foster more communication and
cooperation between Trilinos and APP developers.  This means that Trilinos
developers will have to have a closer relationship with their customers.

{}\textit{\textbf{Provide for instantaneous releasabiliy of Trilinos to
important customers}}: In order to allow for the option of the Dev versions of
the APP and Trilinos to be tagged, branched, and released together, the
Trilinos release process for such customers needs to be doable in only a few
days at most.  This will require a change in a great many of the current
Trilinos practices.  For example, this will require that we port Trilinos Dev
to various platforms where the APP runs on a frequent basis (perhaps every few
months or less).  Also, this will require that we have completely automated
tarball testing and installation testing.  There are other issues that would
also need to be changed and/or improved in how we develop Trilinos.  If a
staggered release of Trilinos and APP is performed, then the instantaneous
releasabiliy is not really need (but is useful as a general principle in any
case).


%
\section{Experience from the ASC Vertical Integration Milestone with Charon + Trilinos Dev}
%

The purpose of this section is to describe what was done in the FY07 ASC
Level-2 Vertical Integration Milestone with Charon + Trilinos Dev.  The
milestone work served as the inspiration for APP + Trilinos Dev and served as
a prototype and case study for APP + Trilinos Dev integration in practice.  I
will describe what we did, what worked well, and what needs to be improved
this this and other such efforts.  Many of the suggested practicies given in
Section {}\ref{sec:suggested-practices} came from feedback provided by the
Charon + Trilinos Dev relationship conducted during the milestone.  As a
contrast, I also describe a smaller effort that involved an integration of
Aria/SIERRA + Trilinos Dev.


%
\subsection{Charon + Trilinos Dev nightly testing building and testing}
%

Here I describe the basic elements of the Charon + Trilinos Dev nightly
building and testing process that we had in place for the milestone.  First,
note that Charon has had its own native nightly test harness in place for many
years.  However, the Charon test harness was only set up to checkout and build
Charon against a static set of third party libraries (TPLs), including
Trilinos, and it was not clear how these scripts would be updated to allow for
building with Trilinos Dev updated daily.  Also, lack of Charon tool developer
support made it difficult to see how to add this capability.  Therefore, the
decision was made to develop a new minimal test harness framework specifically
for nightly building and testing of the Dev versions of Charon and Trilinos.
Starting in February of 2007, we set up and ran nightly building and testing
Charon + Trilinos Dev in debug (dbg) and optimized (opt) mode on my own 64
bit, 4-core, AMD, Linux workstation using GCC for the compiler.

The Charon + Trilinos Dev test harness itself used the native
Nevada/Alegra/Charon test harness to define and run the tests.  Therefore, we
did not reinvent the core test harness since the existing Nevada/Alegra/Charon
test harness is quite good in many ways.  The Charon + Trilinos Dev test
harness shell scripts just focused on checking-out/updating the sources,
running the builds, invoking the Nevada/Alegra/Charon test harness,
interpreting the results, archiving the results, and sending out e-mail
notifications.

A set of shell-based (i.e.\ {}\texttt{sh}) scripts were written to perform all
tasks of the test harness.  A top-level script was written that was directly
invoked by a crontab job.  Every night (starting at midnight), the test
harness script checks out the Dev versions of Charon, the Charon TPLs, and
Trilinos (within the Charon TPL directory structure) and builds and tests
various builds of Charon + Trilinos Dev.  These scripts were not only designed
to be used from the automated test harness but were also designed to be used
to document how to perform various development and testing tasks.  The scripts
that are called are designed to show developers how to run the various tools
to checkout, build, and test Charon with the Charon TPLs (which have Trilinos
Dev embedded in them).  Therefore, the nightly test harness not only tests
Charon + Trilinos Dev, but it also tests the scripts that document the various
tasks and developers can use to perform these tasks.

Two directory trees were established and built from.  An {}\texttt{Updated}
base directory was used where CVS updates were done into an existing tree for
Charon, Charon\_TPL, and Trilinos.  From this tree, the optimized (opt)
version of Charon + Trilinos Dev was built and tested.  This is an important
use case for continuing developers that will typically update existing working
directories and then rebuild the code.  A second {}\texttt{FromScratch} base
directory was used where the working directories for code, object files,
libraries, and executables were all deleted.  Then, all of the source code was
checked out from scratch, and built and tested.  This is also an important use
case since it ensures that new developers can checkout and build all of the
code from scratch at any time.

As mentioned above, the Charon + Trilinos Dev test harness not only builds
Charon + Trilinos Dev every night, it also builds all of the Charon TPLs
(including Trilinos).  Early on in the milestone, a change was made to one of
the Charon TPLs that resulted in Charon to break.  However, the existing
Charon nightly test harness maintained by the Charon developers themselves on
another machine did not (and still does not at the time of this writing) build
the Charon TPLs every day.  This delayed integration represents an increased
risk (as all delayed integration does) which the Charon + Trilinos Dev test
harness has addressed.  A similar delayed integration for Charon exists with
the Nevada/Alegra framework itself but that integration risk is beyond the
scope of this discussion.

The test harness shell scripts also archive the test results to a
web-accessible directory tree.  The links to these directories for each test
are given in passed/failed e-mail notifications that are sent out at the end
of each build \& test invocation for each build (i.e.\ opt vs. dbg) of Charon
+ Trilinos Dev.  All of the test input and output files are saved in this
archive directory tree.  However, to avoid massive increases in storage, all
files above 1M are deleted.  This removes mostly large mesh input and exodus
input and output files but maintains algorithm output traces which are the
most useful in diagnosing failed tests.  The test harness output clearly
records the exact time and date that the source code is checked out which
allows one to check out exactly the same versions at a latter time to
reproduce the test results.  Given that we archive all of the test results
and a clear time stamp is given to each build, we can go back in time (months
in some cases) to examine the behavior of a test and can then checkout the
versions of the code for that build and should be able to more-or-less
reproduce the test output.  Of course, many issues make it almost impossible
to do this if too much time passes (e.g.\ the OS and other system tools on the
test machine may be updated, the Alegra TPLs (which are not rebuilt by the
Charon + Trilinos Dev test harness) may have changed, etc).

The notification e-mails sent out every night give the URL on the test machine
running an Apache server to the archive directory that contain the test
results.  Therefore, a few simple clicks of the mouse are all that is needed
to view the details of the test results from the night before (or for any
prior nightly test for that matter).  Mornings where there were new test
failures, I was able to use the archived test results on the web server to
diagnose most failed tests in less than 10 minutes.  In many cases, I was the
cause of the failed Charon ``research'' test due to a change in Trilinos that
I committed that was not verified against Charon before being checked
in\footnote{Note that later in the milestone, I never checked in code into
Trilinos that I did not first build and test Charon against.  This massively
cut down on the number of failed builds and tests.}.  In these cases, I was
responsible for addressing the failed tests.  In other cases, I sent off a
short e-mail to the person or persons that I suspected was the cause of the
problem and gave them some tips on what I suspected the problem might be given
my brief analysis\footnote{Adding some suggestion on what the problem might be
proved to be a very effective catalyst for getting people to address the
problems in a timely manner.}.

All in all, the nightly building and testing process that we set up for Charon
+ Trilinos Dev was very effective at keeping Charon + Trilinos Dev on track
and avoiding backslides in functionality that we were adding during the
milestone.  However, there were a few aspects of the process that caused some
unnecessary effort.  Some of these problems are described below.

One problem that we had was that we did not build and test Charon + Trilinos
7.0 (the current Trilinos release at the time) and Charon + Trilinos Dev on
the same platforms.  As a result, there were several instances where a new
``production'' test was created and checked into Charon that worked just fine
when run as part of Charon + Trilinos 7.0 on the 32 bit platform, where the
same test failed or diffed on the 64 bit platform where Charon + Trilinos Dev
was being built and tested.  As a result, I spent a fairly significant amount
diagnosing failing ``production'' tests that were unrelated to any changes in
Trilinos and unrelated to the milestone effort.  This has since been addressed
by adding nightly building and testing of Charon + Trilinos 7.0 on the same 64
bit machine where the rest of the Charon + Trilinos Dev tests are run (see the
Section {}\ref{sec:suggested-practices} about this practice).

Another problem involved the transition to the post-milestone period.  The end
of the milestone was marked by the release of Trilinos 8.0 which included new
milestone-related capabilities.  This would have been an ideal time to upgrade
Charon from Trilinos 7.0 to Trilinos 8.0.  Since all of the Charon
``production'' tests were already passing against Trilinos 8.0 which where run
every day , performing the upgrade would have been as simple as removing a few
{}\texttt{ifdefs} and removing the {}\texttt{tridev} keyword from all of the
``research'' tests that were ready to become ``production'' tests.  However,
Charon has a dependency on an internal circuit simulation code called Xyce
which was not being built and tested nightly against Trilinos Dev and
typically takes several months or longer to be upgraded to a new release of
Trilinos.  Therefore, we could not simply upgrade Charon to Trilinos 8.0 since
it would have broken the combined Charon + Xyce + Trilinos application.
Therefore, we decided to go ahead and put in place nightly testing of Charon
against both Trilinos 8.0 and Trilinos Dev.  To accomplish this, we had to add
a new {}\texttt{define} macro {}\texttt{CHARON\_TRI8} and a new test keyword
{}\texttt{tri8}, and we had to add new test builds for Charon + Trilinos 8.0
to the nightly test harness.  At the time of this writing, Xyce and Charon
still had not been upgraded to Trilinos 8.0 and therefore my personal computer
is still running the following five builds of Charon + Trilinos every night:

\begin{enumerate}
%
{}\item Charon + Trilinos Dev (opt, all tests, updated sources)
%
{}\item Charon + Trilinos Dev (dbg, only tridev tests, sources checked out
from scratch)
%
{}\item Charon + Trilinos 8.0 (opt, all tests, updated sources)
%
{}\item Charon + Trilinos 8.0 (dbg, only tri8 tests, updated sources)
%
{}\item Charon + Trilinos 7.0 (opt, all tests except tridev and tri8 tests,
static Charon TPLs (including Trilinos 7.0))
%
\end{enumerate}

Above, only the {}\texttt{tridev} and {}\texttt{tri8} tests were run in the
{}\texttt{dbg} builds since running all of the tests would have taken too long
to complete in a single night.  Even with this, the builds and tests took from
midnight to after 6:00 AM to fully complete.  Note that the existing Charon
test harness did not do nightly building or testing for a {}\texttt{dbg} build
and therefore the milestone test harness actually improves their testing.  By
performing all of these nightly builds, we will be guaranteed that when Charon
is upgraded to Trilinos 8.0 that this will go smoothly with few if any issues.

With all of the new milestone-related ``research'' tests constantly being
built and tested every night, these tests will be preserved such that when we
come back to Charon to advance our work, we will be assured that we have a
solid foundation to augment functionality.  As a result, Charon has become a
very attractive foundation for our future algorithmic research.  Without the
nightly building and testing in place, we could experience significant
problems getting our problems running again and the difficulties that we could
encounter would likely be enough to cause us to delay or abandon our efforts.
This was exactly what happened in our earlier efforts (for example, with
MPSalsa).


%
\subsection{Aria/SIERRA + Trilinos Dev}
%

To contrast our experience with Charon + Trilinos Dev, consider the auxiliary
effort where Aria/SIERRA was updated to build against Trilinos Dev and MOOCHO
[???] was interfaced to Aria to solve a prototype design problem.  This effort
was meant to demonstrate that the milestone work was more general than just
Charon and also served a number of other purposes as well.  However, while
Charon + Trilinos Dev was supported by nightly testing and Charon + Trilinos
Dev was (and still is) instantly available to anyone with SRN access,
Aria/SIERRA + Trilinos Dev was only periodically built by a single developer
and was not easily accessible to others.  The immediate impact of this
approach was that some amount of effort was required to get Aria/SIERRA +
Trilinos Dev to build again each time development paused for a time and then
was continued.  Also, the combined Aria/SIERRA + Trilinos Dev application was
not easily accessible to the MOOCHO expert on the milestone team so when
difficulties solving one of the problems surfaced, they where not be addresses
as well.

For course, the long term implication of not having Aria/SIERRA + Trilinos Dev
building and testing in place is that the developed capability could break
without anyone ever knowing it.  It is very likely that the code may not even
build by the time SIERRA upgrades to Trilinos 8.0.  Therefore, the developed
MOOCHO/Aria capability is fragile and is susceptible to being broken and may be
lost if too much time goes by\footnote{At the time of this writing, planning
is under way to establish nightly building and testing of Aria/SIERRA +
Trilinos Dev that will protect MOOCHO/Aria and other future algorithmic
developments.}.


%
\section{Conclusions}
%

There are a number of conclusions that we have drawn as a result of this
milestone work in relation to the proposed nightly building and testing of the
development versions of an application and Trilinos:

\begin{itemize}

{}\item Nightly building and testing of the development versions of the
application and Trilinos:

  \begin{itemize}

  {}\item results in better production capabilities and better research,

  {}\item brings algorithm developers and application developers closer
  together allowing for a better exchange of ideas and concerns,

  {}\item refocuses Trilinos developers on customer efforts,

  {}\item helps drive research-quality algorithm development, and
        
  {}\item reduces barriers for new algorithms to have impact on production
  applications.

  \end{itemize}

{}\item Other application projects and scientific support software projects
should consider adopting the type of continuous integration that is used with
Charon + Trilinos that was developed as part of the ASC Vertical Integration
Milestone work.

\end{itemize}

% ---------------------------------------------------------------------- %
% References
%
\clearpage
\bibliographystyle{plain}
\bibliography{references}
\addcontentsline{toc}{section}{References}

% ---------------------------------------------------------------------- %
% Appendices should be stand-alone for SAND reports. If there is only
% one appendix, put \setcounter{secnumdepth}{0} after \appendix
%
\appendix


%
\section*{APP + Trilinos Dev Checklist}
\label{sec:checkist}
%

$\Box$ Do you have {}\texttt{ifdef}s is in place in APP code that are needed
to build against Trilinos Dev and against the release (or multiple releases)
of Trilinos?

$\Box$ Have you separated ``production'' and ``research'' tests so that you
can better differentiate APP defects from Trilinos defects?

$\Box$ Have you appointed an official APP + Trilinos Representative to make
sure APP + Trilinos Dev is maintained and is responsible for making sure
issues are forwarded to the appropriate parties?

$\Box$ Have you set up a dedicated machine to do nightly building and testing
of APP + Trilinos Dev?

$\Box$ Have you provided easy access to APP and Trilinos developers to
immediately build a private venison of APP + Trilinos Dev to try out new
algorithmic capabilities?

$\Box$ Do you have appropriate processes in place to avoid unnecessary cross
communication between APP and Trilinos developers?

$\Box$ Do you fix failing builds of APP + Trilinos Dev right away, with no
exceptions?

$\Box$ Do you address failing ``production'' and ``research'' tests with an
urgency that is appropriate for the nature of APP and the APP + Trilinos
collaboration?

$\Box$ Do you archive test results long enough to allow developers to diagnose
failing tests?

$\Box$ After each major release of Trilinos, do you upgrade APP to the new
release in a timely way?

$\Box$ During the transitionary period between when Trilinos is branched for a
release and when the APP finally gets updated for the new Trilinos release, do
you build APP against the old Trilinos release, the current Trilinos release,
and Trilinos Dev?


%\begin{SANDdistribution}
%\end{SANDdistribution}

\end{document}
