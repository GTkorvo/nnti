\section{Introduction}

One area of steady improvement in large-scale engineering and
scientific applications is the increased modularity of application
design and development.  Specification of publicly-defined interfaces,
combined with the use of third-party software to satisfy critical
technology needs in areas such as mesh generation, data partitioning
and solution methods have been generally positive developments in
application design.  While the use of third party software introduces
dependencies from the application developer's perspective, it also
gives the application access to the latest technology in these areas,
amortizes library and tool development across multiple applications
and, if properly designed, gives the application easy access to more
than one option for each critical technology area, e.g., access to
multiple linear solver packages.

One category of modules that is becoming
increasingly important is abstract numerical algorithms (ANAs).  ANAs
such as linear and nonlinear equation solvers,
methods for stability and bifurcation analysis,
uncertainty quantification methods and nonlinear programming solvers
for optimization are typically mathematically 
sophisticated but have surprisingly little essential dependence on the
details of what computer system is being used or how matrices and
vectors are stored and computed.  Thus, by using abstract
interface capabilities in languages such as C++, we can implement ANA
software such that it will work, unchanged, with a variety of
applications and linear algebra libraries.  Such an approach is often
referred to as {\it generic programming}~\cite{ref:boost_generic_programming}.

In this paper we propose the use of a new stripped down
and enhanced version of the
Trilinos Solver Framework (TSF) called TSFCore as the common interface
for (i) ANA development, (ii) the integration of an ANA into an
application (APP) and (iii) providing services to the ANA from a
linear algebra library (LAL).  By agreeing on a simple minimal common
interface layer such as TSFCore, we eliminate the many-to-many
dependency problem of ANA/APP interfaces.  The goal of TSFCore is not
to replace the use of other types of more established linear algebra
interfaces such as TSF \cite{ref:TSF}, \textit{AbstractLinAlgPack}
(the linear algebra interfaces in MOOCHO \cite{ref:moochouserguide}),
or HCL \cite{ref:hcl} as the interfaces that are directly used in the
development of ANAs.  Instead, TSFCore is designed to make it easier
for developers to provide the basic functionality from APPs and LALs
required by these existing ANA-specific interfaces.

While TSFCore provides a mechanism to express all of the functionality
required to be directly used in ANA development it does not attempt to
provide a full collection of methods that directly support the
anticipated functionality needs of ANAs.  Instead TSFCore relies on a
simple but powerful reduction and transformation operator
mechanism~\cite{ref:rtop_toms} that can be used to express any
element-wise vector reduction or transformation operation.  More
direct and convenient access to functionality that might be desired by
a given ANA is provided by interfaces such as TSF and
\textit{AbstractLinAlgPack}.  This extended functionality can be very
helpful in developing ANA code.  Section
\ref{tsfcore:sec:convenience_functionality} discusses this topic in
detail.

It is difficult to describe a set of linear algebra interfaces outside
of the context of some class of numerical problems.  For this purpose,
we will consider numerical algorithms where it is possible to
implement all of the required operations exclusively through well
defined interfaces to vectors, vector spaces and linear operators.
Here we consider only the type of functionality such as is required in
the numerical solution of optimal control problems as described in
\cite{ref:opt_ctrl_itfc}.

We assume that the reader has a basic understanding of vector
reduction/transformation operators (RTOp) (see \cite{ref:rtop_toms}),
is comfortable with object-orientation \cite{ref:gama_et_al_1995} and
C++, and knows how to read basic Unified Modeling Language (UML)
\cite{ref:uml_distilled_2nd_ed} class diagrams.  We also assume
that the reader has some background in large-scale numerics and will
therefore be able to appreciate the challenges that are addressed by
TSFCore.

To motivate TSFCore, we discuss the context for TSFCore in large-scale
(both in lines of code and in problem dimensionality) numerical
software in Section \ref{tsfcore:sec:classification_of_lin_alg_itfc}.
The major requirements for TSFCore are spelled out in Section
\ref{tsfcore:sec:TSFCore_requirements}.  This is followed by an
overview of the TSFCore linear algebra interfaces in Section
\ref{tsfcore:sec:TSFCore_core_overview} and a detailed discussion
of the design of the TSFCore linear algebra interfaces in Section
\ref{tsfcore:sec:TSFCore_Details} including numerous examples.  A
complete example ANA for the iterative solution of simultaneous
systems of linear equations (using a simple BiCG method) is described
in Section \ref{tsfcore:sec:ANA_iter_solver_example}.  A discussion of
some of the object-oriented and other general software design concepts
and principles that have gone into the development of TSFCore is
deferred to Section \ref{tsfcore:sec:general_software_concepts}.  Some
of the nonessential but convenient functionality that is useful to
direct ANA developers that is missing in TSFCore is described in
Section \ref{tsfcore:sec:convenience_functionality}. Finally, a few
comments about making the most of TSFCore by developing adapters is
described in Section \ref{tsfcore:sec:adapters}.

%
\section{Classification of linear algebra interfaces}
\label{tsfcore:sec:classification_of_lin_alg_itfc}
%

Although we will discuss APPs, ANAs and LALs in detail later in this
section, we want to briefly introduce these terms here to make them
clear.  Also, although there are certainly other types of modules in a
large-scale application, we only focus on these three.
\begin{itemize}
\item Application (APP):  The modules of an application that are not
ANA or LAL modules.  Typically this includes the code that is unique
to the application itself such as the code that formulates and
generates the discrete problem.  In general it would also include
other third-party software that is not an ANA or LAL module.
\item Abstract Numerical Algorithm (ANA):  Software that drives a 
solution process, e.g., an iterative linear or nonlinear solver.  This
type of package provides solutions to and requires services from the
APP, and utilizes services from one or more LALs.  It can usually be
written so that it does not depend on the details of the computer
platform, or the details of how the APP and LALs are implemented, so
that an ANA can be used across many APPs and with many LALs.
\item Linear Algebra Library (LAL): Software that provides the 
ability to construct
concrete linear algebra objects such as matrices and vectors.  
A LAL can also be a specific linear solver or preconditioner.
\end{itemize}

An important focus of this paper is to clearly identify the interfaces
between APPs, ANAs and LALs for the purposes of defining the TSFCore
interface.

The requirements for the linear algebra objects as imposed by an ANA
are very different from the requirements imposed by an application
code.  In order to differentiate the various types of interfaces and
the requirements associated with each, consider Figure
\ref{tsfcore:fig:ANA_LAL_APP}.  This figure shows the three major
categories of software modules that make up a complete numerical
application.  The first category is application (APP) software in
which the underlying data is defined for the problem.  This could be
something as simple as the right-hand-side and matrix coefficients of
a single linear system or as complex as a finite-element method for a
3-D nonlinear PDE-constrained optimization problem.  The second
category is linear algebra library (LAL) software that implements
basic linear algebra operations \cite{ref:demmel_1997,
ref:anderson_1995, ref:blackford_et_al_1997, ref:aztec, ref:petsc,
ref:trilinos}. These types of software include primarily matrix-vector
multiplication, the creation of a preconditioner (e.g.~ILU), and may
even include several different types of linear solvers.  The third
category is ANA software that drives the main solution process and
includes such algorithms as iterative methods for linear and nonlinear
systems; explicit and implicit methods for ODEs and DAEs; and
nonlinear programming (NLP) solvers \cite{ref:nocedal_wright_1999}.
There are many example software packages
\cite{ref:petsc,ref:aztec,ref:trilinos,ref:pvode,ref:tao} that contain
ANA software.

{\bsinglespace
\begin{figure}[t]
\begin{center}
%\fbox{
\includegraphics*[bb= 0.245in 2.95in 10.85in 8.60in,angle=0,scale=0.50
]{analal}
%}
\end{center}
\caption{
\label{tsfcore:fig:ANA_LAL_APP}
UML \cite{ref:booch_et_al_1999} class diagram : Interfaces between abstract numerical algorithm
(ANA), linear algebra library (LAL), and application (APP) software.
}
\end{figure}
\esinglespace}

The types of ANAs described here only require operations like
matrix-vector multiplication, linear solves and certain types of
vector reduction and transformation operations.  All of these
operations can be performed with only a very abstract view of vectors,
vector spaces and linear operators.

An application code, however, has the responsibility of populating
vector and matrix objects and requires the passing of explicit
function and gradient value entries, sometimes in a distributed
parallel environment.  This is the purpose of a APP/LAL interface.
This involves a very different set of requirements than those
described above for the ANA/APP and ANA/LAL interfaces.  Examples of
APP/LAL interfaces include the FEI \cite{ref:fei} and much of the
current TSF.

Figure \ref{tsfcore:fig:ANA_LAL_APP} also shows a set of LAL/LAL
interfaces that allows linear algebra objects from one LAL to
collaborate with the objects from another LAL.  Theses interfaces are
very similar to the APP/LAL interfaces and the requirements for this
type of interface is also not addressed by TSFCore.  The ESI
\cite{ref:esi_2001} and much of the current TSF contain examples of
LAL/LAL interfaces.

TSFCore, as described in this paper, specifies only the
ANA/LAL interface.  TSFCore-based ANA/APP interfaces are
described elsewhere (e.g. \cite{ref:TSFCore::Nonlin}).

%
\section{TSFCore: Basic Requirements}
\label{tsfcore:sec:TSFCore_requirements}
%

Before describing the C++ interfaces for TSFCore, some basic
requirements are stated.

\begin{enumerate}

\item
TSFCore interfaces should be portable to all the ASCI
\cite{ref:doe_asci} platforms where SIERRA
\cite{ref:SIERRA} and other ASCI applications might run.  However, a
platform where C++ templates are fundamentally broken will not be a
supported platform for TSFCore.

\item
TSFCore interfaces should provide for stable and accurate numerical
computations at a fundamental level.

\item
TSFCore should provide a minimal, but complete, interface that
addresses all the basic efficiency needs (in both speed and storage)
which will result in near-optimal implementations of all of the linear
algebra objects and all of the above mentioned ANA algorithms that use
these objects.  All other types of nonessential but convenient
functionality (e.g.~Matlab-like syntax using operator overloading, see
Section \ref{tsfcore:sec:operator_overloading}) will not be addressed
by TSFCore.  This extra functionality can be built on top the basic
TSFCore abstractions (e.g.~using TSF).

\item
ANAs developed with TSFCore should be able to transparently utilize
different types of computing environments such as SPMD\footnote{Single
Program Multiple Data (SPMD): A single program running in a
distributed-memory environment on multiple parallel processors},
client/server\footnote{Client/Server: The ANA runs in a process on a
client computer and the APP and LAL run in processors on a server} and
out-of-core\footnote{Out-of-core: The data for the problem is stored
on disk and is read from and written to back disk as needed}
implementations.  A hand-coded program (e.g.~using Fortran and MPI)
should not provide any significant gains in performance in any of the
above categories in any computing environment.  This is critical for
the use of TSFCore in scientific computing.

\item
The work required to implement adapter subclasses (see the ``Adapter''
pattern in \cite{ref:gama_et_al_1995}) for and with TSFCore should be
minimal and straightforward for all of the existing related linear
algebra and ANA interfaces (e.g.~the linear algebra interfaces in
MOOCHO \cite{ref:moochouserguide} and NOX \cite{ref:nox}, see
Section \ref{tsfcore:sec:adapters}).  This requirement is facilitated
by the fact that the TSFCore interfaces are minimal.

\end{enumerate}

%
\section{TSFCore: Overview}
\label{tsfcore:sec:TSFCore_core_overview}
%

The basic linear algebra abstractions that make up TSFCore are shown in
Figure \ref{tsfcore:fig:tsfl_basic}.  Complete C++ class declarations
for these interfaces are given in Appendix \ref{app:tsfcore_classes}.
The key abstractions include
vectors, vector spaces and linear operators.  All of the interfaces
are templated on the \texttt{Scalar} type (the UML notation for
templated classes is not used in the figure for the sake of improving
readability).  A vector space is the foundation for all other linear
algebra abstractions.  Vector spaces are abstracted through the
\texttt{\textit{VectorSpace}} interface.  A
\texttt{\textit{VectorSpace}} object acts primarily as an ``abstract
factory'' \cite{ref:gama_et_al_1995} that creates vector objects
(which are the products in the ``abstract factory'' design pattern).
Vectors are abstracted through the \texttt{\textit{Vector}} interface.
The \texttt{\textit{Vector}} interface is very minimal and really only
defines one nontrivial method -- \texttt{\textit{applyOp(\-...)}}.  The
\texttt{\textit{applyOp(\-...)}} method accepts user-defined
(i.e.~ANA-defined) reduction/transformation operator (RTOp) objects
through the templated RTOp C++ interface
\texttt{\textit{RTOpPack::RTOpT}}.  A set of standard vector
operations is provided as nonmember functions using standard RTOp
subclasses (see Section \ref{tsfcore:sec:vector}).  The set of
operations is also easily extensible.  Every \texttt{\textit{Vector}}
object provides access to its \texttt{\textit{VectorSpace}} (that was
used to create the \texttt{\textit{Vector}} object) through the method
\texttt{space()} (shown in Figure \ref{tsfcore:fig:tsfl_basic} as the
role name \texttt{space} on the association connecting the
\texttt{\textit{Vector}} and \texttt{\textit{VectorSpace}} classes).
The \texttt{\textit{VectorSpace}} interface also provides the ability
to create \texttt{\textit{Multi\-Vector}} objects through the
\texttt{\textit{createMembers(numMembers)}} method.  A
\texttt{\textit{Multi\-Vector}} is a tall thin dense matrix where each
column in the matrix is a
\texttt{\textit{Vector}} object which is accessible through the
\texttt{\textit{col(...)}} method.  \texttt{\textit{Multi\-Vector}}s
are needed for near-optimal processor cache performance (in serial and
parallel programs) and to minimize the number of global communications
in a distributed parallel environment.  The
\texttt{\textit{Multi\-Vector}} interface is useful in many different
types ANAs as described later. \texttt{\textit{VectorSpace}} also
declares a virtual method called
\texttt{\textit{scalarProd(x,y)}} which computes the scalar product
$<x,y>$ for the vector space. This method has a default implementation
based on the dot product $x^T y$.  Subclasses can override the
\texttt{\textit{scalarProd(x,y)}} method for other, more specialized,
application-specific definitions of the scalar product. There is also
a \texttt{\textit{Multi\-Vector}} version
\texttt{\textit{VectorSpace\-::scalarProds(...)}} (not shown in the
figure).  Finally, \texttt{\textit{VectorSpace}} also includes the
ability to determine the compatibility of vectors from different
vector spaces through the method
\texttt{\textit{isCompatible(vecSpc)}} (see Section
\ref{tsfcore:sec:vec_spc_compatibility}).  The concepts behind the design
of the \texttt{\textit{VectorSpace}},
\texttt{\textit{Vector}} and
\texttt{\textit{Multi\-Vector}} interfaces are discussed later in Sections
\ref{tsfcore:sec:vec_space}, \ref{tsfcore:sec:vector} and \ref{tsfcore:sec:multi_vec}
respectively.

{\bsinglespace
\begin{figure}[t]
\begin{center}
%\fbox{
\includegraphics*[bb= 0.0in 0.0in 3.3in 4.4in,scale=0.40
]{UML1}
%}%fbox
%\fbox{
\includegraphics*[bb= 0.0in 0.0in 6.55in 4.6in,scale=0.65
]{TSFCore}
%}%fbox
\end{center}
\caption{
\label{tsfcore:fig:tsfl_basic}
UML class diagram : Major components of the TSF
interface to linear algebra
}
\end{figure}
\esinglespace}

Another important type of linear algebra abstraction is a linear
operator which is represented by the interface class
\texttt{\textit{LinearOp}}.  The \texttt{\textit{LinearOp}} interface
is used to represent quantities such as the Jacobian matrix
$\frac{\partial c}{\partial y}$. A \texttt{\textit{LinearOp}} object
defines a linear mapping from vectors in one vector space (called the
\texttt{domain}) to vectors in another vector space (called the
\texttt{range}).  Every \texttt{\textit{LinearOp}} object provides
access to these vector spaces through the methods \texttt{domain()} and
\texttt{range()} (shown as the role names \texttt{domain} and
\texttt{range} on the associations linking the
\texttt{\textit{OpBase}} and \texttt{\textit{VectorSpace}} classes).
The exact form of this mapping, as implemented by the
method \texttt{\textit{apply(\-...)}}, is
%
\begin{equation}
y = \alpha \, op(M) \, x + \beta y
\label{tsfcore:equ:apply_vec}
\end{equation}
%
where $M$ is a \texttt{\textit{LinearOp}} object; $x$ and $y$ are
\texttt{\textit{Vector}} objects; and $\alpha$ and $\beta$ are \texttt{Scalar}
objects.  Note that the linear operator in (\ref{tsfcore:equ:apply_vec})
is shown as $op(M)$ where $op(M) = M$ or $M^T$ (depending on the
argument \texttt{M\_trans}). This implies that both the non-transposed
and transposed (i.e.~adjoint) linear mappings can be performed.
However, support for transposed (adjoint) operations by a
\texttt{\textit{LinearOp}} object are only optional.  If an operation is
not supported then the method \texttt{\textit{opSupported(M\_trans)}}
will return \texttt{false} (see Section
\ref{tsfcore:sec:linear_op_adjoints}).  Note that when $op(M) = M^T$,
then $x$ and $y$ must lie in the \texttt{range} and
\texttt{domain} spaces respectively which is the opposite for the case
where $op(M) = M$.

In addition to implementing linear mappings for single
\texttt{\textit{Vector}} objects, the
\texttt{\textit{LinearOp}} interface also provides linear mappings of
\texttt{\textit{Multi\-Vector}} objects through an overloaded method
\texttt{\textit{apply(\-...)}} which performs
%
\begin{equation}
Y = \alpha \, op(M) \, X + \beta Y
\label{tsfcore:equ:apply_multi_vec}
\end{equation}
%
where $X$ and $Y$ are \texttt{\textit{Multi\-Vector}} objects.  The
\texttt{\textit{Multi\-Vector}} version of the \texttt{\textit{apply(\-...)}} method has a
default implementation based on the \texttt{\textit{Vector}} version.
The \texttt{\textit{Vector}} version
\texttt{\textit{apply(\-...)}}  is a pure virtual method and therefore must be
overridden by subclasses.  The issues associated with supporting the
\texttt{\textit{Multi\-Vector}} version verses the \texttt{\textit{Vector}}
version of this method are described in Section
\ref{tsfcore:sec:vector_vs_multivector}.

Section \ref{tsfcore:sec:TSFCore_Details} goes into much more detail behind
the design philosophy for the core interfaces and the use of these
interfaces by both clients and subclass developers.

%
\section{TSFCore: Details and Examples}
\label{tsfcore:sec:TSFCore_Details}
%

A basic overview of the interface classes shown in Figure
\ref{tsfcore:fig:tsfl_basic} was provided in Section
\ref{tsfcore:sec:TSFCore_core_overview}.  In the following sections, we go into
more detail about the design of these interfaces and give examples of
the use of these classes.  Note that in all the below code examples it
is assumed that the code is in a source file which include the
appropriate header files.

%
\subsection{A motivating example sub-ANA : Compact limited-memory BFGS}
\label{tsfcore:sec:LBFGS}
%

To motivate the following discussion and to provide examples, we
consider the issues involved in using TSFCore to implement an ANA for the
compact limited-memory BFGS (LBFGS) method described in
\cite{ref:byrd_et_all_lbfgs_1994}.  BFGS and other variable-metric
quasi-Newton methods are used to approximate a Hessian matrix $B$ of
second derivatives.  This approximation is then used to generate
search directions for various types of optimization algorithms.  The
Hessian matrix $B$ and/or its inverse $H = B^{-1}$ is approximated
using only changes in the gradient $y = \nabla f(x_{k+1}) - \nabla
f(x_k)$ of some multi-variable scalar function $f(x)$ for changes
in the variables $s = x_{k+1} - x_k$.  A set of matrix
approximations $B_k$ are formed using rank-2 updates where each update
takes the form
%
\begin{equation}
B_{k+1} = B_k - \frac{B_k s_k s_k^T B_k}{s_k^T B_k s_k} + \frac{y_k y_k^T}{y_k^T s_k}.
\end{equation}

In a limited-memory BFGS method, only a fixed maximum number
$m_{\tiny\mbox{max}}$ of updates are stored
%
\begin{eqnarray}
S & = & {\bmat{cccc} s_1 & s_2 & \ldots & s_{m} \emat} \label{tsfcore:eqn:LBSFGS_S}\\
Y & = & {\bmat{cccc} y_1 & y_2 & \ldots & y_{m} \emat} \label{tsfcore:eqn:LBSFGS_Y}
\end{eqnarray}
%
where $m \le m_{\tiny\mbox{max}}$ is the current number of stored
updates and $S$ and $Y$ are multi-vectors (note that the subscripts in
(\ref{tsfcore:eqn:LBSFGS_S})--(\ref{tsfcore:eqn:LBSFGS_Y}) correspond
to column indexes in the multi-vector objects, not iteration counters
$k$).  When an optimization algorithm begins, $m=0$ and $m$
incremented each iteration until $m = m_{\tiny\mbox{max}}$ after which
the method starts dropping older update pairs $(s,y)$ to make room for
newer ones.  In a compact LBFGS method, the inverse $H$ (shown in
Figure \ref{tsfcore:fig:LBFGS}) of the quasi-Newton matrix $B$ (where
when the index $k$ is dropped, it implicitly refers to the current
iteration $B_k$) on is approximated using the tall thin multi-vectors
$S$ and $Y$ along with a small (serial) coordinating matrix $Q$ (which
is computed and updated from $S$ and $Y$).  The scalar $g$ is chosen
for scaling reasons and $H_0 = B_0^{-1} = g I$ represents the initial
matrix approximation from which the updates are performed.  A similar
compact formula also exists for $B$ which involves the same matrices
(and requires solves with $Q$).  In an SPMD configuration, the
multi-vectors $Y$ and $S$ may contain vector elements spread over many
processors.  However, the number of columns $m$ in $S$ and $Y$ is
usually less than $40$.  Because of the small number of columns in $S$
and $Y$, all of the linear algebra performed with the matrix $Q$ is
performed serially using dense methods (i.e.~BLAS and LAPACK).  A
parallel version of the compact LBFGS method is implemented, for
example, as an option in MOOCHO.  TSFCore supports efficient versions
of all of the operations needed for a near-optimal parallel
implementation of this LBFGS method.

{\bsinglespace
\begin{figure}[t]
\begin{center}
%\fbox{
\includegraphics*[bb= 0.0in 0.0in 7.0in 3.7in,angle=0,scale=0.60
]{LBFGS}
%}%fbox
\end{center}
\caption{
\label{tsfcore:fig:LBFGS}
A compact limited-memory representation of the inverse of a BFGS matrix.
}
\end{figure}
\esinglespace}

The requirements for this sub-ANA will be mentioned in several of the
following sections along with example code.

%
\subsection{\texttt{\textit{VectorSpace}}}
\label{tsfcore:sec:vec_space}
%

The basic design of the \texttt{\textit{VectorSpace}} interface was
taken directly from HCL which is also used in TSF and
\textit{AbstractLinAlgPack}.

We now show a simple code example as to the use of the
\texttt{\textit{VectorSpace}} and
\texttt{\textit{Vector}} interfaces.  The following code snippet shows
a function that performs several types of tasks:

{\scriptsize\begin{verbatim}
temaplate<class Scalar>
void TSFCore::foo0( const VectorSpace<Scalar>& vecSpc, const LinearOp<Scalar>& M )
{
    namespace mmp = MemMngPack;
    THROW_EXCEPTION(!vecSpc.isCompatible(*M.domain()),std::logic_error,"Error!"); // Check compatibility
    mmp::ref_count_ptr<Vector<Scalar> > x = vecSpc.createMember();                // Create new vector x
    mmp::ref_count_ptr<Vector<Scalar> > y = M.range()->createMember();            // Create new vector y
    assign(x.get(),1.0);                                                          // x = 1.0
    M.apply(NOTRANS,*x,y.get());                                                  // y = M*x
    M.apply(TRANS,*y,x.get(),0.5,0.1);                                            // x = 0.5*M*y + 0.1*x
}
\end{verbatim}}

\noindent The above code snippet shows how memory management in TSFCore is
handled -- through the templated smart reference-counted pointer class
\texttt{MemMngPack\-::ref\_count\_ptr<>} (see Section
\ref{tsfcore:sec:general_software_concepts}).  The vector objects pointed
to by the objects \texttt{x} and \texttt{y} are accessed in various
ways in the last three lines.  For instance, in the statement

{\scriptsize\begin{verbatim}
    assign(x.get(),1.0);
\end{verbatim}}

\noindent
the raw C++ pointer (of type \texttt{Vector<Scalar>*}) to the
underlying vector object is returned using the method
\texttt{ref\_count\_ptr<>\-::get()}.  The function
\texttt{assign(...)} is implemented through an RTOp object and its
implementation is shown in Section \ref{tsfcore:sec:vec_apply_op}.
The next statement

{\scriptsize\begin{verbatim}
    M.apply(NOTRANS,*x,y.get());
\end{verbatim}}

\noindent shows the created vectors being passed into the \texttt{apply(\-...)}
method of a \texttt{\textit{LinearOp}} object.  The expression
\texttt{*x} invokes the method
\texttt{ref\_count\_ptr<>\-::operator*()} which returns a reference
(of type \texttt{Vector<Scalar>\&}) to the underlying vector object.

%
\subsubsection{General compatibility of \texttt{\textit{Vector}} objects}
\label{tsfcore:sec:vec_spc_compatibility}
%

There is one important aspect that distinguishes
\texttt{TSFCore\-::\textit{VectorSpace}} from vector space interfaces in
HCL and TSF for instance.  In HCL 1.0, the compatibility of vector
spaces is tested with a virtual \texttt{operator==(...)}  method.
This implies that vector spaces will be compatible only if they are of
the same concrete type and have the same setup.  Ideally, however, we
do not want to require that only vectors and vector spaces with the
same {\em concrete} type to be compatible but instead we would like to
allow vectors and vector spaces of the same {\em general} type be
compatible.  To see the difference, consider parallel programs running
in an SPMD configuration where vector elements are partitioned across
processors and communication is handled using MPI
\cite{ref:mpi}.  There are several different linear algebra libraries
that are designed to work in such an environment such as Aztec
\cite{ref:aztec}, Epetra \cite{ref:Epetra} and PETSc \cite{ref:petsc}.  TSFCore
adapter subclasses would be created for vectors and vector spaces for
each of these packages.  In principle, all implementations of SPMD MPI
vectors that have the same partitioning of elements to processors
should be compatible, regardless of which underlying libraries
are involved.  The RTOp design, given the appropriate
\texttt{\textit{VectorSpace}} and \texttt{\textit{Vector}} interfaces,
allows the seamless integration of vectors of different {\em concrete}
types given the same {\em general} type.  If all of these adapter
subclasses inherited from the node interface classes
\texttt{\textit{MPIVectorSpaceBase}} and
\texttt{\textit{MPIVectorBase}} (see the Doxygen documentation) which
include an appropriate set of abstract methods (like determining
compatibility of maps and access to local vector data), then Epetra
vectors should be transparently compatible with PETSc and Aztec
vectors and so on.  This type of interoperability is demonstrated for
serial vectors and vector spaces in Section
\ref{tsfcore:sec:serial_vecs}

%
\subsection{\texttt{\textit{Vector}}}
\label{tsfcore:sec:vector}
%

The core design principles behind the \texttt{\textit{Vector}}
interface and the \texttt{\textit{applyOp(\-...)}} method (which accepts
RTOp objects) are described in \cite{ref:rtop_toms}.  The benefits of
the RTOp approach can be summarized as follows.

\begin{enumerate}
\item LAL developers need only implement one operation ---
\textit{\texttt{applyOp(\-...)}} --- and not a large collection of
primitive vector operations.  
\item ANA developers can implement
\textit{specialized} vector operations without needing any support
from LAL maintainers.  
\item ANA developers can optimize time
consuming vector operations on their own for the platforms they work
with.  
\item Reduction/transformation operators are more efficient
than using primitive operations and temporary vectors.  
\item ANA-appropriate vector
interfaces that desire built-in standard vector operations (i.e.~axpy
and norms) can use RTOp operators for the default implementations of
these operations (see \textit{AbstractLinAlgPack\-::\texttt{Vector}}).
\end{enumerate}

{\bsinglespace
\begin{figure}[t]
\begin{minipage}{\textwidth}
{\scriptsize\begin{verbatim}
----------------------------------------------------------------------------------------------------
// TSFCoreVectorStdOpsDecl.hpp
...
namespace TSFCore {
template<class Scalar> Scalar sum( const Vector<Scalar>& v );                   // result = sum(v(i))
template<class Scalar> Scalar norm_1( const Vector<Scalar>& v );                // result = ||v||1
template<class Scalar> Scalar norm_2( const Vector<Scalar>& v );                // result = ||v||2
template<class Scalar> Scalar norm_inf( const Vector<Scalar>& v_rhs );          // result = ||v||inf
template<class Scalar> Scalar dot( const Vector<Scalar>& x
                                   ,const Vector<Scalar>& y );                  // result = x'*y
template<class Scalar> Scalar get_ele( const Vector<Scalar>& v, Index i );      // result = v(i)
template<class Scalar> void set_ele( Index i, Scalar alpha
                                     ,Vector<Scalar>* v );                      // v(i) = alpha
template<class Scalar> void assign( Vector<Scalar>* y, const Scalar& alpha );   // y = alpha
template<class Scalar> void assign( Vector<Scalar>* y
                                    ,const Vector<Scalar>& x );                 // y = x
template<class Scalar> void Vp_S( Vector<Scalar>* y, const Scalar& alpha );     // y += alpha
template<class Scalar> void Vt_S( Vector<Scalar>* y, const Scalar& alpha );     // y *= alpha
template<class Scalar> void Vp_StV( Vector<Scalar>* y, const Scalar& alpha
                                    ,const Vector<Scalar>& x );                 // y = alpha*x + y
template<class Scalar> void ele_wise_prod( const Scalar& alpha
    ,const Vector<Scalar>& x, const Vector<Scalar>& v, Vector<Scalar>* y );     // y(i)+=alpha*x(i)*v(i)
template<class Scalar> void ele_wise_divide( const Scalar& alpha
    ,const Vector<Scalar>& x, const Vector<Scalar>& v, Vector<Scalar>* y );     // y(i)=alpha*x(i)/v(i)
template<class Scalar> void seed_randomize( unsigned int );                     // Seed for randomize()
template<class Scalar> void randomize( Scalar l, Scalar u, Vector<Scalar>* v ); // v(i) = random(l,u)
} // end namespace TSFCore
----------------------------------------------------------------------------------------------------
\end{verbatim}}
\end{minipage}
\caption{
\label{tsfcore:fig:std_vec_ops}
Some standard vector operations declared in the header file \texttt{TSFCore\-Vector\-Std\-Ops\-Decl.hpp}
and defined in the \texttt{TSFCore\-Vector\-Std\-Ops.hpp} header file.
}
\end{figure}
\esinglespace}

The \texttt{\textit{applyOp(\-...)}}  method is described in more detail
in Section \ref{tsfcore:sec:vec_apply_op}.  Note that this approach
does not hinder the development of convenience functions in any way.
In fact, a set of basic operations is already available in the header
file \texttt{TSFCore\-Vector\-Std\-Ops\-Decl.hpp}.  The declarations for the
functions in this file are shown in Figure
\ref{tsfcore:fig:std_vec_ops}.  Note, to use these template functions
you should include the definitions from
\texttt{TSFCore\-Vector\-Std\-Ops.hpp} (never directly \texttt{\#include} a
\texttt{xxxDecl.hpp} file unless you know what you
are doing, instead, include the \texttt{xxx.hpp} file for all of the
TSFCore code).  Using one of these non-member vector functions is
transparently obvious and there is not even one hint that the method
\texttt{\textit{Vector::applyOp(\-...)}} is involved.

%
\subsubsection{\texttt{\textit{Vector::applyOp(\-...)}}}
\label{tsfcore:sec:vec_apply_op}
%

Several important issues regarding the specification of the
\texttt{\textit{Vector::applyOp(\-...)}} method were not discussed in
\cite{ref:rtop_toms}.  Before describing these issues, note that the
\texttt{\textit{Vector\-::applyOp(\-...)}} method is not directly called
by a client (it is protected) but instead is called through a
non-member (friend) function of the same name.  This is done to
provide a uniform way to deal with all of the allowed permutations of
the number and types of vector arguments to this function when the
function is called by the client.  Therefore, we will only consider
the prototype for the non-member function
\texttt{TSFCore::appyOp(...)}  which is

{\scriptsize\begin{verbatim}
template<class Scalar>
void TSFCore::applyOp(
    const RTOpPack::RTOpT<Scalar> &op
    ,const size_t num_vecs, const Vector<Scalar>* vecs[]
    ,const size_t num_targ_vecs , Vector<Scalar>* targ_vecs[]
    ,RTOp_ReductTarget reduct_obj
    ,const Index first_ele = 1, const Index sub_dim = 0, const Index global_offset = 0
    );
\end{verbatim}}

\noindent and has nine arguments: the RTOp object that defines the
reduction/transformation operation to be performed \texttt{op}; the
non-mutable input vectors specified by \texttt{num\_vecs} and
\texttt{vecs[]} (\texttt{num\_vecs==0} and \texttt{vecs==NULL}
allowed); the mutable input/output vectors specified by
\texttt{num\_targ\_vecs} and \texttt{targ\-\_vecs[]}
(\texttt{num\_targ\_vecs==0} and \texttt{targ\_vecs==NULL} allowed);
the input/output opaque reduction target object \texttt{reduct\_obj}
(must be set to the value \texttt{RTOp\_REDUCT\_OBJ\_NULL} if no
reduction is defined); the range of elements defining the sub-vector
to apply the operator to specified by \texttt{first\_ele} and
\texttt{sub\_dim}; and the global offset \texttt{global\_offset} to
use when applying coordinate-variant operators.

The role of the first five arguments in \texttt{TSFCore::applyOp(\-...)}
should be clear from the discussion in \cite{ref:rtop_toms}.  However,
the special handling of the object \texttt{reduct\_obj} and the use
cases where the last three arguments are important need to be
carefully explained since they are critical to the success of this
design.  In short, what this specification allows is the ability to
take \texttt{\textit{Vector}} objects and then be able to put together
abstract compositions of them to create new (logical) vector
\texttt{\textit{Vector}} objects.  There are primarily four use cases
that this specification is designed to support: (a) treating all of
the elements in a \texttt{\textit{Vector}} object a a single logical
vector, (b) targeting an RTOp operator to a specific element or range
of elements, (c) creating a sub-view of an existing vector and
treating it as a vector in its own right, and (d) creating a new,
larger composite (i.e.~block, or product) abstract vector out of a
collection of other vector objects.

The first use case (a), where all of the elements in a
\texttt{\textit{Vector}} object are treated as a single logical
vector, is the most common one.  Here, the default argument values of
\texttt{first\_ele=1}, \texttt{sub\_dim=0} (the value \texttt{0} is a
flag to indicate that all of the remaining elements should be
included) and \texttt{global\_offset=0} are used and
\texttt{TSFCore::applyOp(\-...)} is called with the vector arguments.
For example, consider the invocation of an assignment-to-scalar
transformation operator in the following function.

{\scriptsize\begin{verbatim}
template<class Scalar>
void TSFCore::assign( Vector<Scalar>* y, const Scalar& alpha )
{
    THROW_EXCEPTION(y==NULL,std::logic_error,"assign(...), Error!");           // Validate input
    RTOpPack::TOpAssignScalar<Scalar> assign_scalar_op;                        // Create the operator
    Vector<Scalar>* targ_vecs[] = { y };                                       // Set up vector args
    applyOp<Scalar>(assign_scalar_op,0,NULL,1,targ_vecs,RTOp_REDUCT_OBJ_NULL); // Invoke operator
}
\end{verbatim}}

\noindent In the above function, the operator \texttt{assign\_scalar\_op} of type
\texttt{RTOpPack::RTOpAssignScalar} only performs a transformation which
does not require a reduction object.  In these cases the special value
of \texttt{RTOp\_REDUCT\_OBJ\_NULL} must be passed in for the
opaque reduction object \texttt{reduct\_obj}.

If a reduction is being performed, the reduction object is initialized
prior to a single call to \texttt{TSFCore::applyOp(\-...)} and then the
reduction value is extracted.  The following function shows an example
where the norm $||.||_2$ is computed

{\scriptsize\begin{verbatim}
template<class Scalar>
Scalar TSFCore::norm_2( const Vector<Scalar>& v )
{
    RTOpPack::ROpNorm2<Scalar>        norm_2_op;                     // Create the RTOp operator object
    RTOpPack::ReductTargetT<Scalar>   norm_2_targ(norm_2_op);        // Create (init) reduction object
    const Vector<Scalar>* vecs[] = { &v };                           // Set up non-mutable vector args
    applyOp<Scalar>(norm_2_op,1,vecs,0,NULL,norm_2_targ.obj();       // Invoke the reduction operator
    return norm_2_op(norm_2_targ);                                   // Extract reduction value
}
\end{verbatim}}

\noindent A great many implementations of \texttt{RTOp} operator subclasses are
already available and wrapper functions to several of the more
standard operations, including the above functions
\texttt{assign( y, alpha )} and \texttt{norm\_2(v)},
are defined in the header file \texttt{TSFCore\-Vector\-Std\-Ops.hpp}
shown in Figure \ref{tsfcore:fig:std_vec_ops}.

The second use case (b) is where the client targets an RTOp operator
for a specific element or set of elements in a \texttt{Vector} object.
Two important examples are getting and setting individual vector
elements.  This can be accomplished without having to write specialized
RTOp subclasses for these cases.  For example, getting an element
can be performed using a standard RTOp subclass as is done
in the following function.

{\scriptsize\begin{verbatim}
template<class Scalar>
Scalar TSFCore::get_ele( const Vector<Scalar>& v, Index i )
{
    RTOpPack::ROpSum<Scalar>          sum_op;                 // Create RTOp operator object
    RTOpPack::ReductTargetT<Scalar>   sum_targ(sum_op);       // Create (init) reduction object
    const Vector<Scalar>* vecs[1] = { &v };                   // Set up non-mutable vector args
    applyOp<Scalar>(sum_op,1,vecs,0,NULL,sum_targ.obj(),i,1); // Invoke the reduction operator
    return sum_opt(sum_targ);                                 // Extract reduction value
}
\end{verbatim}}

\noindent In the above call to \texttt{TSFCore::applyOp(\-...)}, the argument
\texttt{global\_offset} is left at its default value of \texttt{0},
since this argument is ignored by the RTOp object \texttt{sum\_op}
anyway (the sum operator is coordinate invariant).

Setting a vector element is performed in a similar manner using the
same transformation RTOp operator subclass for assigning the elements
of a vector that was used in the \texttt{assign(...)} function shown
above.  The following function shows how setting a vector element
is performed using this transformation operator.

{\scriptsize\begin{verbatim}
template<class Scalar>
void TSFCore::set_ele( Index i, Scalar alpha, Vector<Scalar>* v )
{
    THROW_EXCEPTION(v==NULL,std::logic_error,"set_ele(...), Error!");               // Validate input
    RTOpPack::TOpAssignScalar<Scalar> assign_scalar_op;                             // Create op object
    Vector<Scalar>* targ_vecs[1] = { v };                                           // Set up vector args
    applyOp<Scalar>(assign_scalar_op,0,NULL,1,targ_vecs,RTOp_REDUCT_OBJ_NULL,i,1);  // Invoke operator
}
\end{verbatim}}

\noindent Again, since the assignment operator is also coordinate invariant, the
\texttt{assign\_scalar\_op} object ignores the \texttt{global\_offset}
argument so \texttt{global\_offset} is left at its default value in
the call to \texttt{TSFCore::applyOp(\-...)}.

For an example of the third use case (c), where a sub-view of an
existing vector is treating as a vector in its own right, consider an
optimization algorithm where the state $y$ and design $u$ variables
are physically concatenated into a single serial vector $x^T =
{\bmat{cc} y^T & u^T \emat}$.  For example, if $n_y = 10$ and $n_u =
5$, then the dimension of the vector $x$ would be $n_x = 15$.  There
are parts of the algorithm where it is most convenient to treat all of
the variables $x$ the same and there are others where access to the
individual state $y$ and design $u$ sub-vectors of $x$ is required.
Now suppose that a \texttt{\textit{Vector}} object \texttt{x} is
directly used by an optimization algorithm.  When the optimization
algorithm needs to apply an RTOp operator to the state variables $y$, it
sets \texttt{first\_ele=1} and \texttt{sub\_dim=10} and then calls
\texttt{TSFCore::applyOp(\-...)} (leaving the default value of
\texttt{global\_offset=0}).  When the algorithm needs to apply an
RTOp operator to the design variables $u$, it sets
\texttt{first\_ele=11} and \texttt{sub\_dim=5} and then calls
\texttt{TSFCore::applyOp(\-...)} (also leaving the default value of
\texttt{global\_offset=0}).  In each case, if a reduction is being performed,
the reduction object is initialized prior to a single call to
\texttt{TSFCore::applyOp(\-...)} and then the reduction value is extracted
just as in the first use case (a).  For example, the following
function computes the $||.||_2$ norms for the state and design
sub-vectors given the vector object \texttt{x}.

{\scriptsize\begin{verbatim}
template<class Scalar>
void TSFCore::compute_norm_2( const Vector<Scalar>& x, Index ny, Scalar* nrm_2_y, Scalar* nrm_2_u )
{
    const Index  nx = x.space()->dim(), nu = nx - ny;                   // Get dimensions
    RTOpPack::ROpNorm2<Scalar>        norm_2_op;                        // Create op object
    RTOpPack::ReductTargetT<Scalar>   norm_2_targ(norm_2_op);           // Create (init) reduction object
    const Vector<Scalar>* vecs[1] = { &x };                             // Set up non-mutable vector args
    applyOp<Scalar>(norm_2_op,1,vecs,0,NULL,norm_2_targ.obj(),1,ny);    // Invoke the operator for y
    *nrm_2_y = norm_2_op(norm_2_targ);                                  // Extract the value of ||y||2
    norm_2_targ.reinit()                                                // Reinitialize reduction object
    applyOp<Scalar>(norm_2_op,1,vecs,0,NULL,norm_2_targ.obj(),ny+1,ny+nu);// Invoke operator for u
    *nrm_2_u = norm_2_op(norm_2_targ);                                  // Extract the value of ||u||2
}
\end{verbatim}}

\noindent Finally, as an example of the fourth use case (d), where a new larger
composite (i.e.~block) abstract vector is created out of a collection
of other abstract vectors, we use the same optimization example as
above, except this time the vector $x$ is actually represented as two
separate \texttt{\textit{Vector}} objects \texttt{y} and \texttt{u}.
In this case, a new composite blocked or product vector
%
\[
x = {\bmat{c} y \\ u \emat}
\]
%
is abstractly created which lies in a new product vector space
$\mathcal{X} = \mathcal{Y} \times \mathcal{U}$.  With that said,
consider how the element with the maximum absolute value and its index
can be determined for the full vector $x$ given separate
\texttt{Vector} objects for the state $y$ and design $u$ variables.
This can be done with the predefined RTOp subclass
\texttt{ROpMax\-AbsEle} which is applied in the following
function.

{\scriptsize\begin{verbatim}
template<class Scalar>
void TSFCore::compute_max_abs_ele( const Vector<Scalar>& y, const Vector<Scalar>& u
    ,Scalar* x_max, Index* x_i )
{
    const Index ny = y.space()->dim(), nu = u.space()->dim();           // Get dimensions
    RTOpPack::ROpMaxAbsEle<Scalar>    max_abs_ele_op;                   // Create op object
    RTOpPack::ReductTargetT<Scalar>   max_abs_ele_targ(max_abe_ele_op); // Create (init) reduct object
    const Vector<Scalar>* vecs[1];                                      // Declare array
    vecs[0] = &y;                                                       // Set pointer to y
    applyOp<Scalar>(max_abs_ele_op,1,vecs,0,NULL,max_abs_ele_targ.obj(),1,0,0);// Reduce over y
    vecs[0] = &u;                                                       // Set pointer to u
    applyOp(max_abs_ele_op,1,vecs,0,NULL,max_abs_ele_targ.obj(),1,0,ny);// Combine with reduction over u
    *x_max = max_abs_ele_op(max_abse_ele_targ).x_max();                 // Extract reduction values
    *x_i   = max_abs_ele_op(max_abse_ele_targ).x_i();                   // ...
}
\end{verbatim}}

\noindent The above reduction operation is not coordinate invariant and
therefore the value of \texttt{global\_offset} is critical in the
calls to \texttt{TSFCore\-::applyOp(\-...)}.

Note that optimization algorithms are not the only ANAs that require
the (logical) composition of individual \texttt{\textit{Vector}}
objects into a single vector.  For example, SFE methods form a large
blocked SFE system out of several smaller deterministic systems
\cite{ref:sfe}.  There can also be multiple levels of blocking such as
embedding a blocked SFE set of state vectors $y^T = \bmat{cccc}
\tilde{y}_1^T & \tilde{y}_2^T & \ldots & \tilde{y}_N^T \emat$ into the blocked set of
optimization variables $x^T = \bmat{cc} y^T & u^T \emat$.  The basic
functionality in \texttt{\textit{Vector\-::applyOp(\-...)}} supports
all of these examples through the above use cases.

%
\subsubsection{Explicit access to \texttt{\textit{Vector}} elements}
\label{tsfcore:sec:explicit_vec_access}
%

Another important feature of the \texttt{\textit{Vector}} interface
regards the methods that can be used to gain explicit access to the
vector elements (which are not shown in the UML diagram in Figure
\ref{tsfcore:fig:tsfl_basic}) .  First, it should be noted that requesting
explicit access to vector elements is ill-advised in general
(especially in an SPMD or client-server environment).  However, there
are instances where this is perfectly appropriate.  One example is
when one needs to access elements for vectors in the domain space of a
\texttt{\textit{Multi\-Vector}} object.  This, for example, is needed in the
implementation of the compact LBFGS method described in Section
\ref{tsfcore:sec:LBFGS} above.  For the implementation of this compact
LBFGS matrix, it is critical to be able to explicitly access elements
in the domain space of $Y$ and $S$ in order to compute and update the
coordinating matrix $Q$.  Another situation when explicit access to
vector elements is appropriate and needed is when the vector is in a
small dimensional design space in an optimization problem and where
the ANA uses dense quasi-Newton methods to approximate the reduced
Hessian of the Lagrangian (e.g.~this is one option in MOOCHO).

The methods in \texttt{\textit{Vector}} support three different types
of use cases with respect to explicit element access: (a) extracting a
non-mutable view of the vector elements; (b) extracting a mutable view
of the vector elements and then committing the changes back to the
vector object; and finally, (c) explicitly setting the elements in the
vector.  The prototypes for these methods are shown below.

{\scriptsize\begin{verbatim}
namespace TSFCore {
teamplate<class Scalar>
class Vector {
public:
    ...
    virtual bool isInCore() const;
    virtual void getSubVector( const Range1D& rng, RTOpPack::SubVectorT<Scalar>* sub_vec ) const;
    virtual void freeSubVector( RTOpPack::SubVectorT<Scalar>* sub_vec ) const;
    virtual void getSubVector( const Range1D& rng, RTOpPack::MutableSubVectorT<Scalar>* sub_vec );
    virtual void commitSubVector( RTOpPack::MutableSubVectorT<Scalar>* sub_vec );
    virtual void setSubVector( const RTOpPack::SparseSubVectorT<Scalar>& sub_vec );
    ...
};
} // namespace TSFCore
\end{verbatim}}

\noindent All of these methods have reasonably efficient default implementations
based on fairly sophisticated RTOp subclasses and
\texttt{\textit{Vector::applyOp(\-...)}}.  The default implementations of the
\texttt{\textit{getSubVector(...)}} methods require dynamic memory allocation.
For most use cases, \texttt{\textit{Vector}} subclasses usually do not
need to override these methods for the sake of efficiency but may need
to override them for other reasons (see the subclass
\texttt{SerialVector} in Section \ref{tsfcore:sec:serial_vecs} and
the interface \texttt{\textit{MPI\-Vector\-Base}} in the Doxygen
documentation).  The method \texttt{\textit{isInCore()}} returns true
if all of the vector's elements are easily accessible is all of the
calling processes and therefore these explicit vector access methods
are an efficient way to get at the explicit elements.  This method
should not generally be called by typical client code but instead is
designed to be used by more specialized types of purposes (e.g.~see the
class \texttt{\textit{MPI\-Vector\-Space\-Base}} in the Doxygen
documentation).

In the first use case (a), extracting and releasing a non-mutable view of the
vector elements involves calling the \texttt{const} methods
\texttt{getSubVector(...)} and
\texttt{freeSubVector(...)} respectively.  These methods use the C++ class
\texttt{RTOpPack::\-SubVectorT<>} that is build into the C++ interfaces for RTOp
and was therefore a natural choice for this purpose.  To demonstrate
the use of these methods the following example function copies the
elements from a \texttt{\textit{Vector}} object into a raw C++ array.

{\scriptsize\begin{verbatim}
teamplate<class Scalar>
void foo1( const Vector<Scalar>& x, Scalar v[] )
{
    RTOpPack::SubVectorT<Scalar> sub_vec;             // Create (int) subvector view object
    x.getSubVector(Range1D(),&sub_vec);               // Initialize the view object
    for( Index i = 0; i < sub_vec.subDim(); ++i )     // Loop through the explicit elements
        v[i] = sub_vec(i+1);                          //     Extract values
    x.freeSubVector(&sub_vec);                        // Free the view of the vector x
}
\end{verbatim}}

\noindent In the statement

{\scriptsize\begin{verbatim}
    x.getSubVector(Range1D(),&sub_vec);
\end{verbatim}}

\noindent the constructed \texttt{Range1D()} object represents the full range of
vector elements (this is similar to the colon '\texttt{:}' syntax
in Matlab).  Note that this method call may require dynamic memory
allocation in order to create a strided view of the vector elements
that is represented in the output argument \texttt{sub\_vec}.  The
data pointed to by \texttt{sub\_vec.values} may be dynamically
allocated which is why it is necessary to call

{\scriptsize\begin{verbatim}
    x.freeSubVector(&sub_vec);
\end{verbatim}}

\noindent after the view in \texttt{sub\_vec} is no longer needed in order to possibly
free dynamically allocated memory.

The process of extracting, modifying and committing a mutable view of
vector elements, in the second use case (b), involves the
non-\texttt{const} methods \texttt{getSubVector(...)} and
\texttt{commit\-Sub\-Vector(...)} respectively.  These methods use the
RTOp C++ class \texttt{RTOpPack::\-Mutable\-Sub\-VectorT<>}.  As an
example, consider the following function that accepts a raw C++ array
of values and then adds them to a \texttt{\textit{Vector}} object's
elements.

{\scriptsize\begin{verbatim}
template<class Scalar>
void foo2( const Scalar v[], Vector<Scalar>* x )
{
    RTOpPack::MutableSubVectorT<Scalar> sub_vec;      // Create (init) subvector view object
    x->getSubVector(Range1D(),&sub_vec);              // Initialize the view object
    for( Index i = 0; i < sub_vec.subDim(); ++i )     // Loop through the explict elements
        sub_vec(i+1) += v[i];                         //      add v[] to elements
    x->commitSubVector(&sub_vec);                     // Commit and free the view of x
}
\end{verbatim}}

The last use case (c) is where a client simply wants to set elements
without creating a view.  This is accomplished through the
non-\texttt{const} method \texttt{set\-Sub\-Vector(...)}.  This method
uses yet another built-in RTOp C++ class called
\texttt{RTOpPack::\-Sparse\-Sub\-VectorT<>}.  This class is different from the
\texttt{RTOpPack::\-SubVectorT<>} and
\texttt{RTOpPack::\-Mutable\-Sub\-VectorT<>} classes in that
\texttt{RTOpPack::\-Sparse\-Sub\-VectorT<>} also allows the representation of 
sparse vectors.  This is very useful for quickly and efficiently
setting up sparse \texttt{\textit{Vector}} objects.  For example, one
way to initialize a \texttt{\textit{Vector}} object to represent a
column of identity (i.e.~an ``eta'' vector $e_i$) is to use a function
like the following.

{\scriptsize\begin{verbatim}
template<class Scalar>
void set_eta_vec( Index i, Vector<Scalar>* e_i )
{
    const Scalar av[] = { 1.0 };                      // Create array for the values
    const Index  ai[] = { i   };                      // Create array for the indexes
    RTOpPack::SparseSubVectorT<Scalar> sub_vec(       // Initialize sub_vec with sparse ele arrays
        0,e_i->dim(),1,av,1,ai,1,0,1);                // ...
    x->setSubVector(sub_vec);                         // Set all x = 0 except x(i) = 1.0
}
\end{verbatim}}

%
\subsubsection{Serial vectors and vector spaces}
\label{tsfcore:sec:serial_vecs}
%

One of the remarkable features of the design of the
\texttt{\textit{VectorSpace}} and \texttt{\textit{Vector}} interfaces
is that they allow, in principle, for all serial vectors of the same
dimension to be automatically compatible with little work.  Here we
use the term serial to mean that all of the vector elements are stored
in core in the same process where the ANA is running.  While this may
not sound remarkable at first thought consider the fact that there
exist numerous C++ classes libraries that contain some concept of a
serial vector \cite{ref:lumsdaine_and_siek_1998, ref:tnt,
ref:roberts_et_al_1996, ref:math++_1996} which are all largely
incompatible (except perhaps through explicit element access using
\texttt{operator[]} or \texttt{operator()} but certainty only
through compile time polymorphism (i.e.~C++ templates)).  With
TSFCore, these incompatibilities are not an issue.  The way that this
works is exemplified by the subclasses \texttt{SerialVectorSpace} and
\texttt{SerialVector} which are derived from the node subclasses
\texttt{Serial\-VectorSpace\-Base} and
\texttt{SerialVectorBase} respectively.

The first step is for every serial \texttt{\textit{VectorSpace}}
subclass to implement the \texttt{\textit{isCompatible(\-...)}}  method
in the same way as shown below (using \texttt{SerialVectorSpaceBase}
as the example).

{\scriptsize\begin{verbatim}
template<class Scalar>
bool SerialVectorSpaceBase<Scalar>::isCompatible( const VectorSpace<Scalar>& aVecSpc ) const
{
    return this->dim() == aVecSpc.dim() && this->isInCore() && aVecSpc.isInCore();
}
\end{verbatim}}

\noindent The above implementation makes the assumption that if the dimensions
of the vector spaces are the same and both vectors are stored in core,
then the vectors themselves should also be compatible (through the
efficient use of the explicit sub-vector element access methods,
first introduced in Section \ref{tsfcore:sec:explicit_vec_access}, as
described below).  This also technically assumes consistent
definitions of the scalar product but this will generally not be an
issue.

The second critical step is to have every serial
\texttt{\textit{Vector}} subclass override of the explicit sub-vector
access methods \texttt{getSubVector(...)} (both the \texttt{const} and
non-\texttt{const} versions), \texttt{free\-Sub\-Vector(...)} and
\texttt{commit\-Sub\-Vector(...)} to perform these operations without
calling the \texttt{applyOp(\-...)} method (see the subclass
\texttt{SerialVector}).

The third step is to have every serial \texttt{Vector} subclass
override and implement the method \texttt{applyOp(\-...)} in the
same way as shown below (using the \texttt{SerialVectorBase} node
subclass as the example).

{\scriptsize\begin{verbatim}
template<class Scalar>
void TSFCore::SerialVectorBase::applyOp(
    const RTOpPack::RTOpT<Scalar> &op, const size_t num_vecs, const Vector<Scalar>* vecs[]
    ,const size_t num_targ_vecs, Vector<Scalar>* targ_vecs[]
    ,RTOp_ReductTarget reduct_obj
    ,const Index first_ele, const Index sub_dim, const Index global_offset
    ) const
{
    ...
    in_applyOp_ = true;
    TSFCore::apply_op_serial(
        op,num_vecs,vecs,num_targ_vecs,targ_vecs,reduct_obj
        ,first_ele,sub_dim,global_offset
        );
    in_applyOp_ = false;
}
\end{verbatim}}

\noindent The implementation of the above \texttt{applyOp(\-...)} method is really
quite simple and it uses a helper function
\texttt{apply\_op\_serial(...)}  that takes care of all of the details
of calling the sub-vector extraction methods on the \texttt{Vector}
objects.  No dynamic casting is performed during this process and in
the case of \texttt{SerialVector}, no dynamic memory allocation is
performed either.  Therefore, for sufficiently large serial vectors,
the overhead of these function calls will be swamped by computation in
the RTOp operators, yielding near-optimal performance.

There are cases where it can not be determined until runtime whether a
vector is serial or not.  In these cases the concrete subclasses can
not simply derive from the \texttt{Serial\-VectorSpace\-Base} and
\texttt{SerialVectorBase} node subclasses but must instead implement
this this functionality themselves to be used when it is determined
that the vectors are indeed serial (see the Epetra TSFCore adapter
subclasses \texttt{TSFCore::EpetraVectorSpace} and
\texttt{TSFCore::EpetraVector} for instance).

By using this simple approach to developing serial
\texttt{\textit{VectorSpace}} and \texttt{\textit{Vector}} subclass,
the details of putting together many different types of numerical
algorithms becomes much easier.

%
\subsection{\texttt{\textit{LinearOp}}}
\label{tsfcore:sec:linear_op}
%

This section continues the discussion started in Section
\ref{tsfcore:sec:TSFCore_core_overview} for the
\texttt{\textit{LinearOp}} interface and includes some examples.

%
\subsubsection{\texttt{\textit{LinearOp::apply(\-...)}}}
\label{tsfcore:sec:linear_op_apply}
%

The C++ prototype for the \texttt{\textit{Vector}} version of
\texttt{\textit{LinearOp\-::apply(\-...)}} is

{\scriptsize\begin{verbatim}
namespace TSFCore{
template<class Scalar>
class LinearOp : public virtual OpBase<Scalar> {
public:
    ...
    virtual void apply(
        ETransp M_trans, const Vector<Scalar> &x, Vector<Scalar> *y
        ,Scalar alpha = 1.0, Scalar beta = 0.0
        ) const = 0;
    ...
};
} // namespace TSFCore
\end{verbatim}}

\noindent where the type \texttt{ETransp} is the C++ \texttt{enum}

{\scriptsize\begin{verbatim}
enum ETransp { NOTRANS, TRANS, CONJTRANS };
\end{verbatim}}

\noindent The use of an \texttt{enum} instead of a simple \texttt{bool} for the
\texttt{M\_trans} argument is very important.  The use of an \texttt{enum}
disallows the implicit conversion from other types like \texttt{char},
\texttt{int}, \texttt{double} and any type of pointer.  Using
\texttt{enum}s instead of \texttt{bool}s requires more typing but
greatly helps to avoid introducing bugs into the program that are
extremely difficult to track down.  In addition, the use of an
\texttt{enum} allows for more than just two values such as is shown
for the third value \texttt{CONJTRANS} which signifies the complex
conjugate.

The \texttt{\textit{Multi\-Vector}} version of
\texttt{\textit{LinearOp\-::apply(\-...)}} has an identical prototype
except the \texttt{\textit{Vector}} arguments are replaced with
\texttt{\textit{Multi\-Vector}} arguments.  The \texttt{\textit{Multi\-Vector}}
version has a default implementation based on the
\texttt{\textit{Vector}} version as described in Section
\ref{tsfcore:sec:vector_vs_multivector}.

In the above prototype, the scalars $\alpha$ and $\beta$ default to
$1.0$ and $0.0$ respectively.  Therefore, by leaving the default values,
the default operation becomes
%
\[
y = op(M) x
\]
%
which is the same form that is declared in
\texttt{\textit{HCL\-\_Linear\-Operator\-::apply(\-...)}}.  However, the scalars
$\alpha$ and $\beta$ provide direct calls to BLAS functions and remove
the need to create temporaries when performing long operations (see
Section \ref{tsfcore:sec:multi_vec_linear_op}).  For example, consider
the following long expression
%
\[
y = A u + \gamma B^T v + \eta C w
\]
%
where $A$, $B$ and $C$ are \texttt{\textit{LinearOp}} objects; and $y$,
$u$, $v$ and $w$ are \texttt{\textit{Vector}} objects.  Using TSFCore, this
long operation can be performed as follows

{\scriptsize\begin{verbatim}
template<class Scalar>
void TSFCore::long_expression(
    const LinearOp<Scalar>& A, const Vector<Scalar>& u
    ,Scalar gamma, const LinearOp<Scalar>& B, const Vector<Scalar>& u
    ,Scalar eta, const LinearOp<Scalar>& C, const Vector<Scalar>& w
    ,Vector<Scalar>* y
    )
{
    A.apply(NOTRANS,u,y);          // y  =  A*u
    B.apply(TRANS,v,y,gamma,1.0);  // y +=  gamma*B'*v
    C.apply(NOTRANS,w,y,eta,1.0);  // y +=  eta*C*w
}
\end{verbatim}}

\noindent where no temporary vectors are required.  Note that if the arguments
\texttt{alpha=1.0} and \texttt{beta=0.0} where fixed (as they are
in HCL for instance), the above operation would have to be implemented
as:

{\scriptsize\begin{verbatim}
template<class Scalar>
void TSFCore::bad_long_expression(
    const LinearOp<Scalar>& A, const Vector<Scalar>& u
    ,Scalar gamma, const LinearOp<Scalar>& B, const Vector<Scalar>& u
    ,Scalar eta, const LinearOp<Scalar>& C, const Vector<Scalar>& w
    ,Vector<Scalar>* y
    )
{
    MemMngPack::ref_count_ptr<Vector<Scalar> >
        t = A.range()->createMember(); // Create a temporary to store the intermediate  products
    A.apply(NOTRANS,u,y);              // y  =  A*u
    B.apply(TRANS,v,t.get());          // t  =  B'*v
    axpy(gamma,*t,y);                  // y +=  gamma*t
    C.apply(NOTRANS,w,t.get());        // t  =  C*w
    axpy(eta,*t,y);                    // y +=  eta*t
}
\end{verbatim}}

Not only is the function \texttt{bad\-\_long\-\_expression(\-...)}
slightly less efficient than \texttt{long\-\_expression(\-...)} but it
is also longer and more difficult to write.  The arguments
\texttt{alpha} and \texttt{beta} are important to achieve a near-optimal
implementation and for ease of use.

Note that some implementations of \texttt{\textit{LinearOp}} may not be
able to apply the operator with a value of $\beta \ne 0$ without
creating at least one temporary vector (or multi-vector).  However,
this is a minor performance issue in most use cases.

%
\subsubsection{Optional support for adjoints}
\label{tsfcore:sec:linear_op_adjoints}
%

The \texttt{\textit{LinearOp}} interface only optionally supports
transposed (adjoint) matrix-vector multiplications and linear solves.
If the method \texttt{\textit{opSupported(M\_trans)}} returns
\texttt{false}, then the argument \texttt{M\_trans}, when
passed to \texttt{\textit{apply(\-...)}}, will result in an
\texttt{OpNotSupported} exception being thrown.
This specification, while not ideal from an object-orientation purest
point of view, does satisfy the basic principles outlined in Section
\ref{tsfcore:sec:general_software_concepts}.

%
\subsection{\texttt{\textit{Multi\-Vector}}}
\label{tsfcore:sec:multi_vec}
%

While the concepts of a \texttt{\textit{VectorSpace}} and
\texttt{\textit{Vector}} are well established, the
concept of a multi-vector is fairly new.  The idea of a multi-vector
was motivated by the library Epetra \cite{ref:Epetra} which contains
mostly concrete implementations of distributed-memory linear algebra
classes using MPI \cite{ref:mpi}.  A key issue is how multi-vectors
and vectors relate to each other.  In Epetra, the vector class is a
specialization of the multi-vector class.  This make sense from an
implementation point of view.  The Epetra approach takes the view that
a vector {\em is a} type of multi-vector.  An arguably more natural
view from an abstract mathematical perspective is that multi-vectors
are composed out of a set of vectors where each vector represents a
column of the multi-vector.  This is the view that multi-vectors {\em
have} or {\em contain} vectors and this is the approach that has been
adopted for TSFCore as shown in Figure \ref{tsfcore:fig:tsfl_basic}.

Note that a multi-vector is not the same thing as a blocked or product
vector.  In fact, multi-vectors and product vectors are orthogonal
concepts and it is possible to have product multi-vectors.  Product
vectors and vector spaces are discussed in Sections
\ref{tsfcore:sec:vec_apply_op} and \ref{tsfcore:sec:composite_abstractions}.

All of the below examples will involve the compact LBFGS
implementation described above in Section \ref{tsfcore:sec:LBFGS}.
For these examples we will consider interactions with the two
principle
\texttt{\textit{Multi\-Vector}} objects \texttt{Y\_store} and
\texttt{S\_store} which each have $m_{\tiny\mbox{max}}$ columns.

%
\subsubsection{Accessing columns of \texttt{\textit{Multi\-Vector}}
as \texttt{\textit{Vector}} objects}
%

The columns of a \texttt{\textit{Multi\-Vector}} object can be accessed
using the \texttt{const} or non-\texttt{const}
\texttt{\textit{col(j)}} methods which return
\texttt{ref\_count\_ptr<>} objects which points to an abstract
\texttt{\textit{Vector}} view of a column.  The prototypes for these
methods are shown below.

{\scriptsize\begin{verbatim}
namespace TSFCore{
template<class Scalar>
class MultiVector : virtual public LinearOp<Scalar> {
public:
    ...
    virtual MemMngPack::ref_count_ptr<Vector<Scalar> >        col(const Index j) = 0;
    virtual MemMngPack::ref_count_ptr<const Vector<Scalar> >  col(const Index j) const;
    ...
};
} // namespace TSFCore
\end{verbatim}}

\noindent Actually, the non-\texttt{const} version of \texttt{\textit{col(...)}}
is the only pure virtual function in \texttt{\textit{Multi\-Vector}} and
therefore the only function that must be overridden in order to create
a concrete (but suboptimal) \texttt{\textit{Multi\-Vector}} subclass.
All of the other virtual methods in \texttt{\textit{Multi\-Vector}} have
default implementations based on this method and
\texttt{\textit{Vector\-::applyOp(\-...)}}.

The following example function copies the most recent update vectors
\texttt{s} and \texttt{y} into the multi-vectors \texttt{S\_store}
and \texttt{Y\_store} and increments the counter \texttt{m} for a
compact LBFGS implementation.

{\scriptsize\begin{verbatim}
template<class Scalar>
void TSFCore::update_S_Y( const Vector<Scalar>& s, const Vector<Scalar>& y
                          ,MultiVector<Scalar>* S_store, MultiVector<Scalar>* Y_store, int* m )
{
    const int m_max = S_store->domain()->dim(); // Get the maximum number of updates allowed
    if(*m < m_max) {
        ++(*m);                                 // Increment the number of updates
        assign(S_store->col(*m).get(),s);       // Copy in s into S(:,m)         
        assign(Y_store->col(*m).get(),y);       // Copy in y into Y(:,m)
    }
    else {
        // We must drop the oldest pair (s,y) and copy in the newest pair
        ...
    }
}
\end{verbatim}}

\noindent Note that the \texttt{\textit{Multi\-Vector}} object that
\texttt{\textit{col(...)}} is called on is not guaranteed to be
updated until the returned \texttt{\textit{Vector}} object is
destroyed when the \texttt{ref\_count\_ptr<>} object returned from
\texttt{\textit{col(...)}} goes out of scope.  The use in the above function
guarantees that this happens after each call to the
\texttt{assign(...)} function.

%
\subsubsection{\texttt{\textit{Multi\-Vector}} sub-views}
%

In addition to being able to access the columns of a
\texttt{\textit{Multi\-Vector}} object one column at a time, a client
can also create \texttt{const} and non-\texttt{const}
\texttt{\textit{Multi\-Vector}} views of the columns
using one of the \texttt{\textit{subView(...)}} methods shown below.

{\scriptsize\begin{verbatim}
namespace TSFCore {
template<class Scalar>
class MultiVector : virtual public LinearOp<Scalar> {
public:
    ...
    virtual MemMngPack::ref_count_ptr<MultiVector<Scalar> >       subView(const Range1D& col_rng);
    virtual MemMngPack::ref_count_ptr<const MultiVector<Scalar> > subView(const Range1D& col_rng) const;
    virtual MemMngPack::ref_count_ptr<MultiVector<Scalar> >       subView(const int numCols
                                                                          ,const int cols[]);
    virtual MemMngPack::ref_count_ptr<const MultiVector<Scalar> > subView(const int numCols
                                                                          ,const int cols[]) const;
    ...
};
} // namespace TSFCore
\end{verbatim}}

\noindent The ability to extract a \texttt{\textit{Multi\-Vector}} sub-view of a
contiguous set of columns of a \texttt{\textit{Multi\-Vector}} object,
which is supported by the first two methods, is required in order to
implement certain types of numerical methods.  For example, the
implementation of the compact LBFGS method described above in Section
\ref{tsfcore:sec:LBFGS} requires this functionality.
The following example function shows how the contiguous
\texttt{\textit{subView(...)}} method is used in an LBFGS
implementation where \texttt{\textit{Multi\-Vector}} storage objects
\texttt{S\_store} and \texttt{Y\_store} are used to create
\texttt{\textit{Multi\-Vector}} view objects \texttt{S} and \texttt{Y}
for only the number of updates currently stored.  These sub-view
objects are used in later example code.

{\scriptsize\begin{verbatim}
template<class Scalar>
MemMngPack::ref_count_ptr<const TSFCore::MultiVector<Scalar> >
TSFCore::get_updated( const MultiVector<Scalar>& Store, int m )
{
    return Store.subView(Range1D(1,m));
}
\end{verbatim}}

The second form of the \texttt{\textit{subView(...)}} method takes a
list of (possibly unsorted but unique) column indexes \texttt{cols[]}
and returns a \texttt{\textit{Multi\-Vector}} view object of those
columns.  This functionality is very useful in the development of some
types of ANAs (e.g.~block Krylov iterative linear equation solvers).

Note that both forms of the \texttt{\textit{subView(...)}} method have
(suboptimal) default implementations based on the
\texttt{MultiVectorCols} utility subclass.  This
\texttt{MultiVectorCols} class, coincidentally, is also used
to provide a general (but suboptimal) implementation of
\texttt{\textit{Multi\-Vector}} just given an implementation of
\texttt{\textit{Vector}}.  This utility subclass is also used to
provide default implementations for many of the
\texttt{\textit{Multi\-Vector}}-related methods which includes
the default implementation of the
\texttt{\textit{VectorSpace\-::createMembers(numMembers)}} method.

%
\subsubsection{\texttt{\textit{Multi\-Vector}} support for \texttt{\textit{applyOp(\-...)}}}
\label{tsfcore:sec:multi_vec_apply_op}
%

RTOp operators can be applied to the columns of a
\texttt{\textit{Multi\-Vector}} object one column at a time
using the \texttt{\textit{col(...)}} method.  However, a potentially
more efficient approach is to allow the
\texttt{\textit{Multi\-Vector}} object to apply the \texttt{RTOp} operator itself.
This is supported by the \texttt{\textit{applyOp(\-...)}} methods on
\texttt{\textit{Multi\-Vector}}.  The \texttt{\textit{applyOp(\-...)}} methods are not
called directly (they are protected) but instead are called by
non-member (friend) methods \texttt{\textit{applyOp(\-...)}} which then
invoke the member functions.  This
approach allows a more natural way to invoke a
reduction/transformation operation in line with the mathematical
description in \cite{ref:rtop_toms}.

There are two versions of
\texttt{\textit{Multi\-Vector\-::applyOp(\-...)}}: one that returns a list
of reduction objects (one for each column of the multi-vector) and
another that uses two \texttt{RTOp} operators to reduce all of the
reduction objects over each column into single reduction object which
is returned.  Both versions of the
\texttt{\textit{Multi\-Vector\-::applyOp(\-...)}} have default implementations
that are based on \texttt{\textit{Multi\-Vector\-::col(...)}} and
\texttt{\textit{Vector\-::applyOp(\-...)}}.

Below, two example operations, which are defined in the header
\texttt{TSFCore\-Multi\-Vector\-Std\-Ops.hpp}, are shown that are needed
by various ANAs.

The first example is the update operator $\alpha U + V \rightarrow V$
and is implemented in the following function.

{\scriptsize\begin{verbatim}
template<class Scalar>
void TSFCore::update( Scalar alpha, const MultiVector<Scalar>& U, MultiVector<Scalar>* V )
{
    THROW_EXCEPTION(V==NULL,std::logic_error,"axpy(...), Error!");    // Validate input
    RTOpPack::TOpAxpy<Scalar> axpy_op(alpha);                         // Create (init) op object
    const MultiVector<Scalar>* multi_vecs[]       = { &U };           // Set up non-mutable mv args
    MultiVector<Scalar>*       targ_multi_vecs[]  = { V  };           // Set up mutable mv args
    applyOp<Scalar>(axpy_op,1,multi_vecs,1,targ_multi_vecs,NULL);     // Invoke the transformation operator
}
\end{verbatim}}

\noindent In the above call to \texttt{applyOp(\-...)}, a \texttt{NULL} pointer is
passed in for the array of reduction objects which is allowed since
this RTOp operator does not perform a reduction.

The second example is a column-wise dot product operation and is
implemented in the following function.

{\scriptsize\begin{verbatim}
template<class Scalar>
void TSFCore::dot( const MultiVector<Scalar>& V1, const MultiVector<Scalar>& V2, Scalar dot[] )
{
    const int m = V1.domain()->dim();                                        // Get the num cols
    RTOpPack::ROpDot<Scalar> dot_op;                                         // Create op object
    std::vector<RTOp_ReductTarget>  dot_targs(m);                            // Array of reduct objects
    for( int kc = 0; kc < m; ++kc )                                          // For each column:
        dot_op.reduct_obj_create_raw(&(dot_targs[kc]=RTOp_REDUCT_OBJ_NULL)); //   Create reduct object
    const MultiVector<Scalar>* multi_vecs[] = { &V1, &V2 };                  // Set up non-mutable mv args
    applyOp(dot_op,2,multi_vecs,0,NULL,&dot_targs[0]);                       // Invoke the reduction operator
    for( int kc = 0; kc < m; ++kc ) {                                        // For each column:
        dot[kc] = dot_op(dot_targs[kc]);                                     //   Extract dot product val
        dot_op.reduct_obj_free_raw(&(dot_targs[kc]));                        //   Free each reduction object
    }
}
\end{verbatim}}

\noindent Note that the above reduction operation will be performed with a
single global reduction when performed on a distributed-memory
parallel computer (using MPI).  Without the concept of a
\texttt{\textit{Multi\-Vector}} or support for the
\texttt{\textit{applyOp(\-...)}} method, this type of multi-vector
reduction operation would require $m$ separate global reductions,
where $m$ is the number of columns in the multi-vector.  The
presence of this method is
critical for a near-optimal implementation with respect to
minimizing communication in a distributed memory program.

%
\subsubsection{\texttt{\textit{Vector}} and \texttt{\textit{Multi\-Vector}} correspondence}
\label{tsfcore:sec:vector_vs_multivector}
%

The interface class \texttt{\textit{LinearOp}} takes the perspective
that most subclasses will naturally prefer to implement the
\texttt{\textit{Vector}} version of the method
\texttt{\textit{apply(\-...)}} and let the default implementation of the
\texttt{\textit{Multi\-Vector}} version of this method deal with
\texttt{\textit{Multi\-Vector}} objects.  There are many cases where there is no
way to provide more specialized implementations of these operations
for multi-vectors.  For example, while the BLAS and LAPACK are
designed from the ground up to be more efficient with multiple
right-hand-side vectors, most current implementations of sparse direct
linear solvers unfortunately only support the solution of single
linear systems (e.g.~the Harwell solvers such as MA47 and MA48
\cite{ref:hsl_1995}).  This realization provides the motivation for
choosing the \texttt{\textit{Vector}} versions of these methods as the
default methods for subclasses to override.  With that said, if a
\texttt{\textit{LinearOp}} subclass can
provide an optimized implementation of the \texttt{\textit{Multi\-Vector}}
version of the \texttt{\textit{apply(\-...)}} method, does such a subclass
also have to provide a completely independent implementation of the
\texttt{\textit{Vector}} version of this method?  The answer is no.
By using the provided utility subclass
\texttt{MultiVectorCols}, a \texttt{\textit{Multi\-Vector}} wrapper can
easily be created for any \texttt{\textit{Vector}} object.  The
following example shows how a
\texttt{\textit{LinearOp}} subclass, for instance, can easily provide
support for the \texttt{\textit{Vector}} version of
\texttt{\textit{apply(\-...)}} when providing an optimized
implementation of the \texttt{\textit{Multi\-Vector}} version.

{\scriptsize\begin{verbatim}
namespace TSFCore {
template<class Scalar>
class MyLinearOp : public LinearOp<Scalar> {
public:
    ...
    void apply( ETransp M_trans, const Vector<Scalar> &x, Vector<Scalar> *y, Scalar alpha
               ,Scalar beta ) const
    {
        namespace mmp = MemMngPack;
        const MultiVectorCols<Scalar>
            X(mmp::rcp(const_cast<Vector<Scalar>*>(&x),false)); // Create mv views
        MultiVectorCols<Scalar>
            Y(mmp::rcp(y,false));                               // ...
        apply(alpha,M_trans,X,&Y,beta);                         // Call mv version
    }
    void apply( ETransp M_trans, const MultiVector<Scalar> &X, MultiVector<Scalar> *Y, Scalar alpha
                ,Scalar beta ) const
    {
        // Optimized implementation for multi-vectors
        ...
    }
    ...
};
} // namespace TSFCore
\end{verbatim}}

\noindent Note that the constructor for the the class \texttt{MultiVectorCols},
for instance called in the line

{\scriptsize\begin{verbatim}
      MultiVectorCols<Scalar>
          Y(mmp::rcp(y,false));
\end{verbatim}}

\noindent takes a \texttt{ref\_count\_ptr<const Vector<Scalar> >} object.  In
order to call this constructor with memory not owned by the client
(which is the case here), the \texttt{rcp(...)} function must be
called with the argument \texttt{owns\_mem = false} so that the last
\texttt{ref\_count\_ptr<const Vector<Scalar> >} object to be destroyed
will not try to free the vector argument.

%
\subsubsection{\texttt{\textit{Multi\-Vector}} acting as a \texttt{\textit{LinearOp}}}
\label{tsfcore:sec:multi_vec_linear_op}
%

The last issues to discuss with regard to
\texttt{\textit{Multi\-Vector}} relate to where it fits in the class
hierarchy.  The decision adopted for TSFCore was to make
\texttt{\textit{Multi\-Vector}} specialize \texttt{\textit{LinearOp}}.
In other words, a \texttt{\textit{Multi\-Vector}} object can also act as
a \texttt{\textit{LinearOp}} object.

As an example where this is needed, consider using the LBFGS inverse
matrix $H$ shown in Figure \ref{tsfcore:fig:LBFGS} as a linear
operator which acts on multi-vector arguments $U$ and $V$ in an
operation of the form
%
\begin{eqnarray*}
U & = & \alpha B^{-1} V \\
  & = & \alpha H V \\
  & = & \alpha g V + \alpha
                            {\bmat{cc} S & g Y \emat}
                            {\bmat{cc} Q_{ss} & Q_{sy} \\ Q_{sy}^T & Q_{yy} \emat}
                            {\bmat{c} S^T \\ g Y^T \emat} V
\end{eqnarray*}
%
where the matrices $Q_{ss}$, $Q_{ys}$ and $Q_{yy}$ are stored as small
\texttt{\textit{Multi\-Vector}} objects.  A multi-vector solve using
the inverse $H = B^{-1}$ might be used, for instance, in an active-set
optimization algorithm where $V$ represents the $p$ gradient vectors
of the active constraints.  This is an important operation in the
formation of a Schur complement of the KKT system in the QP subproblem
of an reduced-space SQP method \cite{RABartlett_2001}.  This
multi-vector operation using $H$ can be performed with the following
operations
%
\begin{eqnarray*}
T_1 & = & S^T V \\
T_2 & = & Y^T V \\
T_3 & = & Q_{ss} T_1 + g Q_{sy} T_2 \\
T_4 & = & Q_{sy}^T T_1 + g Q_{yy} T_2 \\
U   & = & \alpha g V + \alpha S T_3 + \alpha g Y T_4
\end{eqnarray*}
%
where $T_1$, $T_2$, $T_3$ and $T_4$ are all temporary
\texttt{\textit{Multi\-Vector}} objects of dimension $m \times p$.  The
following function shows how the above operations are performed in
order to implement the overall multi-vector solve.

{\scriptsize\begin{verbatim}
template<class Scalar>
void TSFCore::LBFGS_solve(
    int m, Scalar g, const MultiVector<Scalar>& S_store, const MultiVector<Scalar>& Y_store
    ,const MultiVector<Scalar>& Q_ss, const MultiVector<Scalar>& Q_sy, const MultiVector<Scalar>& Q_yy
    ,const MultiVector<Scalar>& V, MultiVector<Scalar>* U, Scalar alpha = 1.0, Scalar = beta = 0.0
    )
{
    // validate input
    ...
    const int p = V.domain()->dim();                // Get number of columns in V and U
    MemMngPack::ref_count_ptr<const MultiVector<Scalar> >
        S = get_updated(S_store,m),                 // Get view of only stored columns in S_store
        Y = get_updated(Y_store,m);                 // Get view of only stored columns in Y_store
    MemMngPack::ref_count_ptr<MultiVector<Scalar> >
        T_1 = S->domain()->createMembers(p),        // Create the tempoarary multi-vectors
        T_2 = Y->domain()->createMembers(p),        // ...
        T_3 = Q_ss->range()->createMembers(p),      // ...
        T_4 = Q_yy->range()->createMembers(p);      // ...
    S->apply(TRANS,V,T_1->get());                   // T_1  =  S'*V
    Y->apply(TRANS,V,T_2->get());                   // T_2  =  Y'*V
    Q_ss.apply(NOTRANS,*T_1,T_3->get());            // T_3  =  Q_ss*T_1
    Q_sy.apply(NOTRANS,*T_2,T_3->get(),g,1.0);      // T_3 +=  g*Q_sy*T_2
    Q_sy.apply(TRANS,  *T_1,T_4->get());            // T_4  =  Q_sy'*T_1
    Q_yy.apply(NOTRANS,*T_2,T_4->get(),g,1.0);      // T_4 +=  g*Q_yy*T_2
    S->apply(NOTRANS,*T_3,U,alpha);                 // U    =  alpha*S*T_3
    Y->apply(NOTRANS,*T_4,U,alpha*g,1.0);           // U   +=  alpha*g*Y*T_4
    axpy(alpha*g,V,U);                              // U   +=  alpha*g*V
}
\end{verbatim}}

Consider the use of the above function in an SPMD environment where
the ANA runs in duplicate and in parallel on each processor.  Here,
the elements for the multi-vector objects \texttt{S\_store},
\texttt{Y\_store}, \texttt{V} and \texttt{U} are distributed across
many different processors.  Note that in this case all of the elements
in the multi-vector objects \texttt{Q\_ss}, \texttt{Q\_sy}, \texttt{Q\_yy},
\texttt{T\_1}, \texttt{T\_2}, \texttt{T\_3} and \texttt{T\_4} are stored
locally and in duplicate on each processor.  Now let us consider the
performance of this set of operations in this context.  Note that
there are principally three different types of operations with
multi-vectors that are performed through the
\texttt{\textit{Multi\-Vector\-::apply(\-...)}} method.

The first type of operation performed by
\texttt{\textit{Multi\-Vector\-::apply(\-...)}} is the parallel/parallel
matrix-matrix products performed in the lines

{\scriptsize\begin{verbatim}
    S->apply(TRANS,V,T_1->get());
    Y->apply(TRANS,V,T_2->get());
\end{verbatim}}

\noindent where the results are stored in the local multi-vectors 
\texttt{T\_1} and \texttt{T\_2}.  These two operations only
require a single global reduction each, independent of the number of
updates $m$ represented in $S$ and $Y$ or columns $p$ in $V$.  Note
that if there was no concept of a multi-vector and these matrix-matrix
products had to be performed one set of vectors at a time, then these
two parallel matrix-matrix products would require a whopping $2 m p$
global reductions.  For $m = 40$ and $p = 20$ this would result in $2
m p = 2(40)(20) = 1600$ global reductions!  Clearly this many global
reductions would destroy the parallel scalability of the overall ANA.
It is in this type of operation that the concept of a
\texttt{\textit{Multi\-Vector}} is most critical for near-optimal
performance in parallel programs.  In addition to mimimizing
communication overhead, the \texttt{\textit{Multi\-Vector}}
implementation can utilize level-3 BLAS to perform the local processor
matrix-matrix multiplications yielding near-optimal cache performance
on most systems.

The second type of operation performed by
\texttt{\textit{Multi\-Vector\-::apply(\-...)}} is the local/local matrix-matrix
products of small local \texttt{\textit{Multi\-Vector}} objects in the
lines

{\scriptsize\begin{verbatim}
    Q_ss.apply(NOTRANS,*T_1,T_3->get());
    Q_sy.apply(NOTRANS,*T_2,T_3->get(),g,1.0);
    Q_sy.apply(TRANS,  *T_1,T_4->get());
    Q_yy.apply(NOTRANS,*T_2,T_4->get(),g,1.0);
\end{verbatim}}

\noindent Note that these types of local computations classify as serial
overhead and therefore it is critical that the cost of these
operations be kept to a minimum or they could cripple the parallel
scalability of the overall ANA.  Each of these four matrix-matrix
multiplications involve only one virtual function call and the
matrix-matrix multiplication itself can be performed with level-3
BLAS, achieving the fastest possible flop rate attainable on most
processors \cite{ref:demmel_1997}.

The third type of operation performed by
\texttt{\textit{Multi\-Vector\-::apply(\-...)}} is local/parallel matrix-matrix
multiplications performed in the lines

{\scriptsize\begin{verbatim}
    S->apply(NOTRANS,*T_3,U,alpha);
    Y->apply(NOTRANS,*T_4,U,alpha*g,1.0);
\end{verbatim}}

\noindent This type of operation involves fully scalable work with no
communication or synchronization required.  Here, a vector-by-vector
implementation will not be a bottleneck from a standpoint of global
communication.  However, this operation will utilize level-3 BLAS and
yield near-optimal local cache performance where a vector-by-vector
implementation would not.

The last type of operation performed in the above
\texttt{LBFGS\_solve(...)}  function does not involve
\texttt{\textit{Multi\-Vector\-::apply(\-...)}} and is shown in the line

{\scriptsize\begin{verbatim}
    axpy(alpha*g,V,U);
\end{verbatim}}

\noindent The implementation of this function uses an RTOp transformation
operator with the \texttt{\textit{Multi\-Vector\-::applyOp(\-...)}}
method.  Note that this function only involves transformation
operations (i.e.~no communication) which are fully scalable.

%
\subsubsection{Aliasing of \texttt{\textit{Vector}} and \texttt{\textit{Multi\-Vector}} arguments}
\label{tsfcore:sec:aliasing}
%

It has not been stated specifically yet but in all
\texttt{\textit{Vector}}, \texttt{\textit{Multi\-Vector}} and
\texttt{\textit{LinearOp}} methods where a \texttt{\textit{Vector}} or
\texttt{\textit{Multi\-Vector}} object may be modified, it is strictly
forbidden for any of the mutable objects to alias any of the other
objects of the same type in the same method.  For example, code like the
following is strictly forbidden.

{\scriptsize\begin{verbatim}
template<class Scalar>
void foo3( const LinearOp& M, ETransp M_trans, Vector<Scalar>* x )
{
    M.apply(M_trans,*x,x);  // Error!!!!!!!!!!
}
\end{verbatim}}

\noindent Note that typically the above function would not even get to the numerics
(where it would most likely compute the wrong results) because
\texttt{M.range()->isCompatible(*M.domain())==false} in general.
Instead, this operation must be implemented as follows.

{\scriptsize\begin{verbatim}
template<class Scalar>
void foo4( const LinearOp& M, ETransp M_trans, Vector<Scalar>* x )
{
    MemMngPack::ref_count_ptr<Vector<Scalar> > x_tmp = x->clone();   // Create a copy
    M.apply(M_trans,*x_tmp,x);                                       // Okay!
}
\end{verbatim}}

\noindent Allowing client code to pass in aliased arguments would greatly
complicate the implementation of most RTOp,
\texttt{\textit{Multi\-Vector}} and \texttt{\textit{LinearOp}}
subclasses and would introduce the possibility of many different types
of bugs that would be extremely difficult to track down.  This is an
issue that is usually not well defined in most linear algebra
interfaces but it is a very important issue.  Allowing ANA developers
to alias objects in these methods does not provide any new
functionality and is considered to be only nonessential but convenient
functionality and is therefore not included in TSFCore.  In general,
it is not possible to determine, from the abstract interfaces for the
objects themselves, if objects alias each other.  To perform this type
of test would require special methods be added to the
\texttt{\textit{Vector}} and
\texttt{\textit{Multi\-Vector}} interfaces and implementing these test
methods would complicate the development of these types of subclasses
greatly.

Note that aliasing of input data with output data is not strictly
forbidden, and is allowd as long as this is built into the operation.
For example, in the \texttt{\textit{LinearOp\-::apply(\-...)}} method,
the vector $y$ both supplies data for the operation (if $\beta \ne 0$)
and stores the output for the operation as shown in
(\ref{tsfcore:equ:apply_vec}).  The same applies to several of the
RTOp-based vector operations shown in Figure
\ref{tsfcore:fig:std_vec_ops} (i.e.~\texttt{Vp\_S(...)},
\texttt{Vt\_S(...)}, \texttt{Vp\_S(...)}, \texttt{Vp\_StV(...)}
and \texttt{ele\_wise\_prod(...)}).  Allowing vectors and
multi-vectors to both supply data for an operation and store output
from an operation is fine as long as the operation has been
specifically designed to handle this as the above mentioned operations
have.

In summary, do not alias output arguments with each other or with
other input arguments in any of the TSFCore interface methods.

%
\section{An Example Abstract Numerical Algorithm : An Iterative Linear Solver}
\label{tsfcore:sec:ANA_iter_solver_example}
%

In this section we describe how TSFCore can be directly used to build
ANAs and while this is not the primary role TSFCore is designed for, this
example shows that TSFCore provides all of the needed functionality for
near-optimaly performing implementations.  Code for a partial ANA in
the form of a compact LBFGS method was described in Section
\ref{tsfcore:sec:LBFGS}.  In this section, we will describe the
implementation of a simple block BiCG
\cite{ref:tmpls_for_iter_systems} method.  BiCG was chosen for this
example was because it requires adjoints and is fairly simple.  Other
types of block iterative linear solvers such as methods as CG,
BiCGStab, GMRES and QMR \cite{ref:tmpls_for_iter_systems} can be
implemented in a similar manner.

The subclass \texttt{BiCG\-Solver} implements a simple block BiCG
method.  A listing for a single-vector version of the BiCG method is
shown in Figure \ref{tsfcore:fig:BiCG}.  This listing is identical to
the listing in \cite{ref:tmpls_for_iter_systems} except for the
substitutions $A = op(M)$, $M = op(\tilde{M})$ and $b =a y$ (where $a$
is a scalar multiplier).  The multi-vector version, as implemented
using TSFCore in code, follows in a straightforward manner.  This
implementation does not take advantage of any potential linear
dependence in the right-hand-side vectors in an attempt to accelerate
the method such as is described in [???].  Such an enhanced
multi-vector version could be implemented in a similar manner.

\begin{figure}
\begin{center}
\fbox{
\begin{minipage}{\textwidth}
{\bsinglespace
\begin{tabbing}
\hspace{4ex}\=\hspace{4ex}\=\hspace{4ex}\=\hspace{4ex} \\
\>	Compute $r^{(0)} = a y - op(M) x^{(0)}$ for the initial guess $x^{(0)}$.\hspace{4ex} \\
\>	Choose $\tilde{r}^{(0)}$ (for example, $\tilde{r}^{(0)} = \mbox{randomize}(-1,+1)$).\hspace{4ex} \\
\>	\textbf{for} $i = 1, 2, \ldots$ \\
\>	\>	solve $op(\tilde{M}) z^{(i-1)} = r^{(i-1)}$ \\
\>	\>	solve $op(\tilde{M})^T \tilde{z}^{(i-1)} = \tilde{r}^{(i-1)}$ \\
\>	\>	$\rho_{i-1} = z^{{(i-1)}^T} \tilde{r}^{(i-1)}$ \\
\>	\>	\textbf{if} $\rho_{i-1} = 0$, \textbf{method fails} \\
\>	\>	\textbf{if} $i = 1$ \\
\>	\>	\>	$p^{(i)} = z^{(i-1)}$ \\
\>	\>	\>	$\tilde{p}^{(i)} = \tilde{z}^{(i-1)}$ \\
\>	\>	\textbf{else} \\
\>	\>	\>	$\beta_{i-1} = \rho_{i-1}/\rho_{i-2}$ \\
\>	\>	\>	$p^{(i)} = z^{(i-1)} + \beta_{i-1} p^{(i-1)}$ \\
\>	\>	\>	$\tilde{p}^{(i)} = \tilde{z}^{(i-1)} + \beta_{i-1} \tilde{p}^{(i-1)}$ \\
\>	\>	\textbf{endif} \\
\>	\>	$q^{(i)} = op(M) p^{(i)}$ \\
\>	\>	$\tilde{q}^{(i)} = op(M)^T \tilde{p}^{(i)}$ \\
\>	\>	$\gamma_{i} = \tilde{p}^{{(i)}^T} q^{(i)}$ \\
\>	\>	$\alpha_{i} = \rho_{i-1}/\gamma_{i}$ \\
\>	\>	$x^{(i)} = x^{(i-1)} + \alpha_{i-1} p^{(i)}$ \\
\>	\>	$r^{(i)} = r^{(i-1)} - \alpha_{i-1} q^{(i)}$ \\
\>	\>	$\tilde{r}^{(i)} = \tilde{r}^{(i-1)} - \alpha_{i-1} \tilde{q}^{(i)}$ \\
\>	\>	check convergence; continue if necessary \\
\>	\textbf{end}
\end{tabbing}
\esinglespace}
\end{minipage}
}%fbox
\end{center}
\caption{
\label{tsfcore:fig:BiCG}
A single-vector version of the preconditioned bi-conjugate gradient method (BiCG).
}
\end{figure}

Figure \ref{tsfcore:fig:BiCG_code} shows a partial listing for the
\texttt{BiCGSolver\-::doIteration(...)} method (which implements
a single iteration of the BiCG method) as implemented in the file
\texttt{TSFCore\-Solvers\-BiCG\-Solver.hpp}.
%
{\bsinglespace
\begin{figure}
\begin{minipage}{\textwidth}
{\scriptsize\begin{verbatim}
 00273 template<class Scalar>
 00274 void BiCGSolver<Scalar>::doIteration(
 00275     const LinearOp<Scalar> &M, ETransp opM_notrans, ETransp opM_trans, MultiVector<Scalar> *X, Scalar a
 00276     ,const LinearOp<Scalar> *M_tilde_inv, ETransp opM_tilde_inv_notrans, ETransp opM_tilde_inv_trans
 00277     ) const
 00278 {
 00285     const Index m = currNumSystems_;
 00286     int j;
 00287     if( M_tilde_inv ) {
 00288         M_tilde_inv->apply( opM_tilde_inv_notrans, *R_,       Z_.get()       );
 00289         M_tilde_inv->apply( opM_tilde_inv_trans,   *R_tilde_, Z_tilde_.get() );
 00290     }
 00291     else {
 00292         assign( Z_.get(),       *R_        );
 00293         assign( Z_tilde_.get(), *R_tilde_  );
 00294     }
 00299     dot( *Z_, *R_tilde_, &rho_[0] );
 00303     for(j=0;j<m;++j) {
 00304         THROW_EXCEPTION(
 00305             rho_[j] == 0.0, Exceptions::SolverBreakdown
 00306             ,"BiCGSolver<Scalar>::solve(...): Error, rho["<<j<<"] = 0.0, the method has failed!"
 00307             );
 00308     }
 00309     if( currIteration_ == 1 ) {
 00310         assign( P_.get(),       *Z_       );
 00311         assign( P_tilde_.get(), *Z_tilde_ );
 00312     }
 00313     else {
 00314         for(j=0;j<m;++j) beta_[j] = rho_[j]/rho_old_[j];
 00315         update( *Z_,       &beta_[0], 1.0, P_.get()       );
 00316         update( *Z_tilde_, &beta_[0], 1.0, P_tilde_.get() );
 00317     }
 00322     M.apply(opM_notrans, *P_,       Q_.get()       );
 00323     M.apply(opM_trans,   *P_tilde_, Q_tilde_.get() );
 00328     dot( *P_tilde_, *Q_, &gamma_[0] );
 00329     for(j=0;j<m;++j) alpha_[j] = rho_[j]/gamma_[j];
 00334     for(j=0;j<m;++j) {
 00335         THROW_EXCEPTION(
 00336             alpha_[j] == 0.0 || RTOp_is_nan_inf(alpha_[j]), Exceptions::SolverBreakdown
 00337             ,"BiCGSolver<Scalar>::solve(...): Error, rho["<<j<<"] = 0.0, the method has failed!"
 00338             );
 00339     }
 00340     update( &alpha_[0], +1.0, *P_, X );
 00341     update( &alpha_[0], -1.0, *Q_, R_.get() );
 00342     update( &alpha_[0], -1.0, *Q_tilde_, R_tilde_.get() );
 00348 }
\end{verbatim}}
\end{minipage}
\caption{
\label{tsfcore:fig:BiCG_code}
Implementation of an iteration of a multi-vector version of BiCG.
}
\end{figure}
\esinglespace}
%
All of the functions and methods called in the C++ code shown in
Figure \ref{tsfcore:fig:BiCG_code} have already been described except
for the non-member functions \texttt{assign(...)} (lines 292, 293, 310
and 311) and \texttt{update(...)} (lines 315, 316 and 340--342) which
are defined in the header \texttt{TSFCore\-Multi\-Vector\-Std\-Ops.hpp}.
There are two assignment functions \texttt{assign(...)}: one that
assigns a \texttt{\textit{Multi\-Vector}} object to a \texttt{Scalar},
and another that assigns one \texttt{\textit{Multi\-Vector}} object to
another.  Both of these methods are implemented through
\texttt{\textit{Multi\-Vector\-::applyOp(\-...)}} and use already-defined
RTOp operators.  The two versions of the \texttt{update(...)} method
used in this code, however, can not use
\texttt{\textit{Multi\-Vector\-::applyOp(\-...)}} and instead are
implemented column-by-column as, for instance
%
\[
(\alpha_{(j)} \beta) U_{(:,j)} + V_{(:,j)} \rightarrow V_{(:,j)}, \; \mbox{for} \; j = 1 \ldots m
\]
%
in the function

{\scriptsize\begin{verbatim}
template<class Scalar>
void TSFCore::update( Scalar alpha[], Scalar beta, const MultiVector<Scalar>& U, MultiVector<Scalar>* V )
{
    ...
    const int m = U.domain()->dim();
    for( int j = 1; j <= m; ++j )
        Vp_StV( V->col(j).get(), alpha[j-1]*beta, *U.col(j) );
}
\end{verbatim}}

\noindent where the \texttt{Vp\_StV(...)} function is the axpy operation for
vectors and is declared in the header
\texttt{TSFCore\-Vector\-StdOps\-Decl.hpp}.  Note that when running the
above BiCG method in an SPMD configuration (where the ANA runs in
parallel and in duplicate in each process) this implementation of
\texttt{update(...)} does
not involve any communication or require any synchronization and
therefore will not affect the performance of the algorithm for a
communication point of view.  However, when running in a master-slave
configuration (where the ANA runs on the master and the linear algebra
runs in the $N_p$ slave process) every method invocation of a method
on a nonlocal TSFCore object involves communication, including each
call to
\texttt{\textit{Multi\-Vector\-::col(j)}}.  While the number of method
invocations on TSFCore objects for all of the other operations shown in
Figure \ref{tsfcore:fig:BiCG_code} are independent of the number of
right-hand-sides $m$, this is not true for the above implementation of
the \texttt{update(...)} function.  However, from a local cache
performance point of view, note that this is a level-1 BLAS operation
so there is no real performance motivation for providing a
multi-vector version.

The reason that this operation is performed column-by-column is that
it is not well supported by the methods
\texttt{\textit{Multi\-Vector\-::applyOp(\-...)}} or
\texttt{\textit{Multi\-Vector\-::apply(\-...)}}.  The problem is that in
the current design of RTOp and
\texttt{\textit{Multi\-Vector\-::applyOp(\-...)}}, an RTOp operator object
does not have any way to distinguish between different columns of a
multi-vector in order to apply different values of $\alpha_{(j)}$ for
each column $j$.  To allow this would require changing the design of
RTOp to deal with multi-vectors directly instead of just individual
vectors.

This operation could be implemented with the
\texttt{\textit{Multi\-Vector\-::apply(\-...)}} method using a
\texttt{\textit{Multi\-Vector}} object
%
\[
A = {\bmat{cccc} \alpha_{(1)} \beta \\ & \alpha_{(2)} \beta \\ & & \ddots \\ & & & \alpha_{(m)} \beta \emat}
\]
%
and then performing
%
\[
U A + V \rightarrow V.
\]
%
But, since it would generally be assumed that the local multi-vector
$A$ is dense, this would likely cost $O(n m^2)$ flops instead of the
$O(n m)$ flops of the actual update operation (where $n$ is the global
number of unknowns in each linear system).

To yield a near-optimal implementation in all computing environments,
this type of update operation would have to be added directly to the
\texttt{\textit{Multi\-Vector}} interface.  However, it is not clear
that this is justified since iterative linear solvers such as this
BiCG method are likely to only run in SPMD mode.

With that said, assuming that the BiCG method shown if Figure
\ref{tsfcore:fig:BiCG_code} is run in SPMD mode, the entire
algorithm only involves three global reductions per BiCG iteration --
independent of the number of linear systems $m$ that are being solved.
These three global reductions include the two multi-vector dot
products on lines 299 and 328 along with a multi-vector norm
calculation for the convergence check which is performed in a
calling function.  The two preconditioner solves on lines 288--289
and the two multi-vector operator applications in lines 322--323
likely involve global communication also, so in general there will be
a total of seven parallel synchronizations per BiCG iteration (or only
five is no preconditioner is used) --- independent of the number of
linear systems being solved.  Therefore, this implementation allows
for near-optimal performance both in terms of minimizing the number of
global synchronizations and in local cache performance (because of the
use of block operations with multi-vectors).

%
\section{General Object-Oriented Software Design Concepts and Principles}
\label{tsfcore:sec:general_software_concepts}
%
 
In this section we discuss some of the basic C++ idioms and design
patterns that have been used to construct the TSFCore C++ classes.  The
primary issues relate to modern approaches to general memory
management for object-oriented programming in C++ and to object
allocation verses initialization.  There is also a short discussion of
proper object-oriented design principles.

The basic design patterns used for memory management in TSFCore are the
``abstract factory'' and the ``prototype'' patterns as described in
the well known ``gang-of-four'' book \cite{ref:gama_et_al_1995}.  When
combined with the C++ idiom of smart reference-counted pointers for
automatic garbage collection (see \cite[Items 28-29]{ref:meyers_1996})
these design patterns become very powerful and greatly help C++
developers to dodge many of the pitfalls of dynamic memory allocation
in C++.  The basic memory management infrastructure is defined in a
namespace called \textit{MemMngPack} which is external to TSFCore.  By
far the most important class in \textit{MemMngPack} (see
\cite{ref:moochodevguide}) is the templated smart reference-counted pointer class
\texttt{ref\_count\_ptr<>}.  This templated class is very close to the
templated class \texttt{shared\_ptr<>} that is provided in the
\texttt{boost} library \cite{ref:boost}.  The use of the class
\texttt{ref\_count\_ptr<>} is described very well in the Doxygen
documentation so it will not be described here.  However, example C++
code that uses this class was shown in the above sections.

All memory management issues associated with abstract objects, which
include instantiations of all of the classes shown in Figure
\ref{tsfcore:fig:tsfl_basic}, are handled using
\texttt{ref\_count\_ptr<>}.  In this way, a client never needs to
explicitly delete any of these objects.  An object will be
automatically deleted once all of the
\texttt{ref\_count\_ptr<>} objects that point to the object go out of
scope.  The methods
\texttt{\textit{VectorSpace\-::createMember()}} and
\texttt{\textit{VectorSpace\-::createMembers(...)}}, as well as may others that
(may) have to allocate new objects, all return pointers to these
objects embedded in \texttt{ref\_count\_ptr<>} objects.  Note that
there are many types of C++ client code, such as functions and
methods, that simply collaborate with preallocated objects for a short
period of time and do not need to assume any responsibilities for
memory management.  In these cases, the reference or raw pointer to the
underlying object can be extracted from the
\texttt{ref\_count\_ptr<>} object which is then passed on to C++ code
that accepts only references or raw pointers.  There are several
examples of this type of usage in the code examples in the previous
sections.

The ``abstract factory'' design pattern (as implemented by
\texttt{\textit{VectorSpace}} for instance) enabled with \texttt{ref\_count\_ptr<>}
effectively relieves clients from having to deal with how objects are
created and destroyed but there is another type of memory management
task that is also required in some use cases.  To describe the
problem, suppose that a C++ client has a handle to a
\texttt{\textit{LinearOp}} object (either through a smart or raw
pointer) and that client wants to copy the object so that some other
client will not modify the object before said client is finished with
the current \texttt{\textit{LinearOp}} object.  This is a classical
problem with the use of objects with {\em reference} (or {\em
pointer}) semantics which does not occur with objects that use {\em
value} semantics \cite{ref:stroustrup_1997}.  This use case requires
the ability to ``clone'' an object which is the basis of the
``prototype'' design pattern.  Every abstract interface shown in
Figure \ref{tsfcore:fig:tsfl_basic} defines some type of
\texttt{\textit{clone()}} method which return
\texttt{ref\_count\_ptr<>} objects pointing to the cloned (or copied)
object.  In some cases the concrete subclass does not have to
override the \texttt{\textit{clone()}} method in order achieve this
functionality (i.e.~\texttt{\textit{Vector}} and
\texttt{\textit{Multi\-Vector}}) while in other cases it does
(i.e.~\texttt{\textit{LinearOp}}).  In cases
where a meaningful default implementation for the
\texttt{\textit{clone()}} method can not be provided, a default implementation
returning a null \texttt{ref\_count\_ptr<>} object is provided.
The implication of this approach is that while the \texttt{\textit{clone()}}
method is a useful feature, it is considered an optional feature where
subclasses are not required to provide an implementation.  However,
every good subclass implementation should provide an implementation of
the \texttt{\textit{clone()}} method since it makes the work of the client much
easier in some use cases.
	
Another set of issues that are related to the memory management issues
described above are issues concerning object allocation verses object
initialization.  Scott Myers \cite{ref:meyers_1996} and others
advocate the ``object initialization on construction'' style of
developing subclasses on the basis that is makes the subclasses easier
to write.  However, this approach is not optimal for the reusability of
a subclass in different use cases from the ones for which the subclass
was originally designed.  To maximize ease of use by clients and
maximize reusability, another style of developing subclasses
``independent object allocation and initialization'' is to be
preferred.  This latter style of developing subclasses is the approach
that is adopted by all of the TSFCore concrete subclasses.  To support
this, every concrete subclass has a default constructor (which
constructs to an uninitialized state) and a set of
\texttt{initialize(...)}  functions that are used to actually
initialize the object.  In order to also support the ``object
initialization on construction'' style (which is useful in many
different cases) there are also a corresponding set of constructors
that call these \texttt{initialize(...)} methods using the same
arguments.  For an example of this style, see the concrete subclass
\texttt{MultiVectorCols} in the Doxygen documentation.

Error handling in TSFCore uses built-in exception handling in C++.
All exceptions thrown by TSFCore code are derived from
\texttt{std::exception}.  Exceptions are thrown using the macro
\texttt{THROW\_EXCEPTION(...)} which results in the \texttt{std::exception::what()}
method containing an error message with the file name and line number
from where the exception was thrown.  This type of information is very
helpful in debugging.  In many cases, armed with just this information
and a good programmer-developed error message, a bug can be found,
diagnosed and fixed without even needing to run a debugger.  The use
of the macro \texttt{THROW\_EXCEPTION(...)} was shown in several of
the above example code snippets.

Finally, a few comments on proper object-oriented design are in order.
It is generally accepted that object-oriented interfaces should be
minimal and every method in an interface should be implementable by
every concrete implementation \cite[Section
24.4.3]{ref:stroustrup_1997}.  However, there are some cases where the
goals of simplicity and strict conformance to this principle of ideal
object-oriented design are at odds.  Finding the proper balance of
simplicity and strict object-oriented correctness requires knowledge,
experience and taste.  In all but one case, the TSFCore interfaces
strictly conform to this ideal principle of object-oriented design.
The one exception is the support of transposed (adjoint) operations.
If an operation may not be supportable by an implementation then the
interface should provide a way for the client to discern this without
having to actually invoke the operation.  This is related to another
principle of proper object-oriented design that absolutely every
interface and method in TSFCore adheres to and this is the principle
that every method should have its preconditions (see
\cite{ref:uml_distilled_2nd_ed} for a decision of pre- and
postconditions) clearly stated and the client should be able to check
the preconditions before the method is called.  Failure to use this
principle makes the use of such software very difficult and results in
a lot of unexpected runtime errors.  If an operation can not be
performed by an object because of the violation of a precondition,
then a good way to handle this is for the method to throw an
exception.  However, proper object-oriented design does not require
this since it is the responsibility of the client to ensure that
preconditions are satisfied (see
\cite{ref:uml_distilled_2nd_ed}).  In practice, however, defensive
programming practices (see \cite{ref:stroustrup_1997}) dictate that
clients should be considered to be unreliable and therefore all
preconditions should be checked by every major method implementation
(at least in a debug build) and if a precondition is found to be
violated then an exception should be thrown which contain a detailed
error message that describes the problem (i.e.~as returned from
\texttt{std::exception::what()}).  If the preconditions are met before
the method is called and the method can not satisfy the postconditions
for some reason then the method should throw an exception in general.
This latter type of exception is the primary reason that exception
handling was added to the C++ standard in the first place
\cite{ref:design_evol_cpp}.

Another desirable principle of object-oriented design is that an
interface should provide declarations for all important methods for
which if specialized implementations for all of these methods were
provided, then the resulting overall software implementation would be
near-optimal with respect to storage and runtime efficiency.  Again,
knowledge, experience and taste are required in the selection of the
appropriate set of methods.  However, there is conflict between the
goals of declaring many methods for the sake of near-optimal
performance and the desire to keep the number of methods to a minimum
to ease subclass development.  The approach that each TSFCore
interface takes to this issue is that the (nearly) full set of methods
needed for a near-optimal implementation are declared in the interface
but reasonable (suboptimal) default implementations are provided for
as many of the methods as possible.  Examples of the application of
this principle are mentioned for every major TSFCore interface (for
example, the default implementation of the
\texttt{\textit{Multi\-Vector}} version of the method
\texttt{\textit{LinearOp\-::apply(\-...)}} which is based on
the \texttt{\textit{Vector}} version).

%
\section{Nonessential but Convenient Functionality Missing in TSFCore}
\label{tsfcore:sec:convenience_functionality}
%

While TSFCore provides all of the functionality required to be
directly used in ANA development it lacks much nonessential but convenient
functionality that is very helpful in developing ANA code.  This
nonessential but convenient functionality can be built on top of the core
functionality which is precisely the type of extra functionality that
TSF and \textit{AbstractLinAlgPack} provide.  In this section, several
different examples of nonessential but convenient functionality are given along
with references to where this functionality exists in TSF and
\textit{AbstractLinAlgPack}.

%
\subsection{Sub-vector views as \texttt{\textit{Vector}} objects}
%

In Section \ref{tsfcore:sec:vec_apply_op}, the use case where the
sub-vectors of a \texttt{\textit{Vector}} object are treated as logical
vector was discussed.  The example in that section got the job done
but a better approach to providing access to sub-vectors is to create a
sub-view decorator subclass (see the ``decorator'' pattern in
\cite{ref:gama_et_al_1995}) that allows the creation of a
\texttt{Vector} view object of a contiguous range of elements in
another \texttt{Vector} object.  Such a subclass is included in
\textit{AbstractLinAlgPack} (see \texttt{VectorSubView} and
\texttt{Vector\-Mutable\-Sub\-View}) and is very useful for high-level ANA
code.  These ``sub-view'' subclasses can be easily implemented through the
\texttt{\textit{Vector\-::applyOp(\-...)}} method.

%
\subsection{Composition of \texttt{\textit{Vector}} and \texttt{\textit{LinearOp}} objects}
\label{tsfcore:sec:composite_abstractions}
%

The ideal way to represent composite blocked or product vector
objects, such as described in Section \ref{tsfcore:sec:vec_apply_op},
is to create a composite blocked or product vector subclass such as
\texttt{TSF\-::TSF\-Product\-Vector} in TSF or
\texttt{Abstract\-Lin\-Alg\-Pack\-::Vector\-Blocked} in
\textit{Abstract\-Lin\-AlgP\-ack}.  Associated with these product
(or blocked) vector subclasses are product vector spaces subclasses.
These subclasses are called
\texttt{TSF\-::TSF\-Product\-Space} in TSF and
\texttt{Abstract\-Lin\-Alg\-Pack\-::Vector\-Space\-Blocked} in
\textit{Abstract\-Lin\-Alg\-Pack}.  These types of composite
product \texttt{\textit{Vector}} and \texttt{\textit{VectorSpace}}
subclasses are easy to develop because of the specification of
\texttt{\textit{Vector\-::applyOp(\-...)}}.

Note that a product vector such as
%
\[
\tilde{x} = {\bmat{c} x_1 \\ x_2 \\ \vdots \\ x_N \emat}
\]
%
with $N$ block vectors is distictly different from a multi-vector
%
\[
Y = {\bmat{cccc} y_1 & y_2 & \ldots & y_N \emat}
\]
%
with $N$ colunns.  In the multi-vector $Y$, each of the column vectors
$y_j$ lie in the same vector space (i.e.~the range space of the linear
operator represented by the multi-vector) which may not be the case
for the vector blocks $x_j$ of $\tilde{x}$ which may lie in distictly
different vector spaces $\mathcal{X}_j$.  While it may seem that the
mathematical differences between a multi-vector and a product vector
are subtle, they are distictly different from a software implication
point of view.  Multi-vectors are ment to represent tall, thin dense
matrices such as for multiple right-hand-sides that are passed to a
linear solver or for performing mulitple linear operator applications
(with the same linear operator) while product vectors and product
vector spaces are ment to represent single vector objects which are
composed of individual vector blocks such as would be used for the
composite unknowns in an SFE method or a multi-period design problem.
For example, a product vector space would be able to create a product
multi-vector such as
%
\[
\tilde{Y} = {\bmat{c} Y_1 \\ Y_2 \\ \vdots \\ Y_N \emat}
\]
%
where each constituent multi-vector $Y_j$ may have a different range
space but all must have the same domain space obviously.  For an
example of a product (or blocked) multi-vector, see
\texttt{Abstract\-Lin\-Alg\-Pack\-::Multi\-Vector\-Mutable\-Blocked}.

Similar generic composition subclasses also exist for linear operators
in TSF (see \texttt{TSF\-::TSF\-Block\-Linear\-Operator} and
\texttt{TSF\-::TSF\-Sum\-Operator}) and \textit{AbstractLinAlgPack} (see
\texttt{Abstract\-Lin\-Alg\-Pack\-::Matrix\-Op\-Blocked} and
\texttt{Abstract\-Lin\-Alg\-Pack\-::Matrix\-Op\-Composite}).  In addition, more
application-specific composite \texttt{\textit{LinearOp}} subclasses
can also be developed (for example, for the SFE system in
\cite{ref:sfe}).

%
\subsection{Matlab-like notation and handle classes for linear algebra
using operator overloading}
\label{tsfcore:sec:operator_overloading}
%

TSFCore contains abstractions for linear algebra objects.  Mathematicians
use a precise syntax to describe linear algebra operations.  Matlab
\cite{ref:matlab} has established a useful convention for mathematical
linear algebra syntax using only ASCII characters.  C++ has operator
overloading.  When you put all of this together it seems obvious, at
first glance, that operator overloading in C++ should be used to
specify linear algebra operations like
%
\[
y = A u + \gamma B^T v + \eta C w
\]
%
in C++ as
%
\begin{verbatim}
    y = A*u + gamma*trans(B)*v + eta*w;
\end{verbatim}

\noindent However, providing a near-optimal implementation (i.e.~no unnecessary
temporaries or multiple memory accesses) of operator overloading for
linear algebra in C++ is nontrivial.  While this type of syntax is
desirable, it does not provide any new functionality and is only
nonessential but convenient functionality and is therefore not included in TSFCore.
An efficient operator overloading mechanism in C++ is hard to
implement and is difficult for C++ novices to debug through.  If
operator overloading is to be built on top of TSFCore (e.g.~using TSF for
instance) then this implementation must be bullet proof and provide
unmatched exception handling so that users must never need to debug
through this code.  TSF has started to implement linear algebra
operations using operator overloading but at the present time only
vector-vector operations are supported.

Closely associated with operator overloading is the concept of handle
classes \cite{ref:advanced_c++_coplien}.  Handles assume the same type
of role as a smart pointers except all of the method forwarding (which
is performed automatically with the operator function
\texttt{ref\_count\_ptr<>\-::operator->()}) must be performed manually
in handle class (which must be written an maintained for every method
on every class by some developer).  Handles make the implementation of
linear algebra operations with operator overloading much easier.
Handles are used extensively in TSF.  Since TSFCore does not implement
operator overloading, handles classify as nonessential but convenient
functionality and are therefore not included in TSFCore.

%
\section{Making the most of TSFCore : Adapters}
\label{tsfcore:sec:adapters}
%

To leverage TSFCore to its fullest benefit, TSFCore should be used as the
standard basic set of linear algebra abstractions that form the basis of
every ANA/LAL and ANA/APP interface.  In addition, every set of
compatible linear algebra interfaces like TSF, HCL and
\textit{AbstractLinAlgPack} should provide adapters to and from TSFCore.
For example, there already exist adapter subclasses that implement the
\textit{AbstractLinAlgPack} interfaces using TSFCore objects
(i.e.~TSFCore-to-\textit{AbstractLinAlgPack}).  There are also adapter
subclasses that implement the TSFCore interfaces using
\textit{AbstractLinAlgPack} objects
(i.e.~\textit{AbstractLinAlgPack}-to-TSFCore).  TSF is built on
top of TSFCore so there is no need for TSFCore/TSF adapters.
If these same set of
adapters are also developed for HCL, and other similar
interfaces, then scenarios such as the
following are possible.

Consider an advanced transient PDE-constrained optimization problem
where the basic PDE constraints are modeled and discretized (in space)
using Sundance \cite{krlong:Sundance}.  Sundance uses TSF for all of
its linear algebra needs.  If the adapters from
TSFCore-to-HCL are available, then the adjoint-sensitivity time
integrator described in \cite{Gockenbach:2002:EAI} could be used to
compute transient adjoint-sensitivities for objective and auxiliary
constraint functions.  Then, with HCL-to-TSFCore and
TSFCore-to-\textit{AbstractLinAlgPack} adapters available, these
adjoint sensitivities to could be used in an optimization algorithm in
MOOCHO.  In turn, MOOCHO may solve for Newton steps with the Hessian
(using a LBFGS matrix as described in Section \ref{tsfcore:sec:LBFGS},
implemented using
\textit{AbstractLinAlgPack}, as a preconditioner) using an iterative
conjugate gradient method as implemented using TSF as provided in
Trilinos.  This would be easy if adapters for
\textit{AbstractLinAlgPack}-to-TSFCore were implemented.
Without going into any more detail about the above optimization
scenario, it should be clear how the adoption of TSFCore as a standard
basic minimal set of linear algebra interfaces would make such
advanced examples of reuse possible.

%
\section{Summary}
%

TSFCore provides the intersection of all of the functionality required
by a variety of abstract numerical algorithms ranging from iterative
linear solvers all the way up to optimizers.  By adapting TSFCore as a
standard interface layer, interoperability between applications,
linear algebra libraries and abstract numerical algorithms can become
a reality.  An extension of the basic TSFCore interfaces for nonlinear
problems is described in \cite{ref:TSFCore::Nonlin}.
