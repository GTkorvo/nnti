% @HEADER
% ***********************************************************************
% 
%            Trilinos: An Object-Oriented Solver Framework
%                 Copyright (2001) Sandia Corporation
% 
% Under terms of Contract DE-AC04-94AL85000, there is a non-exclusive
% license for use of this work by or on behalf of the U.S. Government.
% 
% This library is free software; you can redistribute it and/or modify
% it under the terms of the GNU Lesser General Public License as
% published by the Free Software Foundation; either version 2.1 of the
% License, or (at your option) any later version.
%  
% This library is distributed in the hope that it will be useful, but
% WITHOUT ANY WARRANTY; without even the implied warranty of
% MERCHANTABILITY or FITNESS FOR A PARTICULAR PURPOSE.  See the GNU
% Lesser General Public License for more details.
%  
% You should have received a copy of the GNU Lesser General Public
% License along with this library; if not, write to the Free Software
% Foundation, Inc., 59 Temple Place, Suite 330, Boston, MA 02111-1307
% USA
% Questions? Contact Michael A. Heroux (maherou@sandia.gov) 
% 
% ***********************************************************************
% @HEADER

\section{Interfacing Direct Solvers with Amesos}
\label{chap:amesos}

The Amesos package provides an object-oriented interface to several
direct sparse solvers. Amesos will solve (using a direct factorization
method) the linear systems of equations
\begin{equation}
\label{eq:amesos_ls}
A X = B
\end{equation}
where $A$ is stored as an Epetra\_RowMatrix object, and $X$ and $B$ are
Epetra\_MultiVector objects.

The Amesos package has been designed to face some of the challenges of
direct solution of linear systems. In fact, many solvers have been
proposed in the last years, and often each of them requires different
input formats for the linear system matrix. Moreover, it is not uncommon
that the interface changes between revisions. Amesos aims to solve those
problems, furnishing a clean, consistent interface to many direct
solvers.

Using Amesos, users can interface their codes with a (large) variety of
direct linear solvers, sequential or parallel, simply by a code
instruction of type
\begin{verbatim}
AmesosProblem.Solver();
\end{verbatim}
Amesos will take care of redistributing data among the processors, if
necessary.

Amesos contains several classes: 
\begin{itemize}
\item \verb!Amesos_KLU!: Interface to Amesos's internal solver
  KLU~\cite{KLU}, described in Section~\ref{sec:klu}.
\item \verb!Amesos_Umfpack!: Interface to Tim Davis's
  UMFPACK~\cite{umfpack-home-page}. Section~\ref{sec:umfpack} presents the basic
  functionalities of this class.
\item \verb!Amesos_Dscpack!: Interface to Padma Raghavan's
  DSCPACK~\cite{dscpack-home-page}, presented in Section~\ref{sec:DSCPACK}.
\item \verb!Amesos_Superludist!: Interface to Xiaoye S.~Li's SuperLU
  solver suite, including SuperLU, SuperLU\_DIST 1.0 and SuperLU\_DIST
  2.0~\cite{superlu-home-page}. The SuperLU\_DIST 2.0 interface is
  presented in Section~\ref{sec:superludist}.
\item \verb!Amesos_Mumps!: Interface to MUMPS 4.3.1~\cite{mumps-home-page}, see
  Section~\ref{sec:mumps}.
\item \verb!Amesos_Scalapack!: Interface to ScaLAPACK~\cite{scalapack}, 
  presented in Section~\ref{sec:scalapack}.
\end{itemize}


Note that Amesos does {\sl not} contain interfaces to LAPACK routines.
Other Trilinos packages already offer those routines (Epetra and
Kokkos).

All the Amesos classes are derived from a base class mode,
\verb!Amesos_BaseSolver!. This abstract interface provides the basic
functionalities for all Amesos solvers, and allows users to choose
different direct solvers very easily -- by changing an input scalar
parameter. See Section~\ref{sec:amesos_generic} for more details.

In this Chapter, we will suppose that matrix $A$ in
equation~(\ref{eq:linear_system}) is defined as an Epetra\_RowMatrix, in
principle with nonzero entries on all the processes defined in the
Epetra\_Comm communicator in use. $X$ and $B$, instead, are
Epetra\_MultiVector, defined on the same communicator.  Some of the
supported packages are serial solvers: in this case, if solving with
more than one processor, the linear problem is shipped to processor 0,
solved, then the solution is broadcasted in the solution vector $X$. For
parallel solvers, instead, various options are supported, depending on
the package at hand:
\begin{itemize}
\item The Amesos\_Superludist interface can be used over all the
  processes, as well as on a subset of them. The matrix is kept in
  distributed form over the processes of interest;
\item Amesos\_Mumps can keep the matrix in a distributed form over all
  the processes, or the matrix can be shipped to processor 0. In both
  cases, all the processes in the MPI communicator will be used.
\end{itemize}

This Chapter, we will cover:
\begin{itemize}
\item The installation of the third-part packages required by Amesos, in
  Section~\ref{sec:3pl};
\item The Amesos\_BaseSolver interface to various direct solvers,
  presented (in Section~\ref{sec:amesos_generic}).
\end{itemize}

%%%
%%%
%%%

\subsection{Installation of Third-Party Packages}
\label{sec:3pl}

Amesos is an interface to other packages, mainly developed outside the
Trilinos framework\footnote{Currently, SuperLU is included in the
  Trilinos framework.}. In order to use those packages, the user should
carefully check copyright and licensing of those third party codes.
Please refer to the web page or the documentation of each particular
package for details.

Amesos supports a variety of direct solvers for linear systems of
equations, and you are likely to use Amesos with only few of them. We
suggest to define the shell variable \verb!TRILINOS_3PL!  to define the
directory used to stored third-part packages. For instance, under BASH,
you may have a line of type
\begin{verbatim}
export TRILINOS_3PL=/home/msala/Trilinos3PL
\end{verbatim}
in your \verb!.bashrc! file. Then, you may decide to create a directory
to hold include files and libraries. For instance, to compile under
LINUX with MPI:
\begin{verbatim}
$ mkdir ${TRILINOS_3PL}/LINUX_MPI
$ mkdir ${TRILINOS_3PL}/LINUX_MPI/include
$ mkdir ${TRILINOS_3PL}/LINUX_MPI/lib
\end{verbatim}
(Note that this will reflect the directory structure used by Trilinos,
see Section~\ref{sec:installing}.) While installing a package, you can
now copy all include files and libraries in these directories.

Using this setting, you can configure Amesos with a command of type
\begin{verbatim}
$ cd ${TRILINOS_HOME}/packages/amesos
$ ./configure --prefix=${TRILINOS_HOME}/LINUX_MPI \
  --enable-mpi --with-mpi-compilers \
  --enable-amesos-umfpack \
  --enable-amesos-superludist \
  --with-amesos-superludistlib=\
  "${TRILINOS_3PL}/SuperLU_DIST_2.0/libsuperlu_LINUX.a"
\end{verbatim}
(This command is followed by \verb!make! and \verb!make install!, as
usual.)  This will enable UMFPACK and SuperLU\_DIST, which are the two
packages covered in this Chapter.

For more details about the configuration options of Amesos, please refer
to Amesos documentation.

%%%
%%%
%%%



%%%
%%%
%%%

\subsection{Amesos\_BaseSolver: A Generic Interface to Direct Solvers}
\label{sec:amesos_generic}

All Amesos objects are constructed from the function class
\verb!Amesos_Factory!.  Amesos\_Factory allows a code to delay the
decision about which concrete class to use to implement the
Amesos\_BaseSolver interface. The main goal of this class is to allow
the user to select any supported (and enabled at configuration time)
direct solver, simply changing an input parameter. Another remarkable
advantage of Amesos\_BaseSolver is that, using this class, users does
not have to include the header files of the 3-part libraries in their
code\footnote{Using Amesos\_BaseSolver, 3-part libraries header files
  are required in the compilation of Amesos only.}.

An example of use of this class is as follows. First, the following
header files must be included:
\begin{verbatim}
  #include "Amesos_Factory.h" 
  #include "AmesosClassType.h"
\end{verbatim}
Then, let \verb!A! be an Epetra\_RowMatrix object (for instance, and
Epetra\_CrsMatrix). We need to define a linear problem,
\begin{verbatim}
  Epetra_LinearProblem * Amesos_LinearProblem = 
                         new Epetra_LinearProblem;
  Amesos_LinearProblem->SetOperator( A ) ; 
\end{verbatim}
and to create an Amesos parameter list (which can be empty):
\begin{verbatim}
  Teuchos::ParameterList ParamList ;
\end{verbatim}
Now, let \verb!Choice! be an AmesosClassType variable, with one of the
following values: 
\begin{itemize}
\item AMESOS\_KLU;
\item AMESOS\_UMFPACK;
\item AMESOS\_MUMPS;
\item AMESOS\_SUPERLUDIST;
\item AMESOS\_DSCPACK.
\end{itemize}
We can construct an \verb!Amesos_BaseSolver! object as follows:
\begin{verbatim}
  Amesos_BaseSolver * A_Base;
  Amesos_Factory A_Factory;

  A_Base = A_Factory.Create(Choice, *Amesos_LinearProblem, 
                            ParamList );
  assert(A_Base!=0);
\end{verbatim}
Symbolic and numeric factorizations are computed using methods
\begin{verbatim}
  A_Base->SymbolicFactorization();
  A_Base->NumericFactorization();
\end{verbatim}
The numeric factorization phase will checl whether a symbolic
factorization exists or not. If not, method
\verb!SymbolicFactorization()! is invoked.  Solution is computed (after
setting of LHS and RHS in the linear problem), using
\begin{verbatim}
  A_Base->Solve();
\end{verbatim}
The solution phase will checl whether a numeric factorization exists or
not. If not, method \verb!SymbolicFactorization()! is called.

Users must provide the nonzero structure of the matrix for the symbolic
phase, and the actual nonzero values for the numeric
factorization. Right-hand side and solution vectors must be set before
the solution phase, for instance using
\begin{verbatim}
  Amesos_LinearProblem->SetLHS(x);
  Amesos_LinearProblem->SetRHS(b);
\end{verbatim}

A common ingredient to all the Amesos classes is the Teuchos parameters'
list. This object, whose definition requires the input file
\verb!Teuchos_ParameterList.hpp!, is used to specify the parameters that
affect the 3-part libraries. For a detailed presentation of Teuchos, we
refer to~\cite{Teuchos-home-page}. Here, we simply recall that the
parameters' list can be created as
\begin{verbatim}
  Teuchos::ParameterList AmesosList;
\end{verbatim}
and parameters can be set as
\begin{verbatim}
  AmesosList.setParameter(ParameterName,ParameterValue);
\end{verbatim}
Here, \verb!ParameterName! is a string containing the parameter name,
and \verb!ParameterValue! is any valid C++ object that specifies the
parameter value (for instance, an integer, a pointer to an array or to
an object).

Amesos has to level of parameters: 
\begin{enumerate}
\item a first level refers to parameters that affect all solvers;
\item a second level refers to parameters that are specific to a
  particular solver.
\end{enumerate}


For a detailed list of parameters we refer to the Amesos documentation.