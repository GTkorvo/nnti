%@HEADER
% ************************************************************************
% 
%          Trilinos: An Object-Oriented Solver Framework
%              Copyright (2001) Sandia Corporation
% 
% Under terms of Contract DE-AC04-94AL85000, there is a non-exclusive
% license for use of this work by or on behalf of the U.S. Government.
% 
% This program is free software; you can redistribute it and/or modify
% it under the terms of the GNU General Public License as published by
% the Free Software Foundation; either version 2, or (at your option)
% any later version.
%   
% This program is distributed in the hope that it will be useful, but
% WITHOUT ANY WARRANTY; without even the implied warranty of
% MERCHANTABILITY or FITNESS FOR A PARTICULAR PURPOSE.  See the GNU
% General Public License for more details.
%   
% You should have received a copy of the GNU General Public License
% along with this program; if not, write to the Free Software
% Foundation, Inc., 675 Mass Ave, Cambridge, MA 02139, USA.
% 
% Questions? Contact Michael A. Heroux (maherou@sandia.gov)
% 
% ************************************************************************
%@HEADER

\chapter{Working with Epetra Vectors}
\label{chap:epetra_vec}

Probably, the first mathematical entities defined by a numerical method
is a vector. Within the Trilinos framework, vectors are usually
constructed starting from Epetra Classes.

Epetra vectors can be used to store double values (like the solution of
a PDE problem, the right-hand side of a linear system, or the nodal
coordinates), as well as integer data values (such as a set of indexes).

Epetra vectors can be {\em serial} or {\em distributed}. Serial vectors
are usually small, so that it is not convenient to distribute them
across the processes. Possibly, serial vectors are replicated across the
processes. On the other hand, distributed vectors tend to be
significantly larger, and therefore their elements are distributed
across the processors. In this latter case, users must specify the
partition they intend to use.  In Epetra, this is done by specifying a
communicator (introduced in Section~\ref{sec:comm}) and an Epetra object
called map (introduced in Section~\ref{sec:map}). A map is basically a
partitioning of a list of global IDs.

\medskip

This Chapter will show some of the Trilinos capabilities to work with
vectors. Vector classed can be used to perform common vector operations,
as dot products, vector scalings and norms, or fill with constant or
random values. 

\medskip

During the Chapter, the user be introduced to:
\begin{itemize}
\item The Epetra\_Comm object (in Section~\ref{sec:comm});
\item The Epetra\_Map object (in Section~\ref{sec:map});
\item Creating and assembling Epetra vectors (in
  Sections~\ref{sec:serial_vec} and \ref{sec:distr_vec});
\item Redistributing vectors (in Section~\ref{sec:import_export}).
\end{itemize}

%%%
%%%
%%%

\section{Epetra Communicator Objects}
\label{sec:comm}

The Epetra\_Comm class is an interface that encapsulates the general
information and services needed for the other Epetra classes to run on a
parallel computer. An Epetra\_Comm object is required for building all
Epetra\_Map objects, which in turn are required for all other Epetra
classes.

Epetra\_Comm has two basic implementations:
\begin{itemize}
\item Epetra\_SerialComm (for serial executions);
\item Epetra\_MpiComm (for MPI distributed memory executions).
\end{itemize}

For most basic applications, the user can create an Epetra\_Comm object
using the following code:
\begin{verbatim}
#include "Epetra_config.h"
#ifdef HAVE_MPI
#include "mpi.h"
#include "Epetra_MpiComm.h"
#else
#include "Epetra_SerialComm.h"
#endif
// .. other include files and others ...
int main( int argv, char *argv[]) {
 // .. some declarations here ...
#ifdef HAVE_MPI
  MPI_Init(&argc, &argv);
  Epetra_MpiComm Comm(MPI_COMM_WORLD);
#else
  Epetra_SerialComm Comm;
#endif
// ... other code follows ...
\end{verbatim}
Note that the \verb!MPI_Init()! call and the
\begin{verbatim}
#ifdef HAVE_MPI
  MPI_Finalize();
#endif
\end{verbatim}
call, are likely to be the {\em only} MPI calls users have to explicitly
introduce in their code.

Most of Epetra\_Comm methods are similar to MPI functions. The class
provides methods as \verb!MyPID()!, \verb!NumProc()!, \verb!Barrier()!,
\verb!Broadcast()!, \verb!SumAll()!, \verb!GatherAll()!,
\verb!MaxAll()!, \verb!MinAll()!, \verb!ScanSum()!.  For instance, the
number of processes in the communicator, \verb!NumProc!, and the ID of
the calling process, \verb!MyPID!, can be obtained as
\begin{verbatim}
int NumProc = Comm.NumProc();
int MyPID = Comm.MyPID();
\end{verbatim}

File \TriExe{epetra/ex1.cpp} presents the use of some of the above
introduced functions.  For a description of the syntax, please refer to
the Epetra Class Documentation.

%%%
%%%
%%%

\section{Defining a Map}
\label{sec:map}

Very often, various distributed objects such as matrices or vectors,
have identical distribution of elements among the processes.  This
distribution of elements (or points) is here called a {\sl map}, and its
actual implementation within the Trilinos project is given by the
Epetra\_Map class (or, more generally, by an Epetra\_BlockMap).
Basically, the class handles the definition of:
\begin{itemize}
\item global number of elements (called \verb!NumGlobalPoints!);
\item the local number of elements (called \verb!NumMyPoints!);
\item the global numbering of all local nodes (an integer vector of size
  \verb!NumMyPoints!, called \verb!MyGlobalElements!).
\end{itemize}

There are essentially three ways to define an map. The easiest way is to
specify the global number of elements:
\begin{verbatim}
Epetra_Map Map(NumGlobalPoints,0,Comm);
\end{verbatim}
In this case, the constructor takes the global dimension of the vector
(here indicated as \verb!NumGlobalPoints!), the base index (\verb!0! for
C or C++ arrays, \verb!1! for FORTRAN arrays, but it can be any number),
and an \verb!Epetra_Comm!  object (introduced in
Section~\ref{sec:comm}). As a result, each process will be assigned a
contiguous list of elements.

Another way to build the Epetra\_Comm object is to furnish the local
number of elements:
\begin{verbatim}
Epetra_Map Map(-1,NumMyPoints,0,Comm);
\end{verbatim}
This will create a vector of size $\sum_{i=0}^{NumProc-1}$
\verb!NumMyPoints!. Each process will get a contiguous set of elements.
These two approached are coded in file \TriExe{epetra/ex2.cpp}.

Another, more involved way, to create an Epetra\_Map, is to specify on
each process both the number of local elements, and the global numbering
of each local element. To better explain this, let us consider the
following code, in which a vector, of global dimension 5, is split among
2 processes \verb!p0! and \verb!p1!. \verb!p0! owns nodes 0 an 4, while
\verb!p1! nodes 1, 2, and 3.
\begin{verbatim}
MyPID = Comm.MyPID();
switch( MyPID ) {
case 0:
  MyElements = 2;
  MyGlobalElements = new int[MyElements];
  MyGlobalElements[0] = 0;
  MyGlobalElements[1] = 4;
  break;
case 1:
  MyElements = 3;
  MyGlobalElements = new int[MyElements];
  MyGlobalElements[0] = 1;
  MyGlobalElements[1] = 2;
  MyGlobalElements[2] = 3;
  break;
}

Epetra_Map Map(-1,MyElements,MyGlobalElements,0,Comm);
\end{verbatim}
The complete code is reported in \TriExe{epetra/ex3.cpp}.

A Map object can be queried for the global and local number of elements,
using
\begin{verbatim}
int NumGlobalElements = Map.NumGlobalElements();
int NumMyElements = Map.NumMyElements();
\end{verbatim}
and for the global ID of local elements, using
\begin{verbatim}
int * MyGlobalElements = Map.MyGlobalElements();
\end{verbatim}
or, equivalently,
\begin{verbatim}
int MyGlobalElements[NumMyElements];
Map.MyGlobalElements(MyGlobalElements);
\end{verbatim}

\bigskip

The class Epetra\_Map is derived from Epetra\_BlockMap. This class keeps
information that describes the distribution of objects that have block
elements (for example, one or more contiguous entries of a vector). This
situation is common in applications like multiple-unknown PDE problems.
A variety of constructors are available for this class. An example of
use of block maps is reported in \TriExe{epetra/ex23.cpp}.

\smallskip

Note that different maps can coexist in the same part of the code. This
allows the user to easily define vectors with different distributions
(even for vectors of the same size).  Two classes are provided to
transfer data from one map to an other. Those classes (Epetra\_Import
and Epetra\_Export) are discussed in Section~\ref{sec:import_export}.

\begin{remark}
Most Epetra objects overload the \verb!<<! operator. For example, to
visualize information about the \verb!Map!, one can simply write
\begin{verbatim}
cout << Map;
\end{verbatim}
\end{remark}

This Section has presented the construction of very basic map objects.
However, map objects of very general form can be constructed. First,
element numbers are only labels, and they do not have to be consecutive.
This means that we can define a map with elements 1, 100 and 10000 on
process 0, and elements 2, 200 and 20000 on process 1. This map,
composed by 6 elements, is perfectly legal. Second, each element can be
assigned to more than one process. Examples \TriExe{epetra/ex20.cpp} and
\TriExe{epetra/ex21.cpp} can be used to better understand the potentiality
of Epetra\_Maps.

\begin{remark}
  The use of ``distributed directory'' technology facilitates arbitrary
  global ID support.
\end{remark}

%%%
%%%
%%%

\section{Creating and Assembling Serial Vectors}
\label{sec:serial_vec}

Within Epetra, it is possible to define {\em sequential} vectors, for
serial or for parallel runs. A sequential vector is a vector which, in
the opinion of the programmer, does not need to be partitioned among the
processes.  Note that each process defines its own sequential vectors,
and that changing an element of this vector on this process will {\em
  not} directly affect the vectors stored on other processes (if any
have been defined).

To create a sequential vector containing {\tt Length} elements, one can
use the following command:
\begin{verbatim}
Epetra_SerialDenseVector x(Length);
\end{verbatim}
Other constructors are available; check the Epetra Class
Documentation.

The class Epetra\_SerialDenseVector enables the construction and use of
real-valued, double-precision dense vectors. The
Epetra\_SerialDenseVector class is intended to provide convenient vector
notation but derives all significant functionality from
Epetra\_SerialDenseMatrix class.  The vector can be filled using the
\verb![]! or \verb!()! operators.  Both methods return the specified
element of the vector.  However, using \verb!()!, bounds, checking is
enforced. Using using \verb![]!, no bounds checking is done unless
Epetra is compiled with \verb!EPETRA_ARRAY_BOUNDS_CHECK!.

\begin{remark}
  To construct replicated Epetra objects on distributed memory machines,
  the user may consider the class Epetra\_LocalMap. This class allows
  the constructions of those replicated local objects and keeps
  information that describe the distribution.
\end{remark}

File \TriExe{epetra/ex4.cpp} shows some basic operations on dense vectors.

%%%
%%%
%%%

\section{Creating and Assembling a Distributed Vector}
\label{sec:distr_vec}

To create a distributed vector, the first step is to define a map.
(Actually, this is true for all distributed Epetra objects.) After that,
an Epetra\_Vector object can be constructed with an instruction of type
\begin{verbatim}
Epetra_Vector x(Map);
\end{verbatim}
This constructor allocates space for the vector and set all the elements
to zero. A copy constructor can be used as well:
\begin{verbatim}
Epetra_Vector y(x);
\end{verbatim}
Alternatively, the user can pass a pointer to an array of double
precision values:
\begin{verbatim}
Epetra_Vector x(Copy,Map,LocalValues);
\end{verbatim}
Note the word \verb!Copy! is input to the constructor. Epetra allows two
data access modes:
\begin{enumerate}
\item \verb!Copy! mode: Allocates memory and makes a copy of the
  user-provided data. In this case, the user data is not needed after
  construction;
\item \verb!View! mode: Creates a ``view'' of the user's data. In this
  case, the user data is required to remain untouched for the life of the
  vector. It is worth noting that the View mode is very dangerous from
  a data hiding perspective. Therefore, users are strongly encouraged to
  develop code using Copy mode first and only use View mode in a
  secondary optimization phase. To use the View mode, the user has to
  define the vector entries using a double vector (of appropriate size),
  than construct an Epetra\_Vector with an instruction of type  
\begin{verbatim}
  Epetra_Vector z(View,Map,double_vector);
\end{verbatim}
  where \verb!double_vector! is a pointer to the vector of doubles.
\end{enumerate}

Regardless of how a vector has been created, one can use the \verb![]!
operator to access a vector element:
\begin{verbatim}
x[i] = 1.0*i;
\end{verbatim}
where \verb!i! is in the local index space. 

Epetra also defines some functions to set vector elements in local or
global index space.  \verb!ReplaceMyValues! or \verb!SumIntoMyValues!
will replace or sum values into a vector with a given indexed list of
values, with indexes in the {\em local} index space;
\verb!ReplaceGlobalValues! or \verb!SumIntoGlobalValues! will replace or
sum values into a vector with a given indexed list of values in the {\em
  global} index space. It is important to note that a process {\sl
  cannot} set a vector entries locally owner by another process. In
other words, both global and local insert and replace functions refers
to the part of a vector assigned to the calling process. Intra-process
communications can be performed using Import and Export objects, covered
in Section~\ref{sec:import_export}.

Another way is to put vector values in a user-provided array. For
instance, one may have:
\begin{verbatim}
double *x_values;
x_values = new double[MyLength];
x.ExtractCopy( x_values );
for( int i=0 ; i<MyLength ; ++i ) x_values[i] *= 10;
for( int i=0 ; i<MyLength ; ++i ) x.ReplaceMyValues( 1, 0, x_values+i, &i );
\end{verbatim}
(The complete source is reported in file \TriExe{epetra/ex5.cpp}.)  It
is important to note that \verb!ExtractCopy! does not give access to the
vector elements, but only copies them into the user-provided array.  The
user must commit those changes to the vector object, using, for
instance, \verb!ReplaceMyValues!.

A further, computationally efficient way, is to extract a ``view'' of the
(multi-)vector internal data.  To that aim, one has to call
\begin{verbatim}
double * pointer;
x.ExtractView( &pointer );
\end{verbatim}
Now, modifying the values of \verb!pointer! will affect the internal
data of the Epetra\_Vector \verb!x!.  An example of the use of
\verb!ExtractView! is reported in file \TriExe{epetra/ex6.cpp}.

\begin{remark}
  The class Epetra\_Vector is derived from Epetra\_MultiVector. Roughly
  speaking, a multi-vector is a collection of one or more vectors, all
  having the same length and distribution. The reader may look to the
  file \TriExe{epetra/ex7.cpp} for an example of use of multi-vectors.
\end{remark}

The user can also consider the function \verb!ResetView!, which allows a
(very) light-weight replacement of multi-vector values, created using
the Epetra\_DataMode \verb!View!. Note that no checking is performed to
see if the values passed in contain valid data. This method can be
extremely useful in situation where a vector is needed for use with an
Epetra operator or matrix, and the user is not passing in a
multi-vector. Use this method with caution as it could be extremely
dangerous.
A simple example is reported in \TriExe{epetra/ex8.cpp}

\medskip

It is possible to perform a certain number of operations on vector
objects. Some of them are reported in Table~\ref{tab:distr_vec}.
Example \TriExe{epetra/ex18.cpp} works with some of the functions reported in
the table.

\begin{table}
\begin{center}
\begin{tabular}{ | p{12cm} | }
\hline
\verb!int NumMyELement()! \\
\hspace{1cm} returns the local vector length on the
calling processor \\
\verb!int NumGlobalElements()! \\
\hspace{1cm} returns the  global length\\
\verb!int Norm1(double *Result) const! \\
\hspace{1cm} returns the 1-norm (defined as $\sum_i^n |
  x_i|$ (see also \verb!Norm2! and \verb!NormInf!)\\
\verb!Normweigthed(double *Result) const! \\
\hspace{1cm} returns the  2-norm, defined as
$\sqrt{ \frac{1}{n} \sum_{j=1}^{n} (w_j x_j)^2}$) \\
\verb!int Dot(const Epetra MultiVector A, double *Result) const! \\
\hspace{1cm} computes
the dot product of each corresponding pair of vectors \\
\verb!int Scale(double ScalarA, const Epetra MultiVector &A! \\
\hspace{1cm} 
Replace multi-vector values with scaled values of A,
\verb!this=ScalarA*A! \\
\verb!int MinValue(double *Result) const! \\
\hspace{1cm} compute minimum value of
each vector in multi-vector (see also \verb!MaxValue! and \verb!MeanValue!\\
\verb!int PutScalar(double Scalar)! \\
\hspace{1cm} Initialize all values in a
multi-vector with constant value \\
\verb!int Random()! \\
\hspace{1cm}  set multi-vector values to random numbers \\

\hline
\end{tabular}
\caption{Some methods of the class {\tt Epetra\_Vector}}
\label{tab:distr_vec}
\end{center}
\end{table}

%%%
%%%
%%%

\section{Epetra\_Import and Epetra\_Export}
\label{sec:import_export}

Epetra\_Import and Epetra\_Export are two classes meant for efficient
importing of off-processors elements. Epetra\_Import and Epetra\_Export
are used to construct a communication plan that can be called repeatedly
by computational classes such the Epetra multi-vectors of the Epetra
matrices.

Currently, those classes have one constructor, taking two Epetra\_Map or
Epetra\_BlockMap objects. The first map specifies the global IDs that
are owned by the calling processor. The second map specifies the global
IDs of  elements that we want to import later.

Using an Epetra\_Import object means that the calling process knows what
it wants to receive, while an Epetra\_Export object means that it knows
what it wants to send. An Epetra\_Import object can be used to do an
Export as a reserve operation (and equivalently an Epetra\_Export can be
used to do an Import). In the particular case of bijective maps, either
Epetra\_Import or Epetra\_Export is appropriate.

\medskip

To better illustrate the functionalities of these two classes, we
consider the following example. Suppose that vector \verb!x! of global
length 4, is distributed over two processes. Process 0 own nodes 0,1,2,
while process 1 owns nodes 1,2,3. This means that nodes 1 and 2 are
replicated over the two processes. Suppose that we want to bring all the
components of \verb!x! to process 0, summing up the contributions of
node 1 and 2 from the 2 processes. This is done in the following example
(the complete code is reported in \TriExe{epetra/ex9.cpp}).
\begin{verbatim}
  int NumGlobalElements = 4; // global dimension of the problem

  int NumMyElements; // local nodes
  Epetra_IntSerialDenseVector MyGlobalElements;

  if( Comm.MyPID() == 0 ) {
    NumMyElements = 3;
    MyGlobalElements.Size(NumMyElements);
    MyGlobalElements[0] = 0;
    MyGlobalElements[1] = 1;
    MyGlobalElements[2] = 2;
  } else {
    NumMyElements = 3;
    MyGlobalElements.Size(NumMyElements);
    MyGlobalElements[0] = 1;
    MyGlobalElements[1] = 2;
    MyGlobalElements[2] = 3;
  }

  // create a map
  Epetra_Map Map(-1,MyGlobalElements.Length(),
                 MyGlobalElements.Values(),0, Comm);

  // create a vector based on map
  Epetra_Vector x(Map);
  for( int i=0 ; i<NumMyElements ; ++i )
    x[i] = 10*( Comm.MyPID()+1 );
  cout << x;

  // create a target map, in which all the elements are on proc 0
  int NumMyElements_target;

  if( Comm.MyPID() == 0 )
    NumMyElements_target = NumGlobalElements;
  else
    NumMyElements_target = 0;

  Epetra_Map TargetMap(-1,NumMyElements_target,0,Comm);

  Epetra_Export Exporter(Map,TargetMap);

  // work on vectors
  Epetra_Vector y(TargetMap);

  y.Export(x,Exporter,Add);
  cout << y;
\end{verbatim}

Running this code with 2 processors, the output will be approximatively
the following:
\begin{verbatim}
[msala:epetra]> mpirun -np 2 ./ex31.exe
Epetra::Vector
     MyPID           GID               Value
         0             0                      10
         0             1                      10
         0             2                      10
Epetra::Vector
         1             1                      20
         1             2                      20
         1             3                      20
Epetra::Vector
Epetra::Vector
     MyPID           GID               Value
         0             0                      10
         0             1                      30
         0             2                      30
         0             3                      20
\end{verbatim}

%%%
%%%
%%%

