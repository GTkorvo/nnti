%@HEADER
% ************************************************************************
% 
%          Trilinos: An Object-Oriented Solver Framework
%              Copyright (2001) Sandia Corporation
% 
% Under terms of Contract DE-AC04-94AL85000, there is a non-exclusive
% license for use of this work by or on behalf of the U.S. Government.
% 
% This program is free software; you can redistribute it and/or modify
% it under the terms of the GNU General Public License as published by
% the Free Software Foundation; either version 2, or (at your option)
% any later version.
%   
% This program is distributed in the hope that it will be useful, but
% WITHOUT ANY WARRANTY; without even the implied warranty of
% MERCHANTABILITY or FITNESS FOR A PARTICULAR PURPOSE.  See the GNU
% General Public License for more details.
%   
% You should have received a copy of the GNU General Public License
% along with this program; if not, write to the Free Software
% Foundation, Inc., 675 Mass Ave, Cambridge, MA 02139, USA.
% 
% Questions? Contact Michael A. Heroux (maherou@sandia.gov)
% 
% ************************************************************************
%@HEADER

\section{Eigenvalues and Eigenvectors Computations with Anasazi}
\label{chap:anasazi}

The Anasazi package has been desigend as an extensible and interoperable
framework for large-scale eigenvalue algorithms. Anasazi provides a
generic interface to a collection of algorithms for solving large-scale
eigenvalue problems. Anasazi requires matrices and vectors to be derived
from abstract classes, and therefore it can be used with any matrix and
vector classes that implemenents this interface. In this section, we
present how to use Anasazi with Epetra vectors and matrices.

The algorithm that is currently available through Anasazi is block
implicitly restarted Arnoldi, for symmetric and non-symmetric problems.

In this Chapter we present:
\begin{itemize}
\item the Epetra interface to Anasazi objects (in
  Section~\ref{sec:anasazi_interface});
\item the use of Anasazi to the solution of an eigenvalue problem (in
  Section~\ref{sec:anasazi_solver}).
\end{itemize}

%%%
%%%
%%%

\subsection{Epetra interface to Anasazi Objects}
\label{sec:anasazi_interface}

In order to use the Epetra interface to Anasazi, the user must include
the following files:
\begin{verbatim}
#include "AnasaziPetraInterface.hpp"
#include "AnasaziBlockArnoldi.hpp"
#include "AnasaziConfigDefs.hpp"
#include "AnasaziReturnType.hpp"
#include "AnasaziMultiVec.hpp"
\end{verbatim}
Let \verb!Problem! be an Epetra\_LinearProblem, and \verb!Map! an
Epetra\_BlockMap, \verb!A! an Epetra\_RowMatrix, and \verb!X! an
Epetra\_MultiVector. Anasazi vectors and matrices are defined as
follows:
\begin{verbatim}
Anasazi::PetraVec<double> Vector(X);
Anasazi::PetraMat<double> Mat(A);
\end{verbatim}
The user can also create Anasazi vectors that do no rely on any
Epetra\_MultiVector's, as
\begin{verbatim}
Anasazi::PetraVec<double> ivec(Map, block);
\end{verbatim}

%%%
%%%
%%%

\subsection{Solving Eigenvalue Problems}
\label{sec:anasazi_solver}

Anasazi requires the definition of an Eigenproblem object, that
specifies the linear system matrix and the right-hand side, as
\begin{verbatim}
Anasazi::Eigenproblem<double> MyProblem(&Amat, &ivec);
Anasazi::BlockArnoldi<double> MyBlockArnoldi(Problem, tol, nev, 
                                             length, block, 
                                             which, step, restarts);
\end{verbatim}
Before solving the eigenproblem, the user must define a set of
parameters. For instance, one might have:
\begin{verbatim}
int block = 1;
int length = 20;
int nev = 4;
double tol = 1.0e-14;
string which="SM";
int restarts = 300;
int step = restarts*length*block;
MyBlockArnoldi.setSymmetric(true);
MyBlockArnoldi.setDebugLevel(0);
\end{verbatim}
Here, \verb!SM! means that we are interested in the \verb!nev!
eigenvalues with smallest magniture. Other choices are reported in
Table~\ref{tab:anasazi}. (Please refer to the Anasazi user's guide more
a comprehensive list.)


\begin{table}
\begin{center}
\begin{tabular}{| p{2cm} l |}
\hline
option & action \\
\hline

{\tt SM} & Sort eigenvalues in increasing order of magnitude \\

{\tt SR} & Sort eigenvalues in increasing order of real part \\

{\tt SI} & Sort eigenvalues in increasing order of imaginary part \\

{\tt LM} & Sort eigenvalues in decreasing order of magnitude \\

{\tt LR} & Sort eigenvalues in decreasing order of real part \\

{\tt LI} & Sort eigenvalues in decreasing order of imaginary part \\
\hline
\end{tabular}
\caption{Options for Anasazi.}
\label{tab:anasazi}
\end{center}
\end{table}


The eigenproblem is solved with the instruction
\begin{verbatim}
MyBlockArnoldi.solve();
\end{verbatim}
A summary of results can be printed on the screen, simply typing
\begin{verbatim}
MyBlockArnoldi.currentStatus();
\end{verbatim}
Eigenvectors and eigenvalues can be retrived using
\begin{verbatim}
double * resids = MyBlockArnoldi.getResiduals();
double * evalr = MyBlockArnoldi.getEvals(); 
double * evali = MyBlockArnoldi.getiEvals();

Anasazi::PetraVec<double> evecr(Map, nev);
MyBlockArnoldi.getEvecs( evecr );
Anasazi::PetraVec<double> eveci(Map, nev);
MyBlockArnoldi.getiEvecs( eveci )
\end{verbatim}

%%%
%%%
%%%




\subsection{Concluding Remarks on Anasazi}
\label{sec:anasazi_concluding}

More documentation on the Anasazi package can be found in
\cite{Anasazi-Ref-Guide,Anasazi-User-Guide}.

