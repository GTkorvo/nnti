% @HEADER
% ***********************************************************************
% 
%            Trilinos: An Object-Oriented Solver Framework
%                 Copyright (2001) Sandia Corporation
% 
% Under terms of Contract DE-AC04-94AL85000, there is a non-exclusive
% license for use of this work by or on behalf of the U.S. Government.
% 
% This library is free software; you can redistribute it and/or modify
% it under the terms of the GNU Lesser General Public License as
% published by the Free Software Foundation; either version 2.1 of the
% License, or (at your option) any later version.
%  
% This library is distributed in the hope that it will be useful, but
% WITHOUT ANY WARRANTY; without even the implied warranty of
% MERCHANTABILITY or FITNESS FOR A PARTICULAR PURPOSE.  See the GNU
% Lesser General Public License for more details.
%  
% You should have received a copy of the GNU Lesser General Public
% License along with this library; if not, write to the Free Software
% Foundation, Inc., 59 Temple Place, Suite 330, Boston, MA 02111-1307
% USA
% Questions? Contact Michael A. Heroux (maherou@sandia.gov) 
% 
% ***********************************************************************
% @HEADER

\section{Introduction}

The Trilinos Project is an effort to facilitate the design, development,
integration and ongoing support of mathematical software libraries.  The
goal of the Trilinos Project is develop parallel solver algorithms and
libraries within an object-oriented software framework for the solution
of large-scale, complex multiphysics engineering and scientific
applications. The emphasis is on developing robust, scalable algorithms
in a software framework, using abstract interfaces for flexible
interoperability of components while providing a full-featured set of
concrete classes that implement all the abstract interfaces.

%%%
%%%
%%%

\subsection{Getting Started}
\label{sec:getting}

The Trilinos Project uses a two-level software structure designed around
collections of packages. A Trilinos package is an integral unit, usually
developed to solve a specific task, by a (relatively) small group of
experts.  Packages exist underneath the Trilinos top level, which
provides a common look-and-feel. Each package has its own structure,
documentation and set of examples. A Trilinos packages may also be
available as an independent package. However, each package is even more
valuable when combined with other Trilinos packages.

\smallskip

Trilinos is a large software project, and currently includes about
twenty packages are included. Fully understanding all the
functionalities of the Trilinos packages requires time. The entire set
of packages covers a wide range of numerical methods for large scale
computing, as well as a large set of utilities to improve the
development of software for scientific computing.

Each package offers sophisticated features, that cannot be ``unleashed''
initially.  For each package, we will outline only the basic features,
and we refer to the documentation of each package for a more involved
usage. Our goal is to present enough material so that the reader can
successfully use the described packages.  In fact, for new users, it is
neither easy, nor necessary, to manage all the Trilinos functionalities.
At the beginning, it is more important to understand how to manipulate
the basic classes, such as vector, matrix and linear system classes.
However, for fine-tuning, users still must look through individual
package's documentation and examples.

\medskip

We will describe the following subset of the Trilinos packages.
\begin{itemize} 
\item {\bf Epetra}. The package defines the basic classes for
  distributed matrices and vectors, linear operators and linear
  problems. Epetra classes are the common language spoken by all the
  Trilinos packages (even if some packages can ``speak'' other
  languages). Each Trilinos package accepts as input Epetra objects.
  This allows powerful combinations among the various Trilinos
  functionalities.
\item {\bf AztecOO}. This is a linear solve package based on
  preconditioned Krylov methods. It supports all the Aztec interfaces
  and functionality, but also provides significant new functionality.
\item {\bf IFPACK}. The package performs various incomplete
  factorizations, and is here used with AztecOO.
\item {\bf ML}. The algebraic multilevel preconditioner package provides
  scalable preconditioning capabilities for a variety of problem
  classes. It is here used with AztecOO.
\item {\bf Amesos}. The package provides a common interface to certain
  sparse direct linear solvers (generally available outside the Trilinos
  framework), both sequential and parallel.
\item {\bf Anasazi}. The package provides a common interface to
  parallel eigenvalues and eigenvectors solvers, for both symmetric and
  non-symmetric linear problems.
\item {\bf NOX}. This is a collection of nonlinear solvers, designed to
  be easily integrated into an application and used with many different
  linear solvers.
\item {\bf Teuchos}. This is a collection of classes that can be
  essential for code development.
\item {\bf Triutils}. This is a collection of utilities, that
  can be useful in some phases of software development.
\end{itemize}

Table~\ref{tab:tripackages} gives a partial overview of what can be
accomplished using Trilinos.
\begin{table}[htbp]
  \centering
  \begin{tabular}{| p{10cm} | p{3cm} |}
    \hline
    {\bf Task} & {\bf Package} \\
    \hline
    Light-weight interface to BLAS and LAPACK: & Epetra, Teuchos \\\hline
    Definition of serial dense or sparse matrices: & Epetra \\\hline
    Utilities for Epetra vectors and sparse matrices: & EpetraExt \\\hline
    Definition of distributed sparse matrices:& Epetra \\\hline
    Solve a linear system with preconditioned Krylov accelerators, like
    CG, GMRES, Bi-CGSTAB, TFQMR:& AztecOO, Belos$^\star$ \\\hline
    Definition of incomplete factorizations:& AztecOO, \newline IFPACK \\\hline
    Definition of a multilevel preconditioner:& ML \\\hline
    Definition of a one-level Schwarz preconditioner (overlapping domain
    decomposition):& AztecOO, \newline IFPACK \\\hline
    Definition a two-level Schwarz preconditioner, with coarse grid based on
    aggregation:& AztecOO+ML \\\hline
    Solution of  systems of nonlinear equations:& NOX \\\hline
    Interface with various direct solvers, as UMFPACK, MUMPS, SuperLU
    and others :& Amesos \\\hline
    Computation of eigenvalue of large, sparse matrices:& Anasazi
    \\\hline
    Solution of complex linear equations (using equivalent real formulation):&
    Komplex$^\star$ \\\hline
    Definition of segregated preconditioners and block preconditioners (for
    instance, for the incompressible Navier-Stokes equations):&
    Meros$^\star$ \\\hline
    Templated interface to BLAS and LAPACK, arbitrary-precision
    arithmetic, parameters; list, smart pointers:& Teuchos \\\hline
    Definition of abstract interfaces to vectors, linear operators, and solvers:& TSF$^\star$, TSFCore$^\star$, TSFExtended$^\star$    \\
    \hline
  \end{tabular}
  \caption{Partial overview of intended uses of  Trilinos. $\star$:
    not covered in this tutorial.}
  \label{tab:tripackages}
\end{table}

\begin{remark}
  As already pointed out, Epetra objects are meant to be the ``common
  language'' spoken by all the Trilinos packages, and are a natural
  starting point for new users. Therefore we suggest to read Chapters
  \ref{chap:epetra_vec}-\ref{chap:epetra_others} before considering
  other Trilinos packages. Also, Chapter~\ref{chap:aztecoo} should be
  read before Chapters~\ref{chap:ifpack} and~\ref{chap:ml} (even if both
  IFPACK and ML can be compiled and run without AztecOO).
\end{remark}

The only prerequisites assumed in this tutorial are some familiarities
with numerical methods for PDEs, and iterative linear and nonlinear
solvers. Although not strictly necessary, the reader is assumed to have
some familiarity with distributed memory computing and, to a lesser
extent, with MPI.

\smallskip

Note that this tutorial is not a substitute for individual packages'
documentation. Also, for an overview of all the Trilinos packages, the
Trilinos philosophy, and a description of the packages provided by
Trilinos, the reader is referred to \cite{Trilinos-Overview}.
Developers should also consider the Trilinos Developers' Guide, which
addresses many topics, including the development tools used by Trilinos'
developers, and how to include a new package\footnote{ Trilinos provides
  a variety of services to a developer wanting to integrate a package
  into Trilinos.  The services include Autoconf~\cite{Autoconf},
  Automake~\cite{Automake} and Libtool~\cite{Libtool}. The tools provide
  a robust, full-featured set of tools for building software across a
  broad set of platforms.  The tools are not officially standards, but
  are widely used.  All existing Trilinos packages use Autoconf and
  Automake.  Libtool support will be added in future releases.}.

%%%
%%%
%%%

\subsection{Installation}
\label{sec:installing}

To obtain Trilinos, please refers to the instructions reported at the
following web site:
\begin{verbatim}
http://software.sandia.gov/Trilinos
\end{verbatim}

Trilinos has been compiled on a variety of architectures, including
Linux, Sun Solaris, SGI Irix, DEC, and many others. Trilinos has been
designed to support parallel applications. However, it can be compiled
and run on serial computer.  Detailed comments on the installation, and
an exhaustive list of FAQs, can be found at the web pages:
\begin{verbatim}
http://software.sandia.gov/Trilinos/installing_manual.html
http://software.sandia.gov/Trilinos/faq.html
\end{verbatim}


Suppose that we will compile on a LINUX platform with MPI. Here, we {\sl
  suggest} to set the environmental
variables \verb!TRILINOS_HOME!, indicating the full path of the Trilinos
directory, \verb!TRILINOS_LIB!, indicating the location of the compiled
Trilinos library, and \verb!TRILINOS_ARCH!, containing the architecture
and the communicator currently used.  For example, using the BASH shell,
command lines of the form
\begin{verbatim}
export TRILINOS_HOME=/home/msala/Trilinos
export TRILINOS_ARCH=LINUX.MPI
export TRILINOS_LIB=${TRILINOS_HOME}/${TRILINOS_ARCH}
\end{verbatim}
can be places in the users' \verb!.bashrc! file.

\smallskip

Each Trilinos' package can be enabled or disabled at configuration time.
A procedure one may follow in order to compile Trilinos with AztecOO,
ML, IFPACK, ANASAZI, NOX, and triutil is reported below. \%indicates the
shell prompt. The \verb!tee! command is used to write the output to
standard output and to the specified file, and can be omitted.  More
details about the installation of Trilinos can be found in
\cite{Trilinos-Users-Guide}.
\begin{verbatim}
% cd ${TRILINOS_HOME}
% mkdir ${TRILINOS_ARCH}
% cd ${TRILINOS_ARCH}
% ../configure --prefix=${TRILINOS_HOME}/${TRILINOS_ARCH} \
  --enable-mpi --with-mpi-compilers \
  --enable-triutils --enable-aztecoo \
  --enable-ifpack --enable-anasazi \
  --enable-ml --enable-nox | tee configure_${TRILINOS_ARCH}.log
% make | tee make_${TRILINOS_ARCH}.log
% make install | tee make_install_${TRILINOS_ARCH}.log
\end{verbatim}
For serial configuration, set
\begin{verbatim}
export TRILINOS_ARCH=LINUX.SERIAL
\end{verbatim}
and delete the \verb!--enable-mpi --with-mpi-compilers! options.
\begin{remark}
  All Trilinos packages can be build to run with or without MPI. If MPI
  is enabled (using \verb!--enable-mpi!), the users must know the
  procedure for beginning MPI jobs on their computer system(s). If maybe
  necessary to specify on the configure line the location of MPI include
  files and libraries.
\end{remark}

%%%
%%%
%%%

\subsection{Compiling and Linking a program using Trilinos}
\label{sec:intro_compiling}

In order to compile and link a code that makes use of Trilinos, the user
may decide to use a Makefile similar to the one reported below. 
Suppose that our code requires Epetra, AztecOO and IFPACK, and that
the procedure outlined in Section~\ref{sec:installing} has been followed.
The UNIX makefile can be as follows:
\begin{verbatim}
 1: TRILINOS_HOME = /home/msala/Trilinos/
 2: TRILINOS_ARCH - LINUX_MPI
 3: TRILINOS_LIB = $(TRILINOS_HOME)$(TRILINOS_ARCH)
 4: 
 5: include $(TRILINOS_HOME)/build/makefile.$(TRILINOS_ARCH)
 6: 
 7: MY_COMPILER_FLAGS = -DHAVE_CONFIG_H $(CXXFLAGS) -c -g\
 8:                    -I$(TRILINOS_LIB)/include/
 9:
10: MY_LINKER_FLAGS = $(LDFLAGS) $(TEST_C_OBJ) \
11:         -L$(TRILINOS_LIB)/lib/ \
12:         -lifpack -laztecoo -lepetra $(ARCH_LIBS)
14:
15: ex1: ex1.cpp
16:         $(CXX)     ex1.cpp $(MY_COMPILER_FLAGS)
17:         $(LINKER)  ex1.o   $(MY_LINKER_FLAGS)    -o ex1.exe
\end{verbatim}

Line number are reported for reader's convenience.

Here we assume that the reader knows about UNIX makefiles.  Line 5
includes basic definitions of Trilinos. (Note that, on some
architectures, one may need to use \verb!gmake! instead of \verb!make!.)
In line 7, the variable \verb!HAVE_CONFIG_H! is defined. Linker flags of
lines 10-13 defines the library to link (other libraries may be needed).
The variable \verb!ARCH_LIBS! is defined in line 5.

The name of the executable is \verb!ex1.exe!. 
The command to run this example on 2 processors is often
\begin{verbatim}
$ mpirun -np 2 ./ex1.exe
\end{verbatim}
Please check the local MPI documentation for more details. 

%%%
%%%
%%%

\subsection{Copyright and Licensing of Trilinos}
\label{sec:copyright}

Trilinos is released under the Lesser GPL GNU Licence.

Trilinos is copyrighted by Sandia Corporation. Under the terms of
Contract DE-AC04-94AL85000, there is a non-exclusive license for use of
this work by or on behalf of the U.S. Government.  Export of this
program may require a license from the United States Government.

NOTICE: The United States Government is granted for itself and others
acting on its behalf a paid-up, nonexclusive, irrevocable worldwide
license in ths data to reproduce, prepare derivative works, and perform
publicly and display publicly.  Beginning five (5) years from July 25,
2001, the United States Government is granted for itself and others
acting on its behalf a paid-up, nonexclusive, irrevocable worldwide
license in this data to reproduce, prepare derivative works, distribute
copies to the public, perform publicly and display publicly, and to
permit others to do so.

NEITHER THE UNITED STATES GOVERNMENT, NOR THE UNITED STATES DEPARTMENT
OF ENERGY, NOR SANDIA CORPORATION, NOR ANY OF THEIR EMPLOYEES, MAKES ANY
WARRANTY, EXPRESS OR IMPLIED, OR ASSUMES ANY LEGAL LIABILITY OR
RESPONSIBILITY FOR THE ACCURACY, COMPLETENESS, OR USEFULNESS OF ANY
INFORMATION, APPARATUS, PRODUCT, OR PROCESS DISCLOSED, OR REPRESENTS
THAT ITS USE WOULD NOT INFRINGE PRIVATELY OWNED RIGHTS.

\medskip

Some parts of Trilinos are dependent on a third party code. Each third
party code comes with its own copyright and/or licensing requirements.
It is responsibility of the user to understand these requirements.

%%%
%%%
%%%

\subsection{Programming Language Used in this Tutorial}
\label{sec:language}

Trilinos is written in C++ (for most packages), and in C. Some
interfaces are provided to FORTRAN code (mainly BLAS and LAPACK
routines). Even if limited support is included for C programs (and a
more limited for FORTRAN code), to unleash the full power of Trilinos we
recommend C++. All the example programs contained in this tutorial are
in C++; some packages contain examples in C.

%%%
%%%
%%%

\subsection{Referencing Trilinos}
\label{sec:referencing}

The Trilinos project can be referenced by using the following BiBTeX
citation information:
\begin{verbatim}
@techreport{Trilinos-Overview,
title = "{An Overview of Trilinos}",
author = "Michael Heroux and Roscoe Bartlett and Vicki Howle
Robert Hoekstra and Jonathan Hu and Tamara Kolda and
Richard Lehoucq and Kevin Long and Roger Pawlowski and
Eric Phipps and Andrew Salinger and Heidi Thornquist and
Ray Tuminaro and James Willenbring and Alan Williams ",
institution = "Sandia National Laboratories",
number = "SAND2003-2927",
year = 2003}

@techreport{Trilinos-Dev-Guide,
title = "{Trilinos Developers Guide}",
author = "Michael A. Heroux and James M. Willenbring and Robert Heaphy",
institution = "Sandia National Laboratories",
number = "SAND2003-1898",
year = 2003}

@techreport{Trilinos-Dev-Guide-II,
title = "{Trilinos Developers Guide Part II: ASCI Software Quality
Engineering Practices Version 1.0}",
author = "Michael A. Heroux and James M. Willenbring and Robert Heaphy",
institution = "Sandia National Laboratories",
number = "SAND2003-1899",
year = 2003}

@techreport{Trilinos-Users-Guide,
title = "{Trilinos Users Guide}",
author = "Michael A. Heroux and James M. Willenbring",
institution = "Sandia National Laboratories",
number = "SAND2003-2952",
year = 2003}
\end{verbatim}
The BiBTeX information is available at the web page
\begin{verb}
http://software.sandia.gov/Trilinos/citing.html
\end{verb}

%%%
%%%
%%%

\subsection{A Note on the Directory Structure}
\label{sec:into_note}

Each Trilinos package in contained in the subdirectory
\begin{verbatim}
${TRILINOS_HOME}/packages
\end{verbatim}
Each package contains sources, examples, tests and documentation subdirectories:
\begin{verbatim}
${TRILINOS_HOME}/packages/<package-name>/src
${TRILINOS_HOME}/packages/<package-name>/examples
${TRILINOS_HOME}/packages/<package-name>/test
${TRILINOS_HOME}/packages/<package-name>/doc
\end{verbatim}
Trilinos developers use Doxygen\footnote{Copyright \copyright 1997-2003
  by Dimitri van Heesch. More information can by found at the web
  address {\tt http://www.stack.nl/~dimitri/doxygen/}.}. The command
to create For instance, to create the documentation for Epetra are
\begin{verbatim}
$ cd ${TRILINOS_HOME}/packages/epetra/doc
Generally, both HTML and \LaTeX documentation is created by Doxygen. To
compile the \LaTeX sources, the commands are:
\begin{verbatim}
$ cd ${TRILINOS_HOME}/packages/epetra/doc/latex
$ make
\end{verbatim}

%%%
%%%
%%%

\subsection{List of Trilinos Developers}
\label{sec:intro_incomplete}

The Trilinos' developers as of May 2004 are (in alphabetical order):

Roscoe A. Bartlett,
Jason A. Cross,
David M. Day,
Robert Heaphy,
Michael A. Heroux (project leader),
Russell Hooper,
Vicki E. Howle,
Robert J. Hoekstra,
Jonathan J. Hu,
Tamara G. Kolda,
Richard B. Lehoucq,
Paul Lin,
Kevin R. Long,
Roger P. Pawlowski,
Michael N. Phenow,
Eric T. Phipps,
Andrew J. Rothfuss,
Marzio Sala,
Andrew G. Salinger,
Paul M. Sexton,
Kendall S. Stanley,
Heidi K. Thornquist,
Ray S. Tuminaro,
James M. Willenbring, and
Alan Williams.

