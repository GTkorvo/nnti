%
\section{\texttt{RefCountPtr<>} and multiple inheritance and virtual base classes}
\label{rcp:apdx:mivbc}
%

In this section we discuss issues related to the proper deletion of
objects that are from types that use virtual base classes and why it
is important to follow Commandment \ref{rcp:cmd:rcp-new} involving the
use of the template function {}\texttt{rcp()} when using
{}\texttt{new} to allocate an object to construct a
{}\texttt{RefCountPtr<>} object.  The issue considered here should be
handled properly by ANSI C++ standard implementations and should not
necessitate the use of template function {}\texttt{rcp()} but there
are older C++ implementations that do not handle the deletion of these
types of objects properly therefore requiring the use of
{}\texttt{rcp()}.  Here we describe what the issues are using a
concrete example and describe why some C++ implementations have
trouble getting this correct and how {}\texttt{rcp()} fixes the
problem.

Consider the following class hierarchy:

{\small\begin{verbatim}
class A {
  private: int A_g_, A_f_;
  public:  A(); virtual ~A(); virtual int A_g(); virtual int A_f();
};

class B1 : virtual public A {
  private: int B1_g_, B1_f_;
  public:  B1(); virtual int B1_g(); virtual int B1_f();
};

class B2 : virtual public A {
  private: int B2_g_, B2_f_;
  public:  B2(); virtual int B2_g(); virtual int B2_f();
};

class C : virtual public B1, virtual public B2
{
  private: int C_g_, C_f_;
  public:  C(); virtual int C_g(); virtual int C_f();
};
\end{verbatim}}

An object of type C should have one type B1 object, one type B2 object
and one (since A is a virtual base class) type A object.  Each of
these objects must have at least storage for two {}\texttt{int} data
members.  On a 32 bit system, each {}\texttt{int} should require four
bytes.  In addition, each object should have a pointer to a virtual
function table which on a 32 bit system is four bytes per pointer
which is the same size as an {}\texttt{int}.  Therefore, an object of
type {}\texttt{C} should have a minimum size (as returned by
{}\texttt{sizeof(C)}) of

\[
(4 \, \mbox{objects per C object})(3 \, \mbox{ints per object})(4 \, \mbox{bytes per int})
= 48 \, \mbox{bytes per C object}.
\]

While the layout of objects is compiler dependent, many C++
implementations sets up objects in memory in this manner.

A simple test program (this is the test program for
{}\texttt{RefCountPtr<>} in Teuchos) was written to reveal the layout of
an object of type {}\texttt{C} in memory.  When run under the g++
version 3.2 the test program output reveals the following structure

{\small\begin{verbatim}
Size of C = 48
Base address of object of type C        = 0xa040948
Offset to address of object of type C   = 0
Offset of B1 object in object of type C = 12
Offset of B2 object in object of type C = 36
Offset of A object in object of type C  = 24
\end{verbatim}}

{}\noindent{}This printout shows that an object of type {}\texttt{C}
is laid out in memory as shown in Figure \ref{rcp:fig:layout-of-C}.

\begin{figure}[h]
\begin{center}
\begin{tabular}{lc}
address 10000 & \framebox[10ex]{C base} \\
address 10012 & \framebox[10ex]{B1 base} \\
address 10024 & \framebox[10ex]{A base} \\
address 10036 & \framebox[10ex]{B2 base}
\end{tabular}
\end{center}
\caption{\label{rcp:fig:layout-of-C}
Layout of object of type {}\texttt{C} using g++ version 3.2
with a base address of (base-10) 10000 for example.}
\end{figure}

Why is the layout in Figure \ref{rcp:fig:layout-of-C} significant?  To
understand why this is important, consider how a C++ implementation is
likely to allocate and delete objects of type {}\texttt{C}.  The way
that an object of type {}\texttt{C} is usually constructed by a C++
implementation is to call {}\texttt{malloc(sizeof(C))} first to allocate
storage for the object and then the sub-objects in
{}\texttt{C} are constructed according to the standard (i.e.~base-class
objects first, followed by subclass objects).

Now consider the following program:

{\small\begin{verbatim}
void delete_A(A* a_ptr)
{
    delete a_ptr;
}

void main()
{
    A *a_ptr = new C;
    delete_A(a_ptr);
}
\end{verbatim}}

Some older C++ implementations (most notably Microsoft Visual C++
version 6.0) implement the above {}\texttt{delete} call in the
function {}\texttt{delete\_A()} as follows:

{\small\begin{verbatim}a_ptr->~a_ptr();
free((void*)a_ptr);
\end{verbatim}}

{}\noindent{}and the program would crash with a runtime error from
{}\texttt{free()} complaining that this memory was not allocated from
{}\texttt{malloc()}.  The reason for the runtime error given the above
(incorrect) implementation is obvious if one thinks about what is
occurring.  Let us assume that the runtime allocates memory for the
object {}\texttt{C} on the line

{\small\begin{verbatim}
A *a_ptr = new C;
\end{verbatim}}

{}\noindent{}with the address 10000 returned from {}\texttt{malloc()}
as shown in Figure \ref{rcp:fig:layout-of-C}.  When the pointer
variable {}\texttt{a\_ptr} is assigned, the address is changed to
100024 as also shown in Figure \ref{rcp:fig:layout-of-C}.  When this
object is passed into the function {}\texttt{delete\_A()} the above
(incorrect) implementation first performs

{\small\begin{verbatim}
a_ptr->~a_ptr();
\end{verbatim}}

{}\noindent{}and the correct destructors are called because this
utilizes the virtual function mechanism.  However, when the next line

{\small\begin{verbatim}
free((void*)a_ptr);
\end{verbatim}}

{}\noindent{}is executed it is equivalent to

{\small\begin{verbatim}
free((void*)10024);
\end{verbatim}}

{}\noindent{}which is not the same address 10000 returned from
{}\texttt{malloc()} and hence the runtime error.

A fully ANSI/ISO C++ standard implementation must therefore perform
some type of runtime lookup of an object to determine its base address
before it calls {}\texttt{free()}.  Hence we see that a proper C++
implementation of {}\texttt{delete} for an object with a virtual
function must always perform a runtime type lookup to execute
correctly and this is unavoidable extra overhead.

The reason that the simplistic (incorrect) implementation (such as
implemented in MS Visual C++ 6.0) shown above generally works just
fine is that if virtual inheritance is not used, then the address for
every interface in an object is the same as the base address for the
entire object in almost every C++ implementation (and certainly for MS
Visual C++ 6.0).  And therefore any interface supported by the object
can be used to delete the object since the addresses are the same.

Now let us consider what a {}\texttt{RefCountPtr<>} implementation of
the above program would look like and how it works even with the
incorrect implementation of {}\texttt{delete} (e.g.~as used in MS Visual
C++ 6.0).

{\small\begin{verbatim}
void delete_A_ptr(Teuchos::RefCountPtr<A>* a_ptr)
{
    *a_ptr = Teuchos::null;
}

void main()
{
    Teuchos::RefCountPtr<A> a_ptr = Teuchos::rcp(new C);
    delete_A_ptr(&a_ptr);
}
\end{verbatim}}

In the above example, the line

{\small\begin{verbatim}
Teuchos::RefCountPtr<A> a_ptr = Teuchos::rcp(new C);
\end{verbatim}}

{}\noindent{}results in the templated function {}\texttt{rcp()} first
creating a {}\texttt{RefCountPtr<C>} object which stores the exact
same address as returned from {}\texttt{new C} in the underlying
reference-count node and then the {}\texttt{RefCountPtr<C>} object is
converted into a {}\texttt{RefCountPtr<A>} object before assignment is
performed (or just a copy constructor in most C++ implementations).
See Appendix {}\ref{rcp:apdx:design} for a description of how this
works.  Then when the line

{\small\begin{verbatim}
*a_ptr = Teuchos::null;
\end{verbatim}}

{}\noindent{}in the function {}\texttt{delete\_A\_ptr()} is executed,
the destructor for the underlying reference-count node calls
{}\texttt{delete} on the exact same address returned from
{}\texttt{new C} which was set when the first
{}\texttt{RefCountPtr<C>} object was constructed by {}\texttt{rcp(new
C)}.  Since this is the same address returned form {}\texttt{new} (and
therefore the same returned from {}\texttt{malloc()}) the incorrect
compiler implementation (i.e.~MS VC++ 6.0) works just fine.  The key
to the use of the templated function {}\texttt{rcp()} is that the
compiler automatically constructs a {}\texttt{RefCountPtr<>} with the
proper type without the programmer having to think about it.

New versions of C++ compilers do not seem to suffer from this problem
and the above referenced g++ version performs the above deletion
correctly as well as several other compilers tested.  At the time of
this writing it is unclear what platforms still suffer from this
problem and whether Commandment \ref{rcp:cmd:rcp-new} and the use of
{}\texttt{rcp()} is really necessary but the use of {}\texttt{rcp()}
is safe in any case and is therefore still recommended.
