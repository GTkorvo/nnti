\documentclass[]{llncs}      % it must be
\pagestyle{myheadings}       % it must be
\usepackage[dvips]{graphicx} % if needed
\begin{document}             % it must be

\title{The Design of Trilinos}

\author{Michael A Heroux\inst{1} and Marzio Sala\inst{1}}
\institute{
Sandia National Laboratories \\
        \{maherou, msala\}@sandia.gov }

\maketitle
\index{Heroux, Micheal A.}
\index{Sala, MArzio}

\markboth{M.A. Heroux and Marzio Sala}{The Design of Trilinos, Extended Abstract}

\section*{Extended Abstract}

\subsection*{Introduction}

Trilinos~\cite{1,2} is a suite of parallel numerical solver libraries
within an object-oriented software framework for the solution of
large-scale, complex multi-physics engineering and scientific
applications in a production computing environment.  The goal of the
project is to facilitate the design, development, integration, and
ongoing support of mathematical software libraries.

Our emphasis is on developing robust and scalable algorithms in a
software framework. We use abstract interfaces for flexible
interoperability of packages while providing a full-featured set of
concrete classes that implement all abstract interfaces.  Furthermore,
Trilinos includes an extensive set of tools and processes that support
software engineering practices.  Specifically, Trilinos provides the
following: 
\begin{itemize}
\item A suite of parallel object-oriented solver libraries for solving
linear systems (using iterative methods and providing interfaces to
third-party sequential and parallel direct solvers), nonlinear systems,
defining incomplete factorizations, domain decomposition and multi-level
preconditioners, solving eigensystem and time-dependent problems.
\item Tools for rapid deployment of new packages.  
\item A scalable model for integrating new solver capabilities.
\item An extensive set of software quality assurance (SQA) and software
quality engineering (SQE) processes and tools for developers.
\end{itemize}

Trilinos packages are primarily written in C++, but we provide some C
and Fortran user interface support.  We provide an open architecture
that allows easy integration of other solver packages and we make our
software available to the outside community via the GNU Lesser General
Public License (LGPL)~\cite{3}.  

%%%
%%%
%%%

\subsection*{Trilinos Package Architecture}

The fundamental concept in Trilinos is the {\sl package}.  Trilinos uses
a two-level software structure designed around collections of packages.
Each package is a numerical library (or collection of libraries) that
is: 
\begin{itemize}
\item Focused on important, state-of-the-art algorithms in its problem
regime;
\item Developed by a small team of domain experts;
\item Self-contained, with minimal dependencies on other packages;
\item Configurable, buildable, tested and documented on its own.
\end{itemize}
These packages can be distributed within Trilinos or separately. The
Trilinos framework provides a common look-and-feel that includes
configuration, documentation, licensing, and bug tracking.  There are
also guidelines for adding new packages to Trilinos.

%%%
%%%
%%%

\subsubsection*{Overview of Trilinos Packages}

The word ``Trilinos'' is Greek for ``string of pearls''. It is a
metaphor for the Trilinos architecture.  Packages are strung together
within Trilinos like pearls on a necklace. While each package is
valuable and whole in its own right, the collection represents even more
than the sum of its parts.  Trilinos began as only three packages but
has rapidly expanded.  Table~\ref{tab:1} lists the Trilinos packages
with a brief description and their availability status. At the bottom of
the table is the package count in each Trilinos release.  General
releases are available via automatic download.  Limited releases are
available upon request.  A full description of Trilinos packages is
available at the Trilinos Home Page~\cite{1}.

\begin{table}
\begin{center}
\begin{tabular}{| l l | c | c | c | c |}
\hline
Package &  Description & \multicolumn{4}{| c |}{Release} \\
        &              & \multicolumn{2}{c}{3.1 (9/2003)} & \multicolumn{2}{c |}{4 (5/2004)} \\
\cline{3-6} 
        &              & 3.1             &  3.1         & 4       &  4        \\
        &              & General         & Limited      & General &   Limited \\
   
\hline 
Amesos      & 3rd Party Direct Solver Suite   &    &  X  &  X &  X   \\
Anasazi     & Eigensolver package             &    &     &    &  X   \\
AztecOO     & Linear Iterative Methods        & X  &  X  &  X &  X   \\
Belos       & Block Linear Solvers            &    &     &    &  X   \\
Epetra      & Basic Linear Algebra            & X  &  X  &  X &  X   \\
EpetraExt   & Extensions to Epetra            &    &  X  &  X &  X   \\
Ifpack      & Algebraic Preconditioners       & X  &  X  &  X &  X   \\
Jpetra      & Java Petra Implementation       &    &     &    &  X   \\
Kokkos      & Sparse Kernels                  &    &     &  X &  X   \\
Komplex     & Complex Linear Methods          & X  &  X  &  X &  X   \\
LOCA        & Bifurcation Analysis Tools      & X  &  X  &  X &  X   \\
Meros       & Segregated Preconditioners      &    &  X  &  X &  X   \\
ML          & Multi-level Preconditioners     & X  &  X  &  X &  X   \\
NewPackage  & Working Package Prototype       & X  &  X  &  X &  X   \\
NOX         & Nonlinear solvers               & X  &  X  &  X &  X   \\
Pliris      & Dense direct Solvers            &    &     &  X &  X   \\
Teuchos     & Common Utilities                &    &     &  X &  X   \\
TSFCore     & Abstract Solver API             &    &     &  X &  X   \\
TSFExt      & Extensions to TSFCore           &    &     &    &  X   \\
Tpetra      & Templated Petra                 &    &     &    &  X   \\
\hline
Totals      &                                 &  8 &  11 & 15 & 20 \\
\hline
\end{tabular}
\caption{Trilinos Package Descriptions and Availability.}
\label{tab:1}
\end{center}
\end{table}

\subsubsection*{Package Interoperability vs. Interdependence}

Trilinos provides configuration, compilation and installation facilities
via GNU Autoconf~\cite{4} and Automake~\cite{5}.  These tools make
Trilinos very portable and easy to install.  The Trilinos-level build
tools allow the user to specify which packages should be configured and
compiled via \verb!--enable-package_name! arguments to the configure command.

A fundamental difference between Trilinos and other mathematical
software frameworks is its emphasis on interoperability.  Via abstract
interfaces and configure-time enabling, packages within Trilinos can
interoperate with each other without being interdependent.  For example,
the nonlinear solver package NOX needs some linear algebra support, but
it does not need to know the implementation details.  Because of this,
it specifies the interface it needs.  At configure time, the user can
specify one or more of three possible linear algebra implementations for
NOX: 
\begin{itemize}
\item {\tt --enable-nox-lapack}:  compile NOX lapack interface libraries
\item {\tt --enable-nox-epetra}:  compile NOX epetra interface libraries
\item {\tt --enable-nox-petsc}:  compile NOX PETSc interface libraries
\end{itemize}
 Alternatively, because the NOX linear algebra interfaces are abstract,
users can provide their own implementation and build NOX completely
independent from Trilinos.  Another example is ML, which can be used as
a stand-alone package. ML is based on a general matrix interface, and
wrappers are provided for epetra matrices, but users can define their
own matrix formats.

%%%
%%%
%%%

\subsection*{The Petra Object Model}

 Scalable parallel implementation of algorithms is extremely
challenging.  Therefore, one of the most important features of Trilinos
is its data model for global objects, called the Petra Object
Model~\cite{6}.  There are three implementation of this model, but the
current production implementation is called Epetra~\cite{7}. Epetra is
written for real-valued double-precision scalar field data, and
restricts itself to a stable core of the C++ language standard.  As
such, Epetra is very portable and stable, and is accessible to Fortran
and C users. Epetra combines in a single package: 
\begin{itemize}
\item {\sl Object-oriented, parallel C++ design}: Epetra has facilitated
rapid development of numerous applications and solvers at Sandia because
Epetra handles many of the complicated issues of working on a parallel,
distributed-memory machine.  Furthermore, Epetra provides a lightweight
set of abstract interfaces and a shallow copy mode that allows existing
solvers and data models to interact with Epetra without creating
redundant copies of data.  
\item {\sl High performance}: Despite the fact that
Epetra provides a very flexible data model, performance has always been
the top priority.  Aztec~\cite{8} won an R\&D 100 award in 1997 and has
also been the critical performance component in two Gordon Bell finalist
entries.  Despite Epetra’s much more general interfaces and data
structures, for problems where Aztec and Epetra can be compared, Epetra
is at least as fast and often faster than Aztec.  Epetra also uses the
BLAS and LAPACK wherever possible, resulting in excellent performance
for dense operations.  
\item {\sl Block algorithm support for next-generation solution
capabilities}: The Petra Object Model and Epetra specifically provide
support for multivectors, including distributed multivector operations.
The level of support that Epetra provides for this feature is unique
among available frameworks, and is essential for supporting
next-generation applications that incorporate features such as
sensitivity analysis and design optimization. 
\item {\sl Generic parallel machine
interfaces}: Epetra does not depend explicitly on MPI.  Instead it has
its own abstract interface for which MPI is one implementation.
Experimental implementations for hybrid distributed/shared memory,
PVM~\cite{9} and UPC~\cite{10} are also available or under development.
\item {\sl Parallel data redistribution}: Epetra provides a distributed object
(DistObject) base class that manages global distributed objects.  Not
only does this base class provide the functionality for common
distributed objects like matrices and vectors, it also supports
distributed graphs, block matrices, and coloring objects.  Furthermore,
any class can derive from DistObject by implementing just a few simple
methods to pack and unpack data.  
\item {\sl Interoperability}: Application
developers can use Epetra to construct and manipulate matrices and
vectors, and then pass these objects to any Trilinos solver
package. Furthermore, solver developers can develop many new algorithms
relying on Epetra classes to handle the intricacies of parallel
execution.  Whether directly or via abstract interfaces, Epetra provides
one of the fundamental interoperability mechanisms across Trilinos
packages.  
\end{itemize}

%%%
%%%
%%%

\subsection*{Trilinos software engineering environment for solver
developers}

Trilinos package developers enjoy a rich set of tools and
well-developed processes that support the design, development and
support of their software.  The following resources are available for
each package team: 
\begin{itemize}
\item {\sl Repository (CVS)}: Trilinos uses CVS~\cite{11} for
source control.  Each package exists both as an independent module as
well as part of the larger whole. 
\item {\sl Issue Tracking (Bugzilla)}:  Trilinos
uses Mozilla’s Bugzilla~\cite{12} for issue tracking.  This includes the
ability to search all previous bug reports and to automatically track
(via email notifications) submissions.  
\item {\sl Communication (Mailman)}:
Trilinos uses Mailman1~\cite{3} for managing mailing lists.  There are
Trilinos mailing lists as well as individual package mailing lists as
follows. Each package has low volume lists for announcements as well as
a general users’ mailing list. Further, each package also has
developer-focused lists for developer discussion, tracking of CVS
commits, and regression test results. Finally, there is a mailing list
for the leaders of the packages, and these leaders also have monthly
teleconferences where any necessary Trilinos-level decisions are made.
\end{itemize}

In addition to the above package-specific infrastructure, the following
resources are also available to package developers: 
\begin{itemize}
\item {\sl Debugging (Bonsai)}:
Mozilla’s Bonsai1~\cite{4} is a web-based GUI for CVS, allowing for
queries and easy comparison of different versions of the code. Bonsai is
integrated with the Trilinos CVS and Bugzilla databases.  
\item {\sl Jumpstart
(NewPackage package)}: One of the fundamental contributions of Trilinos
is “NewPackage”.  This is a legitimate package within Trilinos that does
“Hello World”.  We discuss it in more detail below.  
\item {\sl Developer
Documentation}: Trilinos provides an extensive developers guide~\cite{15}
that provides new developers with detailed information on Trilinos
resources and processes.  All Trilinos documents are available online at
the main Trilinos home page1.  
\item {\sl SQA/SQE support}: The Advanced Scalable
Computing Initiative (ASCI) program within the Department of Energy
(DOE) has strong software quality engineering and assurance
requirements.  At Sandia National Laboratories this has led to the
development of 47 SQE practices that are expected of ASCI-funded
projects~\cite{16}.  Trilinos provides a special developers guide17 for
ASCI-funded Trilinos packages.  For these packages, 32 of the SQE
practices are directly addressed by Trilinos.  The remaining 15 are the
responsibility of the package development team.  However, even for these
15, Trilinos provides significant support.  
\end{itemize}

To summarize, the two-level design of Trilinos allows package developers
to focus on only those aspects of development that are unique to their
package and still enjoy the benefits of a mature software engineering
environment.

%%%
%%%
%%%

\subsection*{NewPackage Package: Fully functional package prototype}

One of the most useful features of Trilinos is the NewPackage package.
 This simple “Hello World” program is a full-fledged Trilinos package
 that illustrates: Portable build processes via Autoconf and Automake.
 Automatic documentation generation using Doxygen~\cite{18}.
 Configuring for interaction with another package (Epetra) including the
 use of M~\cite{4,19} macros to customize package options.  Regression
 testing using special scripts in the package that will be automatically
 executed on regression testing platforms.

In addition, Trilinos provides a NewPackage website that can be easily
customized, requiring only specific package content.  This website is
designed to incorporate the reference documentation that is generated by
Doxygen from the NewPackage source code.

Using NewPackage as a starting point, developers of mathematical
libraries are able to become quickly and easily interoperable with
Trilinos.  It is worth noting that a package becomes
Trilinos-interoperable without sacrificing its independence.  This fact
has made Trilinos attractive to a number of developers.  As a result,
Trilinos is growing not only by new development projects, but also by
incorporation of important, mature projects that want to adopt modern
software engineering capabilities.  This is of critical importance to
applications that depend on these mature packages.

%%%
%%%
%%%

\subsection*{Summary}

Computer modeling and simulation is a credible third approach to
fundamental advances in science and engineering, along with theory and
experimentation.  A critical issue for continued growth is access to
both new and mature mathematical software on high performance computers
in a robust, production environment.  Trilinos attempts to address this
issue in the following ways: Provides a rapidly growing set of unique
solver capabilities: Many new algorithms are being implemented using
Trilinos. Trilinos provides an attractive development environment for
algorithm specialists.  Provides an independent, scalable linear algebra
package: Epetra provides extensive tools for construction and
manipulation of vectors, graphs and matrices, and provides the default
implementation of abstract interfaces across all other Trilinos
packages.  Supports SQA/SQE requirements: First solver framework to
provide explicitly documented processes for SQA/SQE.  An essential
feature for production computing.  Provides an innovative architecture
for modular, scalable growth: The Trilinos package concept and the
NewPackage package provide the framework for rapid development of new
solvers, and the integration and support of critical mature solvers.  Is
designed for interoperability: Designed "from the ground up" to be
compatible with existing code and data structures.  Trilinos is the best
vehicle for making solver packages, whether new or existing,
interoperable with each other.  Uses open source tools: Extensive use of
third-party web tools for documentation, versioning, mailing lists, and
bug tracking. To the best of our knowledge, no competing numerical
libraries offer anything nearly as comprehensive.






\begin{thebibliography}{10}

\bibitem{1} The Trilinos Home Page: {\tt
http://trilinos.sandia.gov}, 16-Feb-04.


\bibitem{2} Michael A. Heroux, et. al., An Overview of
 Trilinos. Technical Report SAND2003-2927, Sandia National Laboratories,
 2003.

\bibitem{3} The GNU Lesser General Public License: {\tt
http://www.gnu.org/copyleft/lesser.html}, 16-Feb-04.

\bibitem{4} GNU Autoconf Home Page: {\tt
 http://www.gnu.org/software/autoconf}, 16-Feb-04.

\bibitem{5} GNU Automake Home Page: {\tt
 http://www.gnu.org/software/automake}, 16-Feb-04.

\bibitem{6} Erik Boman, Karen Devine, Robert Heaphy, Bruce Hendrickson
 and Michael A. Heroux. LDRD Report: Parallel Repartitioning for Optimal
 Solver Performance. Technical Report SAND2004-XXXX, Sandia

\bibitem{7} Epetra Home Page: {\tt
 http://trilinos.sandia.gov/packages/epetra}, 16-Feb-04.

\bibitem{8} Ray S. Tuminaro, Michael A. Heroux, Scott A. Hutchinson and
 John N. Shadid. Official Aztec User’s Guide, Version 2.1. Technical
 Report SAND99-8801J, Sandia National Laboratories, 1999.

\bibitem{9} PVM Home Page: {\tt http://www.csm.ornl.gov/pvm}, 16-Feb-04.

\bibitem{10} UPC Home Page: {\tt http://upc.gwu.edu/}, 16-Feb-04.

\bibitem{11} GNU CVS Home Page: {\tt http://www.gnu.org/software/cvs},
 16-Feb-04.

\bibitem{12} Bugzilla Home Page: {\tt http://www.bugzilla.org},
 16-Feb-04.

\bibitem{13}
 GNU Mailman Home Page: {\tt http://www.gnu.org/software/mailman}, 16-Feb-04.

\bibitem{14} The Bonsai Project Home Page:,
 {\tt http://www.mozilla.org/projects/bonsai}, 16-Feb-04.

\bibitem{15} Michael A. Heroux, James M. Willenbring and Robert Heaphy,
 Trilinos Developers Guide, Technical Report SAND2003-1898, Sandia
 National Laboratories, 2003.

\bibitem{16} J. Zepper, K Aragon, M. Ellis, K. Byle and D. Eaton, Sandia
 National Laboratories ASCI Applications

\bibitem{17} Michael A. Heroux, James M. Willenbring and Robert Heaphy,
 Trilinos Developers Guide Part II: ASCI Software Quality Engineering
 Practices, Version 1.0. Technical Report SAND2003-1899, Sandia National
 Laboratories, 2003.

\bibitem{18} Doxygen Home Page: {\tt
 http://www.stack.nl/$~$dimitri/doxygen}, 16-Feb-04.

\bibitem{19} GNU M4 Home Page: {\tt http://www.gnu.org/software/m4},
 16-Feb-04.

\end{thebibliography}

\end{document}   % it must be
