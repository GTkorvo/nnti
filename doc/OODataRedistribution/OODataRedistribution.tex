%
% This is an example LaTeX file which uses the SANDreport class file.
% It shows how a SAND report should be formatted, what sections and
% elements it should contain, and how to use the SANDreport class.
%
% Build it using
%     latex SANDExample
%     bibtex SANDExample
%     latex SANDExample
%     latex SANDExample
%     dvips -o SANDExample.ps SANDExample.dvi
%     ps2pdf SANDExample.ps SANDExample.pdf
%
% This file and the SANDreport.cls file are based on information
% contained in "Guide to Preparing {SAND} Reports", Sand98-0730, edited
% by Tamara K. Locke.
% Please send corrections and suggestions for improvements to
% Rolf Riesen, Org. 9223, MS 1110, rolf@cs.sandia.gov
%
\documentclass[12pt,relax]{SANDreport}

% If you want to relax some of the SAND98-0730 requirements, use the "relax"
% option. It adds spaces and boldface in the table of contents, and does not
% force the page layout sizes.
% e.g. \documentclass[relax,12pt]{SANDreport}
%
% You can also use the "strict" option, which applies even more of the
% SAND98-0730 guidelines. It gets rid of section numbers which are often
% useful; e.g. \documentclass[strict]{SANDreport}



% ---------------------------------------------------------------------------- %
%
% Set the title, author, and date
%
    \title{An Object-Oriented Model for Distributed Memory Parallel Data Distribution}

    \author{Michael A.~Heroux\\
       Numerical and Applied Mathematics Department \\
	  Sandia National Laboratories\\
	  P.O. Box 5800\\
	  Albuquerque, NM 87185-1110 \\
	  maherou@sandia.gov \\
	 }

    % There is a "Printed" date on the title page of a SAND report, so
    % the generic \date should generally be empty.
    \date{}


% ---------------------------------------------------------------------------- %
% Set some things we need for SAND reports. These are mandatory
%
\SANDnum{SAND2002-xxxx}
\SANDprintDate{November 2002}
\SANDauthor{Michael A.~Heroux, Sandia National Laboratories}


% ---------------------------------------------------------------------------- %
% The following definitions are optional. The values shown are the default
% ones provided by SANDreport.cls
%
\SANDreleaseType{Unlimited Release}


% ---------------------------------------------------------------------------- %
% The following definitions do not have a default value and will not
% print anything, if not defined
%
%\SANDsupersed{SAND00-0000}{January 0000}
\SANDdistcategory{UC-999}	% DOE mandates it, but many reports don't have it


% ---------------------------------------------------------------------------- %
%
% Start the document
%
\begin{document}
    \maketitle

    % ------------------------------------------------------------------------ %
    % An Abstract is required for SAND reports
    %
    \begin{abstract}
This paper presents an object-oriented model for parallel data distribution on
distributed memory parallel computers.
    \end{abstract}


    % ------------------------------------------------------------------------ %
    % An Acknowledgement section is optional but important, if someone made
    % contributions or helped beyond the normal part of a work assignment.
    % Use \section* since we don't want it in the table of context
    %
    \clearpage
    \section*{Acknowledgement}
The author would like to acknowledge the support of the ASCI and LDRD programs
that funded development of Trilinos.


    % ------------------------------------------------------------------------ %
    % The table of contents and list of figures and tables
    % Comment out \listoffigures and \listoftables if there are no
    % figures or tables. Make sure this starts on an odd numbered page
    %
    \clearpage
    \tableofcontents
    \listoffigures
    \listoftables


    % ---------------------------------------------------------------------- %
    % An optional preface or Foreword
%    \clearpage
%    \section{Preface}
%	Although muggles usually have only limited experience with
%	magic, and many even dispute its existence, it is worthwhile
%	to be open minded and explore the possibilities.


    % ---------------------------------------------------------------------- %
    % An optional executive summary
    \clearpage
    \section{Summary}


    % ---------------------------------------------------------------------- %
    % An optional glossary. We don't want it to be numbered
%   \clearpage
%   \section*{Nomenclature}
%   \addcontentsline{toc}{section}{Nomenclature}
%   \begin{itemize}
%\item[Package]
%    a collection of software focused on one primary class of numerical methods
%\item[Trilinos]
%    A Greek term that loosely translated means ``a string of pearls,'' meant
%        to evoke an image that each Trilinos package is a pearl in its own right,
%        but is even more valuable when combined with other packages.
%\item[Petra]
%    A Greek term meaning ``foundation.''  Trilinos has three Petra libraries: Epetra, 
%    Tpetra and Jpetra (discussed in Section~\ref{subsect:PetraObjectModel}) that
%    provide basic classes for constructing and manipulating matrix, graph and vector 
%    objects.
%   \end{itemize}


    % ---------------------------------------------------------------------- %
    % This is where the body of the report begins; usually with an Introduction
    %
    \SANDmain		% Start the main part of the report

\section{Introduction}


\section{Basic Concepts}
\label{sect:concepts}
For our purposes, parallel data redistribution concerns the redistribution of 
data on a parallel distributed memory computer, where all processors on the 
machine are participating in the operation, even though some processor may send
or receive no data..  The data is assumed to be 
partitioned across the machine in some form already, including the case where one
processor owns all of the data.

Given an existing distribution of data, we want a formal mechanism for describing
how the data should be redistributed.  To accomplish this, we define an 
{\it element space} as a collection of labeled elements.  The exact meaning of
an element is determined by the type of object we are redistributing.
The label associated with each element is a signed integer value which we refer to as
the {\it global identifier} or {\it GID} of that element.  If there are no 
repeated GIDs associated with an element space, the element space is said to 
have the one-to-one property.  For many of the redistribution operations, the
one-to-one property is required for one or both of the element spaces involved
in the redistribution operation.

An element space is itself a distributed object.  An Espace is constructed by 
having each processor call the espace constructor, passing in the number of
elements that should be assigned to the processor and a list of related GIDs.

\section{Conclusions}


    % ---------------------------------------------------------------------- %
    % References
    %
    \clearpage
    \bibliographystyle{plain}
    \bibliography{OODataRedistribution}
    \addcontentsline{toc}{section}{References}


    % ---------------------------------------------------------------------- %
    % Appendices should be stand-alone for SAND reports. If there is only
    % one appendix, put \setcounter{secnumdepth}{0} after \appendix
    %
%    \appendix
%\section{}

    % \printindex

    \begin{SANDdistribution}
	\SANDdistExternal{1}{An Address\\ 99 $99^{th}$ street NW\\City, State}
	\SANDdistExternal{3}{Some Address\\ and street\\City, State}
	\bigskip
	\SANDdistExternal{12}{Another Address\\ On a street\\City, State\\U.S.A.}


	\SANDdistInternal{1}{1110}{Rolf Riesen}{9223}

	% Housekeeping copies necessary for every unclassified report:
	\SANDdistInternal{1}{9018}{Central Technical Files}{8940-2}
	\SANDdistInternal{2}{0899}{Technical Library}{4916}
	\SANDdistInternal{2}{0619}{Review \& Approval Desk}{4916}

	% If report has a Patent Caution or Patent Interest, add this:
	\SANDdistInternal{3}{0161}{Patent and Licensing Office}{4916}
    \end{SANDdistribution}

\end{document}
