
% Outlining a Development Model for CSE Software
% Jim Willenbring, Mike Phenow, Mike Heroux

%## Header ####################################################################

\documentclass[12pt,relax]{article}
%\usepackage{doublespace}
%\setlength {\oddsidemargin}{1in}
%\setlength {\evensidemargin}{1in}
\addtolength {\textheight}{0.2in}
\addtolength {\textwidth}{0.2in}
%DisplayCommand
\linespread{1.6}
%\linespread{1.75}
\newcommand{\DisplayCommand}[1]{%
\par\vspace{1ex}%
{\bf Command:}%
{\hspace{0.2 in}} {\tt #1} {\par\vspace{1ex}}}

% InlineCommand
\newcommand{\InlineCommand}[1]{
  {\hspace{0.01 in}} {\tt #1} {\hspace{0.01 in}}}

% InlineDirectory
\newcommand{\InlineDirectory}[1]{
  {\hspace{0.01 in}} {\tt #1} {\hspace{0.01 in}}}

\usepackage{array}

\title{Outlining a Development Model for CSE Software}

\author{
James Willenbring\\
Michael Phenow\\
Michael Heroux\\
}

% There is a "Printed" date on the title page of a SAND report, so
% the generic \date should generally be empty.
\date{\today} % Remove ``\today'' in final version

% Do we really have to repeat the authors?
%\author{}

%## Content ###################################################################

\begin{document}

\maketitle
%\setcounter{page}{3} % Accounts for blank page at beginning

%== Abstract ==================================================================

%\begin{abstract}

Abstract:  (same as intro. needs to be rewritten.  -mnp)

Developing computational science and engineering (CSE) software presents some 
unique challenges.  CSE software projects frequently don't have large budgets 
to spend on advanced software engineering tools, processes, and personnel.  
They often are developed by small groups of experts--experts in the particular 
computational science or engineering domain in question, not experts on 
software engineering or software project management.  The Trilinos Project,
developed at Sandia National Labs, is no different.  It is our belief that 
there are certain low-cost, high-yield processes and tools that tend to work 
well to enable the development of high quality CSE software.  To demonstrate 
this, we will draw from our experiences with the Trilinos Project to first
outline some high-level goals of the project.  We will then present aspects of
a development model that we think do a good job job of enabling developers to
reach those ends.  Finally, we will discuss some general classes (and specific
instances) of tools that enable the development model, and through that, 
the ultimate goals of the project.  Throughout we will use our experiences on 
the Trilinos project to provide specific examples of these tools and processes.

%\end{abstract}

\clearpage

%== Acknowledgements ==========================================================

\section*{Acknowledgments}

The authors would like to acknowledge the support of the ASCI and LDRD programs 
that funded development of Trilinos and recognize all Trilinos contributors:
Michael Heroux (project leader), Teri Barth, Ross Bartlett, Paul Boggs, Jason
Cross, David Day, Clark Dohrmann, Robert Heaphy, Ulrich Hetmaniuk, Robert
Hoekstra, Russell Hooper, Vicki Howle, Jonathan Hu, Tammy Kolda, Kris
Kampshoff, Sarah Knepper, Joe Kotulski, Richard Lehoucq, Kevin Long, Joe
Outzen, Roger Pawlowski, Eric Phipps, Andrew Rothfuss, Marzio Sala, Andrew
Salinger, Paul Sery, Paul Sexton, Ken Stanley, Heidi Thornquist, Ray Tuminaro
and Alan Williams.

\clearpage
\tableofcontents
\listoftables

\clearpage
%The following file is also used in the User Guide
%\input{../CommonFiles/TrilinosNomenclature.tex}
%Nomenclature
%** Fill in

%== Introduction ==============================================================

\section{Introduction}
\label{Section:Introduction}

Developing computational science and engineering (CSE) software presents some 
unique challenges.  CSE software projects frequently don't have large budgets 
to spend on advanced software engineering tools, processes, and personnel.  
They often are developed by small groups of experts--experts in the particular 
computational science or engineering domain in question, not experts on 
software engineering or software project management.  The Trilinos Project,
developed at Sandia National Labs, is no different.  It is our belief that 
there are certain low-cost, high-yield processes and tools that tend to work 
well to enable the development of high quality CSE software.  To demonstrate 
this, we will draw from our experiences with the Trilinos Project to first
outline some high-level goals of the project.  We will then present aspects of
a development model that we think do a good job job of enabling developers to
reach those ends.  Finally, we will discuss some general classes (and specific
instances) of tools that enable the development model, and through that, 
the ultimate goals of the project.  Throughout we will use our experiences on 
the Trilinos project to provide specific examples of these tools and processes.

%-- Typographical Conventions -------------------------------------------------
\subsection{Typographical Conventions}

Typographical conventions used in the paper are found in
Table~\ref{Table:TypoConventions}.
\begin{table}[ht]
\scriptsize
\begin{center}
\begin{tabular}{|l|l|p{2.0in}|} \hline
Notation & Example & Description \\ \hline
\InlineCommand{Verbatim text} & \InlineCommand{../configure --enable-mpi} & 
URL's, commands, directory and file name examples, and other text associated
with text displayed in a computer terminal window. \\ \hline
\InlineCommand{CAPITALIZED\_TEXT} & \InlineCommand{CXXFLAGS} & 
Environment variables used to configure how tools such as compilers behave. \\ \hline
\InlineCommand{<text in angle brackets>} & \InlineCommand{../configure
<user parameters>} & 
Optional parameters. \\ \hline
\end{tabular}
\end{center}
\caption{\label{Table:TypoConventions} Typographical Conventions for This Document.}
\end{table}

\clearpage

%== Goals =====================================================================

\section{Goals}
\label{Section:Goals}

\subsection{Quality}
%-------------------
Trilinos, like all software projects, seeks quality as a primary goal--quality
both in the colloquial meaning of the word and the particular meaning it
carries in the software engineering world, specifically:  a measure of the
degree to which software meets its stated requirements.

\subsection{Modularity}
%----------------------
Individual sets of functionality in Trilinos are contained in individual,
autonomous modules called packages, which are developed by individuals or small
teams.  Keeping logically distinct pieces of functionality modular is critical
to the long-term health of the project.  

\subsection{Interoperability}
%----------------------------
Because of this modular architecture, it is of utmost importance that the
various packages "play well together."  For Trilinos to realize its full
potential, all of the various independent parts need to work together in
concert.

\subsection{Scalability}
%-----------------------
Trilinos started as a collection of three packages.  In a few short years, it
has organically grown to include more than 25 packages.  To maximize the
benefits reaped from economies of scale and to leverage to power of other
codes, scalability, the ability to continue to add more packages, is a primary
concern for Trilinos.  The degree to which the Trilinos architecture scales
is directly dependent on the level of modularity and interoperability achieved.

\subsection{Efficient Use of Expert Time}
%----------------------------------------
Trilinos packages are developed by experts in the particular domain of a
package.  One very critical goal of the Trilinos project is to make efficient
use of these experts' time.  These experts ought to be spending as much time as
possible in their domain of expertise, not bothering with the comparatively
mundane tasks of software project management, which, unchecked, have a way of
crowding out other tasks.

\subsection{Support}
%-------------------
Finally, the ultimate goal of any piece of software is to actually get used.  
It is important to provide support so that all of the energy spent developing
the software is put to good use, but, here again, it is important that the
experts don't have to spend their time helping users install and use the
software.

\clearpage

%== Development Model =========================================================

\section{Development Model}
\label{Section:Development Model}

For many CSE software projects it simply isn't feasible to have a traditional, 
formal development model.  There are many reasons for this:  CSE software tends
to be research-based or exploratory in nature, CSE software projects tend not
to have large budgets or extensive staff, and CSE software projects often don't
have guaranteed funding far into the future.

\subsection{Tight Collaboration}
%-------------------------------

CSE software, like most software, has grown in complexity in recent years.  The
most interesting and challenging problems are generally not solved by an
individual or project team coding an entire solution from scratch.  Instead,
they rely on the work of others to provide some amount of a project's
functionality.  Likewise, one mark of the success for your project is the
amount of interest from others looking to use your code in their project.
And for many projects, these relationships may be even more strongly defined.
It may be the case that an outside collaborator is a significant stakeholder in
the project and whose requirements are of utmost importance.

But how do you gather the requirements of your stakeholders?  What happens
when they change?  Classical development models would prescribe a formal
requirements gathering process to set the direction of the project from the
outset.  From then on, all development has to be traceable back to the
requirements and any deviation from the requirements warrants a formal revision
of them.  

For many CSE projects, this is not a feasible approach.  When your work, or the
work of your stakeholders, is research-intensive or exploratory in nature, the
requirements may not be well known at the outset.  The problem itself may not
even be well defined.  In such cases, attempting to adhere to a classical
development model becomes, instead of necessary bookkeeping, an exercise in 
paper-generation.

How then to communicate effectively with your stakeholders?  Establish a
collaborative relationship with them.  Bring them into the workings of your
project.  This doesn't mean that they have to be concerned with the day to day
activities, but rather, use close collaboration to gather, implement,
integrate, and iterate on their requirements.

Internally, Trilinos takes advantage of close collaboration by establishing
well-defined channels of communication between packages.  Issue-tracking
software, mail lists, and regular meetings all give developers of one package
the means to communicate with other packages to coordinate important design
decisions.

Externally, Trilinos maintains close relationships with its primary
stakeholders, some of whom have a developer working on both Trilinos and the
project in question.  This helps ensure the successful integration of Trilinos
into their codes.  It also provides an effective means of staying abreast of
the changing requirements of these external codes.  Close collaborations of
this nature help Trilinos prevent possible problems before they arise and also
help to steer the project in the right direction.

\subsection{Frequent Iterations}
%-------------------------------

The close collaborations mentioned above facilitate the communication of 
design decisions, requirements, and countless other important bits of
information.  But the ultimate goal of all this communication is to produce
working code and the longer development continues without being integrated and
tested, the more time will have to be sunk into the integration process.  This
has led the Trilinos Project to strive for shorter iterations where possible.
And this is a practice that is valuable in a number of areas, from design,
development, and debugging to building, testing, and integrating.  Below we
will discuss a number of tools that the Trilinos Project relies on to enable
these short iterations.

\subsection{Quality Control \& Process Improvement}
%--------------------------------------------------

Software is always a work in progress.  On any project there are always things
being done well and things that need work.  One of the difficulties of 
software engineering or software project management is to take the realities
of a given project and mold them into a form that is in agreement with the
theories of accepted development models.  Realizing that a wholesale overhaul
of an entire project into compliance with an accepted model was infeasible, but
also that acceptance of sub-optimal processes is inefficient, the Trilinos
Project adopted a model of process improvement by which the processes that drive
the project are always subject to ongoing, incremental revision and
and improvement.  We always keep our eyes open for the "low-hanging fruit," or
those modifications to existing processes that would yield the most benefit
with at the least cost.  

\subsection{Two-Tiered Organizational Architecture}
%--------------------------------------------------

One particularly strong feature of the development model employed by the
Trilinos Project is its two-tiered architecture.  The functionality of Trilinos
is divided into a number of small, autonomous modules, called packages. 
Packages are self-driven and have the freedom to make decisions as they see
fit.  But, at the central Trilinos level, called the framework, centralized
guidelines and services are provided.  The package architecture helps to
encourage modularity throughout Trilinos and also promotes scalability by
providing a sound model for adding new functionality.  The guidelines provided
by the framework help promote interoperability between packages.  The services
provided by the framework help to minimize the overhead that the expert
developers need to concern themselves with.  The framework also provides a
first line of support to effectively shield developers from having to deal
with the trivial support requests.  The net effect of all of these factors is
a higher level of quality throughout the project.

  \subsubsection{Packages/Modules}
  %-------------------------------
  
  Grouping functionality into self-contained packages provides a number of
  benefits:  it keeps teams to small groups of experts, it allows for local
  autonomy, and it allows for local authority.
  
  1. Small expert groups
  
  As mentioned above, Trilinos began as three packages and has since grown to 
  include more than 25.  Each of these packages has a different story.  Some
  were started from scratch within Trilinos.  Many others were existing
  projects that existed elsewhere and were imported into Trilinos.  Of these,
  we find the whole range, from those just off the ground, to mature codes
  that have been in use for years.  In all cases, the development of the 
  code is done by experts in the given domain.  These groups generally consist
  of from one to four or five developers.  This small size keeps the groups
  focused, agile, and accountable.
  
  2. Local autonomy
    
  Packages that join Trilinos after they are relatively mature often wouldn't
  do so if they felt they would be forever dependent on Trilinos.  The design
  of the Trilinos architecture very deliberately seeks to prevent a central
  entity upon which all packages must be dependent.  Many packages came to 
  Trilinos already having an established user base and it is very important
  for some packages to be able to exist either within the Trilinos framework
  or completely without it.
  
  3. Local authority
  
  Similarly, many packages would not be inclined to become a part of Trilinos
  if they had to surrender the control of their package to somebody else.
  Being a part of Trilinos brings with it very few requirements.  Instead,
  there are many guidelines and services that, in practice, are almost all
  adopted by all packages.  Local decisions about the direction or design of
  a package are left in the hands of the developers.

  \subsubsection{Framework/Project}
  %--------------------------------
  
  Above the packages is the Trilinos Framework.  The framework is the string
  holding the pearls of the various packages together.  The framework serves
  two major purposes:  to set forth guidelines and to provide services.
  
  1. Global "guidelines"
  
  2. Global "services"
  

% SAMPLE NOTE-BOX ----------------------
%\framebox{\begin{minipage}{\textwidth}{
%NOTE:\\...
%}\end{minipage}}

\clearpage

%== Tools =====================================================================

\section{Tools}
\label{Section:Tools}

\subsection{Versioning System}
%-----------------------------

\subsection{Mail Lists}
%----------------------

\subsection{Bug-Tracking}
%------------------------

\subsection{Website}
%-------------------

\subsection{Test Harness}
%------------------------

\clearpage

%== Conclusion ================================================================

\section{Conclusion}
\label{Section:Conclusion}

\clearpage

%== More Information ==========================================================

** From Jim's paper, revise, but this shows how to cite papers and prevents the
empty bib problem.

Those who are interested in learning more about the Trilinos project should 
consult {\it An Overview of Trilinos}~\cite{Trilinos-Overview} or the
{\it Trilinos Users Guide}~\cite{Trilinos-Users-Guide}.  Trilinos also has an 
extensive web site that can be found at \newline
\InlineDirectory{http://software.sandia.gov/trilinos}~\cite{Trilinos-home-page}.

\clearpage

%== Bibliography ==============================================================

\bibliographystyle{plain}
%\bibliography{SIAMnews}
\bibliography{../CommonFiles/TrilinosBibliography}
\addcontentsline{toc}{section}{References}

\end{document}

%## Notes #####################################################################

%== Outline ===================================================================

%I.    Introduction                                                                                
%      A.  Motivation (problem)
%      B.  Claim (proposed solution)
%      C.  Preview (teaser, road map)
%                                                                      
%II.   Body                                                                                
%      A.  Goals (problem)
%          1.  Quality
%          2.  Modularity
%          3.  Interoperability
%          4.  Scalability
%          5.  Efficient use of expert time
%          6.  Support                                                                                
%      B.  Development model (proposed solution)
%          1.  Tight collaboration
%          2.  Frequent iterations
%          3.  Quality control, process improvement
%          4.  Two-tiered organizational architecture
%              a.  Packages/modules
%                  1.  Small expert groups
%                  2.  Highly autonomous
%                  3.  Local authority
%              b.  Framework/project
%                  1.  Global "control"
%                  2.  Global "services"
%                  3.  Support filter, common front
%      C.  Tools (to support/enable proposed solution)
%          1.  version system
%          2.  mail lists
%          3.  bug-tracking
%          4.  website
%          5.  test harness
%                                                                                
%III.  Conclusion                                                                                
%      A.  Recapitulate problem
%      B.  Recapitulate solution
%      C.  Claim / action item / prediction (say *something*)

%== Titles ====================================================================

%Modern Software Development practices for CSE
%Impact of Modern Software Tools on CSE
%Exploring the Impact of Modern Development practices on CSE
%Examining the Impact of Modern Software Tools on CSE
%Assessing the Impact of Modern Software Tools on CSE
%Assessing the Impact of Tools and Processes on CSE Software Development
%Assessing the Impact of a Customized Development Processes on ...
%Outlining a Development Model for CSE Software
%Making the Case for Agile Models in CSE Software

%== Thesis ====================================================================

%== Ideas =====================================================================