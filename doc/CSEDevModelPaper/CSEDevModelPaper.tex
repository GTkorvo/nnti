
% Outlining a Development Model for CSE Software
% Jim Willenbring, Mike Phenow, Mike Heroux

%## Header ####################################################################

\documentclass[12pt,relax]{article}
%\usepackage{doublespace}
%\setlength {\oddsidemargin}{1in}
%\setlength {\evensidemargin}{1in}
\addtolength {\textheight}{0.2in}
\addtolength {\textwidth}{0.2in}
%DisplayCommand
\linespread{1.6}
%\linespread{1.75}
\newcommand{\DisplayCommand}[1]{%
\par\vspace{1ex}%
{\bf Command:}%
{\hspace{0.2 in}} {\tt #1} {\par\vspace{1ex}}}

% InlineCommand
\newcommand{\InlineCommand}[1]{
  {\hspace{0.01 in}} {\tt #1} {\hspace{0.01 in}}}

% InlineDirectory
\newcommand{\InlineDirectory}[1]{
  {\hspace{0.01 in}} {\tt #1} {\hspace{0.01 in}}}

\usepackage{array}

\title{Outlining a Development Model for CSE Software}

\author{
James Willenbring\\
Michael Phenow\\
Michael Heroux\\
}

% There is a "Printed" date on the title page of a SAND report, so
% the generic \date should generally be empty.
\date{\today} % Remove ``\today'' in final version

% Do we really have to repeat the authors?
%\author{}

%## Content ###################################################################

\begin{document}

\maketitle
%\setcounter{page}{3} % Accounts for blank page at beginning

%== Abstract ==================================================================

%\begin{abstract}

Abstract:  (same as intro. needs to be rewritten.  -mnp)
Developing computational science and engineering (CSE) software presents some 
unique challenges.  CSE software projects frequently don't have large budgets 
to spend on advanced software engineering tools, processes, and personnel.  
They often are developed by small groups of experts--experts in the particular 
computational science or engineering domain in question, not experts on 
software engineering or software project management.  It is our belief that 
there are certain low-cost, high-yield processes and tools that tend to work 
well to enable the development of high quality CSE software.  To demonstrate 
this, we will first outline a number of very high-level goals for CSE software 
projects.  We will then present a development model that we think does a good 
job of enabling developers to reach those ends.  Finally, we will discuss some 
general classes of tools that enable that development model, and through that, 
the ultimate goals of the project.  Throughout we will use our experiences on 
the Trilinos project to provide specific examples of these tools and processes.

%\end{abstract}

\clearpage

%== Acknowledgements ==========================================================

\section*{Acknowledgments}

The author would like to acknowledge the support of the ASCI and LDRD programs 
that funded development of Trilinos and recognize all Trilinos contributors:
Michael Heroux (project leader), Teri Barth, Ross Bartlett, Paul Boggs, Jason
Cross, David Day, Clark Dohrmann, Robert Heaphy, Ulrich Hetmaniuk, Robert
Hoekstra, Russell Hooper, Vicki Howle, Jonathan Hu, Tammy Kolda, Kris
Kampshoff, Sarah Knepper, Joe Kotulski, Richard Lehoucq, Kevin Long, Joe
Outzen, Roger Pawlowski, Eric Phipps, Andrew Rothfuss, Marzio Sala, Andrew
Salinger, Paul Sery, Paul Sexton, Ken Stanley, Heidi Thornquist, Ray Tuminaro
and Alan Williams.

\clearpage
\tableofcontents
\listoftables

\clearpage
%The following file is also used in the User Guide
%\input{../CommonFiles/TrilinosNomenclature.tex}
%Nomenclature
%** Fill in

%== Introduction ==============================================================

\section{Introduction}
\label{Section:Introduction}

%-- Typographical Conventions -------------------------------------------------
\subsection{Typographical Conventions}

Typographical conventions used in the paper are found in
Table~\ref{Table:TypoConventions}.
\begin{table}[ht]
\scriptsize
\begin{center}
\begin{tabular}{|l|l|p{2.0in}|} \hline
Notation & Example & Description \\ \hline
\InlineCommand{Verbatim text} & \InlineCommand{../configure --enable-mpi} & 
URL's, commands, directory and file name examples, and other text associated
with text displayed in a computer terminal window. \\ \hline
\InlineCommand{CAPITALIZED\_TEXT} & \InlineCommand{CXXFLAGS} & 
Environment variables used to configure how tools such as compilers behave. \\ \hline
\InlineCommand{<text in angle brackets>} & \InlineCommand{../configure
<user parameters>} & 
Optional parameters. \\ \hline
\end{tabular}
\end{center}
\caption{\label{Table:TypoConventions} Typographical Conventions for This Document.}
\end{table}

Developing computational science and engineering (CSE) software presents some 
unique challenges.  CSE software projects frequently don't have large budgets 
to spend on advanced software engineering tools, processes, and personnel.  
They often are developed by small groups of experts--experts in the particular 
computational science or engineering domain in question, not experts on 
software engineering or software project management.  It is our belief that 
there are certain low-cost, high-yield processes and tools that tend to work 
well to enable the development of high quality CSE software.  To demonstrate 
this, we will first outline a number of very high-level goals for CSE software 
projects.  We will then present a development model that we think does a good 
job of enabling developers to reach those ends.  Finally, we will discuss some 
general classes of tools that enable that development model, and through that, 
the ultimate goals of the project.  Throughout we will use our experiences on 
the Trilinos project to provide specific examples of these tools and processes.

\clearpage

%== Goals =====================================================================

\section{Goals}
\label{Section:Goals}

CSE software projects share many of the goals common to all software projects, 
but also require some additional goals.  Of course, the primary goal of all 
software projects is quality--quality both in the colloquial meaning of the 
word and the particular meaning it carries in the software engineering world, 
specifically:  a measure of the degree to which software meets its stated
requirements.

It is important that a CSE software project maintain a certain level of 
modularity.  CSE software functionality is often developed by small expert 
teams, and often there are even a number of small teams or individuals working 
on different functionality with a project.  Keeping logically distinct pieces 
of functionality modular is critical to the longterm health of a project.  

Given the ever-increasing complexity of the problems being addressed by CSE 
software, interoperability is becoming ever-more important.  It is no longer 
feasible for any single application to approach a non-trivial problem and
expect to be able to code it from the ground up.  Even within a project, if it 
is being developed in a modular fashion, internal interoperability needs to be 
a primary concern.

Scalability is a necessary goal in many areas of CSE software.  Clearly, it is 
important for the algorithms themselves to scale to allow for the investigation 
of interesting problems.  Additionally, it is important that the project itself 
and its processes be reasonably scalable.  If a project is being developed by a
number of small expert teams or individuals, it may become necessary to add 
more such groups in the future and a project will need to address that.

As mentioned, many CSE software projects are developed by experts in the 
particular computational science or engineering domain.  One very critical 
goal of CSE software projects, then, is to make efficient use of the experts' 
time.  These experts ought to be spending as much time as possible in their 
domain of expertise, not bothering with the comparatively mundane tasks of 
software project management, which, unchecked, have a way of crouding out 
other tasks.  

Finally, the ultimate goal of any piece of software is to actually get used.  
For many projects, it is important to release the software to the public and 
provide a certain level of support.  It is important to provide this support 
so that all of the energy spent developing the software is put to good use, 
but, here again, it is important that the experts don't spend their time 
helping users install the software.

\clearpage

%== Development Model =========================================================

\section{Development Model}
\label{Section:Development Model}

For many CSE software projects it simply isn't feasible to have a traditional, 
formal development model.

\clearpage

%== Tools =====================================================================

\section{Tools}
\label{Section:Tools}

\clearpage

%== Conclusion ================================================================

\section{Conclusion}
\label{Section:Conclusion}

\clearpage

%== More Information ==========================================================

** From Jim's paper, revise, but this shows how to cite papers and prevents the
empty bib problem.

Those who are interested in learning more about the Trilinos project should 
consult {\it An Overview of Trilinos}~\cite{Trilinos-Overview} or the
{\it Trilinos Users Guide}~\cite{Trilinos-Users-Guide}.  Trilinos also has an 
extensive web site that can be found at \newline
\InlineDirectory{http://software.sandia.gov/trilinos}~\cite{Trilinos-home-page}.

\clearpage

%== Bibliography ==============================================================

\bibliographystyle{plain}
%\bibliography{SIAMnews}
\bibliography{../CommonFiles/TrilinosBibliography}
\addcontentsline{toc}{section}{References}

\end{document}

%## Notes #####################################################################

%== Outline ===================================================================

%I.    Introduction                                                                                
%      A.  Motivation (problem)
%      B.  Claim (proposed solution)
%      C.  Preview (teaser, road map)
%                                                                      
%II.   Body                                                                                
%      A.  Goals (problem)
%          1.  Quality
%          2.  Modularity
%          3.  Interoperability
%          4.  Scalability
%          5.  Efficient use of expert time
%          6.  Support                                                                                
%      B.  Development model (proposed solution)
%          1.  Tight collaboration
%          2.  Frequent iterations
%          3.  Quality control, process improvement
%          4.  Two-tiered organizational architecture
%              a.  Packages/modules
%                  1.  Small expert groups
%                  2.  Highly autonomous
%                  3.  Local authority
%              b.  Framework/project
%                  1.  Global "control"
%                  2.  Global "services"
%                  3.  Support filter, common front%                                                                                
%      C.  Tools (to support/enable proposed solution)
%          1.  version system
%          2.  mail lists
%          3.  bug-tracking
%          4.  website
%          5.  test harness
%                                                                                
%III.  Conclusion                                                                                
%      A.  Recapitulate problem
%      B.  Recapitulate solution
%      C.  Claim / action item / prediction (say *something*)

%== Titles ====================================================================

%Modern Software Development practices for CSE
%Impact of Modern Software Tools on CSE
%Exploring the Impact of Modern Development practices on CSE
%Examining the Impact of Modern Software Tools on CSE
%Assessing the Impact of Modern Software Tools on CSE
%Assessing the Impact of Tools and Processes on CSE Software Development
%Assessing the Impact of a Customized Development Processes on ...
%Outlining a Development Model for CSE Software

%== Thesis ====================================================================

%== Ideas =====================================================================
