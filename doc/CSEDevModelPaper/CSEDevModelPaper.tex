
% Outlining a Development Model for CSE Software
% Mike Heroux, Jim Willenbring, Mike Phenow

%## Header ####################################################################

\documentclass[12pt,relax]{article}
\addtolength {\textheight}{0.2in}
\addtolength {\textwidth}{0.2in}
\linespread{1.6}
\newcommand{\DisplayCommand}[1]{%
\par\vspace{1ex}%
{\bf Command:}%
{\hspace{0.2 in}} {\tt #1} {\par\vspace{1ex}}}

% InlineCommand
\newcommand{\InlineCommand}[1]{
  {\hspace{0.01 in}} {\tt #1} {\hspace{0.01 in}}}

% InlineDirectory
\newcommand{\InlineDirectory}[1]{
  {\hspace{0.01 in}} {\tt #1} {\hspace{0.01 in}}}

\usepackage{array}

%## Content ###################################################################

%== Title Page ================================================================

\title{Improving the Development Process for CSE Software}

\author{
Michael Heroux\\
James Willenbring\\
Michael Phenow\\
}

% There is a "Printed" date on the title page of a SAND report, so
% the generic \date should generally be empty.
\date{\today} % Remove ``\today'' in final version

\begin{document}

\maketitle
%\setcounter{page}{3} % Accounts for blank page at beginning

%== Abstract ==================================================================

\begin{abstract}

Most existing computational science and engineering (CSE) software has been 
developed without the benefit of Software Quality Assurance (SQA) processes.
The increasing complexity of applications is dictating a higher standard for 
CSE software quality; it is no longer sufficient to claim to have 
written high quality software.  However, traditional
software development models can be impractical for CSE projects to implement.
Despite this, CSE software teams can benefit by implementing
valuable SQA processes and tools.  We will outline some the
processes and tools that the Trilinos Project has had success with.  These 
tools and processes have been useful not only in increasing 
\textit{verifiable} software quality, but also have improved 
overall software quality, and the development experience in 
general.

\end{abstract}

\clearpage

%== Acknowledgements ==========================================================

\section*{Acknowledgments}

The authors would like to acknowledge the support of the ASC and LDRD programs 
that funded development of Trilinos and recognize all Trilinos contributors:
Michael Heroux (project leader), Teri Barth, Ross Bartlett, Paul Boggs, Jason
Cross, David Day, Clark Dohrmann, Robert Heaphy, Ulrich Hetmaniuk, Robert
Hoekstra, Russell Hooper, Vicki Howle, Jonathan Hu, Tammy Kolda, Kris
Kampshoff, Sarah Knepper, Joe Kotulski, Richard Lehoucq, Kevin Long, Joe
Outzen, Roger Pawlowski, Eric Phipps, Andrew Rothfuss, Marzio Sala, Andrew
Salinger, Paul Sery, Paul Sexton, Ken Stanley, Heidi Thornquist, Ray Tuminaro
and Alan Williams.

\clearpage
\tableofcontents

\clearpage

%== Introduction ==============================================================

\section{Introduction}
\label{Section:Introduction}


The Trilinos Project, developed primarily at Sandia National Laboratories, is
an effort to develop parallel solver algorithms and libraries within an
object-oriented software framework for the solution of large-scale, complex,
multi-physics engineering and scientific applications.  Trilinos consists of
about thirty packages.  Each package is focused on important, state-of-the-art
algorithms in a particular domain and is developed by a small team of experts.

Four years ago the Trilinos development team was charged with the task of
improving its existing software quality practices.  Since many computational
science and engineering (CSE) software projects are not dedicated to formal
Software Quality Assurance (SQA) practices, a lot of work is required to define
practices that are well suited to the project.

A few characteristics of the project make defining formal practices and
processes especially challenging.  Specifically, the requirements of the
project are elusive and evolving, which makes it exceedingly difficult to
maintain a formal requirements document.  The fact that the Trilinos Team is
spread across the United States places special emphasis on enabling effective
methods of communication.  As with many CSE software projects, the budgetary
focus is on algorithms development, leaving little money to put directly
towards software quality efforts--either dedicated personnel or proprietary
tools.

Now, four years into the process of improving software quality practices, we 
have come to realize that there are certain low-cost, high-yield processes and
tools that tend to work well to enable the development of high quality CSE
software.  To demonstrate this, we will present some high-level goals that 
apply to most CSE software projects and Trilinos in particular.  We will then
present some principles that keep the Trilinos project on the right development
path.  Finally, we will discuss some general classes (and specific instances)
of tools that, guided by our driving principles, help us to achieve the goals
of the project.

Those who are interested in learning more about the Trilinos project should
consult {\it An Overview of Trilinos}~\cite{Trilinos-Overview} or the
{\it Trilinos Users Guide}~\cite{Trilinos-Users-Guide}.  Trilinos also has an
extensive web site that can be found at \newline
\InlineDirectory{http://software.sandia.gov/trilinos}~\cite{Trilinos-home-page}.

%== Goals =====================================================================

\section{Goals}
\label{Section:Goals}

Based on the above project characteristics, it is easy to extract some high-
level project goals.  These goals can be applied to most software projects,
but are described below in the specific context of the Trilinos Project.

\subsection{Quality}
%-------------------
Trilinos, like all software projects, seeks quality as a primary goal--quality
both in the colloquial meaning of the word and the particular meaning it
carries in the software engineering world, specifically:  a measure of the
degree to which software meets its stated requirements.  Additionally, for an 
increasing number of CSE software projects, claiming to do a good job is no 
longer sufficient.  Customers now require artifacts that show that documented 
software processes are followed.

\subsection{Modularity}
%----------------------
Individual sets of functionality in Trilinos are contained in individual,
autonomous modules called packages, which are developed by individuals or small
teams.  Keeping logically distinct pieces of functionality modular is critical
to the long-term health of the project.  

\subsection{Interoperability}
%----------------------------
Because of this modular architecture, it is of utmost importance that the
various packages ``play well together.''  For Trilinos to realize its full
potential, all of the various independent parts need to work together in
concert.  This is an important issue for CSE software in general.  A lot of
excellent existing software is underutilized because it cannot readily be
integrated with other existing software to be brought to bear on a single
problem.

\subsection{Scalability}
%-----------------------
Trilinos started as a collection of three packages.  In a few short years, it
has grown organically to include roughly 30 packages.  To maximize the
benefits reaped from economies of scale and to leverage the power of other
codes, scalability (in this context, the ability to continue to add more
packages) is a primary concern for Trilinos.  The degree to which the Trilinos
architecture scales is directly dependent on the level of modularity and
interoperability achieved.  Another key scalability issue for Trilinos is that
as packages are added, users should be shielded from the additional complexity;
using Trilinos should not become significantly more complicated as Trilinos
grows.

\subsection{Efficient Use of Expert Time}
%----------------------------------------
Trilinos packages are developed by experts in the particular domain of a
package.  One very critical goal of the Trilinos project is to make efficient
use of these experts' time.  These developers ought to be spending as much 
time as possible working within their domain of expertise, leaving 
the vast majority of the software project 
management tasks to those who are specialists in that domain. 
The package domain experts should, however, provide input when selecting 
SQA processes because adopting processes that are not well-suited to a
particular project can decrease, rather than increase, efficiency.

\subsection{Accessibility \& Support}
%-------------------
Finally, the ultimate goal of any piece of software is to actually get used.  
In order for people to use it, the software has to be reasonably accessible to
users.  It is also important to provide support so that all of the energy spent
developing the software is put to good use, but, here again, it is important
that the experts don't have to spend all of their time helping users install
and use the software.

\clearpage

%== Driving Principles ========================================================

\section{Driving Principles}
\label{Section:Driving Principles}

As described in the introduction, the Trilinos project, and likewise many CSE
projects, has a unique set of characteristics within the software world.
Within these constraints, certain goals can then be formulated for the project.
In our ongoing attempts to reach these goals, Trilinos has been guided by a
small set of principles that, when faced with a critical decision, help us to
stay on track.

\subsection{Package Orthogonality}
%---------------------------------

Trilinos was originally created as a way to bring together parallel solvers
to enable effective reuse, enable interoperability, and minimize duplication of
effort by application developers.  The mechanism chosen for containing a solver
was a ``package.''  A Trilinos package is simply a self-contained piece of
software that is developed in the Trilinos source repository, can build within
the Trilinos build system, and can interact with other Trilinos packages.
Trilinos began as a set of three packages and got its name from the Greek 
``trilinos,'' which loosely translated means ``string of pearls.''  The name is
meant to convey the idea that each package is a valuable pearl, but is even
more valuable when combined with the rest of the packages.  This image also
contains the notion of a common thread holding all the packages together.

Having both a collection of packages and a central entity gives Trilinos a
two-tiered architecture.  What we think of as the lower level is simply the
packages, the pearls.  Above that, we have what we call the Trilinos framework,
the string.  The framework is where we seek to capitalize on economies of scale
by providing global services to packages so duplication of effort is minimized
wherever possible.  The two driving principles that keep this two-tiered
architecture in its delicate balance are ``global services,'' which we will
discuss in the next section, and ``package orthogonality''.

One critically important driving principle of the project since its inception
has been the idea that we must do everything we can to preserve package
autonomy.  While it's true that most packages never stray very far from the
pack and may never do so, it has served us well to always design to allow for
it.  This principle of package autonomy has evolved into the more
encompassing principle of package orthogonality.   In this context, 
achieving orthogonality means that the 
relationships between different packages, as well as the relationships
between packages and the framework, are such that a change has
a minimal effect on other components.

Achieving a high level of orthogonality is advantageous for many reasons.  For 
instance, it allows packages to be effectively developed by small groups of 
domain experts without unnecessary hassles external to the algorithmic problem 
at hand.  As mentioned above, Trilinos began as three packages and has since
grown to include more than 25.  Each of these packages has a different story.  
Some were started from scratch within Trilinos.  Many others were existing
projects imported into Trilinos.  Of these, we find the whole range, from
those just off the ground, to mature codes that have been in use for years.
In all cases, the development of the code is done by experts in the given
domain.  These groups generally consist of one to five developers.  This
small size keeps the groups focused, agile, and accountable.
  
Packages that join Trilinos after they are relatively mature often wouldn't
do so if they felt they would be forever dependent on Trilinos.  The design
of the Trilinos architecture very deliberately seeks to prevent a central
entity upon which all packages must be dependent.  Many packages came to 
Trilinos already having an established user base and it is very important
for some packages to be able to exist either within the Trilinos framework
or completely apart from it.

Similarly, many packages would not be inclined to become a part of Trilinos
if they had to surrender the control of their package.  Being a part of
Trilinos brings with it very few requirements.  Instead, there are many
guidelines and services that, in practice, are almost all adopted by all
packages.  Local decisions about the direction or design of a package are
left in the hands of the package developers.

Finally, maintaining a high level of package orthogonality and autonomy
keeps us honest.  We don't just acquire source and hold it hostage under heavy
handed requirements.  Since packages are free to go at any time, we know that
to retain them (and thus to retain the benefits of the functionality they
provide and the expertise of their developers) we must continue to provide
value to the package developers.

\subsection{Global Services}
%---------------------------

In the two-tiered architecture, we generally refer to the upper tier as the
framework.  The framework is the string holding the pearls of the various
packages together.  The framework exists for the benefit of member
packages, providing numerous services and suggested practices.

The Trilinos Framework makes many optional, valuable services available to
member packages.  The vast majority of the costs associated with implementing
or adopting a given tool or service are constant, up-front costs.  The cost of
adding an additional package is usually negligible.  This means that, on their
own, the packages couldn't afford the time, energy, or expertise required to
support such an array of services.  By having the Trilinos framework
provide these services to all of the Trilinos packages, each package gains
access to the whole suite of powerful services and tools--a high level of value
at a cost that is effectively amortized across all packages.

Some of the standard services include source control, an issue reporting and
tracking tool, and mail lists.  More advanced services include a package
webpage prototype and personal assistance in creating and maintaining package
websites as well as a build system that allows all packages to be built as a
part of a single process and helps to ease some porting issues.  A functional
example package called New Package can be used by developers to quickly adapt
an existing piece of sofware to the suggested Trilinos build system or to 
hasten the process of developing portable software from scratch.  Automated 
testing on an expanding number of platforms, as well as the ability to set up 
customized test runs is available via the Trilinos Test Harness.  A relatively 
new service provided by  the framework is the ability to view test results 
on the internet from all of the Trilinos Test Harness testing platforms.

As mentioned above, Trilinos does not impose a large number of requirements
on member packages.  Rather, the Trilinos Framework provides suggested 
practices that, with very few exceptions, are adopted by all packages.
For example, packages are required to complete some sort of organized
process prior to an external release (having a documented release
process is a requirement that is imposed by powers above the Trilinos
Framework).  The Trilinos Framework team has developed a checklist that
packages can complete to satisfy this requirement; however, package teams
are free to develop an alternative process.  At this time, every package
uses the default release checklist, which saves developers the hassle of
developing an individualized process and gives them a release process that
has been hardened through process improvement based on feedback from
member package teams.  Another suggested practice is running tests
associated with a package before committing new code for that package to the 
CVS repository.

With some Trilinos team members concentrated on framework-level issues and
valuable input from numerous Trilinos package developers, the Trilinos 
Framework now provides many valuable services and suggested practices that
are inexpensive for package teams to utilize and that have evolved over 
time based on the experience of dozens of people.  

The impact of the Trilinos Framework has spread beyond the boundaries of 
the project.  The package website template and New Package are available 
at the Trilinos website.  Other project teams at Sandia are also in various
stages of adapting for their own project one or more of many Trilinos
services, including the test harness, homepage, results webpage, 
release checklists, or build system to meet their specific needs.

\subsection{Tight Collaboration}
%-------------------------------

CSE software, like most software, has grown in complexity in recent years.
The most interesting and challenging problems are generally not solved by an
individual or project team working in isolation.  Solid relationships with 
external and internal collaborators are essential.  It may even be the case
that an outside collaborator is a significant stakeholder in the project and 
whose requirements are of utmost importance.

But how do you gather the requirements of your stakeholders?  What happens
when they change?  Classical development models would prescribe a formal
requirements gathering process to set the direction of the project from the
outset.  From then on, all development has to be traceable back to the
requirements and any deviation from the requirements warrants a formal revision
of them.  

For many CSE projects, this is not a reasonable approach.  When your work, or 
the work of your stakeholders, is research-intensive or exploratory in nature, 
the problem may not be understood well enough at the outset of the project to 
make it worthwhile to define traditional formal requirements.  Requirements 
will likely change and evolve very quickly.  In such cases, attempting to
adhere to a classical development model becomes, instead of necessary
bookkeeping, an exercise in paper-generation.

How then to communicate effectively with your stakeholders?  Establish a
collaborative relationship with them.  Bring them into the workings of your
project.  This doesn't mean that they have to be concerned with the day to day
activities, but rather, use close collaboration to gather, implement,
integrate, and iterate on their requirements.  Proactively seek additional
input from stakeholders on a regular basis and keep them well informed.

Between packages, Trilinos takes advantage of close collaboration by 
establishing well-defined channels of communication.  Issue-tracking
software, mail lists, and regular meetings all give developers of one package
the means to communicate with other packages to coordinate important design
decisions.

Outside of the project, Trilinos maintains close relationships with its primary
stakeholders, some of whom have a developer working on both Trilinos and the
project in question.  This helps ensure the successful integration of Trilinos
into their codes.  It also provides an effective means of staying abreast of
the changing requirements of these external codes.  Close collaborations of
this nature help Trilinos prevent possible problems before they arise and also
help to steer the project in the right direction.

\subsection{Iterative Development}
%---------------------------------

Close collaborations facilitate the communication of design decisions, 
requirements, and countless other important bits of information, but the
ultimate goal of all this communication is to produce working code, and the
longer development continues without being integrated and tested, the more
time will have to be sunk into the integration and debugging processes.  This
has led the Trilinos Project to strive for shorter iterations where possible.
While this does not mean that we release as frequently as, say, an aggressive XP
project would, we strive to always look for ways to shorten iterations in all
areas of development.

More important than the time between iterations is the complexity between
iterations.  When we minimize the cost of each iteration, we enable more
iterations, and development can happen step by step, instead of in huge leaps
and bounds.  This makes the development most closely resemble extended
rapid prototyping.  Ideas are worked out in code, which then grows and matures
organically into stable, robust software.  Providing a solid framework for
development allows developers to focus on the essentials and ignore as much of
the inessential details as possible for as long as possible.

This is a practice that is valuable in a number of areas, from design,
development, and debugging to building, testing, and integrating.  In section 4
we will discuss a number of tools that the Trilinos Project relies on to enable
short, simple, inexpensive iterations.

\subsection{Process Improvement}
%-------------------------------

Software processes are always a work in progress.  On any project there are
things being done well and things that need work.  One of the difficulties of 
software engineering or software project management is to take the realities
of a given project and mold them into a form that is in agreement with the
theories of accepted development models.  After realizing that a wholesale 
overhaul of an entire project to bring it into compliance with an accepted
model was infeasible, but also that acceptance of sub-optimal processes is
inefficient, the Trilinos Project adopted a model of process improvement by
which the processes that drive the project are always subject to ongoing,
incremental revision and and improvement.  We always keep our eyes open for the 
``low-hanging fruit,'' or those modifications to existing processes that are
likely to yield the most benefit at the least cost.  Whereas "Iterative
Development" involves primarily the incremental improvement of the software,
"Process Improvement" is the incremental improvement of the processes by which
the software is created.  It is a meta-"Iterative Development."

\clearpage

%== Tools =====================================================================

\section{Development Practices and Tools}
\label{Section:Development Practices and Tools}

We have laid out the goals for our project.  We have described some driving
principles that steer us towards those goals.  Now we will discuss how to meet 
our goals and integrate our principles into the project.  The development
practices and tools listed below address many different software development
issues and have been carefully chosen to serve the needs of the project and to
minimize overhead while producing the most benefit.

%quality and reliability are addressed in some way by each of the tools

\subsection{Source Management}
%-----------------------------

% Quality, assessibility, eff use of exp time, support
% tight collab, iterative dev, is a global service

Source management is a very important aspect of software development. 
Important features of a source management tool include providing backups and
version control.  Version control allows developers to revert to previous
versions of the code with a simple command and also makes managing releases
easier by providing the ability to create release branches that can be
developed concurrently with the main development branch.  Bug fixes can then
easily be applied to both branches.  Having a centralized code repository is
invaluable for multi-person development team as it makes it easy to get the
changes that others have made and provides an easy way to resolve differences
in changes made by two people to the same line of code.  It also allows our
closest collaborators to get instant access to the absolute latest versions
of the code, which reduces the length and complexity of iterations.  The
repository is also put to use to contain files that control some of our other
tools (documentation, website, test harness).  This enables convenient control
in a centralized place, but with decentralized access.

Trilinos source code is maintained in a Concurrent Versions System~\cite{CVS}
(CVS) repository.  While there are now a number of source management tools
with more advanced features, CVS continues to meet our needs well.  The cost
associated with migrating to another source management tool and forcing all
developers to learn a new system can not currently be justified by the small
gain in features.  For a listing of CVS features and a description of 
CVS commands, see the GNU CVS Home Page~\cite{CVS}.  For even more information
we recommend {\it Open Source Development with CVS}~\cite{FogelBarCVS}.

In addition to CVS, we also use Bonsai~\cite{Bonsai}, a web-based interface into the CVS
repository information.  This allows developers to easily track changes to 
source code, who made changes, what log message they supplied, what code branch
it happened on, etc.  Bonsai has proved to be an invaluable supplement to CVS.

\subsection{Communication Channels}
%----------------------------------

% Quality, reliability, iterop, eff use of exp time, support
% global services, tight collab, process imp.
% mail man, TUG, phone based.

The value of open lines of communication within a project cannot be 
overemphasized.  Communicating requirements, design decisions, and timelines 
with all team members can lead to better code that is developed efficiently
and naturally promotes process improvement.  The Trilinos Team uses a simple 
tool called Mailman~\cite{Mailman} to supply mail lists that carry much 
of the electronic communication within the project.

The mail list archives are searchable, which allows new Trilinos developers 
to catch up on interesting events from the past and stay up to date on current
development without the risk of someone for getting to CC them on an email.  
There are separate lists for user and developer conversations as well as
announcements.  Lists are also maintained to track CVS checkins and nightly 
test results.  When checking in changes to files in the CVS repository,
developers are prompted to supply a message describing the change.  These log
messages are included in the email and are available in the CVS repository
either through the command line or via the online Bonsai interface.

\subsection{Requirements \& Issue-Tracking}
%------------------------------------------

% all
% global services, tight collab, it dev, process imp. 

An important step in achieving a high level of software quality is tracking
and organizing issues pertaining to faults in the software and issues related
to enhancement requests.  The Trilinos team has utilized a tool called
Bugzilla~\cite{Bugzilla} to achieve this end.  The interface for entering and
searching for bugs is web-based, user friendly, and customizable.  Dependency
tracking makes it easy to track the relationship between bugs.  The Trilinos
team utilizes the concept of a metabug, which is a larger task that is
dependent on multiple smaller tasks.  Metabugs make it is easy for project
leaders (or management) to track the status of issues that depend on many
smaller tasks that are to be completed by one or more team members.  

Although tracking issues in this way does not help to complete the
necessary tasks any quicker, it does allow tasks to be properly prioritized, 
makes sure that issues are not lost or forgotten, and allows project leaders
to quickly  summarize the current state of the project.

\subsection{Documentation}
%-------------------------

% quality, interop, reliab, support
% tight collab

While the means, style, and content of documentation can be hotly debated
topics, few will argue the need for some form of good source documentation.
In a project of any non-trivial size, merely having comments within the source
is insufficient.  It becomes too inefficient to sift through thousands of lines
of source by hand just to find what arguments a function takes.  In Trilinos,
we have adopted the use of Doxygen~\cite{Doxygen}.  Doxygen allows developers
to maintain documentation inline, but then parses source and generates
browsable output in a number of formats.  The Trilinos framework has taken it a
step farther and set up mechanisms by which documentation is automatically
generated twice daily from the latest versions of the source code and posted
online.  Having this documentation up-to-date and readily accessible online
helps to improve interoperability and maximize the amount of support users can
access themselves without needing to contact the development team.

\subsection{Configuration Management}
%------------------------------------

% all
% pkg orthog, gloabal serv, tight collab, 

Achieving a high level of software quality is complicated when
a software project is required to be portable to a wide array of platforms 
and contains a large number of largely autonomous components.
The current Trilinos build system is based on GNU Autoconf~\cite{Autoconf} and
Automake~\cite{Automake}~\cite{GoatBook}.

Of particular interest to CSE software developers is the fact that Autoconf
provides a macro called F77\_FUNC that can be
automatically defined during the configure stage to the proper Fortran
name mangling scheme for the platform that the configure script is run on.
Knowing the Fortran name mangling scheme is necessary when calling routines
from Fortran libraries such as the BLAS~\cite{BLAS1,BLAS2,BLAS3} and
LAPACK~\cite{lapack} from a C or C++ library.  Aside from a few special cases,
Autoconf is able to detect the proper name mangling scheme automatically.

For CSE developers interested in something closer to a plug and play solution,
there exists a Trilinos package called New Package (see section 3.2) that was
developed to jump start an effort to set up a build system that uses Autoconf
and Automake.  The effort was focused primarily on helping new packages join
Trilinos, but the directions that New Package provides can be applied to more
general cases as well.  New Package can be downloaded from the
Trilinos web site.  

No current tools make configuration management a trivial issue.  However, a
configure and build system using Autoconf and Automake has been a
noticeable improvement over a more traditional system using simple makefiles.

\subsection{Information Distribution}
%------------------------------------

% quality, reliab, eff use exp time, support
% global serv, tight collab, proc imp

In any complex software project, there is inevitably a lot of information that
needs to be transferred from the various creators of this information to the
consumers of it.  This includes everything from contact information,
documents, publications, presentations, bug reports, and frequently asked
questions to the software itself.  The natural choice for the delivery of all
this information is a project website.  It might seem painfully obvious that
this is a good solution to a project's information distribution needs, but it
is woefully underutilized by many CSE software projects.  Like so many others
of these tools, a project website need not be a perfect, polished, professional
thing; it just needs to serve its purpose.  So much of the value of a project
website, whether it's for the development team only, or for the general
public, can be had with a very small time investment and a beginner's
knowledge of HTML.  As the website grows incrementally, the growing pains can
be greatly alleviated with a little bit of CSS~\cite{CSS} and PHP~\cite{PHP}.

One of the greatest benefits to be had from a project website is the ability to
bring together the rest of the project's tools.  If only a very simple list of
links, having a comprehensive starting point from which to reach all of a
project's resources is invaluable.

\subsection{Testing}
%-------------------

% all
% pkg orthog, global serv, it dev, proc imp

The success of any software project is critically dependent on good testing.
Testing can be a painful activity when there is no good system in place to
support it.  Like any other activity, if it has to be done manually and from
scratch every time, it will be prone to errors and it won't happen as often as
it should.  To address this, Trilinos has developed over the years a suite of
scripts to run all tests on a number of different platforms automatically on a
regular schedule.  This system includes a standard interface for adding new
tests, which will then be automatically included in the testing.  This helps
to lower the barrier for developers to write and maintain valuable tests.

With a project the size of Trilinos, in addition to the testing itself, the
collection, organization, and distribution of results are particularly
challenging tasks.  To address this, we have developed a database for results,
which is then queried to display the latest results on the website.  Summary
emails are also generated and sent out each morning.  Providing good
information to developers about the state of the code across all target
platforms every day goes a long way to improving quality by tightening
iterations.  No critical bug should live for more than 24 hours without being 
detected.

This testing infrastructure is constantly being improved upon as resources are
available.  We have expanded into coverage testing and are working towards
memory testing.  Every new piece of information automatically gathered and
reported helps to empower developers at a reasonable cost, as it is essentially
amortized across all packages.

\subsection{Release Process}
%---------------------------

The Trilinos Project has invested a lot of time into improving its release
process.  A release process timeline has been established.  Process 
checklists are completed at the framework and package levels for each release,
and checklists, along with associated issues, are stored in Bugzilla.  All
appropriate dependencies are tracked.  The release candidate code is subjected
to tests on each of the Trilinos nightly test harness platforms, as well as the
tests of some of our most important customers.  The timeline and checklists,
along with the structure provided by Bugzilla, have been key in organizing the
complicated efforts of a large number of developers in such a way that releases
can be provided on time, and with confidence in the code.  More information on
the Trilinos release process can be found at \newline
\InlineDirectory{http://software.sandia.gov/trilinos/developer/sqp/release} 
and \newline
\InlineDirectory{http://software.sandia.gov/trilinos/developer/sqp/checklists}.

%\clearpage

%== Conclusion ================================================================

\section{Conclusion}
\label{Section:Conclusion}

Developing quality CSE software is challenging.  Finding the time, energy, and
resources to improve the processes by which you develop it can be even more so.
Often the biggest obstacle is the mere thought of the daunting task of getting
from where you are to where you should be.  But, through organic integration of
these simple, proven, freely-available tools and techniques, you can
incrementally improve the quality of your project's processes which will, in
turn, improve the quality of your software.

\clearpage

%== More Information ==========================================================

Those who are interested in learning more about the Trilinos project should 
consult {\it An Overview of Trilinos}~\cite{Trilinos-Overview} or the
{\it Trilinos Users Guide}~\cite{Trilinos-Users-Guide}.  Trilinos also has an 
extensive web site that can be found at \newline
\InlineDirectory{http://software.sandia.gov/trilinos}~\cite{Trilinos-home-page}.

\clearpage

%== Bibliography ==============================================================

\bibliographystyle{plain}
%\bibliography{SIAMnews}
\bibliography{../CommonFiles/TrilinosBibliography}
\addcontentsline{toc}{section}{References}

\end{document}
