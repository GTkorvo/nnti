
% Outlining a Development Model for CSE Software
% Jim Willenbring, Mike Phenow, Mike Heroux

%## Header ####################################################################

\documentclass[12pt,relax]{article}
%\usepackage{doublespace}
%\setlength {\oddsidemargin}{1in}
%\setlength {\evensidemargin}{1in}
\addtolength {\textheight}{0.2in}
\addtolength {\textwidth}{0.2in}
%DisplayCommand
\linespread{1.6}
%\linespread{1.75}
\newcommand{\DisplayCommand}[1]{%
\par\vspace{1ex}%
{\bf Command:}%
{\hspace{0.2 in}} {\tt #1} {\par\vspace{1ex}}}

% InlineCommand
\newcommand{\InlineCommand}[1]{
  {\hspace{0.01 in}} {\tt #1} {\hspace{0.01 in}}}

% InlineDirectory
\newcommand{\InlineDirectory}[1]{
  {\hspace{0.01 in}} {\tt #1} {\hspace{0.01 in}}}

\usepackage{array}

\title{Outlining a Development Model for CSE Software}

\author{
James Willenbring\\
Michael Phenow\\
Michael Heroux\\
}

% There is a "Printed" date on the title page of a SAND report, so
% the generic \date should generally be empty.
\date{\today} % Remove ``\today'' in final version

% Do we really have to repeat the authors?
%\author{}

%## Content ###################################################################

\begin{document}

\maketitle
%\setcounter{page}{3} % Accounts for blank page at beginning

%== Abstract ==================================================================

%\begin{abstract}

Abstract:  (same as intro. needs to be rewritten.  -mnp)

Developing computational science and engineering (CSE) software presents some 
unique challenges.  CSE software projects frequently don't have large budgets 
to spend on advanced software engineering tools, processes, and personnel.  
They often are developed by small groups of experts--experts in the particular 
computational science or engineering domain in question, not experts on 
software engineering or software project management.  The Trilinos Project,
developed at Sandia National Labs, is no different.  It is our belief that 
there are certain low-cost, high-yield processes and tools that tend to work 
well to enable the development of high quality CSE software.  To demonstrate 
this, we will draw from our experiences with the Trilinos Project to first
outline some high-level goals of the project.  We will then present aspects of
a development model that we think do a good job job of enabling developers to
reach those ends.  Finally, we will discuss some general classes (and specific
instances) of tools that enable the development model, and through that, 
the ultimate goals of the project.  Throughout we will use our experiences on 
the Trilinos project to provide specific examples of these tools and processes.

%\end{abstract}

\clearpage

%== Acknowledgements ==========================================================

\section*{Acknowledgments}

The authors would like to acknowledge the support of the ASCI and LDRD programs 
that funded development of Trilinos and recognize all Trilinos contributors:
Michael Heroux (project leader), Teri Barth, Ross Bartlett, Paul Boggs, Jason
Cross, David Day, Clark Dohrmann, Robert Heaphy, Ulrich Hetmaniuk, Robert
Hoekstra, Russell Hooper, Vicki Howle, Jonathan Hu, Tammy Kolda, Kris
Kampshoff, Sarah Knepper, Joe Kotulski, Richard Lehoucq, Kevin Long, Joe
Outzen, Roger Pawlowski, Eric Phipps, Andrew Rothfuss, Marzio Sala, Andrew
Salinger, Paul Sery, Paul Sexton, Ken Stanley, Heidi Thornquist, Ray Tuminaro
and Alan Williams.

\clearpage
\tableofcontents
\listoftables

\clearpage
%The following file is also used in the User Guide
%\begin{itemize}
\item[Trilinos]
The name of the project.  Also a Greek term which,
loosely translated means ``a string of pearls,'' 
meant to evoke an image that each Trilinos package is a pearl in its 
own right, but is even more valuable when combined with other 
packages.

\item[Package]
A self-contained collection of software in Trilinos
focused on one primary class of numerical
methods.  Also a fundamental, integral unit in the Trilinos framework.

\item[new\_package] A sample Trilinos package containing all of the
infrastructure to install a new package into the Trilinos framework.
Contains the basic directory structure, a collection of sample
configuration and build files and a sample ``Hello World'' package.
Also a website.

\item[Anasazi]
An extensible and interoperable framework for large-scale eigenvalue
algorithms.The motivation for this framework is to provide a generic
interface to a collection of algorithms for solving large-scale 
eigenvalue problems.

\item[AztecOO] 
Linear solver package based on preconditioned Krylov methods.  A 
follow-on to the Aztec solver package~\cite{Aztec2.1}.  
Supports all Aztec 
interfaces and functionality, but also provides significant new 
functionality.

\item[Belos] A Greek term meaning ``arrow.'' Belos is the next
generation of iterative solvers.  Belos solvers are written using
``generic'' programming techniques.  In other words, Belos is written
using TSF abstract interfaces and therefore has no explicit dependence
on any concrete linear algebra library.  Instead, Belos solvers can be
used with any concrete linear algebra library that implements the TSF
abstract interfaces. 

\item[Ifpack] 
Object-oriented algebraic preconditioner, compatible with 
Epetra and AztecOO.  Supports construction and use of parallel
distributed memory preconditioners such as overlapping Schwarz domain
decomposition, Jacobi scaling and local Gauss-Seidel relaxations.

\item[Komplex] 
Complex linear equation solver using equivalent real 
formulations~\cite{DayHero2000}, built on top of Epetra and AztecOO.

\item[LOCA]
Library of continuation algorithms. A package of scalable stability 
analysis algorithms (available as part of 
the NOX nonlinear solver package). When integrated into an application code, 
LOCA enables the tracking of solution branches as a function of system 
parameters and the direct tracking of bifurcation points.

\item[Meros]
Segregated preconditioning package.  Provides scalable block
preconditioning for problems that couple simultaneous solution
variables such as Navier-Stokes problems. 

\item[ML]
Algebraic multi-level preconditioner package.  Provides scalable
preconditioning capabilities for a variety of problem classes.  

\item[NOX]
A collection of nonlinear solvers, designed to be easily integrated
into an application and used with many different linear solvers.

\item[Petra]
A Greek term meaning ``foundation.''  Trilinos has three Petra 
libraries: Epetra, Tpetra and Jpetra that provide basic classes 
for constructing and manipulating matrix, graph and vector
objects.  Epetra is the current production version that is
split into two packages, one core and one extensions.

\begin{itemize}

\item[Epetra] Current C++ production implementation of the Petra
Object Model.  The ``E'' in Epetra stands for ``essential'' implying
that this version provides the most important capabilities that are
commonly needed by our target application base.  Epetra supports real,
double-precision floating point data only (no single-precision or
complex).  Epetra avoids explicit use of some of the more
advanced features of C++, including templates and the Standard
Template Library, that can be impediments to portability.

\item[Tpetra] The future C++ version of Petra, using templates and
other more advanced features of C++.  Tpetra supports arbitrary scalar
and ordinal types, and makes extensive use of advanced C++ features.

\item[Jpetra] A Java implementation of Petra, supporting real,
double-precision data.  Written in pure Java, it is designed to be
byte-code portable and can be executed across multiple compute nodes.

\end{itemize}

\item[Teuchos] A collection of classes and service software that is
useful to almost almost all Trilinos packages.  Includes
reference-counted pointers, parameter lists, templated interfaces to
BLAS, LAPACK and traits for templates.  

\item[TSF] A collection of abstract interfaces that supports application
access to a variety of Trilinos capabilities, supports
interoperability betweeen Trilinos packages and provides future extensibility.
TSF is composed of several packages.  The primary user packages are:  
\begin{itemize}

\item[TSFCore] TSFCore provides 
a basic collection of abstract interfaces to vectors, linear
operators, solvers, etc.  These interfaces provide a common
interface for applications to access one or more packages that
implement the abstract interface.  These interfaces can also be used
by other packages in Trilinos to accomplish the same purpose.

\item[TSFExtended] TSFExtended builds on top of TSFCore, providing implicit
aggregation capabilities and overloaded operators.  
\end{itemize}
\end{itemize}


%Nomenclature
%** Fill in

%== Introduction ==============================================================

\section{Introduction}
\label{Section:Introduction}

Developing computational science and engineering (CSE) software presents some 
unique challenges.  CSE software projects frequently don't have large budgets 
to spend on advanced software engineering tools, processes, and personnel.  
They often are developed by small groups of experts--experts in the particular 
computational science or engineering domain in question, not experts on 
software engineering or software project management.  The Trilinos Project,
developed at Sandia National Labs, is no different.  It is our belief that 
there are certain low-cost, high-yield processes and tools that tend to work 
well to enable the development of high quality CSE software.  To demonstrate 
this, we will draw from our experiences with the Trilinos Project to first
outline some high-level goals of the project.  We will then present aspects of
a development model that we think do a good job job of enabling developers to
reach those ends.  Finally, we will discuss some general classes (and specific
instances) of tools that enable the development model, and through that, 
the ultimate goals of the project.  Throughout we will use our experiences on 
the Trilinos project to provide specific examples of these tools and processes.

%-- Typographical Conventions -------------------------------------------------
\subsection{Typographical Conventions}

Typographical conventions used in the paper are found in
Table~\ref{Table:TypoConventions}.
\begin{table}[ht]
\scriptsize
\begin{center}
\begin{tabular}{|l|l|p{2.0in}|} \hline
Notation & Example & Description \\ \hline
\InlineCommand{Verbatim text} & \InlineCommand{../configure --enable-mpi} & 
URL's, commands, directory and file name examples, and other text associated
with text displayed in a computer terminal window. \\ \hline
\InlineCommand{CAPITALIZED\_TEXT} & \InlineCommand{CXXFLAGS} & 
Environment variables used to configure how tools such as compilers behave. \\ \hline
\InlineCommand{<text in angle brackets>} & \InlineCommand{../configure
<user parameters>} & 
Optional parameters. \\ \hline
\end{tabular}
\end{center}
\caption{\label{Table:TypoConventions} Typographical Conventions for This Document.}
\end{table}

\clearpage

%== Goals =====================================================================

\section{Goals}
\label{Section:Goals}

\subsection{Quality}
%-------------------
Trilinos, like all software projects, seeks quality as a primary goal--quality
both in the colloquial meaning of the word and the particular meaning it
carries in the software engineering world, specifically:  a measure of the
degree to which software meets its stated requirements.

\subsection{Modularity}
%----------------------
Individual sets of functionality in Trilinos are contained in individual,
autonomous modules called packages, which are developed by individuals or small
teams.  Keeping logically distinct pieces of functionality modular is critical
to the long-term health of the project.  

\subsection{Interoperability}
%----------------------------
Because of this modular architecture, it is of utmost importance that the
various packages "play well together."  For Trilinos to realize its full
potential, all of the various independent parts need to work together in
concert.

\subsection{Scalability}
%-----------------------
Trilinos started as a collection of three packages.  In a few short years, it
has organically grown to include more than 25 packages.  To maximize the
benefits reaped from economies of scale and to leverage to power of other
codes, scalability, the ability to continue to add more packages, is a primary
concern for Trilinos.  The degree to which the Trilinos architecture scales
is directly dependent on the level of modularity and interoperability achieved.

\subsection{Efficient Use of Expert Time}
%----------------------------------------
Trilinos packages are developed by experts in the particular domain of a
package.  One very critical goal of the Trilinos project is to make efficient
use of these experts' time.  These experts ought to be spending as much time as
possible in their domain of expertise, not bothering with the comparatively
mundane tasks of software project management, which, unchecked, have a way of
crowding out other tasks.

\subsection{Support}
%-------------------
Finally, the ultimate goal of any piece of software is to actually get used.  
It is important to provide support so that all of the energy spent developing
the software is put to good use, but, here again, it is important that the
experts don't have to spend their time helping users install and use the
software.

\clearpage

%== Development Model =========================================================

\section{Development Model}
\label{Section:Development Model}

For many CSE software projects it simply isn't efficient to use a traditional, 
formal development model.  There are many reasons for this:  CSE software tends
to be research-based or exploratory in nature, CSE software projects tend not
to be on the same scale as large commercial software efforts, and CSE software 
projects often don't have guaranteed funding far into the future.

\subsection{Tight Collaboration}
%-------------------------------

CSE software, like most software, has grown in complexity in recent years.  The
most interesting and challenging problems are generally not solved by an
individual or project team existing in a bubble.  Solid relationships with 
external and internal collaborators are essential.  Likewise, one mark of the 
success for your project is the
amount of interest from others looking to use your code in their project.
And for many projects, these relationships may be even more strongly defined.
It may be the case that an outside collaborator is a significant stakeholder in
the project and whose requirements are of utmost importance.

But how do you gather the requirements of your stakeholders?  What happens
when they change?  Classical development models would prescribe a formal
requirements gathering process to set the direction of the project from the
outset.  From then on, all development has to be traceable back to the
requirements and any deviation from the requirements warrants a formal revision
of them.  

For many CSE projects, this is not a reasonable approach.  When your work, or 
the work of your stakeholders, is research-intensive or exploratory in nature, 
the problem may not be understood well enough at the onset of the project to 
make it worthwhile to define traditional formal 
requirements.  Requirements will likely change and evolve very quickly.  
In such cases, attempting to adhere to a classical
development model becomes, instead of necessary bookkeeping, an exercise in 
paper-generation.

How then to communicate effectively with your stakeholders?  Establish a
collaborative relationship with them.  Bring them into the workings of your
project.  This doesn't mean that they have to be concerned with the day to day
activities, but rather, use close collaboration to gather, implement,
integrate, and iterate on their requirements.  Proactively seek additional
input from stakeholders on a regular basis and keep them well informed, 
possibly by setting up a mail list for them where important announcements
are sent.

Between packages, Trilinos takes advantage of close collaboration by 
establishing well-defined channels of communication.  Issue-tracking
software, mail lists, and regular meetings all give developers of one package
the means to communicate with other packages to coordinate important design
decisions.

Outside of the project, Trilinos maintains close relationships with its primary
stakeholders, some of whom have a developer working on both Trilinos and the
project in question.  This helps ensure the successful integration of Trilinos
into their codes.  It also provides an effective means of staying abreast of
the changing requirements of these external codes.  Close collaborations of
this nature help Trilinos prevent possible problems before they arise and also
help to steer the project in the right direction.

\subsection{Frequent Iterations}
%-------------------------------

The close collaborations mentioned above facilitate the communication of 
design decisions, requirements, and countless other important bits of
information.  But the ultimate goal of all this communication is to produce
working code and the longer development continues without being integrated and
tested, the more time will have to be sunk into the integration process.  This
has led the Trilinos Project to strive for shorter iterations where possible.
%Trilinos tends to release every 6-9 months, so in the context of some 
%development paradigms, such as XP (cite), Trilinos iterations would not be
%considered short, but Trilinos tends to release more often than many of
%its customers and has shortened some iterations within the release cycle...
%not working now, try to revise later
This is a practice that is valuable in a number of areas, from design,
development, and debugging to building, testing, and integrating.  Below we
will discuss a number of tools that the Trilinos Project relies on to enable
these short iterations.

\subsection{Quality Control \& Process Improvement}
%--------------------------------------------------

Software is always a work in progress.  On any project there are always things
being done well and things that need work.  One of the difficulties of 
software engineering or software project management is to take the realities
of a given project and mold them into a form that is in agreement with the
theories of accepted development models.  Realizing that a wholesale overhaul
of an entire project to bring it into compliance with an accepted model was 
infeasible, but also that acceptance of sub-optimal processes is inefficient, 
the Trilinos Project adopted a model of process improvement by which the 
processes that drive the project are always subject to ongoing, incremental 
revision and and improvement.  We always keep our eyes open for the 
"low-hanging fruit," or those modifications to existing processes that would 
yield the most benefit with at the least cost.  

\subsection{Two-Tiered Organizational Architecture}
%--------------------------------------------------

One particularly strong feature of the development model employed by the
Trilinos Project is its two-tiered architecture.  The functionality of Trilinos
is divided into a number of small, autonomous modules, called packages. 
Packages are self-driven and have the freedom to make decisions as they see
fit.  But, at the central Trilinos level, called the framework, centralized
guidelines and services are provided.  The package architecture helps to
encourage modularity throughout Trilinos and also promotes scalability by
providing a sound model for adding new functionality.  The guidelines provided
by the framework help promote interoperability between packages.  The services
provided by the framework help to minimize the overhead that the expert
developers need to concern themselves with.  The framework also provides a
first line of support to effectively shield developers from having to deal
with the trivial support requests.  The net effect of all of these factors is
a higher level of quality throughout the project.

  \subsubsection{Packages/Modules}
  %-------------------------------
  
  Grouping functionality into self-contained packages provides a number of
  benefits:  it keeps teams to small groups of experts, it allows for local
  autonomy, and it allows for local authority.
  
  \begin{enumerate}
  \item  Small expert groups
  
  As mentioned above, Trilinos began as three packages and has since grown to 
  include more than 25.  Each of these packages has a different story.  Some
  were started from scratch within Trilinos.  Many others were existing
  projects that existed elsewhere and were imported into Trilinos.  Of these,
  we find the whole range, from those just off the ground, to mature codes
  that have been in use for years.  In all cases, the development of the 
  code is done by experts in the given domain.  These groups generally consist
  of one to five developers.  This small size keeps the groups
  focused, agile, and accountable.
  
  \item Local autonomy
    
  Packages that join Trilinos after they are relatively mature often wouldn't
  do so if they felt they would be forever dependent on Trilinos.  The design
  of the Trilinos architecture very deliberately seeks to prevent a central
  entity upon which all packages must be dependent.  Many packages came to 
  Trilinos already having an established user base and it is very important
  for some packages to be able to exist either within the Trilinos framework
  or completely without it.
  
  \item Local authority
  
  Similarly, many packages would not be inclined to become a part of Trilinos
  if they had to surrender the control of their package to somebody else.
  Being a part of Trilinos brings with it very few requirements.  Instead,
  there are many guidelines and services that, in practice, are almost all
  adopted by all packages.  Local decisions about the direction or design of
  a package are left in the hands of the developers.

  \end{enumerate}

  \subsubsection{Framework/Project}
  %--------------------------------
  
  Above the packages is the Trilinos Framework.  The framework is the string
  holding the pearls of the various packages together.  The framework exists
  for the benefit of member packages, providing numerous services and 
  suggested practices.

  The Trilinos Framework makes many optional, valuable servies available to
  member packages.  Focusing on a large number of teams
  makes more advanced servies available to teams that otherwise might be too
  small to be able to afford such an array of services.  Some of the standard
  services include source control, an issue reporting tool, and mail lists.
  More advanced services include a package webpage prototype and personal 
  assistance in creating and maintaining package websites as well as a build 
  system that allows all packages to be built as a part of a single process 
  and helps to ease some porting issues.  A functional example package called
  New Package can be used by developers to quickly adapt an existing package 
  to the suggested Trilinos build system or to expediate the process of 
  building a portable package from scratch.  Automated testing on an expanding 
  number of platforms as well as the ability to set up customized test runs is 
  available via the Trilinos Test Harness.  A relatively new service provided 
  by  the framework is the ablility to view test results on the internet from
  all of the Trilinos Test Harness testing platforms.
  
  As mentioned above, Trilinos does not impose a large number of requirements
  on member packages.  Rather, the Trilinos Framework provides suggested 
  practices that,
  for all practical purposes, are adopted by all packages in most cases.  For
  example, packages are required to complete some sort of organized process
  prior to an external release (having a documented release process is a 
  requirement that is imposed by powers above the Trilinos Framework).  The 
  Trilinos Framework team has developed a checklist that packages can complete
  to satisfy this requirement; however, package teams are free to develop an
  alternative process.  At this time, every package uses the default release
  checklist, which saves developers the hassle of developing an individualized
  process and gives them a release process that has
  been improved by several rounds of process improvement based on feedback from
  member package teams.  Another suggested practice is running tests
  associated with a package before committing new code for that package to the 
  CVS repository.  To enable developers to do this more efficiently, there is
  a make target, written by one of the developers, at the Trilinos level that
  calls the part of the Trilinos Test Harness that actually runs tests.

  With some Trilinos team members concentrated on framework-level issues and
  valuable input from numerous Trilinos package developers, the Trilinos 
  Framework now provides numerous services and suggested practices that are
  inexpensive for package teams to utilize and that have evolved over 
  time based on the experience of dozens of people.

  The impact of the Trilinos Framework has spread beyond the boundaries of 
  the project.  The package website template and New Package are available 
  at the Trilinos website.  Other project teams at Sandia are also in various
  stages of adopting or adapting one or more of many Trilinos services, 
  including the test harness, homepage, results webpage, 
  release checklists, or build system to meet their specific needs.

 

% SAMPLE NOTE-BOX ----------------------
%\framebox{\begin{minipage}{\textwidth}{
%NOTE:\\...
%}\end{minipage}}

\clearpage

%== Tools =====================================================================

\section{Tools}
\label{Section:Tools}

\subsection{Version Control System}
%-----------------------------
One of the first steps that should be taken to improve software quality is the 
adoption of a version control system.  Trilinos source code is maintained in a 
Concurrent Versions System~\cite{CVS}
(CVS) repository.  It is very easy to add new files and directories to
the repository.  Projects that already use CVS and are to be merged with 
another project can even retain their CVS history!

CVS has many more features than those that are described below.  For a more 
complete listing of CVS features and a description of valid CVS commands, 
see the GNU CVS Home Page~\cite{CVS}.  Those who still want more information 
might wish to obtain a copy of 
{\it Open Source Development with CVS}~\cite{FogelBarCVS}.

A CVS repository stores all of the information about current and past versions
of all of the files in a project.  (To provide for a safe backup of the 
project, the CVS repository itself should be backed up on a regular basis.)  
Developers can ``checkout'' a snapshot of 
the repository, edit one or more files and then ``commit'' the changes to the
repository.  Complicated situations arise when multiple developers are 
editing the same files at the same time.  How these potential ``conflicts'' 
are resolved is beyond the scope of this document.  Suffice it to say that 
CVS handles all of the situations that it can.  When it cannot safely handle 
the situation, it provides a developer with a clear description of which 
lines of code are in conflict so that the developer can resolve the 
conflict more quickly.  The ability to checkout a snapshot of the 
repository from the past has many obvious advantages including making it easy 
to revert back to an old version of a piece of code that is found to be not 
portable to an important platform and making it possible to pinpoint the exact 
commit that caused an error.

Some of the most useful features of CVS include the fact that a CVS 
repository can be set up in such a way that it is accessible from anywhere, 
and the ability to create multiple ``branches'' of the same project.  For 
example, a development branch and multiple release branches would be 
appropriate for most software projects.  Patches that are applied 
to a release branch can be ``merged'' back into the development branch.  
In addition, security can be enforced in a fine-grained manner if necessary.  

One aspect of using CVS that deserves to be stressed is that comments can be 
made with each commit.  The Trilinos team has found that these comments are 
very important.  A well thought out comment is very helpful when trying to 
debug at a later time.  Even if a commit doesn't make any major changes to the 
code, it is useful to briefly describe what was done.  Some changes that may 
seem to be minor might not be portable.  At the very least, a comment could 
help a developer to rule out a particular commit as the inception point 
of a bug.

Another endearing feature of CVS is that other software quality tools can 
interact seamlessly with CVS.  For example, CVS commit messages can be sent to
a Mailman mail list for distribution.  Also, Bonsai provides a convenient 
interface for accessing CVS information.  See section~\ref{subsect:Bonsai} 
for a thorough description about how the details of CVS commits are accessible 
via Bonsai.

% revise and add branch /tag info and more about release uses in general

The CVS history of a project can be made accessible via a
web-based interface program called Bonsai~\cite{Bonsai}.  This tool can be 
found on the web at 
\newline
\InlineDirectory{http://www.mozilla.org/bonsai.html}.
Bonsai gives developers the ability to view the changes made to the files in 
the repository. Developers can search 
based on filename, directory, branch, date, user who made the 
change, or any combination of these criteria.  The differences between any two 
versions of a file may also be viewed, which can be very helpful when 
debugging.  

Along with the results of a search, the CVS commit messages corresponding to 
the changes that match the query are also shown.  If any Bugzilla bug numbers 
were referenced in a commit message, the number of each bug is clickable.  
Clicking on a bug number will bring up the bug in question.

One very useful advanced feature that Bonsai provides is called ``blame''.  
Blame allows a developer to view the code within a file and see who was the 
last person to edit each line and which version of the file 
the most recent change to each line was made in.

Bonsai doesn't 
provide any capabilities that are not available via another method.  
Rather, Bonsai makes doing difficult things a lot easier.
For example, having the CVS commit message one click away from the line 
by line changes that
were made to a file can save a lot of time.  Also the file differences are 
presented in such a way that multiple lines before and after the differing 
lines are shown to provide some context for the code that was modified.  
The interface for Bonsai is very intuitive and can be learned in a matter of 
minutes.  The time spent learning how to use Bonsai will be more than made up 
for by using it to track and correct one bug.

\subsection{Mail Lists}
%----------------------

Mailman~\cite{Mailman} is an easy to use tool that helps to facilitate 
communication between
those involved with a project and to extend the usefulness of other 
software development tools.  It can be found on the web at 
\InlineDirectory{http://www.list.org}.  Mailman can be used to establish any 
number of 
email lists for a project.  It is advantageous to set up email lists because exactly 
those people who have declared themselves to be interested in a 
topic can be included in the circulation of the discussion.  Without 
mail lists, those who might have insight into an issue can easily be forgotten 
when sending an email.  Conversely, if people routinely get a lot of emails 
concerning topics that they do not care about, some will begin deleting
their email without reading through it.

A specific example of how Mailman 
has had a positive impact on the Trilinos project is that when a new person
joins the development team or becomes a user, that person is able to 
subscribe to the appropriate mail lists and is instantly involved in the 
conversations that take place.  The mail list archives are also searchable, 
which allows people new to Trilinos to catch up on interesting events from the 
past.  Obviously, the searchable archives are useful for new and existing 
developers, users and management.  The self-documenting feature provided 
by Mailman's searchable archives is also useful for creating artifacts for 
audits.  It creates a ``paper trail'' that tracks major decisions that were 
arrived at and closely follows 
the evolution of the entire project, especially when Mailman is used in 
conjunction with other tools such as CVS.

One advanced feature of Mailman that is worth noting is the digest mode option.
This option gathers all of the emails sent to a particular list for an entire 
day and sends them out together in one email.  For those who are not closely 
involved with the discussions that take place on particular lists, but want to 
keep up with what is happening in general, digest mode can be a very 
attractive option.  For urgent messages, digest mode can 
be overridden.

The Trilinos project maintains more than ninety separate mail lists through 
Mailman.  Mail lists are maintained for Trilinos as a whole and for each 
package.  Additional mail lists can be set up very easily.  Each Trilinos 
package usually has five mailing lists.  The example mailing lists mentioned 
below are to be used for issues relating to all of Trilinos.  
The names for the lists pertaining to individual packages follow the same 
naming scheme, simply replace ``Trilinos'' with the name of the package.  For example, the list for Trilinos users is 
called Trilinos-Users and the email address is 
\InlineCommand{trilinos-users@software.sandia.gov}  The list 
for Epetra users is called Epetra-Users and the associated email address is 
\InlineCommand{epetra-users@software.sandia.gov}.  Following a standard naming
scheme allows the name of the lists to be remembered easily.  Although the 
set of mail lists that a software project uses is fully customizable, the 
Trilinos development team has found the break down of lists that is outlined 
below to be very effective.

\begin{itemize}
\item Trilinos-Announce 
\InlineCommand{trilinos-announce@software.sandia.gov}

All Trilinos release announcements and other major news.  The appropriate 
audience for this list includes developers, users and management.

\item Trilinos-Checkins 
\InlineCommand{trilinos-checkins@software.sandia.gov}

CVS commit log messages that are related to Trilinos in general or packages 
that have not had separate lists established.  The appropriate audience for 
this list includes developers who are very closely involved with particular 
aspects of Trilinos development.  Those who are not as closely involved might 
choose to search the archives as necessary, rather than elect to receive 
each of the CVS commit log messages.

\item Trilinos-Developers 
\InlineCommand{trilinos-developers@software.sandia.gov}

All discussions related to Trilinos-specific development (not specific to a 
Trilinos package) are conducted via this list.  Important development 
decisions that originate in other places (regular email, discussions, etc) 
should also be posted to this list (or to the appropriate package list).  
By doing this, the list archive can provide records that explain why various 
changes were made over time.  As the name implies, the appropriate audience for
this list includes Trilinos developers.

\item Trilinos-regression 
\InlineCommand{trilinos-regression@software.sandia.gov}

All regression test output that is not specific to a package.  The appropriate 
audience for this list includes developers who are interested in the daily 
results of the Trilinos test harness.  In reality, few subscribe to this list.
When problems need to be tracked, the archive can provide all available 
information.  This list gathers the artifacts necessary to prove that 
regression testing takes place on a regular basis.

\item Trilinos-Users 
\InlineCommand{trilinos-users@software.sandia.gov}

List for Trilinos Users.  General discussions about the use of Trilinos.  In 
addition to users, developers also subscribe to this lists to answer user 
questions.

\item Trilinos-Leaders
\InlineCommand{trilinos-leaders@software.sandia.gov}

Mailing list for representatives for each Trilinos package.  There are no 
leaders lists for individual packages.  In addition to package leaders and 
interested developers, management will also want to subscribe to this list.
\end{itemize}

Managing the lists requires a minimal amount of overhead.  List administrators
are assigned to each list.  The administrators handle issues such as accepting
or rejecting requests for list membership and posting requests from 
non-members.  This overhead is covered many times by the fine-grained control
of information dispersal that Mailman provides and the archives that 
Mailman maintains.

\subsection{Issue-Tracking}
%------------------------

An important step in achieving a high level of software quality is tracking
and organizing issues pertaining to faults in the software and issues related 
to enhancement requests.  The Trilinos team has implemented a tool called 
Bugzilla~\cite{Bugzilla} to achieve this end.  Bugzilla  can be found 
on the web at \newline
\InlineDirectory{http://www.bugzilla.org}.

All issues related to any Trilinos package are submitted to Bugzilla.  This 
even applies to cases in which 
one developer diagnoses and fixes a bug within a short period of time.  A bug 
report is still very valuable in this case because it provides an artifact 
that outlines the problem and explains how the problem was fixed.  A bug 
report should be filled out with as much detail as possible.  Likewise, after 
a bug has been resolved, the developer should also provide a detailed 
description of the resolution that was used.  These detailed descriptions have 
been used to quickly diagnose problems that tend to come up again and again.
For example, if a particular supported platform doesn't have a header file 
that is widely considered to be ``standard'', multiple developers might try 
to use a similar piece of code that is not portable.  A detailed bug report 
will easily be found based on key words or another piece of relevent search 
criteria, and will also provide a description of the problem and quite 
possibly an alternative technique that will fix the issue. 

Issues tracked by Bugzilla can be searched by owner, platform, and many other 
criteria.  The 
interface for entering and searching for bugs is both user friendly and 
customizable.  Major pieces of a software project can be 
designated as ``products'', different aspects of those products can be 
listed as ``components'', and versions of the various products can also be 
entered.  This breakdown makes it easy to categorize and find issues in the 
Bugzilla database.  For example, Trilinos is composed of individual 
``packages''.  Each package has been designated as a Trilinos product.  One of 
the packages within Trilinos is called Epetra.  So a user who is searching for 
a fix for a build error while building Epetra on a particular 
machine should select ``Epetra'' in the product field, ``Configuration and 
Building'' in the component field, the applicable hardware and OS in the 
fields provided, and
could also fill in the version field.  Issues can be assigned to any 
individual with a Bugzilla account.  If a 
person who is entering a bug report does not know who to assign the bug to,
the bug can be assigned to a default owner of the selected component 
by leaving the ``Assigned To'' field blank.  

When submitting an issue, a 
priority and severity of the bug can be selected (and later changed by the 
issue owner if desired).  Priority and severity are related, but certainly are 
not the same.  Priority refers to the order in which the issue should be 
resolved relative to other issues.  Severity refers to what degree the issue 
negatively affects the proper functions of the software.  Often there is a 
correlation between priority and severity; however, this does not have to be
the case.  For example, an issue that reports a build failure on a 
machine should be given a severity of ``blocker'' (the highest level of 
severity), but might be given a low priority because the build failure occurs
on a platform that is not supported.  Conversely an ``enhancement'' (the 
lowest severity level) might be a very high priority issue.  Regardless of the 
priority and severity assigned to an issue, the issue will not be forgotten 
about in the way that issues sometimes simply fall through the cracks if not 
properly recorded~\cite{Bugzilla}.

In addition to the basic functionality described above, Bugzilla provides some 
useful advanced features.  The more advanced features include
granular security checking, the ability to integrate information with 
other software development tools including CVS, and inter-bug 
dependencies and dependency tracking~\cite{Bugzilla}.  Dependency tracking 
makes it easy to see the relationship 
between bugs.  The Trilinos team has introduced the concept of a metabug which 
is a larger task that is dependent on multiple smaller tasks.  Each of the 
smaller tasks can be assigned to different people. Metabugs make it is easy 
for project leaders (or management) to track the status of important issues. 

% add release uses here

Bugzilla is an extremely useful tool for developers, users and management.  
Developers will especially appreciate the ability to search for issues 
that have been assigned to them and the database of knowledge that grows over
time.  The Trilinos bugzilla database contains over six hundred issues and 
has served as a quick reference to numerous problems that took hours, days, 
or even weeks to solve the first time that the issue was encountered.  The 
Bugzilla interface has led to a noticeable increase in the quality of 
issue submissions.  When presented with fields that can be filled in, those 
who are 
submitting issues tend to provide more information than they would when just 
sending an email to a developer about a problem.  This results in quicker 
issue resolution and a better record of the issue after it has been resolved.  

Users frequently take advantage of Bugzilla's search capability to check on 
the status of issues that they have submitted.  Users also benefit from the 
easy to use interface for submitting issues and from the ability to submit 
issues to default component owners.  

Bugzilla provides management with a quick snapshot of the status of the 
project.  It can also provide more detailed information about pressing issues.

\subsection{Documentation}
% not a complete documentation solution

Doxygen~\cite{Doxygen} is a tool that can be used to create 
extensive documentation for a software project.  In a very short period of 
time, it is possible to take a well-commented piece 
of code and create nicely styled and easy to navigate documentation using 
Doxygen.  The information below is limited 
to an overview of the features of Doxygen.  For information about how to use 
Doxygen, see the Doxygen 
web site at \InlineDirectory{http://www.doxygen.org}.  

Doxygen is meant to be used with C++, C, Java, and IDL, but can also be used 
in a more limited fashion to create documentation for code written in other
programming languages.  Doxygen documentation 
can be generated in several formats including HTML, \LaTeX, Rich Text Format 
(RTF is compatible Microsoft Word), PostScript, and
Unix man pages.  One of the most impressive features of Doxygen is its ability 
to generate dependency graphs, inheritance diagrams, and collaboration 
diagrams~\cite{Doxygen}.  Another impressive feature of Doxygen is that the 
documentation is very easy to navigate.  For a sample of how easily Doxygen 
documentation can be navigated go to
\newline
\InlineDirectory{http://software.sandia.gov/trilinos/packages/epetra}.  Then
click on \InlineDirectory{Documentation}, and then 
\InlineDirectory{Development Docs}. 
Throughout the documentation, the name of objects, 
the name of files, and line numbers within files are all clickable.

\subsection{Configuration Management - Autoconf and Automake}
% revisit all of this!

Achieving a high level of software quality is complicated immensely when 
a software project is required to be portable to a wide array of platforms.
In the past, the Trilinos development team attempted to maintain a large 
number of platform specific makefiles to improve portability.  For the past 
year and a half, we have instead utilized GNU Autoconf~\cite{Autoconf} and 
Automake~\cite{Automake} for 
the purposes of configuring and building Trilinos, respectively.  Again, 
the below discussion will be limited to some highlighted features of Autoconf 
and Automake.  
For a complete description how to use Autoconf and Automake, see 
{\it GNU Autoconf, Automake, and Libtool}~\cite{GoatBook}, which is 
more commonly referred to as the ``Goat Book''.  (Libtool is the third tool 
in the set of three tools commonly referred to as the ``Autotools''.  Trilinos 
does not yet use Libtool, although we plan to incorporate it into the build 
process in the future.)

Unfortunately, Autoconf and Automake have not eliminated all portability 
problems for Trilinos.  We still have to provide a set of platform specific
``invoke-configure''scripts to feed to Autoconf.  In addition, the learning 
curve for Autoconf and Automake was quite steep.  Despite these facts, 
as a whole, the migration to Autoconf and Automake has been very 
beneficial.  

Autoconf provides a macro called F77\_FUNC that can be 
automatically defined during the configure stage to the proper Fortran
name mangling scheme for the platform that the configure script is run on.  
Knowing the Fortran name mangling scheme is necessary when calling routines 
from Fortran libraries such as the BLAS~\cite{BLAS1,BLAS2,BLAS3} and 
LAPACK~\cite{lapack} from a C or C++ library.  Aside from a few special cases, 
Autoconf is able to detect the proper name mangling scheme automatically.  
This eliminates the need for Trilinos packages to have ifdef's for every 
platform that doesn't use the most common scheme.

In addition, the configure script generated by Autoconf detects which 
header files are present on a particular 
machine.  For example, if \InlineDirectory{cmath} is present on a machine, 
pound include that, otherwise, if \InlineDirectory{math.h} is present, pound 
include that.  If neither are present, exit configure with an error indicating 
that one of the two files is required.  Providing the appropriate ifdef's 
for header files on a per machine basis was a difficult task before the 
public release of Trilinos.  Now with users spread around the world, that 
task would be nearly impossible.  Ideally, every machine would 
have all of the latest standard header files available; in practice, that 
is far from the case.

The configure script generated by Autoconf conducts many other configure 
time tests to determine various 
characteristics of the machine it is running on.  Additional tests can 
easily be written to test for other characteristics.  The results of these 
tests can then be used for various purposes, such as conditionally compiling 
a piece of code, or to define any number of macros.

Automake has greatly simplified the process of writing makefiles.  All a 
developer has to do is provide a few simple pieces of information, and 
Automake will generate another file called Makefile.in which is used to 
generate the appropriate Makefile during the configure stage.

As mentioned above, the learning curve for Autoconf and Automake was quite 
steep.  Fortunately, a couple of Trilinos developers have created a package 
called ``new\_package'' to jump start an effort to set up a configure and 
build system that uses Autoconf and Automake.  The effort was focused 
primarily on helping new packages join Trilinos, but the directions that 
new\_package provides can be applied to more general cases as well.  Several
developers have successfully used new\_package as a starting point for 
putting an additional
package into Trilinos.  It is possible to download new\_package from the 
Trilinos web site.  The download page can be found at \newline 
\InlineDirectory{http://software.sandia.gov/trilinos/downloads.html}.  After
downloading new\_package, direct your attention to the README file.

Included with new\_package is a large set of M4~\cite{M4} macros that can 
be freely used by other projects.  These macros perform many common 
configuration tasks such as locating a valid LAPACK~\cite{lapack} library, 
or checking for additional values to prepend to CXXFLAGS that have been 
specified by a user.  The macros can be found in the \InlineDirectory{config}
subdirectory of the new\_package download.  These macros minimize the amount of
redundant effort in using Autoconf and make it easier to apply a common change 
to multiple projects.  The macros are meant to supplement the large number of 
M4 macros that come with Autoconf.  Additional M4 macros that are designed for 
specialized tasks are also available from various sites on the web.

No current tools make porting a trivial issue.  However, a 
configure and build system using Autoconf and Automake is a 
noticeable improvement over a more traditional system using simple makefiles.

\subsection{Website}
%-------------------

\subsection{Test Harness}
%------------------------

\clearpage

%== Conclusion ================================================================

\section{Conclusion}
\label{Section:Conclusion}

\clearpage

%== More Information ==========================================================

** From Jim's paper, revise, but this shows how to cite papers and prevents the
empty bib problem.

Those who are interested in learning more about the Trilinos project should 
consult {\it An Overview of Trilinos}~\cite{Trilinos-Overview} or the
{\it Trilinos Users Guide}~\cite{Trilinos-Users-Guide}.  Trilinos also has an 
extensive web site that can be found at \newline
\InlineDirectory{http://software.sandia.gov/trilinos}~\cite{Trilinos-home-page}.

\clearpage

%== Bibliography ==============================================================

\bibliographystyle{plain}
%\bibliography{SIAMnews}
\bibliography{../CommonFiles/TrilinosBibliography}
\addcontentsline{toc}{section}{References}

\end{document}

%## Notes #####################################################################

%== Outline ===================================================================

%I.    Introduction                                                                                
%      A.  Motivation (problem)
%      B.  Claim (proposed solution)
%      C.  Preview (teaser, road map)
%                                                                      
%II.   Body                                                                                
%      A.  Goals (problem)
%          1.  Quality
%          2.  Modularity
%          3.  Interoperability
%          4.  Scalability
%          5.  Efficient use of expert time
%          6.  Support                                                                                
%      B.  Development model (proposed solution)
%          1.  Tight collaboration
%          2.  Frequent iterations
%          3.  Quality control, process improvement
%          4.  Two-tiered organizational architecture
%              a.  Packages/modules
%                  1.  Small expert groups
%                  2.  Highly autonomous
%                  3.  Local authority
%              b.  Framework/project
%                  1.  Global "control"
%                  2.  Global "services"
%                  3.  Support filter, common front
%      C.  Tools (to support/enable proposed solution)
%          1.  version system
%          2.  mail lists
%          3.  bug-tracking
%          4.  website
%          5.  test harness
%                                                                                
%III.  Conclusion                                                                                
%      A.  Recapitulate problem
%      B.  Recapitulate solution
%      C.  Claim / action item / prediction (say *something*)

%== Titles ====================================================================

%Modern Software Development practices for CSE
%Impact of Modern Software Tools on CSE
%Exploring the Impact of Modern Development practices on CSE
%Examining the Impact of Modern Software Tools on CSE
%Assessing the Impact of Modern Software Tools on CSE
%Assessing the Impact of Tools and Processes on CSE Software Development
%Assessing the Impact of a Customized Development Processes on ...
%Outlining a Development Model for CSE Software
%Making the Case for Agile Models in CSE Software

%== Thesis ====================================================================

%== Ideas =====================================================================
