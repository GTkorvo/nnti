\documentclass{doublecol}
\usepackage{amssymb}
\usepackage{amsbsy,natbib}
\usepackage{amsthm}
\usepackage{amsfonts,amsmath,bm}
\usepackage[dvipsone]{epsfig}

\begin{document}

\setcounter{page}{1}

\LRH{Improving the Development Process for CSE Software}

\RRH{M.A. Heroux, J.M. Willenbring, M.N. Phenow}

\VOL{I}

\ISSUE{1/2/3}

\PUBYEAR{2004}

\BottomCatch

\title{Improving the Development Process for CSE Software}

\authorA{M.A. Heroux}

\affA{Computational Math/Algorithms Department, \\
Sandia National Laboratories, Albuquerque, NM, USA \\ E-mail: maherou@sandia.gov \\
*Corresponding Author}

\authorB{J.M. Willenbring}

\affB{Computational Math/Algorithms Department, \\
Sandia National Laboratories, Albuquerque, NM, USA \\
E-mail: jmwille@sandia.gov}

\authorC{M.N. Phenow}

\affC{Computational Math/Algorithms Department, \\
Sandia National Laboratories, Albuquerque, NM, USA \\ E-mail:
mnpheno@sandia.gov}



\begin{abstract}

Scientific and engineering programming has been around since the
beginning of computing, often being the driving force for new system
development and innovation.  At the same time a continual focus on
new modeling capabilities, and some apparent cultural issues, find
software processes for many computational science and engineering
(CSE) software projects lacking.  Certainly there are notable
exceptions, but our experience has been that CSE software projects,
although committed to writing high-quality software, have few if any
formal software processes and tools in place, and are often unaware
of formal software quality assurance (SQA) concepts.

Presently, increasing complexity of applications and a broad push to
certify computations are dictating a higher standard for CSE
software quality; it is no longer sufficient to claim to write high
quality software. However, traditional software development models
can be impractical for CSE projects to implement. Despite this, CSE
software teams can benefit by implementing valuable SQA processes
and tools.  In this paper we outline some the processes and tools
that are successfully used by the Trilinos Project. These tools and
processes have been useful not only in increasing
\textit{verifiable} software quality, but also have improved overall
software quality, and the development experience in general.

\end{abstract}

\KEY{numerical solvers; software engineering; software
environments.}

\REF{to this paper should be made as follows: Heroux, M.A.
Willenbring, J.M., Phenow, M.N. (2006) `Improving the Development
Process for CSE Software', Int. J. Computational Science and
Engineering, Vol. 1, Nos. 1/2/3, pp. 1--7.}

\BIO{Michael A. Heroux is a Distinguished Member of the Technical
Staff at Sandia National Laboratories, leading the Trilinos solver
project at Sandia, working on solvers for scientific and engineering
applications and on future parallel computer architectures and
languages. James M. Willenbring is the Trilinos project coordinator
and principal framework developer, and lead the Trilinos software
quality efforts.  Michael N. Phenow is the lead Trilinos web and
test framework developer.
%Prior to joining Sandia, he worked for
%Cray Research and Silicon Graphics, specializing parallel numerical
%linear algebra, large-scale parallel applications and
%high-performance parallel computer architecture. Presently his work
%involves leading the Trilinos solver project at Sandia, working on
%solvers for scientific and engineering applications and on future
%parallel computer architectures and languages.
}


\maketitle

%== Introduction ==============================================================
\section{Introduction}
\label{Section:Introduction}


The Trilinos Project, located primarily at Sandia National
Laboratories, is an effort to develop parallel solver algorithms and
libraries within an object-oriented software framework for the
solution of large-scale, complex, multi-physics engineering and
scientific applications.  Trilinos consists of about thirty
packages.  Each package is focused on important, state-of-the-art
algorithms in a particular domain and is developed by a small team
of experts.

Four years ago the Trilinos project was charged with the task of
improving its existing software quality practices.  Since many
computational science and engineering (CSE) software projects are
not dedicated to formal Software Quality Assurance (SQA) practices,
there is not a large body of literature that we can directly
leverage and a lot of work is required to define practices that are
well suited to the project.

A few characteristics of the project make defining formal practices
and processes especially challenging.  Specifically, the
requirements of the project are multi-faceted, global and local
(often defined at the package level), and evolving, which makes it
exceedingly difficult to maintain a global formal requirements
document. Also, the Trilinos team is geographically distributed, so
we have a special emphasis on enabling effective methods of
distributed communication. Finally, as with many CSE software
projects, the budgetary focus is on algorithms development, leaving
little money to put directly towards software quality efforts.

Now, four years into the process of improving software quality
practices, we have found there are certain low-cost, high-yield
processes and tools that tend to work well to enable the development
of high quality CSE software.  To illustrate this, we present some
high-level goals that apply to most CSE software projects, and
Trilinos in particular.  We then present some principles that keep
the Trilinos project on the right development path.  Finally, we
discuss some general classes (and specific instances) of tools that,
guided by our driving principles, help us achieve project goals.

To learn more about the Trilinos project, visit the Trilinos web
site~\cite{Trilinos-home-page}.

%== Goals =====================================================================

\section{Goals}
\label{Section:Goals}

Given the above project characteristics, we outline the high-level
project goals. These goals can be applied to most software projects,
but are described here in the specific context of the Trilinos
Project.

\subsection{Quality}
%-------------------
Trilinos, like all software projects, seeks quality as a primary
goal--quality both in the colloquial meaning of the word and the
particular meaning it carries in the software engineering world,
specifically:  \emph{a measure of the degree to which software meets
its stated and implied requirements}.  Additionally, for an
increasing number of CSE software projects, claiming to do a good
job or anecdotal user opinions of high quality is no longer
sufficient. Customers increasingly require documented software
processes.

\subsection{Modularity}
%----------------------
New solver and support capabilities in Trilinos are introduced as
individual, autonomous modules called \emph{packages}, developed by
individuals or small teams. Keeping logically distinct pieces of
functionality separate is critical to the long-term growth and
health of the project.

\subsection{Interoperability}
%----------------------------
Because of this modular architecture, it is of utmost importance
that the various packages interact well together.  For Trilinos to
realize its full potential, all of the components need to work
together in concert.  This is an important issue for CSE software in
general.  A lot of excellent existing software is under-utilized
because it cannot readily be integrated with other existing software
and brought to bear on a single problem.

\subsection{Scalability}
%-----------------------
Trilinos started as a collection of three packages.  In a few short
years, it has grown organically to include roughly 30 packages, and
continues to grow. To maximize the benefits reaped from economies of
scale and to leverage the power of other codes, scalability (in this
context, the ability to continue to add more packages) is a primary
concern for Trilinos. The degree to which the Trilinos architecture
scales is directly dependent on the level of modularity and
interoperability achieved. Another key scalability issue for
Trilinos is that as packages are added, users should be shielded
from the additional complexity; using Trilinos should not become
significantly more complicated as Trilinos grows, outside of the
complexity that a growing array of functionality inherently
contains.

\subsection{Efficient Use of Expert Time}
%----------------------------------------
Trilinos packages are developed by experts in the particular domain
of a package.  One very critical goal of the Trilinos project is to
make efficient use of these experts' time.  These developers ought
to spend as much time as possible working within their domain of
expertise, leaving the vast majority of the software project
management tasks to those who are specialists in that domain.  The
package domain experts should, however, provide input when selecting
SQA processes, because adopting processes that are not well-suited
to a particular project can decrease, rather than increase,
efficiency.

\subsection{Accessibility and Support}
%-------------------
Finally, the ultimate goal of any piece of software is to actually
get used. In order for people to use it, the software has to be
reasonably accessible to users.  It is also important to provide
support so that all of the energy spent developing the software is
put to good use, but, here again, it is important that the experts
don't have to spend all of their time helping users install and use
the software.

%== Driving Principles ========================================================

\section{Driving Principles}
\label{Section:Driving Principles}

As described in Section~\ref{Section:Introduction} and like many CSE
projects, Trilinos has a unique set of characteristics within the
software world. Within these constraints, goals can then be
formulated for the project.  In our ongoing attempts to achieve
these goals, Trilinos has been guided by a small set of principles
that help the project stay on track when faced with critical
decisions.

\subsection{Package Orthogonality}
%---------------------------------

Trilinos was originally created as a way to bring together parallel
solvers to enable effective reuse and interoperability, and minimize
duplication of effort by solver developers.  The mechanism chosen
for containing a solver was a ``package.''  A Trilinos package is
simply a self-contained piece of software that is developed in the
Trilinos source repository, can build within the Trilinos build
system, and can interact with other Trilinos packages. The Greek
word ``trilinos'' loosely translated means ``string of pearls.'' The
name is meant to convey the idea that each package is independently
valuable, and even more so when combined with other packages. This
image also contains the notion of a common thread holding all the
packages together.

Having both a collection of packages and a central entity gives
Trilinos a two-tiered architecture.  What we think of as the lower
level is simply the packages.  Above that, we have what we call the
Trilinos framework.  The framework is where we seek to capitalize on
economies of scale by providing global services to packages so
duplication of effort is minimized wherever possible. The two
driving principles that keep this two-tiered architecture in its
delicate balance are ``global services,'' which we will discuss in
the next section, and ``package orthogonality.''

One critically important driving principle of the project since its
inception has been to preserve package autonomy.  While most
packages do not stray far from the pack in terms of tools and
processes, and may never do so, autonomy has served us well to
always design to allow for that possibility.  This principle of
package autonomy has evolved into the more encompassing principle of
package \emph{orthogonality}.  In this context, achieving
orthogonality means that the relationships between different
packages, as well as the relationships between packages and the
framework, are such that a change has a minimal effect on other
components.

Achieving a high degree of orthogonality is advantageous for many
reasons. For instance, it allows packages to be effectively
developed by small groups of domain experts without unnecessary
hassles external to the algorithmic problem at hand.  As mentioned
above, Trilinos began as three packages and has since grown to
include roughly 30. Each package has a unique history. Some were
started from scratch within Trilinos.  Others were existing projects
imported into Trilinos.  Of these, we find the whole range, from
those just off the ground, to mature codes that have been in use for
years. In all cases, the development of the code is done by experts
in the given domain.  These groups generally consist of one to five
developers. This small size keeps the groups focused, agile, and
accountable.

Packages that join Trilinos after they are relatively mature often
wouldn't do so if they felt they would be forever dependent on
Trilinos.  The design of the Trilinos architecture very deliberately
seeks to prevent a central entity upon which all packages must be
dependent.  Many packages came to Trilinos already having an
established user base and it is very important for some packages to
be able to exist either within the Trilinos framework or completely
apart from it.

Similarly, many packages would not be inclined to become a part of
Trilinos if they had to surrender the control of their package.
Being a part of Trilinos brings with it very few requirements.
Instead, there are many guidelines and services that, in practice,
are eventually adopted by all packages, \emph{but at a pace
determined by the package developers}. Local decisions about the
direction or design of a package are left in the hands of the
package developers.

Finally, maintaining a high level of package orthogonality and
autonomy keeps us honest. Since packages teams are free to
disassociate from Trilinos at any time, we know that to retain them
(and thus to retain the benefits of the functionality they provide
and the expertise of their developers) we must continue to provide
value to the package developers.

\subsection{Global Services}
%---------------------------
\label{subsect:GlobalServices}

The Trilinos framework exists for the benefit of member packages,
providing numerous services and suggested practices.  The vast
majority of the costs associated with implementing or adopting a
given tool or service are constant and up-front.  The cost of adding
an additional package to Trilinos is usually negligible.  This means
that, on their own, the packages couldn't afford the time, energy,
or expertise required to support such an array of services.  By
having the Trilinos framework provide these services, each package
gains access to the whole suite of powerful services and tools--a
high level of value at a cost that is effectively amortized across
all packages.

Some of the standard services include source control, an issue
reporting and tracking tool, and mail lists.  More advanced services
include a package webpage template, assistance in creating and
maintaining package websites, and a build system that allows all
packages to be built as a part of a single process and helps to ease
porting issues.  A functional example package called New Package can
be used by developers to quickly adapt an existing piece of software
to the suggested Trilinos build system or to hasten the process of
developing portable software from scratch.  The Trilinos test
harness provides a framework for automated testing on a range of
platforms, the ability to set up customized test runs, and the
ability to view all test results online.

As mentioned above, Trilinos does not impose a large number of
requirements on member packages.  Rather, the Trilinos framework
provides suggested practices that, with very few exceptions, are
adopted by all packages. For example, packages are required to
complete some sort of organized process prior to an external release
(having a documented release process is a requirement that is
imposed by powers above the Trilinos framework).  The Trilinos
framework team has developed a checklist that packages can complete
to satisfy this requirement; however, package teams are free to
develop an alternative process.  At this time, every package uses
the default release checklist, which saves developers the hassle of
developing an individualized process and gives them a release
process that has been hardened through process improvement based on
feedback from member package teams.

\subsection{Tight Collaboration}
%-------------------------------

CSE software, like most software, has grown in complexity in recent
years. The most interesting and challenging problems are generally
not solved by an individual or project team working in isolation.
Solid relationships with external and internal collaborators are
essential.  It may even be the case that an outside collaborator is
a significant stakeholder in the project and whose requirements are
of utmost importance.

But how do you gather the requirements of your stakeholders?  What
happens when they change?  Classical development models would
prescribe a formal requirements-gathering process to set the
direction of the project from the outset.  From then on, all
development has to be traceable back to the requirements and any
deviation from the requirements warrants a formal revision of them.

For many CSE projects, this is not a reasonable approach.  When your
work, or the work of your stakeholders, is research-intensive or
exploratory in nature, the problem may not be understood well enough
at the outset of the project to make it worthwhile to define
traditional formal requirements.  Requirements will likely change
and evolve very quickly.  In such cases, attempting to adhere to a
classical development model becomes unnecessary and time-consuming
overhead instead of necessary bookkeeping.

How then to communicate effectively with your stakeholders?
Establish a collaborative relationship with them.  Bring them into
the workings of your project.  This doesn't mean that they have to
be concerned with the day to day activities, but rather, use close
collaboration to gather, implement, integrate, and iterate on their
requirements.  Proactively seek additional input from stakeholders
on a regular basis and keep them well informed.

Trilinos encourages close collaboration amongst packages by
establishing well-defined channels of communication.  Issue-tracking
software, mail lists, and regular meetings all give developers of
one package the means to communicate with other packages to
coordinate and create records of important design decisions.

Outside of the project, Trilinos maintains close relationships with
its primary stakeholders, some of whom have a developer working on
both Trilinos and the project in question.  This helps ensure the
successful integration of Trilinos into their codes.  It also
provides an effective means of staying abreast of the changing
requirements of these external codes.  Close collaborations of this
nature help Trilinos prevent possible problems before they arise and
also help to steer the project in the right direction.

\subsection{Iterative Development}
%---------------------------------

Close collaborations facilitate the communication of design
decisions, requirements, and countless other important bits of
information, but the ultimate goal of all this communication is to
produce working code, and the longer development continues without
being integrated and tested, the more time will have to be sunk into
the integration and debugging processes.  This has led the Trilinos
Project to strive for shorter iterations where possible. While this
does not mean that we release as frequently as an aggressive Extreme
Programming (XP)~\cite{XP} project would, we're always looking for
ways to shorten iterations in all areas of development.

More important than the time between iterations is the complexity
between iterations.  A complex iteration costs time, energy, and
resources.  When we minimize the cost of each iteration, we enable
more iterations, and development can occur step by step, instead of
in huge leaps and bounds.  This makes the development more closely
resemble extended rapid prototyping.  Ideas are worked out in code,
which then grows and matures organically into stable, robust
software.

This principle of iterative, incremental development is valuable in
a number of areas, from design, implementation, and debugging to
building, testing, and integrating.  In
section~\ref{Section:DevelopmentPracticesTools} we will discuss a
number of tools that the Trilinos Project relies on to enable short,
simple, inexpensive iterations.

\subsection{Process Improvement}
%-------------------------------

Software processes are always a work in progress.  On any project
there are things being done well and some that need work.  One of
the difficulties of software engineering or software project
management is to take the realities of a given project and mold them
into a form that is in agreement with the theories of accepted
development models.  After realizing that a wholesale overhaul of
the entire project to bring it into compliance with an accepted
model was infeasible, but also that acceptance of sub-optimal
processes was inefficient, the Trilinos Project adopted a model of
process improvement by which the processes that drive the project
are always subject to ongoing, incremental revision and improvement.
We always keep our eyes open for the low-hanging fruit, or those
modifications to existing processes that are likely to yield the
most benefit at the least cost.  The principle of process
improvement is similar in spirit to the principle of iterative
development, but while iterative development involves primarily the
incremental improvement of the software, process improvement is the
incremental improvement of the processes by which that software is
developed.



%== Tools =====================================================================

\section{Development Practices and Tools}
\label{Section:DevelopmentPracticesTools}

The goals of the project and the principles that steer us toward
those goals have been outlined.  Next, we discuss the practices and
tools Trilinos uses that, guided by the aforementioned principles,
helps us reach our goals. The development practices and tools listed
here address many different software development issues and have
been carefully chosen to serve the needs of the project and to
minimize overhead while producing the most benefit.

%quality and reliability are addressed in some way by each of the tools

\subsection{Source Management}
%-----------------------------

% Quality, assessibility, eff use of exp time, support
% tight collab, iterative dev, is a global service

Reliable source management is critical to a stable software
development environment.  Important features of a source management
tool include providing backups and version control.  Version control
allows developers to revert to previous versions of the code with a
simple command and also makes managing releases easier by providing
the ability to create release branches that can be developed
concurrently with the main development branch, while allowing one
change to be applied to both branches.  Having a centralized code
repository is invaluable for multi-person development teams as it
makes it easy to get the changes that others have made and provides
an easy way to resolve differences in changes made by two people. It
also allows close collaborators to get instant access to the
absolute latest versions of the code, which reduces the length and
complexity of iterations.  A repository is also useful for storing
files that control other tools (documentation, website, test
harness); this enables convenient control in a centralized place,
but with decentralized access.

Trilinos source code is maintained in a Concurrent Versions
System~\cite{CVS} (CVS) repository.  While there are now a number of
source management tools with attractive features, CVS continues to
meet our needs well.  The cost associated with migrating to another
source management tool and forcing all developers to learn a new
system can not currently be justified by the small gain in features.

In addition to CVS, we also use Bonsai~\cite{Bonsai}, a web-based
interface to the information stored in the CVS repository.  This
allows developers to easily see changes made to source code, who
made the changes, what log message they supplied, what code branch
it happened on, and more.  Bonsai has proved to be an invaluable
supplement to CVS.

\subsection{Communication Channels}
%----------------------------------

% Quality, reliability, iterop, eff use of exp time, support
% global services, tight collab, process imp.
% mail man, TUG, phone based.

The value of open lines of communication within a project cannot be
overemphasized.  Communicating requirements, design decisions, and
timelines with all team members naturally promotes process
improvement and leads to better, more efficiently-developed code.
Much of the electronic communication within the project is carried
by email lists provided by a simple tool called
Mailman~\cite{Mailman}.

Mailman list archives are  searchable, which allows new Trilinos
developers to catch up on interesting events from the past and stay
up to date on current development without the risk of someone
forgetting to CC them on an email. There are separate lists for user
and developer conversations as well as for announcements. CVS
checkins and nightly test results are also send to mail lists that
developers can subscribe to.  When committing changes to files in
the CVS repository, developers are prompted to supply a message
describing the change.  These log messages are included in the email
and are available in the CVS repository either on the command line
or via the online Bonsai interface.

\subsection{Requirements and Issue-Tracking}
%------------------------------------------

% all
% global services, tight collab, it dev, process imp.

An important step in achieving a high level of software quality is
tracking enhancement requests and issues pertaining to faults in the
software.  The Trilinos team uses a tool called
Bugzilla~\cite{Bugzilla} to automate this process.  The interface
for entering and searching for bugs is web-based, user friendly, and
customizable.  Dependency-tracking features simplify the task of
tracking the relationship between bugs.  The Trilinos team uses the
concept of a metabug, which is a larger task that is dependent on
multiple smaller tasks. Metabugs make it is easy for project leaders
(or management) to track the status of issues that depend on many
smaller tasks that are to be completed by one or more team members.

Although tracking issues in this way does not help to complete the
necessary tasks any quicker, it does allow tasks to be properly
prioritized, makes sure that issues are not lost or forgotten, and
allows project leaders to quickly summarize the current state of the
project.

\subsection{Documentation}
%-------------------------

% quality, interop, reliab, support
% tight collab

While the means, style, and content of documentation can be hotly
debated topics, few will argue the need for some form of good source
documentation. In a project of any non-trivial size, merely having
comments within the source is insufficient.  It becomes too
inefficient to sift through thousands of lines of source by hand
just to find what arguments a function takes.  In Trilinos, we have
adopted the use of Doxygen~\cite{Doxygen}.  Doxygen allows
developers to maintain documentation inline, but then parses the
source files and generates browsable output in a number of formats.
The Trilinos framework has taken it a step farther and set up
mechanisms by which documentation is automatically generated twice
daily from the latest versions of the source code and posted online.
Having this documentation up-to-date and readily accessible online
helps to improve interoperability and maximize the amount of support
users can access themselves without needing to contact the
development team.

\subsection{Configuration Management}
%------------------------------------

% all
% pkg orthog, gloabal serv, tight collab,

Achieving a high level of software quality is complicated when a
software project consisting of many largely-autonomous components
needs to run on a wide range of platforms.  The current Trilinos
build system is based on GNU Autoconf~\cite{Autoconf} and
Automake~\cite{Automake}~\cite{GoatBook}, which help to minimize the
amount of work needed to build the software on many platforms.

As mentioned in section~\ref{subsect:GlobalServices}, New Package
can be used to quickly set up an Autoconf- and Automake-based build
system for a new or existing piece of software.  No current tools
make configuration management a trivial issue; however, a configure
and build system using Autoconf and Automake has been a noticeable
improvement over a more traditional system using simple makefiles.

\subsection{Information Distribution}
%------------------------------------

% quality, reliab, eff use exp time, support
% global serv, tight collab, proc imp

In any complex software project, there is inevitably a lot of
information that needs to be transferred from the various creators
of this information to the consumers of it.  This includes
everything from contact information, documents, publications,
presentations, bug reports, and frequently asked questions to the
software itself.  The natural choice for the delivery of all this
information is a project website.  It might seem painfully obvious
that this is a good solution for a project's information
distribution needs, but it is woefully underutilized by many CSE
software projects.  Like all of these tools, a project website need
not be a perfect, polished, professional thing; it just needs to
serve its purpose.  So much of the value of a project website,
whether it's for the development team only or for the general
public, can be had with a very small time investment and a
beginner's knowledge of HTML.  As the website grows incrementally,
the growing pains can be greatly alleviated with a little bit of
CSS~\cite{CSS} and PHP~\cite{PHP}.

One of the greatest benefits to be had from a project website is the
ability to bring together the rest of the project's tools.  If only
a very simple list of links, having a comprehensive starting point
from which to reach all of a project's resources is invaluable.

\subsection{Testing}
%-------------------

% all
% pkg orthog, global serv, it dev, proc imp

The success of any software project is critically dependent on good
testing. Testing can be a painful activity when there is no good
system in place to support it.  Like any other activity, if it has
to be done manually and from scratch every time, it will be prone to
errors and it won't happen as often as it should.  To address this,
Trilinos has developed over the years a suite of scripts to run all
tests on a number of different platforms automatically on a regular
schedule.  This system includes a standard interface for adding new
tests, which then get automatically included in the testing.  This
helps to lower the barrier for developers to write and maintain
valuable tests.

With a project the size of Trilinos, in addition to the testing
itself, the collection, organization, and distribution of results
are particularly challenging tasks.  To address this, we have
developed a database for results, which is then queried to display
the latest results on the website.  Summary emails are also
generated and sent out each morning.  This way, no critical bug
should live for more than 24 hours without being detected. Providing
good information to developers about the state of the code across
all target platforms every day goes a long way to improving quality
by tightening iterations.

\subsection{Release Process}
%---------------------------

The Trilinos Project has invested a lot of time into improving its
release process.  We have established a release process timeline to
ensure that all release activities are accomplished on time. Release
process checklists are completed at the framework and package levels
for each release, and checklists, along with associated issues, are
stored in Bugzilla.  All appropriate dependencies are tracked.  The
release candidate code is subjected to tests on each of the Trilinos
nightly test harness platforms, as well as the acceptance tests of
some of our most important customers.  The timeline and checklists,
along with the structure provided by Bugzilla, have been key in
organizing the complicated efforts of a large number of developers
in such a way that releases can be provided on time, and with
confidence in the code.

The release process has benefited greatly from incremental process
improvement. The initial timeline and process checklists were
created based on the what worked fairly well in the past.  By
guaranteeing that important steps would be completed for future
releases and making the effort to improve the processes after each
release, there has been noticeable improvement to the release
process after every Trilinos release cycle.


%== Conclusion ================================================================

\section{Conclusion}
\label{Section:Conclusion}

Historically, software quality assurance and related software
engineering processes and concepts have not been a primary focus for
CSE software projects.  Furthermore, standard software engineering
approaches used for business applications cannot be naively applied.
At the same time, as CSE applications become increasingly part of
high-risk, predictive decision-making, SQA processes and tools will
be necessary.

Developing quality CSE software is challenging. Finding the time,
energy, and resources to improve the processes by which you develop
it can be even more so. Often the biggest obstacle is the mere
thought of the daunting task of getting from where you are to where
you want to be. But, through organic, just-in-time adoption of these
simple, proven, freely-available tools and techniques, one can
incrementally improve the quality of a project's processes which
will, in turn, improve the quality of the software.  This approach
has been very successful for the Trilinos project and appears to be
appropriate to other projects as well.


%== Bibliography ==============================================================
\sectiona{ACKNOWLEDGMENTS}
%== Acknowledgements ==========================================================

The authors would like to acknowledge the support of the ASC and
LDRD programs that funded development of Trilinos and recognize all
Trilinos contributors: Michael Heroux (project leader), Teri Barth,
Ross Bartlett, Paul Boggs, Jason Cross, David Day, Clark Dohrmann,
Michael Gee, Robert Heaphy, Ulrich Hetmaniuk, Robert Hoekstra,
Russell Hooper, Vicki Howle, Jonathan Hu, Tammy Kolda, Kris
Kampshoff, Sarah Knepper, Joe Kotulski, Richard Lehoucq, Kevin Long,
Joe Outzen, Roger Pawlowski, Eric Phipps, Andrew Rothfuss, Marzio
Sala, Andrew Salinger, Paul Sery, Paul Sexton, Ken Stanley, Heidi
Thornquist, Ray Tuminaro and Alan Williams.

\bibliographystyle{apalike}
\bibliography{../CommonFiles/TrilinosBibliography}

\end{document}
