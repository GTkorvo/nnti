%
% This is an example LaTeX file which uses the SANDreport class file.
% It shows how a SAND report should be formatted, what sections and
% elements it should contain, and how to use the SANDreport class.
%
% Build it using
%     latex SANDExample
%     bibtex SANDExample
%     latex SANDExample
%     latex SANDExample
%     dvips -o SANDExample.ps SANDExample.dvi
%     ps2pdf SANDExample.ps SANDExample.pdf
%
% This file and the SANDreport.cls file are based on information
% contained in "Guide to Preparing {SAND} Reports", Sand98-0730, edited
% by Tamara K. Locke.
% Please send corrections and suggestions for improvements to
% Rolf Riesen, Org. 9223, MS 1110, rolf@cs.sandia.gov
%
\documentclass[12pt,relax]{SANDreport}

% If you want to relax some of the SAND98-0730 requirements, use the "relax"
% option. It adds spaces and boldface in the table of contents, and does not
% force the page layout sizes.
% e.g. \documentclass[relax,12pt]{SANDreport}
%
% You can also use the "strict" option, which applies even more of the
% SAND98-0730 guidelines. It gets rid of section numbers which are often
% useful; e.g. \documentclass[strict]{SANDreport}



% ---------------------------------------------------------------------------- %
%
% Set the title, author, and date
%
    \title{An Overview of the Trilinos Project}

    \author{Michael A.~Heroux\\
	  Sandia National Laboratories\\
	  P.O. Box 5800\\
	  Albuquerque, NM 87185-1110 \\
	  maherou@sandia.gov \\
	 }

    % There is a "Printed" date on the title page of a SAND report, so
    % the generic \date should generally be empty.
    \date{}


% ---------------------------------------------------------------------------- %
% Set some things we need for SAND reports. These are mandatory
%
\SANDnum{SAND2003-xxxx}
\SANDprintDate{February 2003}
\SANDauthor{Michael A.~Heroux, Sandia National Laboratories}


% ---------------------------------------------------------------------------- %
% The following definitions are optional. The values shown are the default
% ones provided by SANDreport.cls
%
\SANDreleaseType{Unlimited Release}


% ---------------------------------------------------------------------------- %
% The following definitions do not have a default value and will not
% print anything, if not defined
%
%\SANDsupersed{SAND00-0000}{January 0000}
\SANDdistcategory{UC-999}	% DOE mandates it, but many reports don't have it


% ---------------------------------------------------------------------------- %
%
% Start the document
%
\begin{document}
    \maketitle

    % ------------------------------------------------------------------------ %
    % An Abstract is required for SAND reports
    %
    \begin{abstract}
The Trilinos Project is an effort to develop parallel solver algorithms and libraries 
within 
an object-oriented software framework for the solution of large-scale, complex
multi-physics engineering and scientific applications.   Our emphasis is on 
developing robust, scalable algorithms in a software framework, using abstract 
interfaces for flexible interoperability of components and providing a 
full-featured set of concrete classes that implement all abstract interfaces. 
Trilinos components are primarily written in C++, but provide essential C and 
Fortran user interface support.  We provide an open architecture that allows 
easy integration with other solver packages and we deliver our software to 
the outside community via the Gnu Lesser General Public License
(LGPL)~\cite{gnu-license-site}.
This report provides an overview of Trilinos, discussing the objectives, history,
current development and future plans of the project.
    \end{abstract}


    % ------------------------------------------------------------------------ %
    % An Acknowledgement section is optional but important, if someone made
    % contributions or helped beyond the normal part of a work assignment.
    % Use \section* since we don't want it in the table of context
    %
    \clearpage
    \section*{Acknowledgement}
The authors would like to acknowledge the support of the ASCI and LDRD programs
that funded development of Trilinos and recognize all Trilinos contributors: 
Teri Barth, David Day, Bob Heaphy, Mike Heroux, Robert Hoekstra, Jonathan Hu, 
Tammy Kolda, Rich Lehoucq,
Kevin Long, Roger Pawlowski, Andrew Rothfuss, Andrew Salinger, Ken Stanley, Ray Tuminaro,
Jim Willenbring and Alan Williams.



    % ------------------------------------------------------------------------ %
    % The table of contents and list of figures and tables
    % Comment out \listoffigures and \listoftables if there are no
    % figures or tables. Make sure this starts on an odd numbered page
    %
    \clearpage
    \tableofcontents
    \listoffigures
    \listoftables


    % ---------------------------------------------------------------------- %
    % An optional preface or Foreword
%    \clearpage
%    \section{Preface}
%	Although muggles usually have only limited experience with
%	magic, and many even dispute its existence, it is worthwhile
%	to be open minded and explore the possibilities.


    % ---------------------------------------------------------------------- %
    % An optional executive summary
    \clearpage
    \section{Summary}

A core requirement of many engineering and scientific applications is the need to solve linear and non-linear 
systems of equations, eigensystems and other related problems.  Thus it is no surprise that any 
part of the application that solves these problems is called a ``solver.'' The exact definition of what
specifically constitutes a solver depends on many factors.  However, a good working definition of a solver is
the following: {\it Any piece of software that finds unknown values for some set of discrete
governing equations in 
an application.}  Another characteristic of solvers is that we can often implement them in such a way that they
are ``general-purpose'', so that the
details of how the discrete problem was formed are not specifically needed for the solver to work (although
information about problem characteristics can often be vital to robust solutions.)

General-purpose linear and eigensolvers have been successfully used across a broad set of applications and 
computer systems.  EISPACK~\cite{eispack}, LINPACK~\cite{linpack} and LAPACK~\cite{lapack} are just a few of
the many packages that have made a tremendous impact, providing robust portable solvers to a broad set of 
applications.  More recently packages such as PETSc~\cite{petsc-home-page,petsc-manual,petsc-efficient} 
and Aztec~\cite{Aztec2.1} have provided a large
benefit to applications by giving users access to parallel distributed memory solvers that are easy-to-use and
robust.

Sandia has historically had efforts to develop scalable solver algorithms and software.  Often this
development has been done within the context of a specific application code, providing a good robust solver
that specifically meets the needs of that application.  Even Aztec, one of the most important general-purpose
solvers developed at Sandia, was developed specifically for MPSalsa~\cite{MPSalsa-User-Guide,MPSalsa-Theory} 
and only later extracted 
for use with other applications.  Unfortunately, even though application-focused solvers 
tend to be very robust and can often be made into very effective general-purpose solvers, the opportunity
to re-use the basic set of tools developed for one solver in the development of another solver becomes very
difficult.

The Trilinos Project grew out of this group of established numerical algorithms
efforts at Sandia, motivated by  a recognition that a modest degree of 
coordination among these efforts could have a large positive impact on the quality and
usability of the software we produce and therefore enhance the
research, development and integration of new solver algorithms into
applications.  Although the project has existed for only two years, the degree of effort
required to develop new parallel solvers has been substantially reduced because our common
infrastructure provides an excellent starting point.  Furthermore, many applications are
standardizing on the Trilinos matrix and vector classes.  As a result, these applications
have access to all Trilinos solver components without any unnecessary interface modifications.

This document provides an overview of the Trilinos project,
focusing on the project philosophy and description, and
providing the reader with an summary of the project in its current state.  


    % ---------------------------------------------------------------------- %
    % An optional glossary. We don't want it to be numbered
    \clearpage
    \section*{Nomenclature}
    \addcontentsline{toc}{section}{Nomenclature}
    \begin{itemize}
	\item[Package]
	    A collection of software focused on one primary class of numerical methods
	\item[Trilinos]
	    A Greek term that loosely translated means ``a string of pearls,'' meant
         to evoke an image that each Trilinos package is a pearl in its own right,
         but is even more valuable when combined with other packages.
	\item[Petra]
	    A Greek term meaning ``foundation.''  Trilinos has three Petra libraries: Epetra, 
	    Tpetra and Jpetra (discussed in Section~\ref{subsect:PetraObjectModel}) that
	    provide basic classes for constructing and manipulating matrix, graph and vector 
	    objects.
        \item[AztecOO] Linear solver package based on preconditioned Krylov methods.  A
	  follow-on to the Aztec solver package.  Supports all Aztec interfaces and
	  functionality, but also provides significant new functionality.
	\item[TSF]
	\item[NOX]
	\item[ML]
	\item[Komplex] Complex linear equation solver using equivalent real formulations,
	  built on top of Aztec.
	\item[Ifpack] Object-oriented algebraic preconditioner, compatible with Epetra and
	  AztecOO.
    \end{itemize}


    % ---------------------------------------------------------------------- %
    % This is where the body of the report begins; usually with an Introduction
    %
    \SANDmain		% Start the main part of the report

\section{Introduction}

Research efforts in advanced solution algorithms and parallel solver
libraries have historically had a large impact on engineering and
scientific computing.  Algorithmic advances increase the range
of tractable problems and reduce the cost of solving existing
problems.  Well-designed solver libraries provide a mechanism for
leveraging solver development across a broad set of applications and
minimize the cost of solver integration.  Emphasis is
required in both new algorithms and new software in order
to achieve the maximum impact of efforts.

The Trilinos project encompasses a variety of efforts that are to some
extent self-contained but at the same time inter-related.  The
Trilinos design allows individual packages to grow and mature
autonomously to the extent the algorithms and package developers
dictate. 

Integration of a package into Trilinos, and what Trilinos can provide
to a package, have multiple possibilities
that will be discussed in Section~\ref{sect:TrilinosDesign}.
Section~\ref{sect:EpetraAndTSF} discusses two special Trilinos
packages: Epetra and TSF.  The general definition of a Trilinos
package is presented in Section~\ref{sect:PackageDefinition}
An overview of current software research and
development is given in Section~\ref{sect:Software}.  
Finally, this document contains
an Appendix~\ref{sect:OOTutorial} which gives a brief
tutorial on object-oriented concepts for readers who are unfamiliar
with the area.  

\section{Trilinos Design Philosophy}
\label{sect:TrilinosDesign}
Each Trilinos package is a self-contained, independent piece
of software with its own set of requirements, its own development team
and group of users.  Because of this,
Trilinos itself is designed to respect the autonomy of packages.
Trilinos offers a variety of ways for a particular package to interact with other
Trilinos packages.  It also offers a set of tools that can
assist package developers with builds across multiple platforms, generating
documentation and regression testing across a set of target platforms.
At the same time, what a package {\it must} do to be called a Trilinos
package is minimal, and varies with each package.

\subsection{Services Provided by Trilinos}

Trilinos provides a variety of services to a developer wanting to
integrate a package into Trilinos.  In particular, the following are
provided:
\begin{itemize}
\item Configuration management:
Autoconf~\cite{Autoconf},  Automake~\cite{Automake} and
Libtool~\cite{Libtool} provide a robust, full-featured set of tools for
building software across a broad set of platforms (see also the ``Goat
Book''~\cite{GoatBook}).  Although these
tools are not official standards, they are commonly used in many
packages.  Nearly all existing
Trilinos packages use Autoconf and Automake (and will soon use
Libtool). 

Package developers who are not currently using autotools, but would like
to, can get a jump start by using a Trilinos package called
``new\_package''.  This trivial package exists for one primary purpose.
It walks a developer through the process of setting up a package to
configure and build using autotools.  This package is not yet complete, 
but is far enough along to be of use to a developer who does not have 
extensive Autotools experience. 
 

Trilinos provides a set of M4~\cite{M4} macros that can be used by any other
package that wants to use Autoconf and Automake for configuring and
building libraries.  These macros perform common configuration tasks such as
locating a valid LAPACK~\cite{lapack} library, or checking for a user-
defined MPI C compiler.  These macros minimize the amount of redudant
 effort in using Autotools, and make it easier to apply a general change to
the configure process for all packages.  However, use of these tools is not 
required.


\item Regression testing: Trilinos provides a variety of regression
testing capabilities.  Within a number of Trilinos packages, we employ
``white box'' testing where detailed information about the software is
used and probed.  In addition, Trilinos performs ``black box'' testing
of packages via the Trilinos Solver Framework (TSF) virtual class
interfaces.  Any package that implements the TSF interfaces (see
Section~\ref{subsect:InteropTSF} below), can be tested via this black box
test environment.  ({\bf NOTE: Black box testing via TSF is not in
place at this time}

\item Periodic Testing: Trilinos Packages that configure and build using 
Autotools can easily utilize the the Trilinos test harness.  On a nightly 
basis, the test harness builds the most recent versions of Trilinos libraries 
and runs any tests that have been integrated into the testharness.  Currently 
the testharness only runs on Linux, IRIX64, and DEC/OSF1, but it will 
eventually run on 5-8 platforms.

\item Portable interface to BLAS and LAPACK: The Basic Linear Algebra
Subprograms (BLAS)~\cite{BLAS1,BLAS2,BLAS3} and LAPACK~\cite{lapack}
provide a large repository of robust, high-performance mathematical
software for serial and shared memory parallel dense linear algebra
computations.  However, the BLAS and LAPACK interfaces are Fortran
specifications, and the mechanism for calling Fortran interfaces from
C and C++ varies across computing platforms.  Epetra (and Tpetra)
provide a set of simple, portable interfaces to the BLAS and LAPACK
that provide uniform access to the BLAS and LAPACK across a broad
set of platforms.  These interfaces are accessible to
other packages.

\item Source code repository and build tools: Trilinos source code is
maintained in a CVS~\cite{CVS} repository that is accessible via a
web-based interface package called Bonsai~\cite{Bonsai}.  Features and bug reports
are tracked using Bugzilla~\cite{Bugzilla}, and email lists are
maintained for Trilinos as a whole and for each package.  Support for new
packages can easily be added.  All tools are accessible from the main
Trilinos website~\cite{Trilinos-home-page}.

\end{itemize}

\section{Epetra and TSF: Two Special Trilinos Packages}
\label{sect:EpetraAndTSF}
In order to understand what Trilinos provides beyond the
contributions of each Trilinos package, we briefly discuss two special
Trilinos packages: Epetra and TSF.  These two packages are complimentary,
with TSF providing a common abstract application
programmer interface (API) for other Trilinos packages and Epetra
providing a common concrete implementation of basic classes used by most
Trilinos packages.

\subsection{Epetra}
Matrices, vectors and graphs are basic objects used in most solver
algorithms. Most Trilinos
packages interact with these kinds of objects via abstract interfaces that
allow a package to define what services and behaviors are expected from the objects,
without enforcing a specific implementation.  However, in order to use
these packages, some concrete
implementation must be selected.  Epetra (and in the future other packages described
in Section~\ref{subsect:PetraObjectModel}) is a collection of concrete
classes that supports the construction and use of vectors, sparse
graphs, and dense and sparse matrices.  It provides serial, parallel and
 distributed memory
capabilities.  It uses the BLAS and LAPACK where possible, and as a
result has good performance characteristics.

\subsection{TSF}
\label{subsect:InteropTSF}
Many different algorithms are available to solve any given numerical
problems.  For example, there are many algorithms for solving a system
of linear equations, and many solver packages are available to solve
linear systems.  Which package is appropriate is a function of
many details about the problem being solved and the platform on which
application is being run. However, even though
there are many different solvers, conceptually, from an abstract view,
these solvers are providing a similar capability, and it is
advantageous to utilize this abstract view.
TSF is a collection of abstract classes that provides an application
programmer interface (API) to perform the most common solver
operations.  It can provide a single interface to many different
solvers and has powerful compositional mechanisms that support the
light-weight construction of composite objects from a set of
existing objects.  As a result, TSF users gain easy access to many
solvers and can bring multiple solvers to bear on a single problem.


\section{Trilinos Package Interoperability Mechanisms}
\label{sect:PackageDefinition}
As mentioned above, what a package {\it must} do to be called a Trilinos
package is minimal, and varies with each package.  In this section we
list the primary mechanisms for a package to become part of Trilinos.
Note that each mechanism is an extension or augmentation of package
capabilities, creating connections between packages.  Thus, a package does 
not need to change its internal structure to become part of Trilinos.

\subsection{Mechanism 1: Package Accepts User Data as Epetra Objects}
All solver packages require some user data (usually in the form of
vectors and matrices) or require the user to supply the action of an
operator on a vector.  Accepting this data in the form of Epetra
objects is the first Trilinos interoperability mechanism.  Any package
that accepts user data this way immediately becomes accessible to an
application that has built its data using Epetra.  We expect every
Trilinos package to implement this mechanism in some way.  Since
Epetra provides a variety of ways to extract data from an Epetra
object, minimally we expect that a package can at least copy data from
the user objects that were built using Epetra.  More often, a well-designed
package can typically encapsulate Epetra objects and ask for services from
the Epetra objects without explicitly copying them.

\subsection{Mechanism 2: Package Callable via TSF Interfaces}
TSF provides a powerful set of abstract interfaces that can be used to
interface to a variety of solver packages.  TSF can accept
pre-constructed solver objects, e.g., preconditioners, iterative
solvers, etc., by simple encapsulation or it can
construct solver objects using one of a variety of factories.  (See
Appendix~\ref{sect:OOTutorial} for the definition of a factory.)  Once
constructed, a solver object can be further modified by passing it a
parameter list containing a list of key-value pairs that can control
solver behavior when it is trying to solve a problem.  For example,
the parameter list could specify a residual tolerance for an iterative solver.

A package is callable via TSF if it implements one or more of the TSF
abstract class interface, making it available to TSF users as one of a
suite of possible solver options.

\subsection{Mechanism 3: Package Can Use Epetra Internally}

Another interoperability mechanism available to a package is that of
using Epetra objects as the
internal objects for storing vector, matrices, etc. that are seldom or
never seen by the user.  In many instances, this mechanism has no
practical advantages.  However, in some instances, there can be a
saving in storage requirements.  Furthermore, by using Epetra objects
internally, a package can in turn use other Trilinos packages to
manipulate its own internal objects.

\subsection{Mechanism 4: Package access solver service via TSF interface}
TSF provides an abstract solver interface with access to multiple concrete solvers. 
A package can access solver services via TSF and therefore be able to use
any solver that implements the TSF interface.

\subsection{Mechanism 5: Package Builds Under Trilinos {\tt configure} Scripts}
Trilinos uses Autoconf~\cite{Autoconf} and Automake~\cite{Automake} to
build libraries and test suites.  The Trilinos directory structure
keeps each Trilinos package completely self-contained.  As such, each
package is free to use its own configuration and build process.  At
the same time, Trilinos has a top-level configure script that searches
the directory structure for the existence of other configure scripts,
executing one if it is found, passing on any parameter definitions
from the top level.  Similarly, the make process is also recursive.

A package may easily be automatically built from the top-level
Trilinos configuration and make process by copying and modifying the
Autoconf and Automake scripts from another package.  The benefit for
doing this is that Autoconf and Automake improve the portability of a
package across a broad set of platforms.  Also, Automake provides a
rich set of targets for building libraries, software distributions,
test suites and installation processes.  If a package adopts the
Trilinos configuration and build process, it will be built
automatically along with a large number of other Trilinos packages.

\section{Overview of Current Package Development}
\label{sect:Software}

\subsection{The Petra Object Model}
\label{subsect:PetraObjectModel}
The Petra class libraries provide a
foundation for all Trilinos solver development.  Petra provides object classes for
constructing and using parallel, distributed memory matrices and vectors.  
Petra exists in
multiple forms.  Its most basic form is as an object model~\cite{HeroHoekWill2002}.
As such, it is an abstract 
description of a variety of vector, matrix and supporting classes, along with a 
description of
how these classes interact.  There are presently three implementations
of the Petra Object Model: Epetra, Tpetra and Jpetra.

\subsection{Epetra: Essential Implementation of Petra Object Model}

Epetra~\cite{Epetra-User-Guide}, the current production version of Petra,
 is written for real-valued double-precision scalar field data only, and
restricts itself to a stable core
of the C++ language standard.  As such, Epetra is very portable and stable, and  
is accessible to Fortran and C users.  
Epetra is unique in that it combines in a single package (i) support for generic parallel
machine descriptions, (ii) extensive use of standard numerical libraries, (iii) use of best
practices in object-oriented C++ programming and (iv) parallel data redistribution.
Other basic linear algebra packages, e.~.g.~ the Template Numerical Toolkit~\cite{TNT-site},
PETSc~\cite{petsc-manual} and the Matrix Template Library~\cite{SiekLums98}, feature one
or two of these aspects only.  The availability of Epetra has facilitated rapid development
of numerous applications and solvers at Sandia because many of the complicated issues of
working on a parallel distributed memory machine are handled by Epetra.

Application developers can use Epetra to construct and manipulate matrices
and vectors, and then pass these objects to most Trilinos solver components.  Furthermore,
solver developers can develop many new algorithms relying solely on Epetra classes to
handle the intricacies of parallel execution.  Epetra also has extensive parallel data  
redistribution capabilities, including an interface to the Zoltan load-balancing
library~\cite{zoltan-ug}.

\subsection{Tpetra: Templated C++ Implementation of Petra Object Model}

In addition to Epetra, we have started development of a templated version of Petra,
called Tpetra, that implements the scalar and ordinal fields as templated types.  
When fully developed, Tpetra will allow
matrices and vectors to be composed of real or complex, and single or double precision scalar
values.  Furthermore, in principle, any abstract data type (ADT) can be used as 
the scalar field type as long
as the ADT supports basic mathematical operations such as addition and multiplication and
inversion. Specifically, we could compute using an interval scalar field, matrices, integers,
etc., without any additional code development in Tpetra.  
Tpetra can also use any size integer for indexing.  Typically the ordinal field would be 
an integral data type such as int or long int.  However,
any ADT that supports an indexing capability can be used, including integers in other bases, 
or cyclic indexing. Additionally, Tpetra also
uses the C++ language standard more fully.  In particular, it utilizes the Standard
Template Library (STL)~\cite{Stroustrup}, to provide maximal
algorithmic efficiency with minimal code
development.

We will fully develop Tpetra as a peer library to Epetra. By using partial
specialization of templates, we will base Tpetra on established libraries such as the
BLAS~\cite{BLAS1,BLAS2,BLAS3} and LAPACK~\cite{lapack} and therefore acquire the
performance and robustness of these libraries.
Like Epetra, Tpetra is written for generic parallel distributed
memory computers whose nodes are
potentially shared memory multiprocessors.

\subsection{Jpetra: Java Implementation of Petra Object Model}

In addition to Tpetra, we have started a Java implementation of Petra.  The primary design
goals of this project are to produce a library that is a high performance, pure Java
implementation of Petra.  By restricting ourselves to Java and avoiding the use of the Java
Native Interface (JNI)~\cite{JNI-site} to link to other libraries, we
get the true portability that Java
promises.  The fundamental implication of these goals is that we cannot rely on
BLAS~\cite{BLAS1,BLAS2,BLAS3}, LAPACK~\cite{lapack} or MPI~\cite{MPI}
since they are not written in Java, and we do not use the JNI.
As such, we must track the development of pure Java equivalents of these libraries.  Several
efforts, including Ninja~\cite{MoreMidkGuptArtiWuAlma2001} and
MPJ~\cite{CarpGetoJuddSkjeFox2000}, provide equivalent functionality to the BLAS, LAPACK and
MPI, but are completely written in Java.

We will fully implement Jpetra as a peer library to Epetra.  By making extensive use of Java
interfaces, we can create loose dependencies on emerging BLAS, LAPACK and MPI replacements as
they become mature and stable.  Recently, several research
efforts~\cite{MoreMidkGuptArtiWuAlma2001,SCIMARK-site}
have shown that there is no fundamental performance bottleneck using Java.  Instead, Java
compilers and user practices have been the issue.  As a result, Java holds much promise as a
high performance computing language.  Jpetra will facilitate adoption of Java in scientific and
engineering applications.   Java also has native graphical user interfaces (GUI) support.  A
significant part of Jpetra will be the development of GUI tools for visualization and
manipulation of Jpetra objects.


\subsection{TSF: The Trilinos Abstract Class Package}

Packages like Epetra, Tpetra and Jpetra provide a broad set of powerful 
capabilities.  As a direct
result, their class and interface specifications can be quite complex and difficult to use
for algorithm developers that only need to work with abstract matrix and vector objects.
Furthermore, implementation details are necessarily exposed in the Petra libraries
in a way that commits users
of these packages to pay attention to the particular way these packages are implemented.
For some types of algorithms (for example, algorithms like incomplete factorizations that 
must work directly with matrix coefficients as efficiently as possible), working at this
concrete level is necessary.  

However, many algorithms have no need for this detailed
information.  Forcing use of these concrete classes is a disservice to these algorithm
developers because they are forced to deal with details that are irrelevant to their work.
This causes undue burden on the developer, and make their code tightly connected to the 
concrete library for no good reason.

The TSF abstract class package addresses this potential problem by defining a collection 
of classes that define the expected behavior of abstract matrix and vector objects from a
user perspective, independent of implementation.  The result is that each TSF class has
a much simpler interface than the corresponding Epetra interface.  Furthermore, the TSF
interface can be implemented using Epetra, or any other package that can provide the 
required functionality.

TSF provides abstract interfaces for vector, matrix, operator and solver objects.  In
addition, it has powerful aggregation mechanisms that allow existing TSF objects to 
be combined in a variety of ways to create new TSF objects.
TSF can be useful in many situations.  For example:
\begin{enumerate}
\item Generic Krylov method implementation:  If a preconditioned Krylov solver 
is implemented using TSF vectors and operators, then any concrete package that implements
the TSF vector and operator interfaces can be used with the Krylov solver.
\item Generic solver driver:  If an application accesses solver services via the TSF
solver interfaces, then any solver that implements the TSF solver interface is 
accessible to that application.
\item Aggregate objects to implicitly construct a Schur complement: 
TSF provides mechanisms to implicitly 
construct a matrix of operators, the sum or composition of two operators, the inverse
of an operator, etc.  Similar aggregation mechanisms are available for vectors, matrices
and solvers.  Given these mechanisms, a TSF operator that describes the action of the
Schur compliment of an operator is easy to define.  Let
\[
A=\left[
\begin{array}{ll}
A_{11} & A_{12} \\
A_{21} & A_{22}
\end{array} \right].
\]
Then the Schur compliment of $A$ with respect to $A_{22}$ is 
$S = A_{22} - A_{21}A_{11}^{-1}A_{12}$.
Assume that A11, A12, A21 and A22 are TSF operator objects containing $A_{11}$, $A_{12}$,
$A_{21}$ and $A_{22}$, respectively.  To very explicit, we can think of these objects
 as Epetra matrices that are encapsulated as TSF operators.  Given these four TSF operator 
objects, we can form aggregate TSF operator objects in a variety of ways.  In particular,
TSF has operator classes that are derived from the base TSF operator class (and are thus
also TSF operator objects) but require existing TSF operators to be well-defined.  For
example, one such derived TSF operator class will create a composed operator by
taking two existing TSF operators as
input arguments to its constructor and recording references to the two existing operators.
When the apply method of the composed operator is called, it will in turn call the apply
methods of the two operators that were passed into its constructor.  Similar aggregate
operators exist for the sum of two existing operators and the inverse of an operator.

The following code fragments illustrate the construction and apply processes for the 
ComposedOperator class.  In this example, {\tt left} and {\tt right} are existing TSF operators
and the {\it this} operator is being constructed and used.

\begin{verbatim}
TSFComposedOperator::TSFComposedOperator(const TSFLinearOperator& left,
                                         const TSFLinearOperator& right)
: TSFLinearOperatorBase(right.domain(), left.range()),
left_(left), right_(right) {
  if (left_.domain() != right_.range()) {
   TSFError::raise("domain-range mismatch in TSFComposedOperator ctor");
  }
}

void TSFComposedOperator::apply(const TSFVector& in, 
                                      TSFVector& out) const {

  /* apply operators in order from right to left */
  TSFVector tmp = right_.range().createMember();
  right_.apply(in, tmp);
  left_.apply(tmp, out);
}

\end{verbatim}


\begin{verbatim}
TSFInverseOperator::TSFInverseOperator(const TSFLinearOperator& op,
                                       const TSFLinearSolver& solver)
: TSFLinearOperatorBase(op.range(), op.domain()), 
op_(op), solver_(solver)
{}

void TSFInverseOperator::apply(const TSFVector& in, TSFVector& out) const {

	solver.solve(*this, in, out);
}

\end{verbatim}

There is a number of other aggregation operators.  More details of all the possible 
aggregations can be found in the TSF User Guide~\cite{TSF-User-Guide}.

Given these aggregation capabilities, one can implicitly form the Schur compliment $S$ via
a sequence of TSF operator constructor calls, first constructing a TSF operator
InvA11 containing $A_{11}^{-1}$ by passing
in $A_{11}$ and some solver to use with $A_{11}$ into the TSFInverseOperator constructor, 
then constructing InvA11timesA12 by passing in InvA11 and A12 into TSFComposedOperator, then 
A21timesInvA11timesA12 similarly, and finally S by passing in A22 and -A21timesInvA11timesA12
into TSFSumOperator.

\end{enumerate}

\subsection{AztecOO: Concrete Preconditioned Iterative Solver Package}

AztecOO is an object-oriented follow-on to Aztec~\cite{Aztec2.1}.  As such, it has
all of the same capabilities as Aztec, but provides a more elegant interface and numerous functionality
extensions.  AztecOO specifically solves a linear system $AX=B$ where $A$ is a linear 
operator, $X$ is a multivector containing one or more initial guesses on entry and the
corresponding solutions on exit, and $B$ contains the corresponding right-hand-sides.

AztecOO accepts user matrices and vectors as Epetra objects.  The operator $A$ and any
preconditioner, say $M \approx A^{-1}$, need not be concrete Epetra objects.
Instead, AztecOO expects $A$ and $M$ to be Epetra\_Operator or Epetra\_RowMatrix objects. 
Both Epetra\_Operator and Epetra\_RowMatrix are pure virtual classes.  Therefore, any other
matrix library can be used to supply $A$ and $M$, as long as that library can implement
the  Epetra\_Operator or Epetra\_RowMatrix interfaces, something that is fairly straight-forward
 for most linear solver libraries.

AztecOO provides scalings, parallel domain decomposition preconditioners, and a very robust
set of Krylov methods.  It runs very efficiently on distributed memory parallel computers or
on serial computers.  Also, AztecOO implements the Epetra\_Operator interface.  Therefore,
an AztecOO solver object can be used as a preconditioner for another AztecOO object.

\subsection{Ifpack: Incomplete Factorizations and Other Algebraic Preconditioners}

Ifpack provides local incomplete factorization preconditioners in a
parallel domain decomposition framework.  It accepts user data as Epetra\_RowMatrix objects
(including Epetra\_CrsMatrix, Epetra\_VbrMatrix and Epetra\_MsrMatrix objects, since these
classes implement the Epetra\_RowMatrix interface)
and can construct a large variety of ILU preconditioners.  Ifpack preconditioners implement
the Epetra\_Operator interface.  Therefore, they can be used as preconditioners for AztecOO.
The current released version of Ifpack provides only a relaxed ILUK preconditioner.

\subsection{Komplex: Solver Suite for Complex-valued Linear Systems}

Komplex solves complex-valued linear systems using equivalent real-valued formulations of 
twice the dimension.  Given the following complex-valued linear system:
\begin{equation}
\label{Complex}
C w = d,
\end{equation}
where $C$ is an $m$-by-$n$ known
complex matrix, $d$ is a known right-hand side and $w$ is unknown, 
we can write Equation~(\ref{Complex})
in its real and imaginary terms,
\begin{equation}\label{linearsystem}
( A + i B )(x+iy) = b+ic.
\end{equation}
Equating the real and imaginary parts of the expanded equation, respectively,
gives rise to four
possible 2-by-2 block formulations.  We list one of these in Equation~(\ref{Komplex-1}).
\paragraph{K1 Formulation}
\begin{equation}\label{Komplex-1}
   \left( \begin{array}{rr}
                                    A & -B  \\
                                    B &  A
                             \end{array}
   \right)
   \left( \begin{array}{r}
                                    x  \\
                                    y
                             \end{array}
   \right)
   =
   \left( \begin{array}{r}
                                    b  \\
                                    c
                             \end{array}
   \right).
\end{equation}

Although most preconditioning and iterative methods are generally well-defined for
complex-valued systems, with real-valued systems being a special case, most widely-available
solver packages focus exclusively on real-valued systems or treat complex-valued systems as
an afterthought.  Therefore, by transforming the complex-valued system into a real-valued 
system, we can immediately leverage all of the investment in real-valued solvers.  KomplexOO
constructs an equivalent real-valued formulation for a given complex-valued linear system
and then calls AztecOO to solve the problem, returning the solution back to the user in
a form compatible with the original complex-valued problem.  Details of mathematical and
practical issues of Komplex can be found in Day and Heroux~\cite{DayHero2001}.

\subsection{NOX: Nonlinear Solver Package}

NOX provides a suite of nonlinear solver methods.  It can be easily integrated
into an application with minimal effort.  Historically, many applications have called
linear solvers as libraries, but have provided their own nonlinear solver software.  NOX
can be an improvement because it provides a much larger collection of nonlinear methods,
and can be easily extended as new nonlinear methods are developed.  

NOX does not depend on any particular linear algebra package, making it easy to install. In
order to interface to NOX, the user needs to supply methods that derive from the following abstract classes: 
\begin{itemize}
\item NOX::Abstract::Vector
\item NOX::Abstract::Group
\end{itemize}
The Vector interface supports basic vector operations such as dot products and vector updates. 
The Group interface supports non-vector linear algebra functionality and contains methods 
to evaluate the function and, optionally, the Jacobian.
Complete details are provided on the NOX website~\cite{NOX-home-page}.

Although users can provide their own Vector and Group implementation, NOX provides three implementations
of its own.  These are:
\begin{enumerate}
\item NOX::LAPACK
\item NOX::Epetra
\item NOX::Petsc
\end{enumerate}
The LAPACK interface is an interface to the BLAS/LAPACK library. It is not intended for large-scale
computations, but to serve as an easy-to-understand example of how one might interface to NOX. 
The Epetra interface is an interface to Epetra.
The PETSc interface is an interface with the PETSc library. 

All NOX solvers are in the NOX::Solver namespace. The solvers are accessed via the NOX::Solver::Manager. The
recommended solver is NOX::Solver::LineSearchBased, which is a basic nonlinear solver based on a line search.
Each solver has a number of options that can be specified, as documented in each class or on the NOX Parameter
Reference Page. 

The search directions are in the NOX::Direction namespace and accessed via the NOX::Direction::Manager. The
default search direction for a line-search based method in NOX::Direction::Newton. 

Several line searches are available, as defined in the NOX::LineSearch, and accessed via the
NOX::LineSearch::Manager class. Examples include 
\begin{enumerate}
\item NOX::LineSearch::FullStep
\item NOX::LineSearch::Backtrack
\item NOX::LineSearch::MoreThuente
\end{enumerate}

Convergence or failure of a given solver method is determined by the status tests defined in the
NOX::StatusTest namespace. Various status tests may be combined via the NOX::StatusTest::Combo object. Users
are free to create additional status tests that derive from the NOX::StatusTest::Generic class. 

\subsection{ML: Multi-level Preconditioner Package}

ML is a multigrid, or more generally, a multi-level preconditioner package for solving linear systems
from partial differential equation (PDE) discretizations. Although any linear system can be used with ML,
problems that have an underlying PDE nature have the best chance of successful use of ML.

ML provides several approaches to constructing and solving the multi-level problem:
\begin{enumerate}
\item Algebraic (Vanek) smoothed aggregation approach:  The matrix graph is colored to 
create ``balls'' of aggregate nodes. Then a projection operator is formed by applying a smoother to the 
aggregate nodes, taking into account the null space of the PDE operator.
\item Adaptive Grid approach: The original grid is used as the coarse grid and the adaptive refinements
determined the fine grid.  Prolongation and restriction operators are determined using simple interpolation
and weighted injection.
\item Two-grid approach: A fine and (very) coarse grid are used.  Graph and spatial coordinates are used, 
but there is no necessary correlation required between the two grids.
\end{enumerate}

ML can be run as a stand-alone solver, providing its own smoothers and iterative methods, or can be used with
Aztec or AztecOO.  ML can accept user matrix data in its own format, or as Epetra matrix objects.  More
information is available at the ML website~\cite{ML-home-page}.

\section{Conclusions}

The Trilinos project provides a framework for integrating independent 
solver packages, making the package inter-operable and providing a common ``look-and-feel'' 
for Trilinos users.  Furthermore, Trilinos provides a collection of useful services for
independent solver developers, making integration of a package into Trilinos 
attractive to developers.
The primary advantages that the Trilinos Project provides are:
\begin{enumerate}
\item A common core of basic linear algebra classes:
We can minimize redundant work and jumpstart a new parallel application
by utilizing Petra class libraries to construct 
and manipulate matrix, graph and vector objects.
\item Extensive use of abstract classes, primarily TSF, to define the interaction between Trilinos
packages:  By using abstract interfaces in Trilinos packages, we are
not explicitly dependent on Petra class for functionality.  This allows us to use any
concrete matrix and vector software with Trilinos packages, including PETSc, the BLAS 
and LAPACK.
\item A collection of common software tools and processes: New packages can be 
integrated into Trilinos very easily.  Furthermore, if a package does not have
its own well-developed set of software engineering tools and processes, the Trilinos
design makes it easy for a package to incorporate Autotools, bug and feature tracking,
source code control and mail lists.
\item A one-to-many API for applications: Application developers who adopt the TSF abstract
interfaces gain access to many solvers via a single mechanism.  Furthermore, additional third
party solvers are easily added as necessary.
\item Solver aggregation capabilities:  Via the TSF aggregation capabilities, it is possible
to combine many solvers and bring them to bear on a single problem.
\end{enumerate}

    % ---------------------------------------------------------------------- %
    % References
    %
    \clearpage
    \bibliographystyle{plain}
    \bibliography{TrilinosOverview}
    \addcontentsline{toc}{section}{References}


    % ---------------------------------------------------------------------- %
    % Appendices should be stand-alone for SAND reports. If there is only
    % one appendix, put \setcounter{secnumdepth}{0} after \appendix
    %
    \appendix
\section{A Brief Overview of Some Object-Oriented Concepts}
\label{sect:OOTutorial}

Much of the discussion in this document assumes some familiarity with
object-oriented	concepts and terminology. We realize that some readers may not
be very familiar with these topics. Therefore, we provide this appendix to
cover some of the basic topics, as we understand them and use them.

\subsection{Object-oriented Programming}
We use the term object-oriented programming (OOP) to refer to a philosophy of 
software engineering where procedures (called methods or functions) and data
that are logically related are kept together in a single logical unit called a
class.  Although it is not always clear which data and methods belong in a
given class, we can generally agree on basic associations.  As an example, one
obvious class for a solver framework is a Vector.  For our purposes, we
consider a vector object to have finite dimension and a basis.  Therefore, it
contains data that can be indexed.  Some obvious vector operations are norms,
dot products and vector updates.  Vectors can also be multiplied by a linear operator,
or more specifically a matrix.  However, we commonly put this kind of method in the matrix
class because matrices tend to be more complicated objects and writing the method in the matrix
class is easier.

Some of the strengths of OOP are a strong emphasis on the interaction of objects with each
other, that is, on interfaces between classes.  By focusing on interfaces we get a variety of
benefits.  First, a well-designed interface prescribes {\it what} should be done by a piece of
software, not {\it how} it should be done.  This fact, combined with the fact that a class
owns its data, allows great flexibility in how methods are implemented.  Even more importantly,
once software is in use, OOP techniques give us great flexibility to change the implementation
of a class without changing the interface.  Since a user only works with the interface 
(methods) of a class, we can change the implementation of a class without requiring any major
change in the user code.  

We have used this flexibility within the Trilinos Project.  In particular, earlier versions
of Petra were based on code from Aztec, which allowed us to get working versions of Petra
very quickly.  Over time we replace the Aztec code with implementations that offered more
flexibility and features.  However, the overall design of many of the Petra classes has
remained fundamentally the same.

\subsubsection{Some Key OOP Terms}
Throughout this paper we have used a number of terms repeatedly, and sometimes
interchangeably. In this section, we define these terms. Note that these terms
(and many more) are discussed in great detail in books by
Stroustrup ~\cite{Stroustrup}, Gamma et. al. ~\cite{Gamma},
Meyers ~\cite{Meyers1,Meyers2} and many others.

\paragraph{Virtual Function, Pure Virtual Function}
Virtual functions (also called virtual methods) are functions defined on a
base class that can be redefined in any derived class. When a derived class
redefines a method from its base class, it is said to override that method.
A pure virtual function is a virtual function that is declared but not
implemented in the base class. Pure virtual functions {\it must} be overridden
by derived classes, while ``non-pure'' virtual functions need not be overridden.

\paragraph{Abstract Class, Pure Virtual Class, Interface, Virtual Class}  
These four terms are used to describe classes that are incomplete, and can not
be constructed directly.  The first term is used to describe any class that has
one or more pure virtual methods.  The second two terms describe classes that
have {\it no} executable code.  These classes contain method prototypes only
and cannot be constructed explicitly.  The term pure virtual class tends to be
associated with C++ programming while interface is formally defined in Java.
We tend to use these two terms interchangeably.  A virtual class, like an
abstract class, is one which has some pure virtual methods (prototypes without
code), but has some methods that have a default implementation (sometimes these
implementations are written in terms of other virtual methods).  These
classes cannot be constructed explicitly either.  All four of these class types
must be inherited by a concrete class that implements the virtual
methods, therefore implementing the interface.

\paragraph{Concrete Class, Implementation}  In order for an abstract class to
be used, some other class must provide an implementation of the undeveloped
methods of the abstract class. This implementation class, often called a
concrete class, provides an implementation of the abstract class interface.
Generally the term concrete class can be used to describe any class that can
be constructed, i.e., any class which contains no pure virtual methods.

\paragraph{Base Class, Derived Class, Specialization} 
A concrete class that implements an abstract class is said to be a derived
class while the abstract class is called a base class.  Unrelated to abstract
and concrete classes, we also mention another form of derived class called
specialization.  One class is a specialization of another (base) class if it
is a subset (or special case) of the base class.  For example, given an
existing matrix class, a vector class can be derived by constructing a matrix
object with one column. In other cases a derived class {\it extends} the base
class, providing methods from the base class as well as methods not in the
base class.

\paragraph{Base Class, Polymorphism, Factory}
An abstract class, and in fact any class containing virtual methods, can be
implemented by multiple concrete classes.  In this situation each concrete
class can be used interchangeably to behave as an instance of the base class.
This interchangeability of the concrete classes that implement a common base
is referred to as polymorphism.  For convenience, and to hide the details of
concrete class construction, we often develop a function or class called a
factory that can construct one of a number of concrete classes that have a
common base class.  Once an instance of the concrete class is constructed, the
object is returned as an object of the base class type. In this way, the
calling code (the scope in which the object is to be used) need not know what
the concrete type of the object actually is.

\paragraph{Multiple Inheritance}
Multiple inheritance describes the case where a single concrete class inherits
more than one base class. This feature of the C++ language is utilized by
several classes in the Epetra package. For example, Epetra\_CrsMatrix is a
concrete compressed row sparse matrix class that implements the abstract
interface Epetra\_RowMatrix as well as Epetra\_DistObject, the interface
specification for import and export operations in distributed-memory parallel
environments. An instance of a class that implements multiple base classes may
be passed as an argument where any of those base classes is expected.

\paragraph{Templates, Traits}
C++ classes (and stand-alone functions) may be written in terms of one or more
generic type parameters. Such classes are called templates. An example could be
a matrix class that may be instantiated with any type of coefficient data --
double-precision floating-point numbers, integers, etc. In a templated class,
the implementation code doesn't know the type of the template parameter. In
many cases this is a severe limitation, for instance if a templated vector
class is to call through to BLAS functions it is necessary to distinguish
between calling 'dnrm2', 'snrm2' or 'dznrm2'. Another example is the need to
associate different MPI data-types with template parameters. This limitation
can be addressed using a template technique called traits ~\cite{MyersTraits}.
Traits are essentially a way of associating a set of types and methods with
the specific type used to instantiate the template. This is accomplished by
using a secondary template which has a specialization for each possible type
that is to be supported. This secondary template is only used internally, and
is not exposed to the end user.

    % \printindex

    \begin{SANDdistribution}
	\SANDdistExternal{1}{An Address\\ 99 $99^{th}$ street NW\\City, State}
	\SANDdistExternal{3}{Some Address\\ and street\\City, State}
	\bigskip
	\SANDdistExternal{12}{Another Address\\ On a street\\City, State\\U.S.A.}


	\SANDdistInternal{1}{1110}{Rolf Riesen}{9223}

	% Housekeeping copies necessary for every unclassified report:
	\SANDdistInternal{1}{9018}{Central Technical Files}{8940-2}
	\SANDdistInternal{2}{0899}{Technical Library}{4916}
	\SANDdistInternal{2}{0619}{Review \& Approval Desk}{4916}

	% If report has a Patent Caution or Patent Interest, add this:
	\SANDdistInternal{3}{0161}{Patent and Licensing Office}{4916}
    \end{SANDdistribution}

\end{document}
