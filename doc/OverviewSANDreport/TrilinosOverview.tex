%
% This is an example LaTeX file which uses the SANDreport class file.
% It shows how a SAND report should be formatted, what sections and
% elements it should contain, and how to use the SANDreport class.
%
% Build it using
%     latex SANDExample
%     bibtex SANDExample
%     latex SANDExample
%     latex SANDExample
%     dvips -o SANDExample.ps SANDExample.dvi
%     ps2pdf SANDExample.ps SANDExample.pdf
%
% This file and the SANDreport.cls file are based on information
% contained in "Guide to Preparing {SAND} Reports", Sand98-0730, edited
% by Tamara K. Locke.
% Please send corrections and suggestions for improvements to
% Rolf Riesen, Org. 9223, MS 1110, rolf@cs.sandia.gov
%
\documentclass[12pt,strict]{SANDreport}

% If you want to relax some of the SAND98-0730 requirements, use the "relax"
% option. It adds spaces and boldface in the table of contents, and does not
% force the page layout sizes.
% e.g. \documentclass[relax,12pt]{SANDreport}
%
% You can also use the "strict" option, which applies even more of the
% SAND98-0730 guidelines. It gets rid of section numbers which are often
% useful; e.g. \documentclass[strict]{SANDreport}



% ---------------------------------------------------------------------------- %
%
% Set the title, author, and date
%
    \title{An Overview of the Trilinos Project}

    \author{Michael A.~Heroux\\
       Numerical and Applied Mathematics Department \\
	  Sandia National Laboratories\\
	  P.O. Box 5800\\
	  Albuquerque, NM 87185-1110 \\
	  maherou@sandia.gov \\
	 }

    % There is a "Printed" date on the title page of a SAND report, so
    % the generic \date should generally be empty.
    \date{}


% ---------------------------------------------------------------------------- %
% Set some things we need for SAND reports. These are mandatory
%
\SANDnum{SAND2002-xxxx}
\SANDprintDate{August 2002}
\SANDauthor{Michael A.~Heroux, Sandia National Laboratories}


% ---------------------------------------------------------------------------- %
% The following definitions are optional. The values shown are the default
% ones provided by SANDreport.cls
%
\SANDreleaseType{Unlimited Release}


% ---------------------------------------------------------------------------- %
% The following definitions do not have a default value and will not
% print anything, if not defined
%
%\SANDsupersed{SAND00-0000}{January 0000}
\SANDdistcategory{UC-999}	% DOE mandates it, but many reports don't have it


% ---------------------------------------------------------------------------- %
%
% Start the document
%
\begin{document}
    \maketitle

    % ------------------------------------------------------------------------ %
    % An Abstract is required for SAND reports
    %
    \begin{abstract}
The Trilinos Project is an effort to develop parallel solver algorithms within 
an object-oriented software framework for the solution of large-scale, complex
multi-physics engineering and scientific applications.   Our emphasis is on 
developing robust, scalable algorithms in a software framework, using abstract 
interfaces for flexible interoperability of components and providing a 
full-featured set of concrete classes that implement all abstract interfaces. 
Trilinos components are primarily written in C++, but provide essential C and 
Fortran user interface support.  We provide an open architecture that allows 
easy integration with other solver packages and we deliver our software to 
the outside community via the Gnu Lesser General Public License
(LGPL)~\cite{gnu-license-site}.
This report provides an overview of Trilinos, discussing the objectives, history,
current development and
future plans of the project.
    \end{abstract}


    % ------------------------------------------------------------------------ %
    % An Acknowledgement section is optional but important, if someone made
    % contributions or helped beyond the normal part of a work assignment.
    % Use \section* since we don't want it in the table of context
    %
    \clearpage
    \section*{Acknowledgement}
The author would like to acknowledge the support of the ASCI and LDRD programs
that funded development of Trilinos and the talented group of Trilinos contributors: 
Robert Hoekstra, Alan Williams, Richard Lehoucq, James Willenbring, Kevin Long, Tamara
Kolda, Roger Pawlowski, David Day, Ray Tuminaro, Jonathan Hu, Mark Adams and Teri
Barth.



    % ------------------------------------------------------------------------ %
    % The table of contents and list of figures and tables
    % Comment out \listoffigures and \listoftables if there are no
    % figures or tables. Make sure this starts on an odd numbered page
    %
    \clearpage
    \tableofcontents
    \listoffigures
    \listoftables


    % ---------------------------------------------------------------------- %
    % An optional preface or Foreword
%    \clearpage
%    \section{Preface}
%	Although muggles usually have only limited experience with
%	magic, and many even dispute its existence, it is worthwhile
%	to be open minded and explore the possibilities.


    % ---------------------------------------------------------------------- %
    % An optional executive summary
    \clearpage
    \section{Summary}
The Trilinos Project grew out of a group of established numerical algorithms
efforts at Sandia, motivated by  a recognition that a modest degree of 
coordination among these efforts could have a large positive impact on the quality and
usability of the software we produce.  This document describes the Trilinos project,
focusing on the project philosophy and description, and
providing the reader with an overview of the project in its current state.  It also
includes brief developer and user guides.


    % ---------------------------------------------------------------------- %
    % An optional glossary. We don't want it to be numbered
    \clearpage
    \section*{Nomenclature}
    \addcontentsline{toc}{section}{Nomenclature}
    \begin{itemize}
	\item[Package]
	    a collection of software focused on one primary class of numerical methods
	\item[Trilinos]
	    A Greek term that loosely translated means ``a string of pearls,'' meant
         to evoke an image that each Trilinos package is a pearl in its own right,
         but is even more valuable when combined with other packages.
    \end{itemize}


    % ---------------------------------------------------------------------- %
    % This is where the body of the report begins; usually with an Introduction
    %
    \SANDmain		% Start the main part of the report

\section{Introduction}

Trilinos is.

\section{A Brief Overview of Some Object-Oriented Concepts}

Object-oriented.

\section{Trilinos Design Philosophy}

Each Trilinos package is a self-contained, independent piece
of software with its own set of requirememts, its own development team
and a group of users.  Because of this,
Trilinos itself is designed to respect the autonomy of packages.
Trilinos offers a variety of ways for a package to interact with other
Trilinos components.  Trilinos also offers a set of tools that can
assist a package with builds across muliple platforms, generating
documentation and regression testing across a set of target platforms.
At the same time, what a package {\it must} do to be called a Trilinos
package is minimal, and varies with each package.

\subsection{Services Provided by Trilinos}

Trilinos provides a variety of services to a developer wanting to
integrate a package into Trilinos.  In particular, the following are
provided:
\begin{itemize}
\item Configuration management:
Autoconf~\cite{AutoconfManual},  Automake~\cite{Automake} and
Libtool~\cite{Libtool} provide a robust, full-featured set of tools for
building software across a broad set of platforms.  Although these
tools are not official standards, they are commonly used in many
packages.  Many existing
Trilinos packages use Autoconf and Automake (and will soon use
Libtool). In order to
minimize the amount of redudant effort, Trilinos provides a set of
M4~\cite{M4} macros that can be used as a starting point for any other
package that wants to use Autoconf and Automake for configuring and
building libraries.  However, use of these tools is not required.

\item Regression testing: Trilinos provides a variety of regression
testing capabilities.  Within a number of Trilinos packages, we employ
``white box'' testing where detailed information about the software is
used and probed.  In addition, Trilinos performs ``black box'' testing
of packages via the Trilinos Solver Framework (TSF) virtual class
interfaces.  Any package that implements the TSF interfaces (see
Section~\cite{Interop_TSF} below), can be tested via this black box
test environment.  ({\bf NOTE: Black box testing via TSF is not in
place at this time}


\item Portable interface to BLAS and LAPACK: The Basic Linear Algebra
Subprograms (BLAS)~\cite{BLAS1,BLAS2,BLAS3} and LAPACK~\cite{lapack}
provide a large repository of robust, high-performance mathematical
software for serial and shared memory parallel dense linear algebra
computations.  However, the BLAS and LAPACK interfaces are Fortran
specifications, and the mechanism for calling Fortran interfaces from
C and C++ varies across computing systems.  Epetra (and Tpetra)
provide a set of simple, portable interfaces to the BLAS and LAPACK
that hide these interface variations.  These interfaces are accessible to
other packages.

\item Source code repository and build tools: Trilinos source code is
maintained in a CVS~\cite{CVS} repository that is accessible via a
web-based interface package called Bonsai~\cite{Bonsai}.  Features and bug reports
are tracked using Bugzilla~\cite{Bugzilla}, and email lists are
maintained for Trilinos as a whole and each package.  Support for new
packages can easily be added.  All tools are accessible from the main
Trilinos website~\cite{Trilinos-home-page}.

\end{itemize}

\section{Epetra and TSF: Two Special Trilinos Packages}

In order to understand what Trilinos provides beyond the
contributions of each Trilinos package, we briefly discuss two special
Trilinos packages: Epetra and TSF.  These two packages are complimentary,
with TSF providing a common abstract application
programmer interface (API) for other Trilinos packages and Epetra
providing a common concrete implementation of basic classes used by most
Trilinos packages.

\subsection{Epetra}
Matrices, vectors and graphs are basic objects used in most solver
algorithms. Most Trilinos
packages interact with these objects via abstract interfaces that
allow a package to define what services and behavior are expected,
without enforcing a specific implementation.  However, in order to use
these packages, some concrete
implementation must be select.  Epetra (and in the future other packages described
in Section~\ref{subsect:PetraObjectModel}) is a collection of concrete
classes that supports the construction and use of vectors, sparse
graphs and dense and sparse matrices.  It provides serial, parallel and distributed memory
capabilities.  It uses the BLAS and LAPACK where possible, and as a
result has good performance characteristics.

\subsection{TSF}
Many different algorithms are available to solve any given numerical
problems.  For example, there are many algorithms for solving a system
of linear equations, and many solver packages are available to solve
linear systems.  Which package is appropriate is a function of
many details about the problem being solved. However, even though
there are many different solvers, conceptually, from an abstract view,
these solvers are providing a similar capability, and it is
advantageous to utilize this abstract view.
TSF is a collection of abstract classes that provides an application
programmer interface (API) to perform the most common solver
operations.  It can provide a single interface to many different
solvers and has powerful compositional mechanisms that support the
light-weight construction of composite objects from a set of
existing objects.  As a result, TSF users gain easy access to many
solvers and can bring multiple solvers to bear on a single problem.


\section{Trilinos Package Interoperability Mechanisms}

As mentioned above, what a package {\it must} do to be called a Trilinos
package is minimal, and varies with each package.  In this section we
list the primary mechanisms for a package to become part of Trilinos.
Note that each mechanism is an extension to or augmentation of package
capabilities that creates connections between packages.  

\subsection{Package Accepts User Data as Epetra Objects}
All solver packages require some user data, usually in the form of
vectors and matrices, or require the user to supply the action of an
operator on a vector.  Accepting this data in the form of Epetra
objects is the first Trilinos interoperability mechanism.  Any package
that accept user data this way immediately becomes accessible to an
application that has built its data using Epetra.  We expect every
Trilinos package to implement this mechanism in some way.  Since
Epetra provides a variety of ways to extract data from an Epetra
object, minimally we expect that a package can at least copy data from
the user objects.

\subsection{Package Callable via TSF Interfaces}

\subsection{Package Can Use Epetra Internally}
\subsection{Package Builds Under Trilinos {\tt configure} Scripts}

\section{Overview of Current Algorithms Research}

\subsection{Robust Incomplete Factorizations}
\subsection{Block Krylov Methods}
\subsection{Complex Valued Linear Systems}
\subsection{Tensor Methods for Nonlinear Systems}

\section{Overview of Current Package Development}

\subsection{The Petra Object Model}
\label{subsect:PetraObjectModel}
\subsection{Epetra}
\subsection{Tpetra}
\subsection{Jpetra}

\subsection{TSF}

\subsection{AztecOO}

\subsection{Ifpack}

\subsection{Komplex}

\subsection{NOX}
	

    \section{Conclusion}


    % ---------------------------------------------------------------------- %
    % References
    %
    \clearpage
    \bibliographystyle{plain}
    \bibliography{TrilinosOverview}
    \addcontentsline{toc}{section}{References}


    % ---------------------------------------------------------------------- %
    % Appendices should be stand-alone for SAND reports. If there is only
    % one appendix, put \setcounter{secnumdepth}{0} after \appendix
    %
    \appendix
    \section{Historical Perspective}
	This is an example of an appendix.

	If we follow~\cite{Sand98-0730} strictly, we would have to
	have a separate bibliography section for each appendix.
	The style file doesn't provide that, but it can be done
	using the {\tt bibunits} and {\tt chapterbib} packages.

	If there are many subsections in an appendix, it should also
	have its own table of contents. Again, the SAND report class
	file does not provide that functionality.

	\subsection{The Past a Long Time Ago}
	    This is where we talk about things so old nobody
	    can verify them. We are safe.

	\subsection{The Past More Recently}
	    Now we have to be a little bit more careful, since
	    records exist from that time, and some people still
	    alive actually lived back then.


    \section{Some Other Appendix}
	Just to show what a second Appendix would look like. It contains
	a table. Each appendix is supposed to be self-contained, so
	tables and figures are not supposed to show up in the main
	table of contents. There can be a separate table of contents
	for each appendix.

	\begin{table}[ht]
	    \centering
	    \caption{A small table}
	    \bigskip

	    \begin{tabular}{|c|c|}
		\hline
		    A & B  \\ \hline
		    C & D  \\ \hline
	    \end{tabular}
	    \label{tab3}
	\end{table}

	\begin{figure}[ht]
	    \centering
	    \begin{picture}(50,50)(0,0)
		\put(25,25){\circle{1}}
		\put(25,25){\circle{5}}
		\put(25,25){\circle{10}}
		\put(25,25){\circle{15}}
		\put(25,25){\circle{20}}
		\put(25,25){\circle{25}}
		\put(25,25){\circle{30}}
		\put(25,25){\circle{35}}
		\put(25,25){\circle{40}}
		\put(25,25){\circle{45}}
		\put(25,25){\circle{50}}
	    \end{picture}
	    \caption{Dizzy yet?}
	    \label{fig4}
	\end{figure}

    % \printindex

    \begin{SANDdistribution}
	\SANDdistExternal{1}{An Address\\ 99 $99^{th}$ street NW\\City, State}
	\SANDdistExternal{3}{Some Address\\ and street\\City, State}
	\bigskip
	\SANDdistExternal{12}{Another Address\\ On a street\\City, State\\U.S.A.}


	\SANDdistInternal{1}{1110}{Rolf Riesen}{9223}

	% Housekeeping copies necessary for every unclassified report:
	\SANDdistInternal{1}{9018}{Central Technical Files}{8940-2}
	\SANDdistInternal{2}{0899}{Technical Library}{4916}
	\SANDdistInternal{2}{0619}{Review \& Approval Desk}{4916}

	% If report has a Patent Caution or Patent Interest, add this:
	\SANDdistInternal{3}{0161}{Patent and Licensing Office}{4916}
    \end{SANDdistribution}

\end{document}
