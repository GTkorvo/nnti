%
% This is an example LaTeX file which uses the SANDreport class file.
% It shows how a SAND report should be formatted, what sections and
% elements it should contain, and how to use the SANDreport class.
%
% Build it using
%     latex SANDExample
%     bibtex SANDExample
%     latex SANDExample
%     latex SANDExample
%     dvips -o SANDExample.ps SANDExample.dvi
%     ps2pdf SANDExample.ps SANDExample.pdf
%
% This file and the SANDreport.cls file are based on information
% contained in "Guide to Preparing {SAND} Reports", Sand98-0730, edited
% by Tamara K. Locke.
% Please send corrections and suggestions for improvements to
% Rolf Riesen, Org. 9223, MS 1110, rolf@cs.sandia.gov
%
\documentclass[12pt,strict]{SANDreport}

% If you want to relax some of the SAND98-0730 requirements, use the "relax"
% option. It adds spaces and boldface in the table of contents, and does not
% force the page layout sizes.
% e.g. \documentclass[relax,12pt]{SANDreport}
%
% You can also use the "strict" option, which applies even more of the
% SAND98-0730 guidelines. It gets rid of section numbers which are often
% useful; e.g. \documentclass[strict]{SANDreport}



% ---------------------------------------------------------------------------- %
%
% Set the title, author, and date
%
    \title{An Overview of the Trilinos Project}

    \author{Michael A.~Heroux\\
       Numerical and Applied Mathematics Department \\
	  Sandia National Laboratories\\
	  P.O. Box 5800\\
	  Albuquerque, NM 87185-1110 \\
	  maherou@sandia.gov \\
	 }

    % There is a "Printed" date on the title page of a SAND report, so
    % the generic \date should generally be empty.
    \date{}


% ---------------------------------------------------------------------------- %
% Set some things we need for SAND reports. These are mandatory
%
\SANDnum{SAND2002-xxxx}
\SANDprintDate{August 2002}
\SANDauthor{Michael A.~Heroux, Sandia}


% ---------------------------------------------------------------------------- %
% The following definitions are optional. The values shown are the default
% ones provided by SANDreport.cls
%
\SANDreleaseType{Unlimited Release}


% ---------------------------------------------------------------------------- %
% The following definitions do not have a default value and will not
% print anything, if not defined
%
%\SANDsupersed{SAND00-0000}{January 0000}
\SANDdistcategory{UC-999}	% DOE mandates it, but many reports don't have it


% ---------------------------------------------------------------------------- %
%
% Start the document
%
\begin{document}
    \maketitle

    % ------------------------------------------------------------------------ %
    % An Abstract is required for SAND reports
    %
    \begin{abstract}
The Trilinos Project is an effort to develop parallel solution algorithms within 
an object-oriented software framework for the solution of large-scale, complex
multi-physics 
engineering and scientific applications.   Our emphasis is on developing robust,
scalable algorithms in a
software framework, using abstract interfaces for flexible interoperability of
components and providing
a full-featured set of concrete classes that implement all 
abstract interfaces. Trilinos components are primarily written in C++, but provide
essential C and Fortran user 
interface support.  We provide an open architecture that allows easy integration with
other solver
packages and we deliver our software to the outside community via the Gnu Lesser
General Public License
(LGPL)~\cite{gnu-license-site}.
This report provides an overview of Trilinos, discussing the objectives, history,
current development and
future plans of the project.
    \end{abstract}


    % ------------------------------------------------------------------------ %
    % An Acknowledgement section is optional but important, if someone made
    % contributions or helped beyond the normal part of a work assignment.
    % Use \section* since we don't want it in the table of context
    %
    \clearpage
    \section*{Acknowledgement}
The author would like to acknowledge the support of the ASCI and LDRD programs
that funded development of Trilinos and the talented group of Trilinos contributors: 
Robert Hoekstra, Alan Williams, Richard Lehoucq, James Willenbring, Kevin Long, Tamara
Kolda, Roger Pawlowski, David Day, Ray Tuminaro, Jonathan Hu, Mark Adams and Teri
Barth.



    % ------------------------------------------------------------------------ %
    % The table of contents and list of figures and tables
    % Comment out \listoffigures and \listoftables if there are no
    % figures or tables. Make sure this starts on an odd numbered page
    %
    \clearpage
    \tableofcontents
    \listoffigures
    \listoftables


    % ---------------------------------------------------------------------- %
    % An optional preface or Foreword
%    \clearpage
%    \section{Preface}
%	Although muggles usually have only limited experience with
%	magic, and many even dispute its existence, it is worthwhile
%	to be open minded and explore the possibilities.


    % ---------------------------------------------------------------------- %
    % An optional executive summary
    \clearpage
    \section{Summary}
The Trilinos Project grew out of a set of established efforts at Sandia combined with
a recognition that a modest degree of coordination among the numerical algorithms
efforts at Sandia and elsewhere could have a large positive impact on the quality and
usability of the software we produce.  This document describes the Trilinos project,
focusing on the project philosophy and definitions and
providing the reader with an overview of the project in its current state.  It also
includes brief developer and user guides.


    % ---------------------------------------------------------------------- %
    % An optional glossary. We don't want it to be numbered
    \clearpage
    \section*{Nomenclature}
    \addcontentsline{toc}{section}{Nomenclature}
    \begin{itemize}
	\item[package]
	    a collection of software focused on one primary class of numerical methods
	\item[Trilinos]
	    A Greek term that loosely translated means ``a string of pearls,'' meant
         to evoke an image that each Trilinos package is a pearl in its own right,
         but is even more valuable when combined with other packages.
    \end{itemize}


    % ---------------------------------------------------------------------- %
    % This is where the body of the report begins; usually with an Introduction
    %
    \SANDmain		% Start the main part of the report

\section{Introduction}

\section{Package Design Philosophy}

\section{Epetra and TSF: Two Special Trilinos Packages}

\section{Trilinos Compatibility}

\subsection{Package Accepts User Data as Epetra Objects}
\subsection{Package Callable via TSF Interfaces}
\subsection{Package Uses Epetra Internally}

	\label{Intro}
	    \begin{table}[ht]
		\centering
		\caption[Magical shapes]{This superb table lists a few
		    of the more important magical shapes and some of
		    their properties. Be aware that this condensed list
		    can by no means describe all the properties or
		    shapes in use by modern magic.}
		\bigskip

		\begin{tabular}{|l|c|l|c|}
		    \hline \hline
		    Name  & Number of & Importance & Shape \\
		          & corners   &            &       \\
		    \hline
		    circle & 0        & high       & $\bigcirc$ \\
		    square & 4        & medium     & $\diamond$ \\
		    triangle & 3      & low        & $\triangle$ \\
		    \hline
		\end{tabular}
		\label{tab1}
	    \end{table}

	    \begin{table}[ht]
		\centering
		\caption{A magic square}
		\bigskip

		\begin{tabular}{|c|c|c|c|}
		    \hline
			1 & 15 & 14 & 4 \\ \hline
			12 & 6 & 7 & 9 \\ \hline
			8 & 10 & 11 & 5 \\ \hline
			13 & 3 & 2 & 16 \\ \hline
		\end{tabular}
		\label{tab2}
	    \end{table}


    \section{A Long Section}

	\newcommand{\myblaA}{bla bla bla bla bla bla bla bla bla bla }

    \section{Conclusion}
	Of course, no report would be complete without some conclusions.
	This section is where they would go, if we had some.


    % ---------------------------------------------------------------------- %
    % References
    %
    \clearpage
    \bibliographystyle{plain}
    \bibliography{TrilinosOverview}
    \addcontentsline{toc}{section}{References}


    % ---------------------------------------------------------------------- %
    % Appendices should be stand-alone for SAND reports. If there is only
    % one appendix, put \setcounter{secnumdepth}{0} after \appendix
    %
    \appendix
    \section{Historical Perspective}
	This is an example of an appendix.

	If we follow~\cite{Sand98-0730} strictly, we would have to
	have a separate bibliography section for each appendix.
	The style file doesn't provide that, but it can be done
	using the {\tt bibunits} and {\tt chapterbib} packages.

	If there are many subsections in an appendix, it should also
	have its own table of contents. Again, the SAND report class
	file does not provide that functionality.

	\subsection{The Past a Long Time Ago}
	    This is where we talk about things so old nobody
	    can verify them. We are safe.

	\subsection{The Past More Recently}
	    Now we have to be a little bit more careful, since
	    records exist from that time, and some people still
	    alive actually lived back then.


    \section{Some Other Appendix}
	Just to show what a second Appendix would look like. It contains
	a table. Each appendix is supposed to be self-contained, so
	tables and figures are not supposed to show up in the main
	table of contents. There can be a separate table of contents
	for each appendix.

	\begin{table}[ht]
	    \centering
	    \caption{A small table}
	    \bigskip

	    \begin{tabular}{|c|c|}
		\hline
		    A & B  \\ \hline
		    C & D  \\ \hline
	    \end{tabular}
	    \label{tab3}
	\end{table}

	\begin{figure}[ht]
	    \centering
	    \begin{picture}(50,50)(0,0)
		\put(25,25){\circle{1}}
		\put(25,25){\circle{5}}
		\put(25,25){\circle{10}}
		\put(25,25){\circle{15}}
		\put(25,25){\circle{20}}
		\put(25,25){\circle{25}}
		\put(25,25){\circle{30}}
		\put(25,25){\circle{35}}
		\put(25,25){\circle{40}}
		\put(25,25){\circle{45}}
		\put(25,25){\circle{50}}
	    \end{picture}
	    \caption{Dizzy yet?}
	    \label{fig4}
	\end{figure}

    % \printindex

    \begin{SANDdistribution}
	\SANDdistExternal{1}{An Address\\ 99 $99^{th}$ street NW\\City, State}
	\SANDdistExternal{3}{Some Address\\ and street\\City, State}
	\bigskip
	\SANDdistExternal{12}{Another Address\\ On a street\\City, State\\U.S.A.}


	\SANDdistInternal{1}{1110}{Rolf Riesen}{9223}

	% Housekeeping copies necessary for every unclassified report:
	\SANDdistInternal{1}{9018}{Central Technical Files}{8940-2}
	\SANDdistInternal{2}{0899}{Technical Library}{4916}
	\SANDdistInternal{2}{0619}{Review \& Approval Desk}{4916}

	% If report has a Patent Caution or Patent Interest, add this:
	\SANDdistInternal{3}{0161}{Patent and Licensing Office}{4916}
    \end{SANDdistribution}

\end{document}
