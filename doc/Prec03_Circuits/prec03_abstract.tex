% mail pc03_org@nersc.gov <= July 15th with
\documentclass[10pt,fleqn]{article}
\setlength{\mathindent}{0.25in}
\usepackage{graphicx}
\usepackage{latexsym}

%*** Column setup
%\setlength{\columnsep}{0.5in}
%\setlength{\columnseprule}{0.01in}

\pagestyle{plain}

%*** Page numbering
%\pagenumbering{arabic}
%\setcounter{page}{1}

%*** Paragraph setup

\setlength{\parindent}{0ex}
\setlength{\parskip}{01.5ex plus 0.3ex minus 0.3ex}
%\renewcommand{\baselinestretch}{1.2}
\renewcommand{\baselinestretch}{1.0}
%\renewcommand{\baselinestretch}{1.5}

%*** Floats
\renewcommand{\textfraction}{0.01}
\setlength{\floatsep}{0.0in}
%\setlength{\textfloatsep}{0.0in}

%*** Page formating

\setlength{\topmargin}{0.0in}
\setlength{\headheight}{0.3in}
\setlength{\headsep}{0.2in}
%\setlength{\textheight}{8.0in}
%\setlength{\footskip}{1in}
\setlength{\textheight}{8.5in}
\setlength{\footskip}{0.5in}

\setlength{\oddsidemargin}{-0.5in}
\setlength{\evensidemargin}{-0.5in}
\setlength{\textwidth}{7in}

%
% Body of document
%
\begin{document}

%
% Title page
%
\title{Preconditioning Techniques in Circuit Simulation}

\author{David M. Day, Michael A. Heroux, Robert J. Hoekstra \\
Sandia National Laboratories\footnote{
Sandia is a multiprogram laboratory operated by Sandia Corporation, a
Lockheed-Martin Company, for the United States Department of Energy
under Contract DE-AC04-94AL85000.}, Albuquerque NM 87185 USA}

\date{\today}

\maketitle

\paragraph{Summary}
This article contains a discussion of
preconditioners for linear systems arising in the 
distributed memory circuit simulation application called
Xyce~\cite{Xyce}. Xyce is under development at Sandia National Laboratories to
meet Sandia's current and future circuit simulation needs.
The computational bottleneck in circuit simulation is the
solution of a sequence of sparse indefinite unsymmetric linear systems.
Iterative linear solvers and distributed memory platforms
are needed to meet the demand to simulate circuits with increasing
numbers of devices.  

In this article we present our approaches to
preconditioning these problems.  We organize the discussion to match
our evolutionary approach to developing preconditioners.
Specifically, we first discuss our early efforts to sove circuit problems
using well-established methods from PDE applications and then proceed
to how we specialized these methods for circuits, and explored
new algorithms that address key subproblems in circuit simulation.

The assumptions regarding preconditioning,
overlap, and load balance differ significantly from discretized
partial differential equations.

The problem is defined and standard approaches are reviewed.  The
limitations of well known preconditioners for circuit problems are
outlined.  Next an adaptive preconditioning algorithm is presented.

Circuit simulation refers to nonlinear transient analyses by implicit
differential-algebraic equation solvers using Newton's method.
There is an initial direct current (DC) operating point calculation.

DC operating point problem is notoriously difficult, but accounts for
little of the overall computation time.  DC operating point calculations 
involving 1000 Newton iterations are routine.  We present linear solvers
that exploit the structure of the DC operating point problem.

\paragraph{Background}
The circuit simulation community relies on sparse direct linear solvers
(~\cite{Ksparse86}) and special Homotopy methods for each device model
(~\cite{MTFW93}).  We do not discuss algorithms that involve modifying the
devices models.  Recently Schur complement preconditioners~\cite{SaadSosonkina00}
 for circuit simulation have appeared~\cite{BCGJH03}.

Domain decomposition preconditioners with partition inherited from a
graph partitioner (e.g. Metis).  A domain decomposition preconditioner
depends on the partition.  Though the dependence is in general complex, 
successful partitioning is a necessary condition for the preconditioning
problems to be tractable.  It is noteworthy that circuit networks 
are not well partitioned by contemporary graph partitioners.

Some of the linear systems are singular for several reasons.  The
nonlinear function is not convex, and has local minima.  Furthermore,
in finite precision arithmetic evaluating device models far from
equilibria is challenging.

\paragraph{Experiments with Standard Preconditioners}
In our initial experiments with circuit simulation, preconditioner
existence, not to mention effectiveness, was often not achievable.
We were forced to rely on the high dimensional Krylov subspaces available
on distributed memory platforms.  Overlapping Schwarz preconditioners
do not apply to matrices with a dense row or column; one subdomain
overlaps the entire matrix.  Also we observed that incomplete
factorizations did not exist due to zero diagonals.  Allowing more
fill-in through thresholds was not an efficient solution.  Using the
Duff-Koster permutation to maximize product of absolute values of the
diagonal entries, we were first able to reliably compute preconditioners. 
We found that a mechanism to regularize the problem is also needed.

\paragraph{Adaptive Preconditioners}
Using permutations and a dual parameter regularization technique, we
significantly increased the range of problems we were able to solve.
However, the regularization technique we used was ad hoc.  Selection
of threshold parameters was by trial-and-error.  On approach to
improve the situation is to introduce an adaptive procedure.
C. diagonal perturbation: 
1. threshold on diagonal entries, (relative and absolute tolerances)
2. Usually will regularize the problem 

D. Seek ILUT parameters adaptively based on information from 
1. condest of preconditioner,
2. projection of preconditioned matrix along Krylov subspace,
3. information from previous linear solves (residuals, stagnation)

\paragraph{Eliminate singleton rows and columns (recursively)}
first suggested by  Basermann

\paragraph{DC Operating Point Calculations}
Determining the DC operating point is a steady-state problem that
produces structured Jacobians.  The Jacobians admit a permutation to
block triangular form with small diagonal blocks.  The Dulmage-Mendelsohn
permutation (linear time and memory costs) finds the permutation.

For example, a digital component comprised mainly of MOSFET transistors
is reducible due to the MOSFET's asymmetry.  A MOSFET 'gate' terminal
controls the currents and conductivity for the 'drain', 'source', 
and 'base'.  In steady-state problems through the 'gate' terminal
no current flows and conductivity is zero.  The stencil for the MOSFET
contains a zero row corresponding to the gate.  After permuting the
Jacobian  to lower triangular form, the diagonal blocks are connected
through columns that correspond to the 'gate' terminals.  Transient
problems involve capacitances that generate currents and conductivities
associated with the 'gate', and the Jacobians are irreducible.

The DC operating point Jacobians are often singular.  Analyses using
the singular value decompositions of the diagonal blocks in combination
with condition number estimates for the block triangular system show
that the the diagonal blocks usually support the gravest singular vectors.
Management of rank deficient systems through least squares methods
enhances the convergence of both the linear and the nonlinear solver.
\bibliographystyle{plain}
\nopagebreak
\scriptsize
\bibliography{circuits}
\end{document}

