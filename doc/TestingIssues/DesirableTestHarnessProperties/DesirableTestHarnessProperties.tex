\documentclass[pdf,ps2pdf,11pt]{SANDreport}
\usepackage{pslatex}
%Local stuff
\usepackage{graphicx}
\usepackage{latexsym}
%\usepackage{color}
\usepackage[all]{draftcopy}
%
% Command to print 1/2 in math mode real nice
%
\newcommand{\myonehalf}{{}^1 \!\!  /  \! {}_2}

%
% Command to print over/under left aligned in math mode
%
\newcommand{\myoverunderleft}[2]{ \begin{array}{l} #1 \\ \scriptstyle #2 \end{array} }

%
% Command to number equations 1.a, 1.b etc.
%
\newcounter{saveeqn}
\newcommand{\alpheqn}{\setcounter{saveeqn}{\value{equation}}
\stepcounter{saveeqn}\setcounter{equation}{0}
\renewcommand{\theequation}
	{\mbox{\arabic{saveeqn}.\alph{equation}}}}
\newcommand{\reseteqn}{\setcounter{equation}{\value{saveeqn}}
\renewcommand{\theequation}{\arabic{equation}}}

%
% Shorthand macros for setting up a matrix or vector
%
\newcommand{\bmat}[1]{\left[ \begin{array}{#1}}
\newcommand{\emat}{\end{array} \right]}

%
% Command for a good looking \Re in math enviornment
%
\newcommand{\RE}{\mbox{\textbf{I}}\hspace{-0.6ex}\mbox{\textbf{R}}}

%
% Commands for Jacobians
%
\newcommand{\Jac}[2]{\displaystyle{\frac{\partial #1}{\partial #2}}}
\newcommand{\jac}[2]{\partial #1 / \partial #2}

%
% Commands for Hessians
%
\newcommand{\Hess}[2]{\displaystyle{\frac{\partial^2 #1}{\partial #2^2}}}
\newcommand{\hess}[2]{\partial^2 #1 / \partial #2^2}
\newcommand{\HessTwo}[3]{\displaystyle{\frac{\partial^2 #1}{\partial #2 \partial #3}}}
\newcommand{\hessTwo}[3]{\partial^2 #1 / (\partial #2 \partial #3)}



%\newcommand{\Hess2}[3]{\displaystyle{\frac{\partial^2 #1}{\partial #2 \partial #3}}}
%\newcommand{\myHess2}{\frac{a}{b}}
%\newcommand{\hess2}[3]{\partial^2 #1 / (\partial #2 \partial #3)}

%
% Shorthand macros for setting up a single tab indent
%
\newcommand{\bifthen}{\begin{tabbing} xxxx\=xxxx\=xxxx\=xxxx\=xxxx\=xxxx\= \kill}
\newcommand{\eifthen}{\end{tabbing}}

%
% Shorthand for inserting four spaces
%
\newcommand{\tb}{\hspace{4ex}}

%
% Commands for beginning and ending single spacing
%
\newcommand{\bsinglespace}{\renewcommand{\baselinestretch}{1.2}\small\normalsize}
\newcommand{\esinglespace}{}



\raggedright

% If you want to relax some of the SAND98-0730 requirements, use the "relax"
% option. It adds spaces and boldface in the table of contents, and does not
% force the page layout sizes.
% e.g. \documentclass[relax,12pt]{SANDreport}
%
% You can also use the "strict" option, which applies even more of the
% SAND98-0730 guidelines. It gets rid of section numbers which are often
% useful; e.g. \documentclass[strict]{SANDreport}

% ---------------------------------------------------------------------------- %
%
% Set the title, author, and date
%

\title{\center
Desirable Properties of a Test Harness for Complex Computational Software}
\author{
Roscoe A. Bartlett \\ Department of Optimization and Uncertainty Estimation \\ \\
Sandia National Laboratories, Albuquerque NM 87185-1318 USA \\ }

% ---------------------------------------------------------------------------- %
% Set some things we need for SAND reports. These are mandatory
%
\SANDnum{SAND2008-xxx}
\SANDprintDate{October 2008}
\SANDauthor{Roscoe A. Bartlett}

% ---------------------------------------------------------------------------- %
% The following definitions are optional. The values shown are the default
% ones provided by SANDreport.cls
%
\SANDreleaseType{Unlimited Release}
%\SANDreleaseType{Not approved for general release}

% ---------------------------------------------------------------------------- %
% The following definition does not have a default value and will not
% print anything, if not defined
%
%\SANDsupersed{SAND1901-0001}{January 1901}

% ---------------------------------------------------------------------------- %
%
% Start the document
%
\begin{document}

\maketitle

%% ------------------------------------------------------------------------ %
%% An Abstract is required for SAND reports
%%
%
%%\clearpage
%
%%
%\begin{abstract}
%%
%
%Blah blah blah ...
%
%%
%\end{abstract}
%%

%% ------------------------------------------------------------------------ %
%% An Acknowledgement section is optional but important, if someone made
%% contributions or helped beyond the normal part of a work assignment.
%% Use \section* since we don't want it in the table of context
%%
%\clearpage
%\section*{Acknowledgments}
%
%Blah blah blah ...

%
%The format of this report is based on information found
%in~\cite{Sand98-0730}.

% ------------------------------------------------------------------------ %
% The table of contents and list of figures and tables
% Comment out \listoffigures and \listoftables if there are no
% figures or tables. Make sure this starts on an odd numbered page
%
\clearpage
\tableofcontents
%\listoffigures
%\listoftables

% ---------------------------------------------------------------------- %
% An optional preface or Foreword
%\clearpage
%\section{Preface}
%Although muggles usually have only limited experience with
%magic, and many even dispute its existence, it is worthwhile
%to be open minded and explore the possibilities.

% ---------------------------------------------------------------------- %
% An optional executive summary

%\clearpage

%\section{Executive Summary}

% ---------------------------------------------------------------------- %
% An optional glossary. We don't want it to be numbered
%\clearpage
%\section*{Nomenclature}
%\addcontentsline{toc}{section}{Nomenclature}
%\begin{itemize}
%\item[alohomora]
%spell to open locked doors and containers
%\end{itemize}

% ---------------------------------------------------------------------- %
% This is where the body of the report begins; usually with an Introduction
%


\SANDmain % Start the main part of the report


%
{}\section{Introduction}
%

In this document, we describe some of desirable properties for test
harness software for computational software.


%
{}\section{Desirable Properties for individual tests}
%

There are several desirable features for individual tests to posses in
order to maximize the effectiveness of a testing process:

\begin{itemize}

{}\item\textit{Tests should check their own results and return
``passed'' or ``failed''}: All tests must check themselves in some way
to determine if they succeeded or not.  Also, it is a good idea to
print and check for an explicit ``FINAL: Passed'' or ``FINAL: Failed''
and not just rely on the return code from the test as the return code
has been found to be unreliable in some MPI implementations.

{}\item\textit{Tests should report why they fail with as much detail
as is reasonable}: Instead of a test just printing ``Failed'' or a
nonzero return value, every test should print out exactly why the test
failed in as much detail as is needed to diagnose the test failure.
For example, if a test failed because a tolerance was not achieved,
the test should print out the details of the test like the name/value
of the error computation, and the name/value of the error tolerance
variable the error is compared to.  Also, for example, if a test fails
because a solver took too many iterations, then the number of
iterations taken must be printed as well as the value/name of the
maximum number of allowed iterations.

{}\item\textit{Tests should provide enough ``knobs'' so that they can
be specialized for different platforms or other special cases}: For
example, a test should provide an --error-tol=VAL command-line
argument that can be adjusted by the test harness for different
platforms.

{}\item\textit{???}:

\end{itemize}


%
{}\section{Desirable properties for a test harness}
%

There are a number of important properties that a test harness should
have to provide maximum value from the testing process.

%
\subsection{Test reporting and notification}
%

Desirable properties for test reporting and notification include:

\begin{itemize}

{}\item\textit{Test results should be posted to a results web page
(dashboard) to allow all to see}: A (semi-)public test results
dashboard provides visibility for the development effort and should be
the primary way to start to diagnose failed builds and tests.

{}\item\textit{Sufficient information should be accessible from the
dashboard to determine exactly why a test failed}: A developer should
not have to go to a different machine or build and run the code
themselves to try to hunt down the output from the test to determine
why the test is failing.

{}\item\textit{Sufficient information for passing tests should be
available to compare against failing tests}: It is not enough to just
provide information for failing tests.  Often, to diagnose a failing
test, you also need to see what the test output looked like when it
passed.  Often, that type of information is critical in helping to
diagnose and fix failing tests.

{}\item\textit{Archive and allow for easy access to older test results
for a sufficient period of time}: It is often the case that failing
tests can not be addressed immediately and may need to be removed from
the test harness (i.e.\ by disabling them and then putting a new story
on the Sprint Backlog or Product Backlog) until a later time when they
can be fixed.  In this situation, it is very important to be able to
archive the test results on consecutive days where a test went from
passing to failing to be able to diagnose and fix the problem.  In
order to not fill up disk storage, a process must be implemented that
will selectively prune out old test results and only keep critical
data needed to diagnose failing builds and tests.

{}\item\textit{Notification of critical failed builds and tests should
be pushed out to everyone who needs to know}: There are builds and
tests that must work in order to support the basic functioning of a
development team.  For example, if a main part of the code does not
build on the primary development platforms where developers do their
development work, then no-one can checkout or checkin any code and
development comes to a halt.  Also, even if the build succeeds, if a
critical foundational functionality of the code is broken, then no-one
can develop new functionality and development stops.  For these types
of critical build and test failures, everyone on the development team
should get some type of automatic notification that there is a problem
so that it can be addressed as soon a possible.  The form of
notification is typically email but other more drastic forms of
notification my be appropriate in many cases (see {}\cite[Chapter
9]{book:continuous-integration}).  For example, every continuous
integration (CI) system must support automatic notifications or it is
not a very useful form of CI (again, see
{}\cite{book:continuous-integration}).  Note that it may not be
appropriate to send out automatic notifications for every secondary
platform and test as these can be broken much more frequently than on
the primary development platforms.
  
{}\item\textit{???}:

\end{itemize}

%
\subsection{Other test harness features}
%

Desirable properties for other miscellaneous testing properties
include:

\begin{itemize}

{}\item\textit{The test harness should use as much of the same
infrastructure for building and running tests that is used in
day-to-day development and deployments as possible}: Major problems
can occur when there is even what might appear to be minor differences
between the automated testing environment and the environments used
for day-to-day development and deployment of the software.  If the
environments are different, then developers may get all passing tests
while the automated testing environment shows failed builds and/or
failing tests.  Whats worse, if there is a mismatch in the automated
testing and deployment environments, then broken object code and
executables may be distrubuted to users.  The best way to ensure that
the development, testing, and deployment environments are the same is
to use the exact same set of tools for all three (see
{}\cite{book:continuous-integration}).

{}\item\textit{The test harness should allow for targeted testing
criteria for different platforms}: One approach to port a test suite
is the ability to allow different tests properties for different
platforms or even entire tests to be targeted or excluded from
specific platforms.

{}\item\textit{???}:

\end{itemize}


% ---------------------------------------------------------------------- %
% References
%
\clearpage
\bibliographystyle{plain}
\bibliography{references}
\addcontentsline{toc}{section}{References}


%% ---------------------------------------------------------------------- %
%% Appendices should be stand-alone for SAND reports. If there is only
%% one appendix, put \setcounter{secnumdepth}{0} after \appendix
%%
%\appendix
%
%%
%\section*{???}
%\label{sec:checkist}
%%



\begin{SANDdistribution}[NM]
% \SANDdistCRADA	% If this report is about CRADA work
% \SANDdistPatent	% If this report has a Patent Caution or Patent Interest
% \SANDdistLDRD	% If this report is about LDRD work
% External Address Format: {num copies}{Address}
%\SANDdistExternal{}{}
%\bigskip
%% The following MUST BE between the external and internal distributions!
%\SANDdistClassified % If this report is classified
% Internal Address Format: {num copies}{Mail stop}{Name}{Org}
%\SANDdistInternal{}{}{}{}
% Mail Channel Address Format: {num copies}{Mail Channel}{Name}{Org}
%\SANDdistInternalM{}{}{}{}
%\SANDdistInternal{2}{MS 9018}{Central Technical Files}{8944}
%\SANDdistInternal{2}{MS 0899}{Technical Library}{4536}
\end{SANDdistribution}

\end{document}
