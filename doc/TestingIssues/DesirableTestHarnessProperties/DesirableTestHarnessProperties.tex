\documentclass[pdf,ps2pdf,11pt]{SANDreport}
\usepackage{pslatex}
%Local stuff
\usepackage{graphicx}
\usepackage{latexsym}
%\usepackage{color}
\usepackage[all]{draftcopy}
\input{rab_commands}

\raggedright

% If you want to relax some of the SAND98-0730 requirements, use the "relax"
% option. It adds spaces and boldface in the table of contents, and does not
% force the page layout sizes.
% e.g. \documentclass[relax,12pt]{SANDreport}
%
% You can also use the "strict" option, which applies even more of the
% SAND98-0730 guidelines. It gets rid of section numbers which are often
% useful; e.g. \documentclass[strict]{SANDreport}

% ---------------------------------------------------------------------------- %
%
% Set the title, author, and date
%

\title{\center
Desirable Properties of a Test Harness for Complex Computational Science Software}
\author{
Roscoe A. Bartlett \\ Department of Optimization and Uncertainty Estimation \\ \\
Sandia National Laboratories, Albuquerque NM 87185-1318 USA \\ }

% ---------------------------------------------------------------------------- %
% Set some things we need for SAND reports. These are mandatory
%
\SANDnum{SAND2008-xxx}
\SANDprintDate{October 2008}
\SANDauthor{Roscoe A. Bartlett}

% ---------------------------------------------------------------------------- %
% The following definitions are optional. The values shown are the default
% ones provided by SANDreport.cls
%
\SANDreleaseType{Unlimited Release}
%\SANDreleaseType{Not approved for general release}

% ---------------------------------------------------------------------------- %
% The following definition does not have a default value and will not
% print anything, if not defined
%
%\SANDsupersed{SAND1901-0001}{January 1901}

% ---------------------------------------------------------------------------- %
%
% Start the document
%
\begin{document}

\maketitle

%% ------------------------------------------------------------------------ %
%% An Abstract is required for SAND reports
%%
%
%%\clearpage
%
%%
%\begin{abstract}
%%
%
%Blah blah blah ...
%
%%
%\end{abstract}
%%

%% ------------------------------------------------------------------------ %
%% An Acknowledgement section is optional but important, if someone made
%% contributions or helped beyond the normal part of a work assignment.
%% Use \section* since we don't want it in the table of context
%%
%\clearpage
%\section*{Acknowledgments}
%
%Blah blah blah ...

%
%The format of this report is based on information found
%in~\cite{Sand98-0730}.

% ------------------------------------------------------------------------ %
% The table of contents and list of figures and tables
% Comment out \listoffigures and \listoftables if there are no
% figures or tables. Make sure this starts on an odd numbered page
%
\clearpage
\tableofcontents
%\listoffigures
%\listoftables

% ---------------------------------------------------------------------- %
% An optional preface or Foreword
%\clearpage
%\section{Preface}
%Although muggles usually have only limited experience with
%magic, and many even dispute its existence, it is worthwhile
%to be open minded and explore the possibilities.

% ---------------------------------------------------------------------- %
% An optional executive summary

%\clearpage

%\section{Executive Summary}

% ---------------------------------------------------------------------- %
% An optional glossary. We don't want it to be numbered
%\clearpage
%\section*{Nomenclature}
%\addcontentsline{toc}{section}{Nomenclature}
%\begin{itemize}
%\item[alohomora]
%spell to open locked doors and containers
%\end{itemize}

% ---------------------------------------------------------------------- %
% This is where the body of the report begins; usually with an Introduction
%


\SANDmain % Start the main part of the report


%
{}\section{Introduction}
%

In this document, we describe some of desirable properties for test
harness software for computational software.


%
{}\section{Desirable Properties for individual tests}
%

There are several desirable features individual tests to posses in
order to maximize the effectiveness of a testing process:

\begin{itemize}

{}\item\textit{Tests should check their own results and return
``passed'' or ``failed''}: All tests must check themselves in some way
to determine if they succeeded or not.  Also, it is a good idea to
print an explicit ``FINAL: Passed'' or ``FINAL: Failed'' and not just
rely on the return code from the test as the return code has been
found to be unreliable in some MPI implementation.

{}\item\textit{Tests should report why they fail}: Instead of a test
just returning ``Failed'' or a nonzero return value, they should print
to STDOUT exactly why the test failed the best they can.  For example,
if a test failed because a tolerance was not achieved, the test should
print out the details of the test like the name/value of the error
computation, and the name/value of the error tolerance variable.
Also, for example, if a test fails because a solver took too many
iterations, then all the number of iterations taken must be printed as
well as the maximum number and the name of the command-line option
that specifies this.

{}\item\textit{Tests should provide enough ``knobs'' so that they can
be specialized for different platforms}: For example, a test should
provide an --error-tol=VAL command-line argument that can be adjusted
by the test harness for different platforms.

{}\item\textit{???}:

\end{itemize}


%
{}\section{Desirable properties for a test harness}
%

There are a number of important properties that a test harness should
have to provide maximum value from the testing process.

%
\subsection{Test reporting and notification}
%

Desirable properties for test reporting and notification include:

\begin{itemize}

{}\item\textit{Test results should be posted to a results web page
(dashboard) to allow all to see}: A (semi)public test results
dashboard provides visibility for the development effort and should be
the primary way to start to diagnose failed builds and failed tests.

{}\item\textit{Sufficient information should be accessible from the
dashboard to determine exactly why a test failing}: A developer should
not have to go to a different machine to try to hunt down the output
from the test to determine why the test is failing.

{}\item\textit{Sufficient information for passing tests should be
available to compare against failing tests}: It is not enough to just
provide information for failing tests.  Often, to diagnose a failing
test, you also need to see what the test output looked like when it
passed.  Often, that type of information is critical in helping to
diagnose and fix failing tests.

{}\item\textit{Archive and allow for easy access to test results for a
sufficient period of time}: It is often the case failing tests can not
be addressed immediately and may need to be removed from the test
harness until they can be fixed later.  In this case, it is very
important to be able to archive the test results on consecutive days
where the test went from passing to failing to be able to diagnose and
fix the problem.  In order to not fill up disk storage, a process must
be implemented that will selectively prune out old test results and
only keep critical data needed to diagnose failing builds and tests.

{}\item\textit{Notification of critical failed builds and tests should
be pushed out to everyone who needs to know}: There are builds and
tests that must work in order to support the basic functioning of a
development team.  For example, if a main part of the code does not
build on the primary development platforms where developers do their
development work, then no-one can checkout or checkin any code and
development comes to a halt.  Also, even if the build succeeds, if a
critical foundational functionality of the code is broken, then no-one
can develop new functionality and development stops.  For these types
of critical build and test failures, everyone on the development team
should get some type of automatic notification that there is a problem
so that it can be addressed as soon a possible.  The form of
notification is typically email but other more drastic forms of
notification my be appropriate in many cases (see
{}\cite{book:continuous-integration}).  For example, every continuous
integration (CI) system must support automatic notifications or it is
not a very useful form of CI (again, see
{}\cite{book:continuous-integration}).  Note that it may not be
appropriate to send out automatic notifications for every secondary
platform and test as these can be broken much more frequently than on
the primary development platforms.
  
{}\item\textit{???}:

\end{itemize}

%
\subsection{Other test harness features}
%

Desirable properties for other miscellaneous testing properties
include:

\begin{itemize}

{}\item\textit{The test harness should allow for targeted testing
criteria for different platforms}: One approach to achieve a portable
test suite is the ability to allow different tests properties or even
entire tests to be targeted or excluded from specific platforms.

{}\item\textit{???}:

\end{itemize}


% ---------------------------------------------------------------------- %
% References
%
\clearpage
\bibliographystyle{plain}
\bibliography{references}
\addcontentsline{toc}{section}{References}


%% ---------------------------------------------------------------------- %
%% Appendices should be stand-alone for SAND reports. If there is only
%% one appendix, put \setcounter{secnumdepth}{0} after \appendix
%%
%\appendix
%
%%
%\section*{???}
%\label{sec:checkist}
%%



\begin{SANDdistribution}[NM]
% \SANDdistCRADA	% If this report is about CRADA work
% \SANDdistPatent	% If this report has a Patent Caution or Patent Interest
% \SANDdistLDRD	% If this report is about LDRD work
% External Address Format: {num copies}{Address}
%\SANDdistExternal{}{}
%\bigskip
%% The following MUST BE between the external and internal distributions!
%\SANDdistClassified % If this report is classified
% Internal Address Format: {num copies}{Mail stop}{Name}{Org}
%\SANDdistInternal{}{}{}{}
% Mail Channel Address Format: {num copies}{Mail Channel}{Name}{Org}
%\SANDdistInternalM{}{}{}{}
%\SANDdistInternal{2}{MS 9018}{Central Technical Files}{8944}
%\SANDdistInternal{2}{MS 0899}{Technical Library}{4536}
\end{SANDdistribution}

\end{document}
